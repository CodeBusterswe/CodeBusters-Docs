\section{Introduzione}
\subsection{Scopo del documento}
Lo scopo di questo documento è descrivere e motivare tutte le scelte architetturali che il gruppo \Gruppo{} ha deciso di fare nella fase di progettazione e codifica del prodotto. Vengono quindi riportati i diagrammi delle classi, dei package e di sequenza per descrivere architettura e funzionalità principali del prodotto.
È poi presente una sezione dedicata ai requisti che il gruppo è riuscito a soddisfare in ingresso alla RQ, così da fornire un'ampia visione sullo stato di avanzamento del lavoro.

\subsection{Scopo del prodotto}
Oggigiorno, anche i programmi più tradizionali gestiscono e memorizzano una grande mole di dati; di conseguenza servono software in grado di eseguire un'analisi e un'interpretazione delle informazioni.\\
Il capitolato C4 ha come obiettivo quello di creare un'applicazione di visualizzazione di dati con numerose dimensioni in modo da renderle comprensibili all'occhio umano.  Lo scopo del prodotto sarà quello di fornire all'utente diversi tipi di visualizzazioni e di algoritmi per la riduzione dimensionale in modo che, attraverso un processo esplorativo, l'utilizzatore del prodotto possa studiare tali dati ed evidenziarne degli eventuali cluster.

\subsection{Riferimenti}
\subsubsection{Riferimenti normativi}
\begin{itemize}
	\item	Capitolato d'appalto C4 - HD Viz: visualizzazione di dati multidimensionali: \\
	\textcolor{blue}{\url{https://www.math.unipd.it/~tullio/IS-1/2020/Progetto/C4.pdf}}
\end{itemize}

\subsubsection{Riferimenti informativi}
\begin{itemize}
	\item \textbf{Slide E1 del corso di Ingengeria del Software - Diagrammi delle classi e dei package}
	\item \textbf{Slide E2 del corso di Ingengeria del Software - Diagrammi delle attività e di sequenza}
	\item \textbf{Slide E10 del corso di Ingengeria del Software - Design pattern comportamentali}
		\begin{itemize}
			\item Da slide 22 a slide 40 (Observer Pattern, Strategy Pattern)
		\end{itemize}
	\item \textbf{Slide L02 del corso di Ingengeria del Software - Design pattern architetturali: Model View Controller e derivati}
\end{itemize}