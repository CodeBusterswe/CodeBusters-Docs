\section{Casi d'uso}
\subsection{Scopo}
Lo scopo di questa sezione è la descrizione in elenco di tutti i casi d'uso individuati dal gruppo, in riferimento alle funzionalità dell'applicazione.
\subsection{Attori}
Come accordato con il proponente, non essendo richiesto alcun servizio di autenticazione attraverso un login o una registrazione, è presente un solo attore che può interagire con l'applicazione web:

\begin{figure}[h]
\includegraphics[width=1cm]{section/Images/Utente.png}
\centering
\caption{Gerarchia attori}
\end{figure}

\begin{description}
\item[Utente]:
Si riferisce all'utente utilizzatore che può accedere alla piattaforma.
\end{description}
Per un eventuale gestione di dati in più sessioni è quindi richiesta la funzionalità di poter salvare il proprio lavoro in un file scaricabile, che può poi essere successivamente caricato sulla piattaforma permettendo la ripresa del lavoro.
\subsection{Elenco casi d'uso}
Di seguito sono riportati tutti i casi d'uso che coinvolgono l'utente:
\begin{figure}[h]
%\includegraphics[width=\linewidth]{section/Images/HDviz.png}
\centering
\caption{Casi d'uso dell'utente}
\end{figure}
\subsubsection{UC1 - Inserimento dati}
\begin{figure}[h]
\includegraphics[width=\linewidth]{Section/Images/UC1InserimentoDati.png}
\centering
\caption{UC1 - Inserimento dati}
\end{figure}
\begin{itemize}
	\item \textbf{Attore primario}: Utente.
	\item \textbf{Precondizioni}: Il sistema è raggiungibile e funzionante.
	\item \textbf{Postcondizioni}: I dati sono stati inseriti nel database.
	\item \textbf{Scenario principale}:
		\begin{enumerate}
			\item L'utente seleziona "inserisci dati";
			\item L'utente può decidere se:
			\begin{enumerate}
			\item caricare un file contenente dati [UC1.1].
			\item scrivere i dati a mano [UC1.2].
			\end{enumerate}
		\end{enumerate}
	\item \textbf{Estensioni}:
\end{itemize}
\subsection{UC2 - Selezione delle dimensioni da utilizzare}
\begin{itemize}
	\item \textbf{Attore primario}: Utente;
	\item \textbf{Precondizioni}: L'utente ha caricato i dati nel sistema [UC1];
	\item \textbf{Postcondizioni}: Le dimensioni scelte vengono aggiornate nel sistema e i dati sono pronti per essere visualizzati [UC6];
	\item \textbf{Scenario principale}:
		\begin{enumerate}
			\item All'utente viene presentata una schermata con tutte le dimensioni presenti nel dataset caricato già selezionate di default;
			\item Per ogni dimensione è presente una cella da selezionare nel caso la si voglia utilizzare o meno;
			\item L'utente seleziona le dimensioni che desidera analizzare.
		\end{enumerate}
	\item \textbf{Estensioni:}
		\begin{enumerate}[(a)]
			\item Nel caso in cui l'utente non abbia selezionato nessuna dimensione:
			\begin{enumerate}[1.]
				\item Le dimensioni non vengono aggiornate nel sistema;
				\item Viene visualizzato un messaggio d'errore esplicativo [UC12].
			\end{enumerate}
		\end{enumerate}
\end{itemize}
\subsection{UC3 - Riduzione dimensionale}
\begin{figure}[h]
\includegraphics[width=14cm]{section/Images/UC3.png}
\centering
\caption{UC3 - Riduzione dimensionale}
\end{figure}
\begin{itemize}
	\item \textbf{Attore primario}: Utente;
	\item \textbf{Precondizioni}: L'utente ha caricato i dati e le dimensioni nel sistema [UC1];
	\item \textbf{Postcondizioni}: Le nuove dimensioni vengono inserite nel sistema e sono disponibili all'utente per la visualizzazione [UC5];
	\item \textbf{Scenario principale}: L'utente può creare nuove dimensioni, a partire da quelle caricate [UC1] ed eventualmente scremate [UC2] o prodotte da iterazioni precedenti, tramite:
	\begin{enumerate}[1.]
		\item Algoritmo di riduzione dimensionale [UC3.1];
		\item Calcolo delle distanze tra i valori delle dimensioni [UC3.2].
	\end{enumerate}
	L'utente potrà selezionare le dimensioni interessate dalla riduzione tramite apposite celle per poi premere il tasto di conferma; le nuove dimensioni saranno visualizzate insieme alle precedenti per eventualmente procedere in modo iterativo.
\end{itemize}
\subsubsection{UC4 - Impostazione dei parametri di personalizzazione}
\begin{figure}[h]
\includegraphics[width=\linewidth]{section/Images/UC4.png}
\centering
\caption{UC4 - Impostazione dei parametri di personalizzazione}
\end{figure}
\begin{itemize}
	\item \textbf{Attore primario}: Utente.
	\item \textbf{Precondizioni}: L'utente ha caricato dei dati nel sistema [UC1], ha scelto le dimensioni da utilizzare [UC2] e ha scelto la visualizzazione che desidera vedere [UC3].
	\item \textbf{Postcondizioni}: Viene aggiornata la visualizzazione scelta con i nuovi parametri impostati dall'utente.
	\item \textbf{Scenario principale}:
		\begin{enumerate}
			\item Viene presentata all'utente una sezione per apportare delle modifiche ai parametri relativi alla visualizzazione scelta;
			\item L'utente decide le dimensioni da visualizzare e le altre funzionalità specifiche del grafico. Se queste non dovessero essere modificate, verranno applicati valori di default tipici per ogni visualizzazione;
			\item Al termine, l'utente dovrà premere sul pulsante per la conferma e l'invio delle nuove preferenze.
		\end{enumerate}
\end{itemize}
\subsection{UC5 - Scelta della \glo{visualizzazione}}
%\begin{figure}[h]
%\includegraphics[width=\linewidth]{section/Images/UC5.png}
%\centering
%\caption{UC5 - Scelta della visualizzazione}
%\end{figure}
\begin{itemize}
	\item \textbf{Attore primario}: Utente.
	\item \textbf{Precondizioni}: L'utente ha caricato dei dati nel sistema [UC1] e ha selezionato le dimensioni da utilizzare [UC2].
	\item \textbf{Postcondizioni}: Viene mostrata la visualizzazione scelta, con possibilità di personalizzazione. La scelta viene salvata nel sistema.
	\item \textbf{Scenario principale}: L'utente seleziona la visualizzazione che vuole utilizzare, tra quelle disponibili.
	\item \textbf{Generalizzazioni}: L'utente seleziona una delle seguenti opzioni:
		\begin{enumerate}
			\item \glo{\textit{Scatter Plot Matrix}} [UC5.1]
			\item \glo{\textit{Heat Map}} [UC5.2]
			\item \glo{\textit{Force Field}} [UC5.3]
			\item \glo{\textit{Proiezione Lineare Multi Asse}} [UC5.4]
		\end{enumerate}

	\item \textbf{Estensioni}:
	\begin{enumerate}[(a)]
		\item Nel caso in cui non è stata scelta alcuna dimensione:
		\begin{enumerate}[1.]
			\item Il grafico non viene visualizzato;
			\item Viene visualizzato un errore esplicativo [UC14].
		\end{enumerate}
	\end{enumerate}
\end{itemize}
\subsubsection{UC5.1 Scelta visualizzazione Scatter Plot Matrix}
\begin{figure}[h]
\centering
\caption{}
\end{figure}
\begin{itemize}
	\item \textbf{Attore primario}: Utente.
	\item \textbf{Precondizioni}:
	\item \textbf{Postcondizioni}:
	\item \textbf{Scenario principale}:
		\begin{enumerate}
			\item L'utente seleziona la visualizzazione Scatter Plot Matrix
		\end{enumerate}
	\item \textbf{Estensioni}:
	\begin{enumerate}[(a)]
		\item Nel caso in cui non è stato caricato alcun dato o non è stata scelta alcuna dimensione:
		\begin{enumerate}[1.]
			\item il grafico non viene visualizzato;
			\item viene visualizzato un errore esplicativo [UCx.3].
		\end{enumerate}
	\end{enumerate}
\end{itemize}
\subsubsection{UC5.2 - Selezionato Heat Map}
\begin{itemize}
	\item \textbf{Attore primario}: Utente.
	\item \textbf{Precondizioni}: L'utente ha caricato dei dati nel sistema e ha selezionato le dimensioni da utilizzare [UC2].
	\item \textbf{Postcondizioni}: Viene mostrata la visualizzazione \glo{\textit{Heat Map}} scelta dall'utente, con possibilità di personalizzazione [UC6.2].
	\item \textbf{Scenario principale}: L'utente seleziona la visualizzazione \glo{\textit{Heat Map}} e il sistema ritorna un grafico con cui si può interagire.

\end{itemize}
\subsubsection{UC5.3 - Selezionato Force Field}
\begin{itemize}
	\item \textbf{Attore primario}: Utente.
	\item \textbf{Precondizioni}: L'utente ha caricato dei dati nel sistema [UC1], ha selezionato le dimensioni da utilizzare [UC2].
	\item \textbf{Postcondizioni}: Viene mostrata la visualizzazione \glo{\textit{Force Field}} scelta dall'utente, con possibilità di personalizzazione.
	\item \textbf{Scenario principale}: L'utente seleziona la visualizzazione \glo{\textit{Force Field}} e il sistema ritorna un grafico con cui si può interagire.
\end{itemize}
\subsubsection{UC5.4 - Selezione Proiezione Lineare Multi Asse}

\begin{itemize}
	\item \textbf{Attore primario}: Utente;
	\item \textbf{Precondizioni}: L'utente ha caricato dei dati nel sistema e ha selezionato le dimensioni da utilizzare;
	\item \textbf{Postcondizioni}: Viene mostrata la visualizzazione \glo{\textit{Proiezione Lineare Multi Asse}} scelta dall'utente, con possibilità di personalizzazione [UC6.4];
	\item \textbf{Scenario principale}: L'utente seleziona la visualizzazione \glo{\textit{Proiezione Lineare Multi Asse}} e il sistema ritorna un grafico con cui si può interagire.
\end{itemize}
\input{Section/Subsection/UCX.tex}
\input{Section/Subsection/UCX2.tex}
\subsubsection{UCX - Visualizzazione errore caricamento file}
\begin{itemize}
	\item \textbf{Attore primario}: Utente.
	\item \textbf{Precondizioni}: L'utente non seleziona alcun tipo di grafico da visualizzare o non carica alcun dato.
	\item \textbf{Postcondizioni}: L'utente visualizza un messaggio di errore e l'operazione fallisce.
	\item \textbf{Scenario principale}:
		\begin{enumerate}
			\item L'utente visualizza un messaggio di errore esplicativo;
			\item L'utente clicca "OK" per continuare.
		\end{enumerate}
\end{itemize}