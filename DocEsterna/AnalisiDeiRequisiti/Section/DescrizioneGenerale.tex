\section{Descrizione Generale}
\subsection{Obbiettivi del prodotto}
L'obbiettivo del progetto è la realizzazione di una applicazione che permette la visualizzazione di dati con molte dimensioni, a supporto della fase esplorativa dell'analisi dei dati, con l'utilizzo di tecnologie web.
\subsection{Funzioni del prodotto}
L'applicazione si occupa di analizzare dati a molte dimensioni (difficilmente interpretabile dall'uomo) inviati dall'utente e restituirli sotto forma di grafico (di diversi tipi a seconda della richiesta) a dimensioni ridotte, che può quindi essere interpretato facilmente dall'utente. \'E richiesta, perciò, la possibilità di ricevere dei file in input (per esempio di tipo csv) contenenti i dati richiesti.
\subsection{Caratteristiche degli utenti}
Il progetto non prevede come requisito la presenza di diverse categorie di utenza e non è necessaria una funzionalità di autenticazione/registrazione dell'utente.
\subsection{Piattaforme di esecuzione}
Il progetto sarà costituito da un insieme di pagine web accessibili dai browser più recenti come Google Chrome o Mozilla Firefox; non è richiesto, come requisito, una completa compatibilità con ulteriori browser.
\subsection{Obblighi di progettazione}
Il prodotto finale è soggetto a vincoli progettuali obbligatori ed opzionali, così come specificato all’interno del \glo{capitolato} C4. I vincoli obbligatori sono però da considerare come dei forti consigli che rendono la formazione del progetto molto più semplice rispetto all'uso di diverse tecnologie e non sono regole precise da dover rispettare a tutti i costi.
I vincoli obbligatori sono i seguenti:
\begin{itemize}
	\item l'applicazione deve essere sviluppata in tecnologia \glo{HTML}/\glo{CSS}/\glo{JavaScript} utilizzando la libreria \glo{D3.js};
	\item la parte server di supporto alla presentazione nel browser e alle query ad un database \glo{SQL} o \glo{NoSQL} potrà essere sviluppata in \glo{Java} con server \glo{Tomcat} o in \glo{Javascript} con server \glo{Node.js};
	\item i dati da visualizzare dovranno poter avere almeno fino a 15 dimensioni (o un numero di dimensioni tale da permettere l'uso della riduzione dimensionale senza l'utilizzo di scorciatoie), ma deve essere possibile anche visualizzare dati con meno dimensioni;
	\item i dati devono poter essere forniti al sistema di visualizzazione sia con query ad un database che da file in formato \glo{CSV} preparati precedentemente;
	\item dovranno essere presentabili almeno le seguenti visualizzazioni:
	\begin{itemize}
		\item \glo{Scatter plot Matrix} (fino ad un massimo di 5 dimensioni);
		\item \glo{Force Field};
		\item \glo{Heat Map};
		\item \glo{Proiezione Lineare Multi Asse}.
	\end{itemize}
	\item l' applicazione dovrà ordinare i punti nel grafico "Heat map" per evidenziare i "cluster" presenti nei dati.
\end{itemize}
Il tema della visualizzazione dei dati multidimensionali è vasto e ricco di spunti, perciò qualunque proposta verrà valutata dall'azienda e accettata come requisito opzionale se ritenuta valida; il proponente elenca comunque delle attività che saranno ben accettate:
\begin{itemize}
	\item Altri grafici adatti alla visualizzazione dei dati con più di tre dimensioni;
	\item Utilizzo di funzioni di calcolo della distanza diverse dalla distanza “Euclidea” in tutte le
visualizzazioni che dipendono da tale concetto;
	\item Utilizzo di funzioni di “forza” diverse da quelle previste in automatico dal grafico “force
based” di \glo{D3.js};
	\item Analisi automatiche per evidenziare situazioni di particolare interesse. Esempi di questa
possibilità si possono vedere in “ggobi” e “Orange Canvas”;
	\item Algoritmi di preparazione del dato per la visualizzazione, cioè anziché eseguire la
trasformazione direttamente nella visualizzazione far precedere un passo di trasformazione;
\end{itemize}