\section{Introduzione}
\subsection{Scopo del Documento}
Questo documento contiene la descrizione degli attori del sistema, definendo poi tutti i casi d'uso individuati a partire dai requisiti, fornendo una visione chiara ai progettisti sul problema da trattare.
\subsection{Scopo del Prodotto}
Oggigiorno, anche i programmi più tradizionali gestiscono e memorizzano una grande mole di dati e di conseguenza serve un software in grado di eseguire un'analisi e una interpretazione delle informazioni.\\
Il \glo{capitolato} C4 ha come obiettivo quello di creare un'applicazione di visualizzazione di dati con numerose dimensioni in un formato comprensibile dall'occhio umano.  A questo scopo è necessario utilizzare algoritmi di intelligenza artificiale, o nel caso svilupparne di nuovi, che, agendo sulla distanza dei vari punti del grafico, riescano a sviluppare un modello semplificato che ne evidenzi i \glo{cluster}. 
L'applicazione dovrà inoltre agire su questi grafici creati evidenziando i dati ottenuti.
\subsection{Glossario}
Per evitare ambiguità relative alle terminologie utilizzare, è stato compilato il \Glossariov{1.0.0}. In questo documento sono riportati tutti i termini di particolare importanza e con un significato particolare. Questi termini sono evidenziati da una 'G' ad apice.
\subsection{Riferimenti}
\subsubsection{Riferimenti normativi}
\begin{itemize}
	\item Norme di Progetto 1.0.0;
	\item Verbale interno 2020/12/14;
	\item Verbale esterno 2020/12/17;
	\item Verbale interno 2020/12/20.
\end{itemize}

\subsubsection{Riferimenti informativi}
\begin{itemize}
	\item Studio di Fattibilità 1.0.0;
	\item Capitolato d’appalto C4 - 2020:
	\url{https://www.math.unipd.it/~tullio/IS-1/2020/Progetto/C4.pdf}
\end{itemize}

