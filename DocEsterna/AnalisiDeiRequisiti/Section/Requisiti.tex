\section{Requisiti}
Ogni requisito è composto dalla seguente struttura: 
\begin{itemize}
	\item \textbf{Codice identificativo:} ogni codice identificativo è univoco e composto dalle seguenti voci:
	\begin{itemize}
	\item \textbf{Importanza:}
	\begin{itemize}
	\item \textbf{1:} requisito obbligatorio, irrinunciabile per qualcuno degli \glo{stakeholder};
	\item \textbf{2:} requisito desiderabile, non strettamente necessario ma a valore aggiunto riconoscibile;
	\item \textbf{3:} requisito opzionale, relativamente utili oppure contrattabili più avanti nel corso del progetto.
	\end{itemize}
	\item \textbf{Tipologia:}
	\begin{itemize}
	\item \textbf{F:} funzionale;
	\item \textbf{Q:} qualitativo;
	\item \textbf{P:} prestazionale;
	\item \textbf{V:} vincolo.
	\end{itemize}
	\item \textbf{Codice:} identificatore univoco del requisito in forma gerarchica padre/figlio
	\end{itemize}
	\begin{center}
	\textbf{R[Importanza][Tipologia][Codice]}
	\end{center}
	\item \textbf{Classificazione:} viene riportata l'importanza del requisito, già presente nel codice, ma in questo modo ne facilità la lettura;
	\item \textbf{Descrizione:} descrizione concisa del requisito;
	\item \textbf{Fonti:}
	\begin{itemize}
	\item \textbf{Capitolato:} requisito individuato dalla lettura del capitolato;
	\item \textbf{Interno:} requisito che gli analisti hanno ritenuto opportuno aggiungere;
	\item \textbf{Caso d'uso:} requisito estrapolato da uno o più \glo{UC}. In questo caso è riportato il codice univoco del UC.
	\item \textbf{Verbale:} requisito individuato a seguito di una richiesta di chiarimento al proponente. Tali informazioni sono riportate nei verbali in cui ogni requisito individuato è segnato da un codice presente nella tabella dei tracciamenti.
	\end{itemize}
\end{itemize}
\newpage
\subsection{Requisiti funzionali}
\renewcommand{\arraystretch}{1.5}
\rowcolors{2}{coloreGrigietto}{white}
\begin{center}
\begin{longtable}{C{\colA} C{\colB} C{\colC} C{\colA}}
		\rowcolor{coloreRosso}
		\textcolor{white}{\textbf{Codice}} & 
		\textcolor{white}{\textbf{Classificazione}} & 
		\textcolor{white}{\textbf{Descrizione}} & 
		\textcolor{white}{\textbf{Fonti}} \\
		\endfirsthead
	    \rowcolor{white}\multicolumn{4}{C{\total}}{\textit{Continua nella pagina successiva...}}\\
	    \endfoot
	    \rowcolor{white}\caption{Tabella dei requisiti funzionali}
	    \endlastfoot

%--------------------------------------------
R1F1 & Desiderabile & Descrizione del primo requisito bla bla bla & Fonti \\
R1F1 & Desiderabile & Descrizione del primo requisito bla bla bla & Fonti \\
R1F1 & Desiderabile & Descrizione del primo requisito bla bla bla & Fonti \\
R1F1 & Desiderabile & Descrizione del primo requisito bla bla bla & Fonti \\
R1F1 & Desiderabile & Descrizione del primo requisito bla bla bla & Fonti \\
R1F1 & Desiderabile & Descrizione del primo requisito bla bla bla & Fonti \\
\end{longtable}
\end{center}




\newpage
\subsection{Requisiti di qualità}
\renewcommand{\arraystretch}{1.5}
\rowcolors{2}{coloreGrigietto}{white}
\begin{center}
\begin{longtable}{C{\colA} C{\colB} C{\colC} C{\colA}}
		\rowcolor{coloreRosso}
		\textcolor{white}{\textbf{Codice}} & 
		\textcolor{white}{\textbf{Classificazione}} & 
		\textcolor{white}{\textbf{Descrizione}} & 
		\textcolor{white}{\textbf{Fonti}} \\
		\endfirsthead
	    \rowcolor{white}\multicolumn{4}{C{\total}}{\textit{Continua nella pagina successiva...}}\\
	    \endfoot
	    \rowcolor{white}\caption{Tabella dei requisiti funzionali}
	    \endlastfoot

%--------------------------------------------
R1F1 & Desiderabile & Descrizione del primo requisito bla bla bla & Fonti \\
R1F1 & Desiderabile & Descrizione del primo requisito bla bla bla & Fonti \\
R1F1 & Desiderabile & Descrizione del primo requisito bla bla bla & Fonti \\
R1F1 & Desiderabile & Descrizione del primo requisito bla bla bla & Fonti \\
R1F1 & Desiderabile & Descrizione del primo requisito bla bla bla & Fonti \\
R1F1 & Desiderabile & Descrizione del primo requisito bla bla bla & Fonti \\
\end{longtable}
\end{center}
\newpage
\subsection{Requisiti di vincolo}
\renewcommand{\arraystretch}{1.5}
\rowcolors{2}{coloreGrigietto}{apricot}
\begin{center}
\begin{longtable}{C{\colA} C{\colB} C{\colC} C{\colA}}
		\rowcolor{coloreRosso}
		\textcolor{white}{\textbf{Codice}} & 
		\textcolor{white}{\textbf{Classificazione}} & 
		\textcolor{white}{\textbf{Descrizione}} & 
		\textcolor{white}{\textbf{Fonti}} \\
		\endfirsthead
	    \rowcolor{white}\multicolumn{4}{C{\total}}{\textit{Continua nella pagina successiva...}}\\
	    \endfoot
	    \rowcolor{white}\caption{Tabella dei requisiti di vincolo}
	    \endlastfoot

%--------------------------------------------
R1V1 & Obbligatorio & L'applicazione HD Viz deve avere il \glo{front-end} sviluppato attraverso l'uso di tecnologie web & Capitolato \\
R1V1.1 & Obbligatorio & Le visualizzazioni dei grafici sono sviluppate in \glo{Javascript} utilizzando la libreria \glo{D3.js} & Capitolato\\
R1V2 & Obbligatorio & Il back-end dovrà essere sviluppato in \glo{Java} con server \glo{Tomcat} o in \glo{Javascript} con server \glo{Node.js} & Capitolato \\
R1V3 & Obbligatorio & Deve essere presente un database \glo{SQL} o \glo{NoSQL} & Capitolato\\
R1V4 & Obbligatorio & L'applicazione deve visualizzare dati a molte dimensioni, fino a 15. & Capitolato\\
R2V5 & Desiderabile & Per il salvataggio e il ripristino della sessione si utilizzerà un file in formato \glo{JSON} & Verbale Interno 5.10\\
R1V6 & Obbligatorio & La visualizzazione \glo{Scatter Plot Matrix} deve avere un massimo di 5 dimensioni & Capitolato\\
R3V7 & Opzionale & L'applicativo deve essere utilizzabile anche da dispositivi mobili, come tablet & Verbale Interno 5.11\\
R1V8 & Obbligatorio & Lo sviluppo deve basarsi su browser aggiornati all'ultima versione disponibile & Verbale Interno 5.12\\



\end{longtable}
\end{center}
\newpage
\subsection{Requisiti prestazionali}

%In questo periodo non sono stati individuati requisiti prestazionali obbligatori. La libreria \glo{D3.js} non presenta problemi di performance attuando una riduzione dimensionale preliminare. Per questo motivo la mole di dati sarà notevolmente ridotta e non causerà problemi evidenti durante l'utilizzo dell'applicativo.\\
%Potrebbe essere necessaria la definizione di requisiti prestazionali nel caso in cui si volessero
%approfondire alcuni dei requisiti opzionali presenti o che si andranno ad individuare in futuro. 

Specifiche hardware per garantire i requisiti riportati:
\begin{itemize}

\item \textbf{Sistema operativo}: Windows 10 a 64 bit;
\item \textbf{Processore}: Quad-Core 3,2 GHz;
\item \textbf{RAM}: 8GB DDR4;

\end{itemize}

La connessione internet ideale è di almeno \textbf{80Mb/s} in download.

\renewcommand{\arraystretch}{1.5}
{
\setlength\arrayrulewidth{1pt}
\begin{longtable}{C{2cm} | C{1.5cm} | C{9cm} | C{3.3cm}}
		\rowcolor{coloreRosso}
		\textcolor{white}{\textbf{Codice}} & 
		\textcolor{white}{\textbf{Classe}} & 
		\textcolor{white}{\textbf{Descrizione}} & 
		\textcolor{white}{\textbf{Fonti}} \\
		\endfirsthead
	    \rowcolor{white}\multicolumn{4}{c}{\textit{Continua nella pagina successiva...}}\\
	    \endfoot
	    \rowcolor{white}\caption{Tabella dei requisiti prestazionali}
	    \endlastfoot
	    
R2P1 & DE & I tempi di risposta della web app per il caricamento dei dati deve essere inferiore ai 2 secondi con un dataset di 2Mb di dimensione & Verbale Interno-9.3\\
R2P2 & DE & Il tempo di risposta della web app per la riduzione dimensionale deve essere inferiore a 7 secondi a fronte di un carico di 4 dimensioni con 500 punti circa & Verbale Interno-9.4\\
R2P3 & DE & Il tempo di risposta della web app per la visualizzazione di un grafico da 500 punti circa deve essere inferiore a 2 secondi & Verbale Interno-9.5\\

\end{longtable}
}


\newpage
\subsection{Tracciamento}
\subsubsection{Fonte - Requisiti}
\renewcommand{\arraystretch}{1.5}
\begin{center}
\begin{longtable}{C{6cm} C{6cm}}
		\rowcolor{coloreRosso}
		\textcolor{white}{\textbf{Fonte}} & 
		\textcolor{white}{\textbf{Requisiti}}\\
		\endfirsthead
	    \rowcolor{white}\multicolumn{2}{C{12cm}}{\textit{Continua nella pagina successiva...}}\\
	    \endfoot
	    \rowcolor{white}\caption{Tabella di tracciamento fonte-requisiti}
	    \endlastfoot

%--------------------------------------------
UC1 &  	R1F1\\
UC1.1.1 & R1F1.1\\
UC1.1.2 & R1F1.2\\
UC1.2 & R2F6\\
UC2 & R1F5\\
UC3 & R2F15\\
UC4 & R2F2\\
UC6 & R3F12\\
UC6.1.1 & R1F7.1.1\\
UC6.2.1 & R1F3\\
UC6.2.2 & R1F7.2.1\\
UC6.3.1 & R1F3\\
UC6.3.2 & R3F7.3.1\\
UC6.4.1 & R2F7.4.1\\
UC7 & R2F6\\
UC5 & R1F7\\
UC5.1 & R1F7.1\\
UC5.2 & R1F7.2\\
UC5.3 & R1F7.3\\
UC5.4 & R1F7.4\\
			  
UC8 \newline UC9 \newline UC10 \newline UC11 \newline UC12 & R1F12 \\

Capitolato &  	R1F7.2.1 \newline
				R3F7.3.1 \newline
				R3F7.5 \newline
				R3F7.6 \newline
				R3F8 \newline
				R3F9 \newline
				R1V1 \newline
				R1V1.1 \newline
				R1V2 \newline
				R1V3 \newline
				R1V4 \newline
				R1V6 \newline
				R2Q1 \newline
				R1Q2 \newline
				R1Q3 \newline
				R1Q4 \\

Verbale Esterno & R2F4 \newline
					R1F5 \newline
					R2F6 \newline
					R3F10 \newline
					R2F11 \newline
					R3F12 \\

Verbale Interno & 	R2F2 \newline
					R1F3 \newline
					R1F7.1.1 \newline
					R2F7.4.1 \newline
					R3F13 \newline
					R1Q5 \newline
					R1Q6 \newline
					R2V5 \newline
					R3V7 \newline
					R1V8 \\

\end{longtable}
\end{center}
\subsubsection{Requisito - Fonti}
\renewcommand{\arraystretch}{1.5}
\rowcolors{2}{coloreGrigietto}{apricot}
\begin{center}
\begin{longtable}{C{6cm} C{6cm}}
		\rowcolor{coloreRosso}
		\textcolor{white}{\textbf{Requisito}} & 
		\textcolor{white}{\textbf{Fonti}}\\
		\endfirsthead
	    \rowcolor{white}\multicolumn{2}{C{12cm}}{\textit{Continua nella pagina successiva...}}\\
	    \endfoot
	    \rowcolor{white}\caption{Tabella di tracciamento requisito-fonti}
	    \endlastfoot

%--------------------------------------------

R1F1 & UC1.1 \\
R1F1.1 & UC1.1.1 \\
R1F1.2 & UC1.1.2 \\
R2F2 & Verbale Interno-5.3, UC4 \\
R1F3 & Verbale Interno-5.5, UC6.2.1, UC6.3.1 \\
R2F4 & Verbale Esterno-1.6 \\
R1F5 & Verbale Esterno-1.7, UC2 \\
R2F6 & Verbale Esterno-1.8, UC1.2, UC7 \\
R1F7 & UC5 \\
R1F7.1 & UC5.1 \\
R1F7.1.1 & Verbale Interno-5.4, UC6.1.1 \\
R1F7.2 & UC5.2 \\
R1F7.2.1 & Capitolato, UC6.2.2 \\
R1F7.3 & UC5.3 \\
R3F7.3.1 & Capitolato, UC6.3.2 \\
R1F7.4 & UC5.4 \\
R2F7.4.1 & Verbale Interno-5.6, UC6.4.1 \\
R3F7.5 & Capitolato \\
R3F7.6 & Capitolato \\
R3F8 & Capitolato \\
R3F9 & Capitolato \\
R3F10 & Verbale Esterno-1.4 \\
R2F11 & Verbale Esterno-1.5 \\
R3F12 & Verbale Esterno-1.10, UC6 \\
R3F13 & Verbale Interno-5.7 \\
R1F14 & UC8, UC9, UC10, UC11, UC12 \\
R2F15 & UC3\\

R2Q1 & Capitolato \\
R1Q2 & Capitolato \\
R1Q3 & Capitolato \\
R1Q4 & Capitolato \\
R1Q5 & Verbale Interno-5.8 \\
R1Q6 & Verbale Interno-5.9 \\


R1V1 & Capitolato \\
R1V1.1 & Capitolato \\
R1V2 & Capitolato \\
R1V3 & Capitolato \\
R1V4 & Capitolato \\
R2V5 & Verbale Interno-5.10 \\
R1V6 & Capitolato \\
R3V7 & Verbale Interno-5.11 \\
R1V8 & Verbale Interno-5.12 \\

\end{longtable}
\end{center}

\subsection{Conclusioni}
I requisiti potranno subire delle variazioni in futuro, per apportare degli aggiornamenti alle
voci presenti o delle migliorie. Nel caso in cui le attività pianificate terminassero prima del
previsto, e dovessero avanzare delle ore di lavoro, potranno essere presi in carico nuovi requisiti
per aggiungere del valore al prodotto. Dunque eventuali espansioni sono lasciate a momenti
futuri.