\section{Requisiti}
Ogni requisito è composto dalla seguente struttura: 
\begin{itemize}
	\item \textbf{Codice identificativo:} ogni codice identificativo è univoco e composto dalle seguenti voci:
	\begin{itemize}
	\item \textbf{Importanza:}
	\begin{itemize}
	\item \textbf{1:} requisito obbligatorio, irrinunciabile per qualcuno degli \glo{stakeholder};
	\item \textbf{2:} requisito desiderabile, non strettamente necessario ma a valore aggiunto riconoscibile;
	\item \textbf{3:} requisito opzionale, relativamente utili oppure contrattabili più avanti nel corso del progetto.
	\end{itemize}
	\item \textbf{Tipologia:}
	\begin{itemize}
	\item \textbf{F:} funzionale;
	\item \textbf{Q:} qualitativo;
	\item \textbf{P:} prestazionale;
	\item \textbf{V:} vincolo.
	\end{itemize}
	\item \textbf{Codice:} identificatore univoco del requisito in forma gerarchica padre/figlio
	\end{itemize}
	\begin{center}
	\textbf{R[Importanza][Tipologia][Codice]}
	\end{center}
	\item \textbf{Classificazione:} viene riportata l'importanza del requisito, già presente nel codice, ma in questo modo ne facilità la lettura;
	\item \textbf{Descrizione:} descrizione concisa del requisito;
	\item \textbf{Fonti:}
	\begin{itemize}
	\item \textbf{Capitolato:} requisito individuato dalla lettura del capitolato;
	\item \textbf{Interno:} requisito che gli analisti hanno ritenuto opportuno aggiungere;
	\item \textbf{Caso d'uso:} requisito estrapolato da uno o più \glo{UC}. In questo caso è riportato il codice univoco del UC.
	\item \textbf{Verbale:} requisito individuato a seguito di una richiesta di chiarimento al proponente. Tali informazioni sono riportate nei verbali in cui ogni requisito individuato è segnato da un codice presente nella tabella dei tracciamenti.
	\end{itemize}
\end{itemize}
\newpage
\subsection{Requisiti funzionali}
\renewcommand{\arraystretch}{1.5}
\rowcolors{2}{coloreGrigietto}{apricot}
\begin{center}
\begin{longtable}{C{\colA} C{\colB} C{\colC} C{\colA}}
		\rowcolor{coloreRosso}
		\textcolor{white}{\textbf{Codice}} & 
		\textcolor{white}{\textbf{Classificazione}} & 
		\textcolor{white}{\textbf{Descrizione}} & 
		\textcolor{white}{\textbf{Fonti}} \\
		\endfirsthead
	    \rowcolor{white}\multicolumn{4}{C{\total}}{\textit{Continua nella pagina successiva...}}\\
	    \endfoot
	    \rowcolor{white}\caption{Tabella dei requisiti funzionali}
	    \endlastfoot

%--------------------------------------------
R1F1 & Obbligatorio & L'utente deve poter caricare dei dati nel sistema & UC1\\
R1F1.1 & Obbligatorio & Caricamento dati attraverso l'invio di un file \glo{CSV} & UC1.1\\
R1F1.2 & Obbligatorio & Caricamento dati attraverso l'interrogazione ad un \glo{database} & UC1.2\\

xxxxxxxxxxxx & Obbligatorio & L'utente deve poter scegliere le dimensioni da analizzare tra quelle da lui caricate nel sistema  & UC2\\

xxxxxxxxxxxx & Desiderevole & L'utente deve poter decidere eventuali parametri di personalizzazione dell'algoritmo scelto & Verbale Interno-5.3\\

xxxxxxxxxxxx & Obbligatorio & L'utente deve poter decidere che dimensioni visualizzare negli assi del grafico \textit{Scatter Plot Matrix} & Verbale Interno-5.4\\

xxxxxxxxxxxx & Obbligatorio & L'utente deve poter decidere che tipo di distanza calcolare nei grafici che dipendono da questo concetto & Verbale Interno-5.5\\

xxxxxxxxxxxx & Desiderevole & L'utente deve poter decidere quali dimensioni visualizzare nella visualizzazione \textit{Proiezione Lineare Multi Asse} & Verbale Interno-5.6\\

 

R2F2 & Desiderevole & Aiuti all'utente attraverso widget & Verbale Esterno-1.6 \\
R1F3 & Obbligatorio & L'applicazione deve permettere la scelta delle dimensioni da visualizzare & Verbale Esterno-1.7\\
R2F4 & Desiderevole & L'utente può salvare la sessione in corso per ripristinarla in un secondo momento & Verbale Esterno-1.8\\
R1F5 & Obbligatorio & L'applicazione deve fornire diverse visualizzazioni per i dati & UC3\\
R1F5.1 & Obbligatorio & L'applicazione deve fornire la visualizzazione \glo{Scatter plot Matrix} & UC3.1\\
R1F5.2 & Obbligatorio & L'applicazione deve fornire la visualizzazione \glo{Heat Map} & UC3.2\\
R1F5.2.1 & Obbligatorio & L'applicazione deve ordinare i punti nella visualizzazione \glo{Heat Map} & Capitolato \\
R1F5.3 & Obbligatorio & L'applicazione deve fornire la visualizzazione \glo{Force Field} & UC3.3\\
R3F5.3.1 & Opzionale & Utilizzo di funzioni di “forza” diverse da quelle previste in automatico dal grafico “forcebased” di \glo{D3.js} & Capitolato\\
R1F5.4 & Obbligatorio & L'applicazione deve fornire la visualizzazione \glo{Proiezione Lineare Multi Asse} & UC3.4 \\
R3F5.5 & Opzionale & L'applicazione deve fornire altre visualizzazioni con più di tre dimensioni & Capitolato\\
R3F5.6 & Opzionale & Utilizzo di funzioni di calcolo della distanza diverse dalla \glo{distanza “Euclidea”} in tutte le visualizzazioni che dipendono da tale concetto & Capitolato \\
R3F6 & Opzionale & Implementare analisi automatiche per evidenziare situazioni di particolare interesse & Capitolato\\
R3F7 & Opzionale & Utilizzo di algoritmi di preparazione del dato per la visualizzazione & Capitolato, UC3\\
R3F8 & Opzionale & Presenza di una guida introduttiva per l'utente & Verbale Esterno-1.4\\
R2F9 & Desiderevole & Possibilità di visualizzare contemporaneamente due grafici per confronti & Verbale Esterno-1.5\\
R3F10 & Opzionale & L'utente può personalizzare i grafici ottenuti & Verbale Esterno-1.10 \\
R3F11 & Opzionale & Ogni personalizzazione del grafico sarà visibile in tempo reale & Verbale Interno-5.7 \\
R1F12 & Obbligatorio & In caso di errori verrà visualizzato un messaggio esplicativo per aiutare l'utente & UC7, UC8, UC9, UC10\\

\end{longtable}
\end{center}




\subsection{Requisiti di qualità}
\renewcommand{\arraystretch}{1.5}
\rowcolors{2}{coloreGrigietto}{white}
\begin{center}
\begin{longtable}{C{\colA} C{\colB} C{\colC} C{\colA}}
		\rowcolor{coloreRosso}
		\textcolor{white}{\textbf{Codice}} & 
		\textcolor{white}{\textbf{Classificazione}} & 
		\textcolor{white}{\textbf{Descrizione}} & 
		\textcolor{white}{\textbf{Fonti}} \\
		\endfirsthead
	    \rowcolor{white}\multicolumn{4}{C{\total}}{\textit{Continua nella pagina successiva...}}\\
	    \endfoot
	    \rowcolor{white}\caption{Tabella dei requisiti funzionali}
	    \endlastfoot

%--------------------------------------------
R2Q1 & Desiderabile & Il progetto deve essere pubblicato su GitHub o altri repository pubblici & Capitolato \\
R1Q2 & Obbligatorio & Il progetto deve essere open-source & Capitolato\\
R1Q3 & Obbligatorio & Dovrà essere fornito un manuale per l'utilizzo & Capitolato\\
R1Q4 & Obbligatorio & Dovrà essere fornito un manuale per la manutenzione e l'estensione dell'applicazione & Capitolato\\
R1Q4 & Obbligatorio & Il prodotto deve essere sviluppato in modo concorde a quanto stabilito nelle \NdP & Decisione interna\\


\end{longtable}
\end{center}
\subsection{Requisiti di vincolo}
\renewcommand{\arraystretch}{1.5}
\rowcolors{2}{coloreGrigietto}{white}
\begin{center}
\begin{longtable}{C{\colA} C{\colB} C{\colC} C{\colA}}
		\rowcolor{coloreRosso}
		\textcolor{white}{\textbf{Codice}} & 
		\textcolor{white}{\textbf{Classificazione}} & 
		\textcolor{white}{\textbf{Descrizione}} & 
		\textcolor{white}{\textbf{Fonti}} \\
		\endfirsthead
	    \rowcolor{white}\multicolumn{4}{C{\total}}{\textit{Continua nella pagina successiva...}}\\
	    \endfoot
	    \rowcolor{white}\caption{Tabella dei requisiti di vincolo}
	    \endlastfoot

%--------------------------------------------
R1V1 & Obbligatorio & L'applicazione \glo{HD Viz} di visualizzaione deve avere il front-end sviluppato attraverso l'uso di tecnologie web & Capitolato \\
R1V1.1 & Obbligatorio & Le visualizzazioni dei grafici sono sviluppate in \glo{Javascript} utilizzando la libreria \glo{D3.js} & Capitolato\\
R1V2 & Obbligatorio & Il back-end dovrà essere sviluppato in \glo{Java} con server \glo{Tomcat} o in \glo{Javascript} con server \glo{Node.js} & Capitolato \\
R1V3 & Desiderabile & Deve essere presente un database \glo{SQL} o \glo{NoSQL} & Capitolato\\
R1V4 & Obbligatorio & L'applicazione deve visualizzare dati a molte dimensioni, fino a 15. & Capitolato\\
\end{longtable}
\end{center}
\subsection{Requisiti prestazionali}

In questo periodo non sono stati individuati requisiti prestazionali obbligatori. La libreria \glo{D3.js} non presenta problemi di performance attuando una riduzione dimensionale preliminare. Per questo motivo la mole di dati sarà notevolmente ridotta e non causerà problemi evidenti durante l'utilizzo dell'applicativo.\\
Potrebbe essere necessaria la definizione di requisiti prestazionali nel caso in cui si volessero
approfondire alcuni dei requisiti opzionali presenti o che si andranno ad individuare in futuro. 
\subsection{Tracciamento}
\subsubsection{Fonte - Requisiti}
\renewcommand{\arraystretch}{1.5}
\rowcolors{2}{coloreGrigietto}{apricot}
\begin{center}
\begin{longtable}{C{6cm} C{6cm}}
		\rowcolor{coloreRosso}
		\textcolor{white}{\textbf{Fonte}} & 
		\textcolor{white}{\textbf{Requisiti}}\\
		\endfirsthead
	    \rowcolor{white}\multicolumn{2}{C{12cm}}{\textit{Continua nella pagina successiva...}}\\
	    \endfoot
	    \rowcolor{white}\caption{Tabella di tracciamento fonte-requisiti}
	    \endlastfoot

%--------------------------------------------
UC1 &  	R1F1\\
UC1.1 & R1F1.1\\
UC1.2 & R1F1.2\\
UC3 & 	R1F5\\
UC3.1 & R1F5.1\\
UC3.2 & R1F5.2\\
UC3.3 & R1F5.3\\
UC3.4 & R1F5.4\\
			  
UC7 \newline UC8 \newline UC9 \newline UC10 \newline & R1F12 \\

Capitolato &  	R1F5.2.1 \newline
				R3F5.3.1 \newline
				R3F5.5 \newline
				R3F5.6 \newline
				R3F6 \newline
				R3F7 \newline
				R1V1 \newline
				R1V1.1 \newline
				R1V2 \newline
				R1V3 \newline
				R1V4 \newline
				R1V6 \newline
				R2Q1 \newline
				R1Q2 \newline
				R1Q3 \newline
				R1Q4 \\

Verbale Interno & R3F11 \newline
					R2V5 \newline
					R3V7 \newline
					R1V8 \newline
					R1Q5 \newline
					R1Q6 \\

Verbale esterno & 	R2F2 \newline
					R1F3 \newline
					R2F4 \newline
					R3F8 \newline
					R2F9 \newline
					R3F10 \\

\end{longtable}
\end{center}
\subsubsection{Requisito - Fonti}
\renewcommand{\arraystretch}{1.5}
\begin{center}
\begin{longtable}{C{2cm} | C{5cm} C{0.2cm} C{2cm} | C{5cm}}
		\rowcolor{coloreRosso}
		\textcolor{white}{\textbf{Requisito}} & 
		\textcolor{white}{\textbf{Fonti}}&
		\cellcolor{white} & 
		\textcolor{white}{\textbf{Requisito}} & 
		\textcolor{white}{\textbf{Fonti}}\\
		\endfirsthead
	    \rowcolor{white}\multicolumn{5}{c}{\textit{Continua nella pagina successiva...}}\\
	    \endfoot
	    \rowcolor{white}\caption{Tabella di tracciamento requisito-fonti}
	    \endlastfoot

%--------------------------------------------

R1F1 & UC1.1 & \cellcolor{white} & R1F15.4 & Verb. Interno-8.4, UC3.1.4 \\

R1F1.1 & UC1.1.1 & \cellcolor{white} & R1F15.4.1 & Verb. Interno-8.5, UC4.2.2.1 \\

R1F1.2 & UC1.1.2 & \cellcolor{white} & R1F15.4.2 & Verb. Interno-8.6, UC4.2.2.2 \\

R1F1.2.1 & UC1.1.2.1 & \cellcolor{white} & R3F15.5 & Verbale Interno-8.7, UC4.2.1 \\

R1F1.2.2 & UC1.1.2.2 & \cellcolor{white} & R2F15.6 & Verbale Interno-8.8, UC4.1 \\

R2F2 & Verbale Interno-5.3, UC4.2 & \cellcolor{white} & R2F15.7 & Verbale Esterno-5.2, UC3.1.5 \\

R1F3 & Verbale Interno-5.5, UC3.2 & \cellcolor{white} & R2F15.8 & Verbale Interno-14.1, UC3.1.6 \\

R2F4 & Verbale Esterno-1.5 & \cellcolor{white} & R1F16 & Verbale Interno-2.1, UC3.2 \\

R1F5 & Verbale Esterno-1.6, UC2 & \cellcolor{white} & R2F16.1 & Verbale Interno-8.9, UC3.2.1 \\

R2F6 & Verbale Esterno-1.7, UC1.2, UC9 & \cellcolor{white} & R2F16.2 & Verbale Interno-8.10, UC3.2.2 \\

R1F7 & UC6, UC7 & \cellcolor{white} & R2F16.3 & Verbale Interno-8.11, UC3.2.3 \\

R1F7.1 & UC6.1 & \cellcolor{white} & R1F16.4 & UC3.2.4 \\

R1F7.1.1 & Verbale Interno-5.4, UC8.1.1 & \cellcolor{white} & R1F17 & Capitolato \\

R2F7.1.2 & Verbale Interno-7.4, UC8.1.1 & \cellcolor{white} & R1F18 & Verbale Esterno-3.3, UC4.3 \\

R2F7.1.3 & Verbale Interno-7.5, UC8.1.1 & \cellcolor{white} & R1F19 & Verbale Esterno-3.3, UC5.2 \\

R1F7.1.4 & Capitolato, UC8.1.1 & \cellcolor{white} & R1F20 & Verbale Interno-14.2 \\

R1F7.1.5 & UC8.1.3 & \cellcolor{white} & R2Q1 & Capitolato \\

R1F7.2 & UC7.1 & \cellcolor{white} & R1Q2 & Capitolato \\

R1F7.2.1 & Capitolato, UC8.2.2 & \cellcolor{white} & R1Q3 & Capitolato \\

R1F7.2.2 & Verbale Interno-13.2, UC8.2.4 & \cellcolor{white} & R1Q4 & Capitolato \\

R1F7.3 & UC7.2 & \cellcolor{white} & R1Q5 & Verbale Interno-5.8\\

R3F7.3.1 & Capitolato, UC8.3.2 & \cellcolor{white} & R1Q6 & Verbale Interno-7.7\\

R1F7.3.2 & Verbale Interno-13.1, UC8.3.3 & \cellcolor{white} & R1Q7 & Verbale Interno-7.8 \\

R1F7.3.3 & Verbale Interno-13.2, UC8.3.4 & \cellcolor{white} & R1V1 & Capitolato\\

R1F7.4 & UC6.2 & \cellcolor{white} & R1V1.1 & Capitolato\\

R2F7.4.1 & Verbale Interno-5.6, UC8.4.1 & \cellcolor{white} & R1V2 & Capitolato\\

R3F7.5 & Capitolato & \cellcolor{white} & R1V3 & Capitolato\\

R3F7.6 & Capitolato & \cellcolor{white} & R2V4 & Verbale Interno-5.10\\

R1F7.7 & Verbale Interno-12.4, Verbale Esterno-4.2 & \cellcolor{white} & R2V5 & Verbale Interno-7.1\\

R1F7.7.1 & Verbale Interno-12.4 & \cellcolor{white} & R2V6 & Verbale Interno-7.2\\

R1F7.7.2 & Verbale Interno-12.4 & \cellcolor{white} & R3V7 & Verbale Interno-5.11\\

R3F8 & Capitolato & \cellcolor{white} & R1V8 & Verbale Interno-5.9\\

R3F9 & Capitolato & \cellcolor{white} & R1V9 & Supporto browser SVG 1.1\\

R3F10 & Verbale Esterno-1.3, UC18 & \cellcolor{white} & R1V10 & Supporto browser SVG 1.1\\

R2F11 & Verbale Esterno-1.4 & \cellcolor{white} & R1V11 & Supporto browser SVG 1.1\\

R3F12 & Verbale Esterno-1.9, UC8 & \cellcolor{white} & R1V12 & Supporto browser SVG 1.1 \\

R3F13 & Verbale Interno-5.7 & \cellcolor{white} & R1V13 & Verbale Interno-7.6\\

R1F14 & UC10 a UC17 & \cellcolor{white} & R1V14 & Verbale Interno-7.3 \\

R1F15 & UC3.1 & \cellcolor{white} & R2P1 & Verbale Interno-9.3 \\

R1F15.1 & Verbale Interno-8.1, UC3.1.1 & \cellcolor{white} & R2P2 & Verbale Interno-9.4 \\

R1F15.2 & Verbale Interno-8.2, UC3.1.2 & \cellcolor{white} & R2P3 & Verbale Interno-9.5 \\

\end{longtable}
\end{center}






























\subsection{Conclusioni}
I requisiti potranno subire delle variazioni in futuro, per apportare degli aggiornamenti alle
voci presenti o delle migliorie. Nel caso in cui le attività pianificate terminassero prima del
previsto, e dovessero avanzare delle ore di lavoro, potranno essere presi in carico nuovi requisiti
per aggiungere del valore al prodotto. Dunque eventuali espansioni sono lasciate a momenti
futuri.