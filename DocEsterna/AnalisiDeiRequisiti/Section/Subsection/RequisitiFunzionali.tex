\subsection{Requisiti funzionali}
\renewcommand{\arraystretch}{1.5}
\rowcolors{2}{coloreGrigietto}{white}
\begin{center}
\begin{longtable}{C{\colA} C{\colB} C{\colC} C{\colA}}
		\rowcolor{coloreRosso}
		\textcolor{white}{\textbf{Codice}} & 
		\textcolor{white}{\textbf{Classificazione}} & 
		\textcolor{white}{\textbf{Descrizione}} & 
		\textcolor{white}{\textbf{Fonti}} \\
		\endfirsthead
	    \rowcolor{white}\multicolumn{4}{C{\total}}{\textit{Continua nella pagina successiva...}}\\
	    \endfoot
	    \rowcolor{white}\caption{Tabella dei requisiti funzionali}
	    \endlastfoot

%--------------------------------------------
R1F1 & Obbligatorio & Caricamento dei dati nel sistema & Capitolato e UC\\
R1F1.1 & Obbligatorio & Caricamento attraverso l'invio di un file CSV & Capitolato e UC\\
R1F1.2 & Obbligatorio & Caricamento attraverso l'interrogazione a un DB & Capitolato e UC\\
R1F1.3 & Obbligatorio & Il caricamento fallisce se il file non è corretto o l'interrogazione al DB fallisce & Interno UC\\
R2F2 & Desiderevole & Aiuti all'utente attraverso widget & Verbale Esterno\\
R1F3 & Obbligatorio & L'applicazione deve fornire diverse visualizzazioni per i dati & Capitolato\\
R1F3.1 & Obbligatorio & L'applicazione deve fornire la visualizzazione \glo{Scatter plot Matrix} & Capitolato\\
R1F3.1.1 & Obbligatorio & L'applicazione deve permettere la scelta delle dimensioni da visualizzare come assi & Verbale esterno\\
R1F3.2 & Obbligatorio & L'applicazione deve fornire la visualizzazione \glo{Heat Map} & Capitolato\\
R1F3.2.1 & Obbligatorio & L'applicazione deve ordinare i punti nella visualizzazione Heat Map & Capitolato \\
R1F3.3 & Obbligatorio & L'applicazione deve fornire la visualizzazione \glo{Force Field} & Capitolato\\
R1F3.4 & Obbligatorio & L'applicazione deve fornire la visualizzazione \glo{Proiezione Lineare Multi Asse} & Capitolato \\
R3F3.5 & Opzionale & L'applicazione deve fornire altre visualizzazioni con più di tre dimensioni & Capitolato\\
R3FX & Opzionale & Utilizzo di funzioni di calcolo della distanza diverse dalla distanza “Euclidea” in tutte le visualizzazioni che dipendono da tale concetto & Capitolato \\
R3FX & Opzionale & Utilizzo di funzioni di “forza” diverse da quelle previste in automatico dal grafico “forcebased” di D3. & Capitolato\\
R3FX & Opzionale & Implemenare analisi automatiche per evidenziare situazioni di particolare interesse & Capitolato\\
R3FX & Opzionale & Utilizzo di algoritmi di preparazione del dato per la visualizzazione & Capitolato\\
RxFx & Desiderevole o opzionale? & Presenza di una guida introduttiva per l'utente & Esterno-1.4\\
RxFx & Desiderevole o opzionale? Penso sia diverso da R2F2 & Possibilità di visualizzare contemporaneamente due grafici per confronti & Esterno-1.5\\

\end{longtable}
\end{center}



