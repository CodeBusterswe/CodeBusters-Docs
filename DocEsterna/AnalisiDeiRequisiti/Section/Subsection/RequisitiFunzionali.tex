\subsection{Requisiti funzionali}
OB = Obbligatorio \quad DE = Desiderevole \quad OP = Opzionale
\renewcommand{\arraystretch}{1.5}
{
\setlength\arrayrulewidth{1pt}
\begin{longtable}{C{2cm} | C{1.5cm} | C{9cm} | C{3.3cm}}
		\rowcolor{coloreRosso}
		\textcolor{white}{\textbf{Codice}} & 
		\textcolor{white}{\textbf{Classe}} & 
		\textcolor{white}{\textbf{Descrizione}} & 
		\textcolor{white}{\textbf{Fonti}} \\
		\endfirsthead
	    \rowcolor{white}\multicolumn{4}{c}{\textit{Continua nella pagina successiva...}}\\
	    \endfoot
	    \rowcolor{white}\caption{Tabella dei requisiti funzionali}
	    \endlastfoot

%--------------------------------------------
R1F1 & OB & L'utente deve poter caricare dei dati nel sistema & UC1.1\\
R1F1.1 & OB & Caricamento dati attraverso l'invio di un file \glo{CSV} & UC1.1.1\\
R1F1.2 & OB & Caricamento dati attraverso l'interrogazione ad un \glo{database} & UC1.1.2\\

R2F2 & DE & L'utente deve poter decidere eventuali parametri di personalizzazione dell'algoritmo scelto & Verbale Interno-5.3 \newline UC4\\

R1F3 & OB & L'utente deve poter decidere che tipo di distanza calcolare nei grafici che dipendono da questo concetto & Verbale Interno-5.5 \newline UC6.2.1, UC6.3.1\\

R2F4 & DE & Aiuti all'utente attraverso widget & Verbale Esterno-1.5 \\
R1F5 & OB & L'applicazione deve permettere la scelta delle dimensioni da visualizzare & Verbale Esterno-1.6 \newline UC2\\
R2F6 & DE & L'utente può salvare la sessione in corso per ripristinarla in un secondo momento & Verbale Esterno-1.7 \newline UC1.2, UC7\\
R1F7 & OB & L'applicazione deve fornire diverse visualizzazioni per i dati & UC5\\
R1F7.1 & OB & L'applicazione deve fornire la visualizzazione \glo{Scatter plot Matrix} & UC5.1\\
R1F7.1.1 & OB & L'utente deve poter decidere che dimensioni visualizzare negli assi del grafico \textit{Scatter Plot Matrix} & Verbale Interno-5.4 \newline UC6.1.1\\
R2F7.1.2 & DE & Nel grafico \textit{Scatter Plot Matrix} l'utente può selezionare dei punti e vedere le relazioni in ogni singolo \textit{Scatter Plot} & Verbale Interno-7.4\\
R2F7.1.3 & DE & L'utente può vedere i valori di un punto passando sopra con il cursore & Verbale Interno-7.5\\

R1F7.1.4 & OB & La visualizzazione \glo{Scatter Plot Matrix} deve avere un massimo di 5 dimensioni & Capitolato\\

R1F7.1.5 & OB & Nello \glo{Scatter Plot Matrix} l'utente deve poter decidere quale dimensione associare al colore & UC6.1.3\\

R1F7.2 & OB & L'applicazione deve fornire la visualizzazione \glo{Heat Map} & UC5.2\\
R1F7.2.1 & OB & L'applicazione deve ordinare i punti nella visualizzazione \glo{Heat Map} e fornire il \glo{dendrogramma} & Capitolato, UC6.2.2 \\
R1F7.3 & OB & L'applicazione deve fornire la visualizzazione \glo{Force Field} & UC5.3\\
R3F7.3.1 & OP & Utilizzo di funzioni di “forza” diverse da quelle previste in automatico dal grafico “forcebased” di \glo{D3.js} & Capitolato, UC6.3.2\\
R1F7.4 & OB & L'applicazione deve fornire la visualizzazione \glo{Proiezione Lineare Multi Asse} & UC5.4 \\
R2F7.4.1 & DE & L'utente deve poter decidere quali dimensioni visualizzare nella visualizzazione \textit{PLMA} & Verbale Interno-5.6 \newline UC6.4.1\\
R3F7.5 & OP & L'applicazione deve fornire altre visualizzazioni con più di tre dimensioni & Capitolato\\
R3F7.6 & OP & Utilizzo di funzioni di calcolo della distanza diverse dalla \glo{distanza “Euclidea”} in tutte le visualizzazioni che dipendono da tale concetto & Capitolato \\
R3F8 & OP & Implementare analisi automatiche per evidenziare situazioni di particolare interesse & Capitolato\\
R3F9 & OP & Utilizzo di algoritmi di preparazione del dato per la visualizzazione & Capitolato\\
R3F10 & OP & Presenza di una guida introduttiva per l'utente & Verbale Esterno-1.3\\
R2F11 & DE & Possibilità di visualizzare contemporaneamente due grafici per confronti & Verbale Esterno-1.4\\
R3F12 & OP & L'utente può personalizzare i grafici ottenuti & Verbale Esterno-1.9 \newline UC6 \\
R3F13 & OP & Ogni personalizzazione del grafico sarà visibile in tempo reale & Verbale Interno-5.7 \\
R1F14 & OB & In caso di errori verrà visualizzato un messaggio esplicativo per aiutare l'utente & UC8, UC9, UC10, UC11, UC12\\
R1F15 & OB & L'utente puó scegliere di utilizzare un algoritmo di riduzione dimensionale & UC3.1 \\
R1F15.1 & OB & L'utente puó utilizzare l'algoritmo di riduzione dimensionale \glo{IsoMap} & Verbale Interno-8.1 \newline UC3.1.1 \\
R1F15.2 & OB & L'utente puó utilizzare l'algoritmo di riduzione dimensionale \glo{LLE} & Verbale Interno-8.2 \newline UC3.1.2\\
R1F15.3 & OB & L'utente puó utilizzare l'algoritmo di riduzione dimensionale \glo{FastMap} & Verbale Interno-8.3 \newline UC3.1.3\\
R1F15.4 & OB & L'utente puó utilizzare l'algoritmo di riduzione dimensionale \glo{t-SNE} & Verbale Interno-8.4 \newline UC3.1.4\\
R3F15.4.1 & OP & L'utente può impostare la \glo{\textit{perplessità}} dell'algoritmo t-SNE & Verbale Interno-8.5 \newline UC4.2.2.1\\
R3F15.4.2 & OP & L'utente può impostare il tasso di apprendimento dell'algoritmo t-SNE & Verbale Interno-8.6 \newline UC4.2.2.2\\
R3F15.5 & OP & L'utente può impostare il numero di \glo{\textit{neighbors}} per la stima approssimativa del \glo{\textit{manifold}} & Verbale Interno-8.7 \newline UC4.2.1\\
R2F15.6 & DE & L'utente può selezionare il numero di dimensioni ottenute come risultato dell'algoritmo di riduzione & Verbale Interno-8.8 \newline UC4.1\\
R1F16 & OB & L'utente puó creare nuove dimensioni tramite calcolo delle distanze & Verbale Esterno-2.1 \newline UC3.2\\ 
R2F16.1 & DE & L'utente può calcolare le distanze tramite distanza di \glo{Manhattan} &  Verbale Interno-8.9 \newline UC3.2.1\\
R2F16.2 & DE & L'utente può calcolare le distanze tramite distanza di \glo{Canberra} & Verb. Interno-8.10 \newline UC3.2.2 \\
R2F16.3 & DE & L'utente può calcolare le distanze tramite distanza di \glo{Chebyshev} &  Verb. Interno-8.11 \newline UC3.2.3\\
R1F17 & OB & L'applicazione deve visualizzare dati a molte dimensioni, fino a 15 & Capitolato\\


\end{longtable}
}







