\subsubsection{UC1 - Caricamento del dataset}

\begin{figure}[h]
%\includegraphics[width=\linewidth]{}
\centering
%\caption{UC1 - Caricamento file}
\end{figure}


\begin{itemize}
	\item \textbf{Attore primario}: Utente.
	\item \textbf{Precondizioni}: Il sistema è raggiungibile e funzionante.
	\item \textbf{Postcondizioni}: Viene visualizzato un messaggio che avvisa l'utente del corretto caricamento dei dati e della loro validità.
	\item \textbf{Scenario principale}:
		\begin{enumerate}
			\item L'utente accede al sistema;
			\item L'utente sceglie come ricavare i dati:
				\begin{enumerate}[(a)]
			\item L'utente seleziona la funzionalità "carica file";
			\item L'utente seleziona un dataset tra quelli disponibili nel database.
				\end{enumerate}
		\end{enumerate}
	\item \textbf{Estensioni}:
	\begin{enumerate}[(a)]
		\item Nel caso in cui il file sia in un formato sbagliato o i dati non sono validi:
		\begin{enumerate}[1.]
			\item i dati non vengono caricati nel sistema;
			\item viene visualizzato un errore esplicativo [UCX].
		\end{enumerate}
	\end{enumerate}
\end{itemize}

\subsubsection{UC1.1 - Caricamento dataset da file}

\begin{itemize}
	\item \textbf{Attore primario}: Utente.
	\item \textbf{Precondizioni}: Il sistema è raggiungibile e funzionante. L'utente ha a disposizione un dataset in formato CSV.
	\item \textbf{Postcondizioni}: I dati presenti nel file vengono caricati nel sistema. Viene visualizzato un messaggio che avvisa l'utente del corretto caricamento e della validità dei dati.
	\item \textbf{Scenario principale}: L'utente sceglie di caricare un dataset personale o ricavato da altre fonti esterne.
	
\end{itemize}

\subsubsection{UC1.2 - Caricamento dataset dal database}

\begin{itemize}
	\item \textbf{Attore primario}: Utente.
	\item \textbf{Precondizioni}: Il sistema è raggiungibile e funzionante. L'utente effettua una query dal database disponibile per prelevare il dataset.
	\item \textbf{Postcondizioni}: I dati vengono caricati nel sistema. Viene visualizzato un messaggio che avvisa l'utente del corretto caricamento e della loro validità.
	\item \textbf{Scenario principale}: L'utente sceglie di caricare un dataset tra quelli presenti nel database.
	
\end{itemize}


  