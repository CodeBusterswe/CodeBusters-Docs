\section*{A}
\markright{}
\addcontentsline{toc}{section}{A}
\subsection*{Android}
Sistema operativo mobile sviluppato da Google, con interfacce utente adattate a diverse tipologie di dispositivi: smartphone, tablet, Android TV, Android Auto, orologi da posto (Wear OS) e molti altri. È basato sul kernel Linux ed è disponibile dal 2008 come progetto open source con controparti proprietarie. Ad oggi è il sistema operativo per dispositivi mobili più diffuso al mondo.
\subsection*{API}
Acronimo di Application Programming Interface. Si indica un  insieme  di  procedure disponibili al programmatore, di solito  raggruppate a formare un set di strumenti specifici per lo svolgimento di un determinato compito all’interno di un certo programma. Talvolta le API sono offerte tramite servizi a pagamento, oppure potrebbero essere funzionalità gratuite, come librerie software disponibili in un certo linguaggio di programmazione.
\subsection*{AWS}
Acronimo di Amazon Web Service. Azienda di proprietà di Amazon che offre diversi servizi di cloud computing su una piattaforma online.
\subsection*{AWS AppSync}
Servizio \glo{AWS} completamente gestito che facilita lo sviluppo di API utilizzando GraphQL, ossia un linguaggio di interrogazione lato server che rende le API più intuitive e flessibili per gli sviluppatori. Lato client permette di generare query e restituire unicamente i dati richiesti dal database (\glo{DynamoDB} per esempio).  
\subsection*{AWS GameLift}
Servizio \glo{AWS} che fornisce una soluzione di hosting di server completamente gestita per giochi multiplayer, favorendo bassa latenza e bassi tempi d'attesa per i giocatori che accedono da tutto il mondo.  
