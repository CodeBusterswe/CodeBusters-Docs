\section*{C}
\markright{}
\addcontentsline{toc}{section}{C}

\subsection*{Capitolato}
Documento redatto dal cliente in cui vengono specificati i vincoli contrattuali (prezzo e scadenze) per lo sviluppo di un determinato prodotto software. Tale documento viene presentato in un bando d'appalto e serve a trovare qualcuno che possa svolgere il lavoro richiesto.

\subsection*{Cloud}
Vasta rete di server remoti ubicati in tutto il mondo, collegati tra loro e che operano come un unico ecosistema. Questi server possono archiviare e gestire dati, eseguire applicazioni o distribuire contenuti o servizi. L'accesso avviene online, da qualsiasi dispositivo con connessione Internet.

\subsection*{CloudFormation}
Strumento di \glo{AWS} che permette di modellare una raccolta di risorse \glo{AWS} e gestirle nell'intero arco di ciclo di vita del prodotto.

\subsection*{Cluster}
Un cluster è un insieme di oggetti che presentano tra loro delle similarità e, allo stesso modo, delle dissimilarità con oggetti in altri cluster.

\subsection*{Code folding}
È una caratteristica di alcuni editor di testo e ambienti di sviluppo. Permette di nascondere delle porzioni di un file di codice mentre si lavora ad altre parti dello stesso file. Ciò permette agli sviluppatori di gestire più comodamente file molto lunghi all'interno di un'unica finestra.

\subsection*{Commit}
In informatica un commit è la creazione di una serie di modifiche provvisorie permanenti, che segna la fine di una transazione.

\subsection*{Compiti}
Attività assegnate ai membri del gruppo, svolgibili anche da un singolo in un breve lasso di tempo.

\subsection*{CSS}
Acronimo di Cascading Style Sheets. Linguaggio usato per definire la formattazione di documenti \glo{HTML}, XHTML e XML, ad esempio i siti web e relative pagine web.

\subsection*{CSV}
Acronimo di comma-separated values. Formato di file basato su file di testo utilizzato per l'importazione ed esportazione (ad esempio da fogli elettronici o \glo{database}) di una tabella di dati. 



