\section*{I}
\markright{}
\addcontentsline{toc}{section}{I}
\subsection*{Incremento}
Incrementare il progetto tipicamente significa aumentarne i tempi, i costi e le risorse, portando all' aggiunta di nuove funzionalità.

\subsection*{Indice Gulpease}
L'Indice Gulpease è un indice di leggibilità di un testo tarato sulla lingua italiana. Rispetto ad altri ha il vantaggio di utilizzare la lunghezza delle parole in lettere anziché in sillabe, semplificandone il calcolo automatico. Oltre alla lunghezza delle parole tiene in considerazione anche la lunghezza della frase rispetto al numero di lettere. I risultati sono compresi tra 0 e 100, dove il valore "100" indica la leggibilità più alta e "0" la leggibilità più bassa.

\subsection*{Information Hiding}
Secondo il concetto di Information Hiding i dettagli implementativi di una classe, o di un costrutto di altro tipo (oggetto, modulo, ecc.), sono nascosti all'utente. Pertanto una parte di un programma può nascondere informazioni incapsulandole in un costrutto dotato di interfaccia.

\subsection*{iOS}
Sistema operativo mobile sviluppato da Apple per i suoi dispositivi iPhone e iPad. È stato rilasciato nella sua prima versione nel 2007 e ad oggi è il secondo sistema operativo più installato al mondo dopo \glo{Android}.

\subsection*{IoT}
Acronimo di Internet of Things. Concetto che si riferisce alla tecnologia che permette a oggetti aventi apparati elettronici di comunicare dati riguardanti il loro funzionamento o cambiare il loro funzionamento in base al contesto esterno. 

\subsection*{ITS}
Acronimo di Issue Tracking System. L'ITS è un sistema informatico che gestisce e registra delle liste di richieste di assistenza o di problemi, organizzato secondo le necessità di chi offre il servizio.