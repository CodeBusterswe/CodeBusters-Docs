\section*{S}
\markright{}
\addcontentsline{toc}{section}{S}

\subsection*{Stakeholder}
Gli stakeholder di progetto sono persone o strutture organizzative attivamente coinvolte nel progetto o i cui interessi possono subire conseguenze dell'esecuzione o dal completamento del progetto, possono quindi influire sugli obiettivi e sui risultati del progetto.

\subsection*{Scatter plot}
Grafico di dispersione a due dimensioni. I dati sono riportati su uno spazio cartesiano: una dimensione sull'asse delle ascisse e una su quello delle ordinate. 

\subsection*{Scatter plot Matrix}
Grafico formato da più \glo{scatter plot} disposti a matrice, dove in ognuno vengono messe in relazione due dimensioni diverse. Esso aiuta l'analista a trovare dimensioni con forti correlazioni e più  dimensioni che danno la stessa informazione. Il grafico è reperibile nella libreria \glo{D3.js}.

\subsection*{SceneKit}
\glo{Framework} a supporto del linguaggio di programmazione \glo{Swift} che permette di costruire grafiche 3D per applicazioni \glo{iOS} e MacOS. Permette di inserire animazioni e comportamenti fisici realistici alle componenti dell'interfaccia dell'applicazione. 

\subsection*{Scikit-learn}
Libreria \glo{open source} di apprendimento automatico per il linguaggio di programmazione \glo{Python}. Contiene algoritmi di classificazione, regressione e clustering, macchine a vettori di supporto, regressione logistica, classificatore bayesiano, k-mean e DBSCAN. È progettato per operare con le librerie NumPy e SciPy. 

\subsection*{Serverless}
Si intende un network la cui gestione non viene incentrata su dei server, ma dislocata fra i vari utenti che utilizzano il network stesso.

\subsection*{Sistematico}
L'ingegneria del software è un approccio sistematico, ossia metodico e rigoroso che studia, usa ed evolve le best practice.

\subsection*{Skype}
Software che consente di effettuare videoconferenze e creare gruppi, facilitando la comunicazione tra più utenti. 

\subsection*{Snapshot}
Rappresenta uno stato del sistema in uno specifico momento, una fase di lavoro che si vuole "fermare" nel caso che le variazioni che si stanno per compiere non ci soddisfino. Le snapshot consentono per esempio di vedere versioni alternative di una stessa immagine, per poterne scegliere la migliore.

\subsection*{Socket.io}
Libreria \glo{JavaScript} per la realizzazione delle applicazioni real-time, eseguita lato server. Consente una comunicazione in tempo reale, basata sugli eventi.

\subsection*{SpriteKit}
\glo{Framework} a supporto del linguaggio di programmazione \glo{Swift} che permette di costruire grafiche 2D per applicazioni \glo{iOS} e MacOS. Permette di animare figure, immagini, testi in ambienti a due dimensioni.

\subsection*{SQL}
Acronimo di Structured Query Language. Linguaggio standardizzato per \glo{database} basati sul modello relazionale (RDBMS), progettato per creare e modificare schemi di database e interrogare i dati inseriti.

\subsection*{Stripe}
Software che permette a privati e aziende di inviare e ricevere pagamenti attraverso la rete Internet.

\subsection*{Swift}
Linguaggio di programmazione object-oriented (OOP) sviluppato da Apple che permette di creare applicazioni \glo{iOS} e MacOS.  

\subsection*{SwiftUI}
\glo{Framework} a supporto del linguaggio di programmazione \glo{Swift}. Facilita lo sviluppo della \glo{UI} per applicazioni \glo{iOS} e MacOS.