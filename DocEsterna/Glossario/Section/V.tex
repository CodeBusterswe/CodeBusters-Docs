\section*{V}
\markright{}
\addcontentsline{toc}{section}{V}

\subsection*{Validazione}
Processo che ci assicura che quanto prodotto nel progetto soddisfi i requisiti. Questo processo viene svolto solo sul prodotto finale e coinvolge fortemente il cliente tramite attività di collaudo.

\subsection*{VCS}
Acronimo di Version Control System. \'E un sistema che registra, nel tempo, i cambiamenti ad un file o ad una serie di file, così da poter richiamare una specifica versione in un secondo momento.

\subsection*{VCS distribuito}
\'E un tipo di \glo{VCS} dove i client non solo controllano lo \glo{snapshot} più recente dei file, ma piuttosto fanno una copia identica dell'archivio, completa di tutta la propria storia. In questo modo se un server smettesse di funzionare e se i sistemi interagissero tramite questo server, il \glo{repository} di un qualsiasi client potrebbe essere copiato sul server per ripristinarlo. Ogni clone è proprio un backup completo di tutti i dati.

\subsection*{Verificatore}
E un ruolo di sviluppo, in particolare di supporto allo sviluppo. Il verificatore si occupa di controllare se le attività e i prodotti sono conformi alle attese. Il verificatore svolge il suo lavoro durante l'intera durata del progetto. Deve avere competenze tecniche, esperienza e conoscenza delle norme. Altre qualità necessarie da non sottovalutare sono la capacità di giudizio per rendere quantificabile il lavoro supervisionato, e di relazione con il team per evitare attriti interni.


