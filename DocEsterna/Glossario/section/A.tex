\section*{A}
\markright{}
\addcontentsline{toc}{section}{A}

\subsection*{Android}
Sistema operativo per dispositivi mobili sviluppato da Google. È un software open source con controparti proprietarie e dal 2017 è il sistema operativo più diffuso al mondo. 
 
\subsection*{Angular}
\glo{Framework} open source per lo sviluppo di applicazioni web eseguite interamente dal web browser dopo essere state scaricate dal web server.

\subsection*{API}
Acronimo di Application Programming Interface. Si indica un  insieme  di  procedure disponibili al programmatore, di solito  raggruppate a formare un set di strumenti specifici per lo svolgimento di un determinato compito all'interno di un certo programma. Talvolta le API sono offerte tramite servizi a pagamento, oppure potrebbero essere funzionalità gratuite, come librerie software disponibili in un certo linguaggio di programmazione.

\subsection*{AWS}
Acronimo di Amazon Web Service. Azienda di proprietà di Amazon che offre diversi servizi di cloud computing su una piattaforma online.

\subsection*{AWS API Gateway}
Servizio \glo{AWS} per la creazione, pubblicazione, gestione e monitoraggio di \glo{API} che accedono ad altri servizi AWS o utilizzano dati archiviati in un AWS Cloud.

\subsection*{AWS Appsync}
Servizio \glo{AWS} utile per velocizzare la creazione di API gestendo attività che prevedono l'utilizzo di un \glo{AWS Dynamo DB} o un \glo{AWS Lambda}.

\subsection*{AWS Cognito Identity}
Servizio \glo{AWS} che permette di aggiungere velocemente diversi metodi di registrazione nell'API. Permette quindi all'utente di registrarsi utilizzando un profilo che già possiede di un altro social network, come per esempio Amazon, Facebook o Google. 

\subsection*{AWS Dynamo DB}
Servizio \glo{AWS} che fornisce un database \glo{NoSQL} completamente gestito. La sua rapidità nel prelevare informazioni, la sua facilità di utilizzo e la sua flessibilità lo hanno reso uno dei database più utilizzati, sia da aziende di piccole che di grandi dimensioni. 

\subsection*{AWS Gamelift}
Servizio \glo{AWS} che si occupa di distribuire, gestire e dimensionare i server cloud utilizzati per giochi multigiocatore. Migliora la latenza per i giocatori connessi ottimizzando i costi.

\subsection*{AWS Lambda}
Servizio \glo{AWS} che permette di gestire codice scritto senza dover occuparsi dell'amministrazione di esso o della gestione dei server. Lambda si occupa anche di calibrare costantemente le risorse necessarie, favorendo efficienza nel loro utilizzo. 

\subsection*{AWS S3}
Servizio \glo{AWS} di storage di oggetti che offre scalabilità, sicurezza e alte prestazioni sui dati che vengono salvati al suo interno. Permette di gestire ed organizzare con facilità una qualsiasi quantità di dati, nella massima sicurezza e per una vasta gamma di casi d'uso.
