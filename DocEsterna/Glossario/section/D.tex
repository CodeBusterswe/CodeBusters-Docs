\section*{D}
\markright{}
\addcontentsline{toc}{section}{D}

\subsection*{Database}
Letteralmente "base di dati". Rappresenta la versione digitale di un archivio di informazioni, ossia memorizza e organizza grandi moli di dati all'interno di dischi rigidi.

\subsection*{D3.js}
È una libreria \glo{JavaScript} per creare visualizzazioni dinamiche ed interattive partendo da dati organizzati, visibili attraverso un comune browser.

\subsection*{Deep learning}
Insieme di tecniche basate su reti neurali artificiali organizzate in diversi strati, dove ogni strato calcola i valori per quello successivo affinché l'informazione venga elaborata in maniera sempre più completa.

\subsection*{Dendrogramma}
Nelle tecniche di clustering viene utilizzato per fornire una rappresentazione grafica del processo di raggruppamento delle istanze.

\subsection*{Design pattern}
Un design pattern descrive una soluzione generale a un problema di progettazione ricorrente, gli attribuisce un nome, astrae e identifica gli aspetti principali della struttura utilizzata per la soluzione del problema, identifica le classi e le istanze partecipanti e la distribuzione delle responsabilità, descrive quando e come può essere applicato. 

\subsection*{Discord}
Applicazione \glo{multipiattaforma} basta sulla tecnologia VoIP (Voice over Internet Protocol) che consente di creare gruppi di messaggistica e videoconferenze. Questa applicazione funziona sia sui dispositivi mobili sia su browser.

\subsection*{Docker}
Progetto open source che automatizza il rilascio di applicazioni all'utente all'interno di contenitori software.


