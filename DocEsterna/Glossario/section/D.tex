\section*{D}
\markright{}
\addcontentsline{toc}{section}{D}
\subsection*{D3.js}
\'E una libreria \glo{JavaScript} per creare visualizzazioni dinamiche ed interattive partendo da dati organizzati, visibili attraverso un comune browser.

\subsection*{Deep learning}
Insieme di tecniche basate su reti neurali artificiali organizzate in diversi strati, dove ogni strato calcola i valori per quello successivo affinché l'informazione venga elaborata in maniera sempre più completa.

\subsection*{Design pattern}
Un design pattern descrive una soluzione generale a un problema di progettazione ricorrente, gli attribuisce un nome, astrae e identifica gli aspetti principali della struttura utilizzata per la soluzione del problema, identifica le classi e le istanze partecipanti e la distribuzione delle responsabilità, descrive quando e come può essere applicato. 

\subsection*{Discord}
Applicazione multipiattaforma basta sulla tecnologia VoIP (Voice over Internet Protocol) che consente di creare gruppi di messaggistica e videoconferenze. Questa applicazione è utilizzabile sia da applicazione (desktop o mobile) sia da browser.

\subsection*{Docker}
Progetto open-source che automatizza il rilascio di applicazioni all'utente all'interno di contenitori software.

\subsection*{DynamoDB}
Database \glo{NoSQL} di Amazon che permette agli sviluppatori di gestire qualsiasi volume di dati per la propria applicazione. DynamoDB permette prestazioni elevate e tempi di latenza bassi per eseguire applicazioni web che altrimenti sovraccaricherebbero i database relazionali tradizionali.

\subsection*{Discord}
Applicazione multipiattaforma basta sulla tecnologia \textbf{VoIP}(Voice over Internet Protocol) che consente di creare gruppi di messaggistica e videoconferenze. Questa applicazione funziona sia sui dispositivi mobili sia su browser.

