\section*{G}
\markright{}
\addcontentsline{toc}{section}{G}

\subsection*{Git}
Software di controllo di versione distribuito utilizzabile da interfaccia a riga di comando, creato da Linus Torvalds nel 2005. 

\subsection*{GitHub}
Servizio di hosting per sviluppatori. Fornisce uno strumento di controllo versione e permette lo sviluppo distribuito del software.

\subsection*{GitLab}
Piattaforma web open source che permette la gestione di \glo{repository} \glo{Git} e di funzioni trouble ticket. Fornisce uno strumento di controllo versione e permette lo sviluppo distribuito del software.

\subsection*{Gantt}
Un diagramma di Gantt è uno strumento utile per la pianificazione dei progetti. Attraverso una panoramica dei compiti programmati, tutte le parti interessate sono a conoscenza dei compiti e delle rispettive scadenze. Un diagramma di Gantt mostra le date di inizio e fine di un progetto, da quali attività è composto il progetto, le attività assegnate a ciascuna persona, le date previste per l'inizio e la fine delle attività, una stima di quanto tempo durerà ogni attività, come le attività si sovrappongono e/o sono collegate tra loro.
