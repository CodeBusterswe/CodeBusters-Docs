\section*{M}
\markright{}
\addcontentsline{toc}{section}{M}

\subsection*{M2M}
Acronimo di Machine-to-machine. Tecnologie e applicazioni di telemetria e telematica che utilizzano le reti wireless.

\subsection*{Machine learning}
Branca dell'intelligenza artificiale che utilizza metodi statistici per migliorare la performance di un algoritmo nell'identificare \glo{pattern} nei dati.

\subsection*{Microsoft Planner}
Applicazione di pianificazione e gestione di progetti. Questa applicazione consente di creare una panoramica di facile utilizzo delle attività che devono essere completate.

\subsection*{MQTT}
Acronimo di Message Queue Telemetry Transport. Protocollo ISO standard (ISO/IEC PRF 20922) di messaggistica leggero di tipo publish-subscribe posizionato in cima a TCP/IP. È stato progettato per le situazioni in cui è richiesto un basso impatto e dove la banda è limitata. 

\subsection*{Multi-piattaforma}
Riferito ad un linguaggio di programmazione, ad un'applicazione software o ad un dispositivo hardware che funziona su più di un sistema o appunto, piattaforma.

\subsection*{MVC}
Acronimo di Model-view-controller. Pattern architetturale molto diffuso, in particolare nell'ambito della programmazione orientata agli oggetti e in applicazioni web, in grado di separare la logica di presentazione dei dati dalla logica.