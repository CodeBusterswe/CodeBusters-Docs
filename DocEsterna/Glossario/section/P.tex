\section*{P}
\markright{}
\addcontentsline{toc}{section}{P}
\subsection*{PaaS}
Acronimo di Platform as a Service. Piattaforme di elaborazione che permettono di sviluppare, sottoporre a test, implementare e gestire le applicazioni aziendali senza i costi e la complessità associati all'acquisto, alla configurazione, all'ottimizzazione e alla gestione dell'hardware e del software di base. 

\subsection*{Pattern}
Rappresenta uno schema ricorrente, una struttura ripetitiva in uno specifico contesto. 

\subsection*{Proiezione lineare multi asse}
Grafico che posiziona i punti dello spazio multidimensionale in un piano cartesiano (con assi "draggabili"), effettuando quindi sui dati una riduzione a due dimensioni. Questo grafico non è tra quelli presenti negli esempi di \glo{D3.js}, ma è visibile nel programma di data mining “Orange Canvas” o nello strumento di visualizzazione “ggobi”.

\subsection*{Product baseline}
Presenta la baseline architetturale del prodotto (design patterns adottati) e va data tramite "allegato tecnico" (una banale cartella compressa per mail) con diagrammi delle classi e di sequenza. In questo momento deve esistere un prodotto, idealmente "finito",e bisogna essere pronti alla \glo{validazione}.

\subsection*{Proof of Concept}
Una realizzazione incompleta o abbozzata di un determinato progetto o metodo, allo scopo di provarne la fattibilità o dimostrare la fondatezza di alcuni principi o concetti costituenti. Un esempio tipico è quello di un prototipo. 

\subsection*{Proponente}
Ente o azienda che compie l’atto di proporre il \glo{capitolato d’appalto} per un progetto.

\subsection*{Protocollo asincrono}
Protocollo per la trasmissione di dati, un byte alla volta (un carattere). Si definiscono protocolli start-stop perché l'informazione da inviare deve essere accompagnata da un bit all'inizio (start della trasmissione) e un bit alla fine (stop della trasmissione).

\subsection*{Python}
Linguaggio di programmazione orientato a oggetti, adatto, tra gli altri usi, a sviluppare applicazioni distribuite, scripting, computazione numerica e system testing.

\subsection*{Pytorch}
Libreria open-source di machine-learning sviluppata da Facebook e basata sulla libreria Torch. È utilizzata in campi come la computer vision e per l'elaborazione del linguaggio naturale.

