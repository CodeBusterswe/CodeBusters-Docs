\section*{P}
\markright{}
\addcontentsline{toc}{section}{P}
\subsection*{PaaS}
Acronimo di Platform as a Service. Piattaforme di elaborazione che permettono di sviluppare, sottoporre a test, implementare e gestire le applicazioni aziendali senza i costi e la complessità associati all'acquisto, alla configurazione, all'ottimizzazione e alla gestione dell'hardware e del software di base. 
\subsection*{proiezione lineare multi asse}
La “proiezione lineare multi asse” posiziona i punti dello spazio multidimensionale in un piano cartesiano, riducendo a 2 dimensioni anche dati con molte più dimensioni. Per far intuire all’analista le strutture ed i raggruppamenti si lasciano spostare gli assi, tipicamente rendendo draggabili le frecce ad essi associate, o si crea un’animazione che ruota gli assi secondo percorsi prestabiliti. Questo grafico non è tra quelli presenti negli esempi di D3, ma è visibile nel programma di data mining “Orange Canvas” o nello strumento di visualizzazione “ggobi”.
\subsection*{Python}
Linguaggio di programmazione orientato a oggetti, adatto, tra gli altri usi, a sviluppare applicazioni distribuite, scripting, computazione numerica e system testing.
\subsection*{Pytorch}
Libreria open-source di machine-learning sviluppata da Facebook e basata sulla libreria Torch. \'E utilizzata in campi come la computer vision e per l'elaborazione del linguaggio naturale.

\subsection*{Microsoft Planner}
Applicazione di pianificazione e gestione di progetti. Questa applicazione consente di creare una panoramica di facile utilizzo delle attività che devono essere completate.