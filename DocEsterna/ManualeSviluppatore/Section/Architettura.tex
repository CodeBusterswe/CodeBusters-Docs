\section{Architettura}
L'architettura di \textit{HDViz} è basata sul design architetturale \textit{Model-View-ViewModel (MVVM)}, derivato dal più comune Model-View-Controller (MVC). 
È stato scelto MVVM per i seguenti motivi: 
\begin{itemize}
	\item Favorisce la separazione tra \textit{business logic} e \textit{presentational logic}, facendo comunicare \textit{Model} e \textit{View} solo attraverso il \textit{ViewModel};  
	\item Permette di non avere un unico controller con cui dover gestire tutta la \textit{application logic}. Essa è infatti suddivisa nei vari componenti che compongono la vista, fornendo diversi vantaggi: 
	\begin{itemize}
		\item Minor numero di conflitti in fase di codifica (non di deve accedere a uno stesso file dove è contenuta tutta la logica);
		\item Performance migliori (viene renderizzato solo il componente che effettivamente subisce modifiche del proprio stato interno).
	\end{itemize}
	\item Adatto per le web app la cui interfaccia utente viene sviluppata con la libreria React.
\end{itemize}

La comunicazione tra \textit{View} e \textit{ViewModel} avviene attraverso un contesto (\textit{Context React}) con il quale la vista è in grado di accedere e modificare il valore dei dati presenti nel \textit{ViewModel}, il quale contiene un'istanza del \textit{Model}. \\
L'utilizzo di un \textit{Context React} permette di non passare \glo{\textit{props}} in tutta la vista (da componente padre a componente figlio), evitando il possibile problema di dover attraversare tutta la gerarchia prima di raggiungere il componente che effettivamente le deve utilizzare.
\begin{figure}[hb]
\includegraphics[width=15.8cm]{Extra/MVVMPattern}
\centering
\caption{Model-View-ViewModel (MVVM)}
\end{figure}