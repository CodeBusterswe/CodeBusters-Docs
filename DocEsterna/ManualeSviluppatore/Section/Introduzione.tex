\section{Introduzione}
\subsection{Scopo del documento}
Questo documento ha lo scopo di servire da linea guida per gli sviluppatori che andranno ad estendere o manutenere il prodotto HDViz. Successivamente lo sviluppatore troverà nel documento tutte le informazioni riguardanti i linguaggi e le tecnologie utilizzate, l'architettura del sistema e le scelte progettuali effettuate per il prodotto; questo documento inoltre ha il fine di illustrare le procedure per l'installazione e lo sviluppo in locale.

\subsection{Scopo del prodotto}
Oggigiorno, anche i programmi più tradizionali gestiscono e memorizzano una grande mole di dati; di conseguenza servono software in grado di eseguire un'analisi e un'interpretazione delle informazioni.\\
Il prodotto HDViz ha come obiettivo quello di creare un'applicazione di visualizzazione di dati con numerose dimensioni in modo da renderle comprensibili all'occhio umano.  Lo scopo del prodotto sarà quello di fornire all'utente diversi tipi di visualizzazioni e di algoritmi per la riduzione dimensionale in modo che, attraverso un processo esplorativo, l'utilizzatore del prodotto possa studiare tali dati ed evidenziarne degli eventuali \glo{cluster}. 

\subsection{Glossario}
Per evitare ambiguità relative alle terminologie utilizzate, queste verranno evidenziati da una 'G' ad apice e riportate nel glossario presente nell'appendice \S A.

\subsection{Riferimenti}

\subsubsection{Riferimenti informativi}
\label{riferimenti}
\begin{itemize}
\item xx
\end{itemize}
	