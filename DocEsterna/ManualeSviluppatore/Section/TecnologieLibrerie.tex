\section{Tecnologie coinvolte}

{
\setlength\arrayrulewidth{0.95pt}
\renewcommand{\arraystretch}{1.5}
\begin{longtable}{C{3cm} C{2cm} C{10cm}}
\rowcolor{coloreRosso}

\textcolor{white}{\textbf{Tecnologia}}&
\textcolor{white}{\textbf{Versione}}&
\textcolor{white}{\textbf{Descrizione}} \\
\endfirsthead
\rowcolor{white}\multicolumn{3}{C{15cm}}{\textit{Continua nella pagina successiva...}}\\
\endfoot
\rowcolor{white}\caption{Tecnologie coinvolte}
\endlastfoot
	
\rowcolor{coloreRossoChiaro}
\multicolumn{3}{|c|}{\textcolor{white}{\textbf{Linguaggi}}} \\

	\textbf{JavaScript} & 
	\glo{ES6} &
	JavaScript è un linguaggio di programmazione utilizzato nella programmazione web per la creazione di effetti dinamici e interattivi tramite eventi invocati dall'utente. È il linguaggio della libreria React, utilizzata per costruire l'UI di \NomeProgetto{}.\\

	\textbf{HTML} &
	5 &
	HTML5 è un linguaggio di markup per la strutturazione di pagine web. Viene utilizzato nel progetto \NomeProgetto{} assieme a React per definire l'interfaccia grafica. In particolare React utilizza HTML e JavaScript in una sintassi chiamata \glo{JSX} che unisce i due linguaggi in singoli statement.\\
	
	\textbf{CSS} &
	3 & 
	CSS è un linguaggio usato per definire la formattazione di documenti HTML5, XHTML e XML, come ad esempio siti web.
Essendo il progetto \NomeProgetto{} composto di elementi HTML viene utilizzato il CSS per definirne lo stile. \\
	
 \rowcolor{coloreRossoChiaro}
\multicolumn{3}{|c|}{\textcolor{white}{\textbf{Strumenti}}} \\

	\textbf{PostgreSQL} & 
	13.x &
	PostgreSQL è un \glo{DBMS} ad oggetti ed offre caratteristiche all'avanguardia nel settore delle basi di dati. Viene utilizzato nel progetto per il caricamento nell'applicazione di basi di dati prefatte a disposizione dell'utente. \\
 
 	\textbf{Npm} & 
	7.x &
	Npm è un gestore di pacchetti per il linguaggio di programmazione JavaScript. Consiste in un client da linea di comando e un database online di pacchetti pubblici e privati, chiamato npm registry. Viene utilizzato dal gruppo per effettuare le operazioni di build del codice. \\
 
 \textbf{ESLint} & 
	7.15.0  &
	ESLint è uno strumento di analisi statica per identificare i modelli problematici trovati nel codice. Le regole in ESLint sono configurabili e personalizzabili a seconda delle esigenze del progetto.
In \NomeProgetto{} è stato utilizzato questo strumento principalmente per la segnalazione degli errori di sintassi, per avere regole  d'indentazione uguali in tutti i file e per notifiche su variabili, funzioni, componenti non utilizzati. \\
 
	\textbf{Babel} & 
	7.11.0 &
	Babel è un transcompiler JavaScript open source che viene utilizzato per convertire il codice ECMAScript 2015+ in una versione retro compatibile, così da rendere \NomeProgetto{} utilizzabile anche in browser non aggiornati. \\
  
 \rowcolor{coloreRossoChiaro}
\multicolumn{3}{|c|}{\textcolor{white}{\textbf{Librerie e framework}}}\\
 
	\textbf{React} & 
	17.0.1 &
	React è una libreria JavaScript per la creazione di UI, utilizzata principalmente come base nello sviluppo di applicazioni web.
Questa libreria è stata scelta per facilitare lo sviluppo del fron-tend e avere performance migliori grazie al suo metodo di renderizzazione dei componenti grafici.\\ 
 
 	\textbf{D3.js} & 
	6.x &
	D3.js è una libreria JavaScript per creare visualizzazioni dinamiche ed interattive partendo da dati organizzati.
Questa libreria è stata utilizzata nel progetto \NomeProgetto{} per la creazione delle visualizzazioni dei dati richiesti.\\

	\textbf{Node.js} & 
	14.16.0 &
	Node.js è un runtime system open source multipiattaforma orientato agli eventi per l'esecuzione di codice JavaScript, costruito sul motore JavaScript V8 di Google Chrome.
Node.js è stato utilizzato per lo sviluppo del progetto \NomeProgetto{} come strumento per l'utilizzo di Javascript lato server.\\
 
	\textbf{React Bootstrap} & 
	1.5.2 &
	Bootstrap è un \glo{framework} che permette la creazione di applicazioni web responsive. In particolare React Bootstrap fornisce componenti React con uno stile già integrato, senza quindi dover scrivere codice CSS. Essa è stata utilizzata per il progetto per componenti dell'interfaccia, come modal, pulsanti e form.\\
		
	\textbf{MobX} & 
	6.1.x &
	MobX è una libreria appositamente creata per React che permette la gestione dello \glo{state} dei componenti in maniera semplice e scalabile. Nel progetto \NomeProgetto{} è stata utilizza questa libreria per l'implementazione del design pattern Observer.\\
	
	\textbf{Druid.js} & 
	0.3.5 &
	DruidJS è una libreria JavaScript per la \glo{riduzione della dimensionalità}. DruidJS semplifica la proiezione di un dataset fornendo metodi di riduzione già implementati.
Nel progetto \NomeProgetto{} questa libreria viene usata per implementare il processo di riduzione dimensionale attraverso gli algoritmi \textit{IsoMap}, \textit{FastMap}, \textit{tSNE} e \textit{LLE}.\\

	\textbf{Express} & 
	4.17.1 &
	Express è il \glo{framework} standard open source per la comunicazione tra front-end e back-end di applicazioni web basate su server Node.js. 
Nel progetto \NomeProgetto{} Express fa da intermediario tra web app e database, agevolandone il collegamento.\\

	\textbf{Jest} & 
	26.x &
	Jest è un \glo{framework} open-source di test JavaScript gestito da Facebook. Funziona con progetti che utilizzano Babel, TypeScript, Node.js, React, Angular, Vue.js e Svelte. In \NomeProgetto{} è utilizzato per l'analisi dinamica del codice.\\

	\textbf{React Testing Library} & 
	11.2.x &
	La libreria React Testing Library è una libreria apposita per testare i componenti React. Anch'essa viene utilizzata in \NomeProgetto{} per la stesura dei test.\\
\end{longtable}	
}



