\section{Installazione}
Per accedere all'applicazione web è necessario:
Per installare il prodotto in locale è necessario seguire i passi qui riportati.
\begin{enumerate}
	\item Scaricare il codice come file .zip direttamente dal repository di CodeBusters-HDViz:
		\begin{center}
			\textcolor{blue}{\url{https://github.com/CodeBusterswe/CodeBusters-HDviz}}
		\end{center}	
	\item Estrarre il contenuto della cartella a piacere;
	\item Localizzare da terminale la cartella in cui si è stato estratto il prodotto:  
		\begin{center}
			\textcolor{darkgray}{\textbf{cd percorso\textbackslash HDViz}}
 		\end{center}	
	\item Entrare nella cartella "client":
		\begin{center}
			\textcolor{darkgray}{\textbf{cd client}}
 		\end{center}	     
	\item A questo punto è necessario installare tutti i "node\_ modules", ossia tutte le dipendenze dichiarate nel "package.json". Questo è possibile lanciando il comando:
		\begin{center}
			\textcolor{darkgray}{\textbf{npm install}}
 		\end{center}		
	Ci vorrà al massimo qualche minuto. 
	\item Terminata l'installazione è subito possibile avviare l'applicazione con il comando:
		\begin{center}
			\textcolor{darkgray}{\textbf{npm start}}
 		\end{center}
	Al termine dell'operazione dovrebbe aprirsi in automatico l'applicazione. In caso contrario aprire un browser e inserire l'indirizzo:
	\begin{center}
		\textbf{localhost:3000} .
	\end{center}		
\end{enumerate}



DA INSERIRE IMMAGINI