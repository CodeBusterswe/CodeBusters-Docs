\section{Analisi dei rischi}

Questa attività richiede attenzione costante e ha l'obiettivo di fare delle previsioni sui problemi che si potrebbero verificare durante l'intero periodo di svolgimento del progetto, classificandoli in base alla loro entità e apportando delle risoluzioni.\\ Grazie ad un monitoraggio continuo è inoltre possibile identificare nuovi rischi e raffinare le strategie pianificate.\\
Di seguito è riportata una tabella che riassume tutte le informazioni, realizzata con le seguenti fasi:
\begin{itemize}
\item \textbf{Identificazione:} attività che permette d'individuare gli eventi che potrebbero causare problemi durante l'avanzamento del progetto;
\item \textbf{Analisi:} studio degli eventi rilevati ed assegnazione di un indice di gravità e di una probabilità di occorrenza, rilevando così l'impatto che avrebbero nel progetto;
\item \textbf{Controllo:} pianificazione di una metodologia per evitare che
si verifichino i rischi individuati, stabilendo come agire nel caso in cui questi si riscontrassero;
\item \textbf{Monitoraggio:} durante lo svolgimento del progetto viene costantemente eseguito un controllo per:
	\begin{itemize}
		\item Rilevare eventuali nuovi indicatori di rischio;
		\item Aggiornare le informazioni già presenti.
	\end{itemize}
Questo fase risulta essere fondamentale perché con il tempo gli effetti sui rischi possono variare ed è necessario riportarli periodicamente all'attenzione di tutto il gruppo.
\end{itemize}
I rischi sono stati suddivisi nelle seguenti categorie:
\begin{itemize}
\item \textbf{Tecnologie scelte}: T;
\item \textbf{Rapporti interpersonali}: I;
\item \textbf{Organizzazione del lavoro}: O.
\end{itemize}
I rischi sono identificati dal seguente codice:
\begin{center}
	\textbf{R\{Iniziale categoria\}\{Numero progressivo\}}
\end{center}
\newpage

\newlength\colA\setlength\colA{2.5cm}
\newlength\colB\setlength\colB{5cm}
\newlength\colC\setlength\colC{3cm}
\newlength\total\setlength\total{\dimexpr\colA+\colB+\colB+\colC+6\tabcolsep\relax}
\renewcommand{\arraystretch}{2}%
\subsection{Rischi relativi alle tecnologie}
\rowcolors{1}{coloreGrigietto}{colorePanna}
\begin{center}
\begin{longtable}{C{\colA} C{\colB} C{\colB} C{\colC}}
		\rowcolor{coloreRosso}
		\textcolor{white}{\textbf{Codice}} & 
		\textcolor{white}{\textbf{Descrizione}} & 
		\textcolor{white}{\textbf{Identificazione}} & 
		\textcolor{white}{\textbf{Grado}} \\
		\endfirsthead
	    \rowcolor{white}\multicolumn{4}{C{\total}}{\textit{Continua nella pagina successiva...}}\\
	    \endfoot
	    \rowcolor{white}\caption{Tabella dei rischi tecnologici}
	    \endlastfoot

%--------------------------------------------
\textbf{RT1} \newline Scarsa esperienza &

Tutti i membri del gruppo non hanno ancora un'esperienza tale da affrontare un progetto di questa complessità senza l'insorgere di problemi operativi. & 

Ciascun membro del gruppo deve comunicare con assoluta trasparenza eventuali difficoltà incontrate.  & 

Pericolosità: \newline \textbf{Elevata} \newline Occorrenza: \newline \textbf{Elevata} \\

\multicolumn{4}{C{\total}}{\textbf{Piano di contingenza:} I compiti con difficoltà maggiore verranno assegnati a più componenti, in modo da favorire l'assistenza reciproca.} \\

%--------------------------------------------
\textbf{RT2} \newline Tecnologie da usare &

La documentazione disponibile per l'utilizzo delle tecnologie interessate è molto ampia. Il tempo di apprendimento potrebbe causare dei ritardi nello svolgimento dei lavori. & 

Il \textit{Responsabile} ha il compito di monitorare la preparazione dei membri rispetto ai compiti assegnati.  & 

Pericolosità: \newline \textbf{Elevata} \newline Occorrenza: \newline \textbf{Media} \\

\multicolumn{4}{C{\total}}{\textbf{Piano di contingenza:} In casi di particolare difficoltà è prevista una ridistribuzione del carico di lavoro.} \\

%--------------------------------------------
\textbf{RT3} \newline Strumenti software &

Il team si affida a strumenti software di terze parti e piattaforme online. Potrebbe esserci il rischio di perdita di dati o non operatività. & 

Qualsiasi membro ha il compito di avvisare il \textit{Responsabile} e gli altri componenti in caso di rilevamento di problemi.  & 

Pericolosità: \newline \textbf{Media-Elevata} \newline Occorrenza: \newline \textbf{Bassa} \\

\multicolumn{4}{C{\total}}{\textbf{Piano di contingenza:} effettuare un backup dei dati periodico su altre piattaforme.} \\


%--------------------------------------------
\textbf{RT4} \newline Problemi hardware &

Tutti i componenti del gruppo utilizzano dispositivi personali per lavorare al progetto. Guasti hardware potrebbero causare notevoli disagi e perdite di tempo. & 

Ciascun membro dovrà, nei limiti del possibile, evitarli ed avvisare il \textit{Responsabile} e gli altri componenti dei problemi riscontrati.  & 

Pericolosità: \newline \textbf{Media} \newline Occorrenza: \newline \textbf{Bassa} \\

\multicolumn{4}{C{\total}}{\textbf{Piano di contingenza:} ogni componente deve rispettare l'utilizzo degli strumenti stabiliti nelle \textit{Norme di progetto} per evitare perdite di dati.} \\

\end{longtable}
\end{center}

\vspace{-1cm}
\subsection{Rischi relativi ai rapporti interpersonali}
\rowcolors{1}{coloreGrigietto}{colorePanna}
\begin{center}
\begin{longtable}{C{\colA} C{\colB} C{\colB} C{\colC}}
		\rowcolor{coloreRosso}
		\textcolor{white}{\textbf{Codice}} & 
		\textcolor{white}{\textbf{Descrizione}} & 
		\textcolor{white}{\textbf{Identificazione}} & 
		\textcolor{white}{\textbf{Grado}} \\
		\endfirsthead
	    \rowcolor{white}\multicolumn{4}{C{\total}}{\textit{Continua nella pagina successiva...}}\\
	    \endfoot
	    \rowcolor{white}\caption{Tabella dei rischi interpersonali}
	    \endlastfoot

%--------------------------------------------
\textbf{RI1} \newline Collaborazione a distanza &

A causa della situazione di emergenza sanitaria dovuta al virus \textit{Covid-19}, il gruppo potrebbe trovarsi in difficoltà a lavorare e a comunicare sia internamente che con il \textit{proponente}, causando pesanti conseguenze nello svolgimento del progetto. & 

Ciascun membro del gruppo deve utilizzare le piattaforme stabilite e mantenere una comunicazione attiva. Inoltre dovrà adattarsi alle politiche interne dell'azienda proponente per garantire la collaborazione.  & 

Pericolosità: \newline \textbf{Elevata} \newline Occorrenza: \newline \textbf{Media} \\

\multicolumn{4}{C{\total}}{\textbf{Piano di contingenza:} il gruppo ha predisposto molteplici canali di comunicazione su dispositivi fissi e mobili. Inoltre si devono effettuare riunioni frequenti sia internamente che con il \textit{proponente}.} \\

%--------------------------------------------
\textbf{RI2} \newline Conflitti decisionali &

Alcuni componenti potrebbero essere in disaccordo rispetto ad alcune decisioni, provocando l'insorgere di conflitti e situazioni spiacevoli. & 

Ciascun membro del gruppo deve esporre con assoluta trasparenza le proprie opinioni, mentre il \textit{Responsabile} dovrà favorire una buona collaborazione.  & 

Pericolosità: \newline \textbf{Media} \newline Occorrenza: \newline \textbf{Media}\\

\multicolumn{4}{C{\total}}{\textbf{Piano di contingenza:} tutte le proposte devono essere valutate e discusse, con l'unico obbiettivo di scegliere quelle più adatte al bene del progetto.}\\
%--------------------------------------------
\textbf{RI3} \newline Comunicazione interna &

In alcuni momenti i membri del team potrebbero non essere reperibili, causando dei ritardi nello svolgimento delle riunioni prefissate. & 

Ciascun componente deve comunicare con dovuto anticipo eventuali momenti di irreperibilità attraverso i canali utilizzati dal gruppo di lavoro.  & 

Pericolosità: \newline \textbf{Media} \newline Occorrenza: \newline \textbf{Bassa}  \\

\multicolumn{4}{C{\total}}{\textbf{Piano di contingenza:} il gruppo ha organizzato degli incontri a cadenza fissa in modo da essere sempre allineati con l'avanzamento del progetto.} \\

%--------------------------------------------
\textbf{RI4} \newline Comunicazione esterna &

Il gruppo potrebbe avere una scarsa comunicazione con il \textit{proponente}, instaurando un cattivo rapporto di collaborazione e causando rallentamenti con l'avanzamento del progetto. & 

Sia il \textit{proponente} che il gruppo devono comunicare eventuali difficoltà che non permettano una normale comunicazione, cercando di trovare dei metodi alternativi a quelli prefissati per mantenerla attiva.  & 

Pericolosità: \newline \textbf{Media} \newline Occorrenza: \newline \textbf{Bassa} \\

\multicolumn{4}{C{\total}}{\textbf{Piano di contingenza:} esporre periodicamente quesiti e dubbi al \textit{proponente}, effettuando delle riunioni e mostrando l'avanzamento del progetto.} \\

%--------------------------------------------
\textbf{RI5} \newline Contrasti &

Come in qualsiasi attività collaborativa con più persone potrebbero crearsi dei conflitti di varia entità e natura. & 

Tutti i componenti devono limitare le tensioni ed evitare che queste influiscano sull'avanzamento del progetto.  & 

Pericolosità: \newline \textbf{Media} \newline Occorrenza: \newline \textbf{Bassa} \\

\multicolumn{4}{C{\total}}{\textbf{Piano di contingenza:} tutti i membri non interessati nella controversia, insieme al \textit{Responsabile}, hanno il compito di sedare i contrasti che si possono instaurare durante tutto il periodo.}\\
\end{longtable}
\end{center}
\subsection{Rischi relativi all'organizzazione}
\rowcolors{1}{coloreGrigietto}{colorePanna}
\begin{center}
\begin{longtable}{C{\colA} C{\colB} C{\colB} C{\colC}}
		\rowcolor{coloreRosso}
		\textcolor{white}{\textbf{Codice}} & 
		\textcolor{white}{\textbf{Descrizione}} & 
		\textcolor{white}{\textbf{Identificazione}} & 
		\textcolor{white}{\textbf{Grado}} \\
		\endfirsthead
	    \rowcolor{white}\multicolumn{4}{C{\total}}{\textit{Continua nella pagina successiva...}}\\
	    \endfoot
	    \rowcolor{white}\caption{Tabella dei rischi organizzativi}
	    \endlastfoot

%--------------------------------------------
\textbf{RO1} \newline Calcolo delle tempistiche &

L'inesperienza e la complessità del progetto potrebbero portare al non rispetto delle scadenze e a continue modifiche nel calcolo del consumo delle risorse. & 

A ciascuno componente saranno affidati dei compiti. Sarà onere di chi appartiene il compito comunicare tutte le difficoltà riscontrate ed eventuali ritardi nel rispetto delle scadenze.  & 

Pericolosità: \newline \textbf{Elevata} \newline Occorrenza: \newline \textbf{Media} \\

\multicolumn{4}{C{\total}}{\textbf{Piano di contingenza:} in caso di problematiche il \textit{Responsabile} ha il compito di assegnare nuove risorse per concludere l'attività. Inoltre tutti i membri devono collaborare per evitare situazioni di questa tipologia.} \\

%--------------------------------------------
\textbf{RO2} \newline Costi &

La pianificazione prevede un costo per ogni attività. Essendo il gruppo inesperto potrebbero essere presi in considerazioni dei valori poco veritieri.  & 

Ciascun componente del gruppo ha il compito di prendere nota delle proprie ore dedicate allo studio personale e al lavoro.  & 

Pericolosità: \newline \textbf{Elevata} \newline Occorrenza: \newline \textbf{Media}\\

\multicolumn{4}{C{\total}}{\textbf{Piano di contingenza:} monitorare la quantità di lavoro svolta da ciascun componente ed aggiornare i valori per tutto l'avanzamento del progetto.}\\

%--------------------------------------------
\textbf{RO3} \newline Impegni &

Tutti i membri del gruppo potrebbero causare problemi all'avanzamento del progetto per impegni sia accademici che personali.  & 

Ciascun componente ha il dovere di comunicare al gruppo tutti gli impegni, in modo da favorire un'organizzazione ottimale delle varie attività.  & 

Pericolosità: \newline \textbf{Bassa} \newline Occorrenza: \newline \textbf{Media}\\

\multicolumn{4}{C{\total}}{\textbf{Piano di contingenza:} utilizzare un calendario condiviso e visibile a tutto il gruppo, in modo da assegnare correttamente incarichi e scadenze. }\\

%--------------------------------------------
\textbf{RO3} \newline Modifiche dei requisiti &

Durante lo sviluppo del prodotto software, l'azienda potrebbe decidere di modificare i requisiti obbligatori, causando problemi interni sull'organizzazione delle attività e scadenze che erano state prefissate.  & 

Il gruppo deve mantenere una comunicazione attiva e un rapporto collaborativo con il \textit{proponente}, in modo da percepire le intenzioni rispetto al prodotto finale.  & 

Pericolosità: \newline \textbf{Elevata} \newline Occorrenza: \newline \textbf{Bassa}\\

\multicolumn{4}{C{\total}}{\textbf{Piano di contingenza:} fare riferimento alle precauzioni stabilite per \textbf{RI4}. }\\
\end{longtable}
\end{center}
