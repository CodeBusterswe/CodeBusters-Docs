\section{Attualizzazione dei rischi}

Questa sezione riporta i rischi che si sono effettivamente verificati e come il gruppo si è dedicato alla loro risoluzione.

\begin{longtable}{C{2.5cm} C{7cm} C{7cm}} 
 	\rowcolor{coloreRosso}
 	\color{white}{\textbf{Rischio}} &
 	\color{white}{\textbf{Descrizione}} &
 	\color{white}{\textbf{Mitigazione}} \\
 	
 	RI1 & A causa della situazione di emergenza sanitaria dovuta al virus \textit{Covid-19} il gruppo ha avuto modo di collaborare solamente a distanza. & Tutti i componenti hanno comunicato il più possibile, aggiornandosi reciprocamente in modo costante sull'andamento dei lavori e su eventuali problemi riscontrati. \\
 	RI4 & Il gruppo ha riscontrato alcuni dubbi in merito alla realizzazione del prodotto finale, causando alcuni rallentamenti durante la redazione dei documenti. & Il team, in seguito a diverse discussioni, ha deciso di mettersi nuovamente in contatto con il \textit{proponente} per avere dei chiarimenti. \\
 	RT3 & Alcuni componenti hanno avuto difficoltà nell'utilizzare \glo{GitHub} per la collaborazione e il versionamento, causando problemi nella struttura impostata. & I membri più esperti si sono dedicati nel sistemare gli inconvenienti e ad insegnare l'utilizzo dello strumento a chi ne aveva bisogno. \\
 	RO1 & La figura del \textit{Responsabile} ha riscontrato difficoltà nel calcolo della mole di lavoro; conseguentemente, sono state necessarie più distribuzioni delle ore di lavoro tra i membri. & \'E stata attuata un'analisi più approfondita con l'aiuto degli altri componenti. \\
 	\rowcolor{white}
 	\caption{Attualizzazione dei rischi}
\end{longtable}