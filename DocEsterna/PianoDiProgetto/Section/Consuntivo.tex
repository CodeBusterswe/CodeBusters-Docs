\section{Consuntivo di periodo}

%Di seguito vengono indicate le spese effettivamente sostenute per ciascun periodo.

\subsection{Fase di analisi}

Le ore di lavoro accumulate durante questa fase non sono rendicontate e vengono considerate come investimento. Pertanto la spesa non sarà a carico del proponente.

{
\setlength\arrayrulewidth{1pt}
\begin{longtable}{ C{4cm} C{3cm} C{3.5cm}} 
 	\rowcolor{coloreRosso}
 	\color{white}{\textbf{Ruolo}} &
 	\color{white}{\textbf{Ore}} &
 	\color{white}{\textbf{Costo €}} \\
 	
 	Responsabile & 25 \color{coloreRosso}{\textbf{(+5)}} & 900 \color{coloreRosso}{\textbf{(+150 €)}}\\
 	Amministratore & 45 & 900\\
 	Analista & 78 \color{coloreRosso}{\textbf{(+9)}} & 2175 \color{coloreRosso}{\textbf{(+225 €)}}\\
 	Progettista & - & -\\
 	Programmatore & - & -\\
 	Verificatore & 62 & 930\\
 	
	\hline 	
 	
 	\textbf{Totale preventivo} &
	210 &
 	4530 € \\		
 	
 	\textbf{Totale consuntivo} &
	224 &
 	4905 € \\	
 	
 	\textbf{Differenza} &
	+14 &
 	+375 € \\	
 	
 	\rowcolor{white}
 	\caption{Consuntivo della fase di analisi}
\end{longtable}
}

\subsubsection{Ragione degli scostamenti}
Come si può notare dalla tabella, durante questa fase sono state richieste più ore rispetto a quelle preventivate.
\begin{itemize}
\item \textbf{Responsabile}: come riportato nella \S B, a causa dell'inesperienza, la seguente figura ha riscontrato problemi nella suddivisione del carico del lavoro, richiedendo un'analisi più minuziosa e diversi cambiamenti durante questo periodo;
\item \textbf{Analista}: come riportato nella \S B, durante la fase di analisi sono sorti diversi dubbi che hanno richiesto sia uno studio più approfondito sia una comunicazione diretta con il proponente.
\end{itemize}

\subsubsection{Considerazioni}
Nonostante la variazione del bilancio, non è necessario adottare alcuna contromisura poiché le ore di lavoro non sono rendicontate. Tuttavia il gruppo s'impegna ad evitare in futuro situazioni che potrebbero causare nuove variazioni al monte ore calcolato.

\subsection{Fase di Progettazione Architetturale}

{
\setlength\arrayrulewidth{1pt}
\begin{longtable}{ C{1.5cm} C{1.5cm} | C{3cm} | C{3cm} C{3cm} | C{2.2cm}} 
 	\rowcolor{coloreRosso}
 	 &
 	 & 
 	 \color{white}{\textbf{Prog. della TB}} & 
 	 \multicolumn{2}{c |}{\color{white}{\textbf{Prog. e codifica del POC}}} & 
 	 \\

	\hline 	
 	
 	\rowcolor{coloreRossoChiaro}
 	\color{white}{\textbf{Ruolo}} &
 	\color{white}{\textbf{Ore}} &
 	\color{white}{\textbf{Periodo unico}} &
 	\color{white}{\textbf{Incr. I}} &
 	\color{white}{\textbf{Incr. II}} & 
 	\color{white}{\textbf{Costo €}}\\
 	
 	RE & 
 	12 & 
 	\color{coloreRosso}{\textbf{}} &
 	\color{coloreRosso}{\textbf{+1}} &
 	\color{coloreRosso}{\textbf{}} &
 	360 \color{coloreRosso}{\textbf{(+30)}}\\
 	
 	AM & 
 	30 & 
 	\color{coloreRosso}{\textbf{}} &
 	\color{coloreRosso}{\textbf{}} &
 	\color{coloreRosso}{\textbf{}} &
 	600 \color{coloreRosso}{\textbf{}}\\
 	
 	AN & 
 	43 & 
 	\color{coloreRosso}{\textbf{-2}} &
 	\color{coloreRosso}{\textbf{}} &
 	\color{coloreRosso}{\textbf{-1}} &
 	1075 \color{coloreRosso}{\textbf{(-75)}}\\
 	
 	PT & 
 	60 & 
 	\color{coloreRosso}{\textbf{}} &
 	\color{coloreRosso}{\textbf{-2}} &
 	\color{coloreRosso}{\textbf{}} &
 	1320 \color{coloreRosso}{\textbf{(-44)}}\\
 	
 	PR & 
 	34 & 
 	\color{coloreRosso}{\textbf{+4}} &
 	\color{coloreRosso}{\textbf{}} &
 	\color{coloreRosso}{\textbf{+2}} &
 	510 \color{coloreRosso}{\textbf{(+90)}}\\
 	
 	VE & 
 	45 & 
 	\color{coloreRosso}{\textbf{-5}} &
 	\color{coloreRosso}{\textbf{}} &
 	\color{coloreRosso}{\textbf{+4}} &
 	675 \color{coloreRosso}{\textbf{(-15)}}\\
 	
 	\hline
 	
 	\cellcolor{white}&
 	\cellcolor{white}&
 	\multicolumn{2}{c |}{\textbf{Totale preventivo}} &
	224 &
 	4540 €\\		
 	
 	\cellcolor{white}&
 	\cellcolor{white}&
 	\multicolumn{2}{c |}{\textbf{Totale consuntivo}} &
	225 &
 	4526 € \\	
 	
 	\cellcolor{white}&
 	\cellcolor{white}&
 	\multicolumn{2}{c |}{\textbf{Differenza}} &
	+1 &
 	-14 €\\	
 		
 	\rowcolor{white}
 	\caption{Consuntivo della fase di progettazione architetturale}
\end{longtable}


}

\subsubsection{Ragione degli scostamenti}
\paragraph{Progettazione della technology baseline} \mbox{}\\ \mbox{}\\
Il team si è impegnato ad apportare le modifiche segnalate alla precedente revisione sulla maggior parte dei documenti, oltre ad integrarli, ed individuare le tecnologie necessarie allo sviluppo del prodotto. 
\begin{itemize}
\item \textbf{Analista (-2 ore)}: grazie alla buona base prodotta nella fase precedente, sono servite meno ore di quelle preventivate in quanto l'\textit{Analisi dei Requisiti} è stata per lo più integrata. In particolare, dopo aver scelto quali algoritmi di riduzione dimensionale utilizzare, sono stati approfonditi i casi d'uso e i requisiti;

\item \textbf{Programmatore (+4 ore)}: la seguente figura ha occupato più tempo di quello che si era previsto per testare le tecnologie individuate. In particolare sono state riscontrate alcune difficoltà nell'utilizzare come base la libreria \glo{React}, scelta per lo sviluppo dell'\glo{UI};

\item \textbf{Verificatore (-5 ore)}: poiché il lavoro degli analisti e degli amministratori è stato svolto con precisione e in breve tempo, anche il lavoro dei verificatori è stato più veloce del previsto.
\end{itemize}

\paragraph{Progettazione e codifica del Proof of Concept} \mbox{}\\

Durante l'incremento I il gruppo si è occupato di terminare i lavori in fase di conclusione e iniziare lo sviluppo del \glo{PoC}.

\begin{itemize}
\item \textbf{Responsabile (+1 ora)}: il tempo richiesto per ultimare il \textit{Piano di Progetto} è stato leggermente più del dovuto. Parte del rallentamento è stata causata anche al complicato coordinamento delle attività dovute a questa fase;

\item \textbf{Progettista (-2 ore)}: la progettazione necessaria all'incremento, grazie all'esperienza maturata in precedenza, ha richiesto meno ore del previsto.

\end{itemize}

Durante l'incremento II il team si è concentrato ad ultimare il \glo{PoC}, ovvero nell'implementazione del primo grafico che il prodotto dovrà offrire.

\begin{itemize}
\item \textbf{Analista (-1 ore)}: grazie ad una comunicazione costante con il proponente, gli analisti hanno individuato rapidamente degli algoritmi di calcolo della distanza tra i dati. Per questo motivo il raffinamento dei requisiti e dei casi d'uso ha richiesto meno tempo;

\item \textbf{Programmatore (+2 ore)}: sono state riscontrate alcune difficoltà nell'implementare la visualizzazione \glo{\textit{Scatter Plot Matrix}} in modo dinamico, ovvero nella creazione automatica dei singoli \glo{\textit{Scatter Plot}} in base al set di dati caricato;

\item \textbf{Verificatore (+4 ore)}: quasi tutte le ore che non erano state utilizzate nel primo periodo sono state impiegate per garantire una maggiore qualità dei documenti.
\end{itemize}

\subsubsection{Considerazioni}

Durante questa prima fase di sviluppo, nonostante il coinvolgimento da parte di tutti i membri, ci sono state delle variazioni sul numero di ore assegnato a ciascuna persona nei vari sottoperiodi; questo si può notare anche per il numero di ore impiegato dalle varie figure, leggermente instabile. Il motivo principale è stata la sessione degli esami che ha occupato i componenti in momenti completamente diversi.\\ Grazie ad un rapporto di lavoro sempre più consolidato, l'andamento accademico dei vari membri è stato assimilato e questo ha permesso di effettuare delle considerazioni sulla pianificazione futura dei lavori.

\subsection{Pianificazione futura}

Per il prossimo periodo di progettazione di dettaglio e codifica si prevede un andamento dei lavori poco allineato tra i membri del team. Questo è sostanzialmente dovuto al fatto che circa metà dei componenti avranno dei corsi universitari da seguire, mentre altri dovranno dedicarsi a recuperare informazioni sullo stage. probabilmente saranno necessarie più ore da programmatore (+3 previste) e più ore da verificatore (+3 previste) visti gli inconvenienti passati sull'implementazione dei grafici e la quantità di documenti e software che bisognerà verificare. \\ Per quanto riguarda l'ultimo periodo di validazione e collaudo, al momento il team prevede di rispettare le ore preventivate grazie anche alla maturità e le skills sviluppate e che verranno apprese.

\subsection{Preventivo a finire}
Di seguito il preventivo a finire, raffinato in base alle considerazioni sulla pianificazione futura con i costi che ad oggi si prevede di sostenere.
{
%\setlength\arrayrulewidth{0.975pt}
\begin{longtable}{ C{3.5cm} | C{5cm} | C{5cm} | C{2cm}} 
 	\rowcolor{coloreRosso}
 	\color{white}{\textbf{Fase}} &
 	\color{white}{\textbf{Preventivo}} &
 	\color{white}{\textbf{Consuntivo}} & 
 	\color{white}{\textbf{Sostenuta}} \\
 	
 	Progettazione architetturale & 4.540,00€ , 224 ore & 4526 € , 225 ore & \begin{LARGE}\redcheck \end{LARGE}\\
 	Progettazione di dettaglio e codifica & 6.210,00€ , 350 ore & 6.300,00€ , 356 ore & Previsione: +90€, +6\\
 	Validazione e collaudo & 2.605,00€ , 140 ore  & 2,605,00€ , 140 ore   & Previsione: invariata\\
 	\textbf{Totale} & \textbf{13.335,00€ , 714 ore} & \textbf{13.431,00€ , 721 ore} &  \\
 	
 	\rowcolor{white}
 	\caption{Preventivo a finire}
\end{longtable}
}

\newpage