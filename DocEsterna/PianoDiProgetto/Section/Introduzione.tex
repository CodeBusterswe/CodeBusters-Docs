\section{Introduzione}

\subsection{Scopo del documento}
Questo documento ha lo scopo di fornire un prospetto della pianificazione dettagliata e delle modalità attraverso le quali verrà sviluppato il progetto. All'interno vengono anche riportate le problematiche che il team potrebbe incontrare lungo tutto il periodo. Al termine sono presenti i preventivi e consuntivi di periodo.

\subsection{Scopo del capitolato}
Oggigiorno, anche i programmi più tradizionali gestiscono e memorizzano una grande mole di dati; di conseguenza servono software in grado di eseguire un'analisi e un'interpretazione delle informazioni.\\
Il \glo{capitolato} C4 ha come obiettivo quello di creare un'applicazione di visualizzazione di dati con numerose dimensioni in modo da renderle comprensibili all'occhio umano.  Lo scopo del prodotto sarà quello di fornire all'utente diversi tipi di visualizzazioni e di algoritmi per la riduzione dimensionale in modo che, attraverso un processo esplorativo, l'utilizzatore del prodotto possa studiare tali dati ed evidenziarne degli eventuali \glo{cluster}. 

\subsection{Glossario}
Per evitare ambiguità relative alle terminologie utilizzare, è stato compilato il \Glossariov{1.0.0}. In questo documento sono riportati tutti i termini di particolare importanza e con un significato particolare. Questi termini sono evidenziati da una 'G' ad apice.

\subsection{Riferimenti}
\subsubsection{Riferimenti normativi}
\begin{itemize}	
	\item \textbf{Norme di Progetto v1.0.0};
	
	\item \textbf{Regolamento organigramma}:\\
	\textcolor{blue}{\url{https://www.math.unipd.it/~tullio/IS-1/2020/Progetto/RO.html}}
\end{itemize}

\subsubsection{Riferimenti informativi}
\begin{itemize}
	\item \textbf{Analisi dei Requisiti v1.0.0};
	
	\item \textbf{Capitolato d'appalto C4 - HD Viz: visualizzazione di dati multidimensionali}:\\
	\textcolor{blue}{\url{https://www.math.unipd.it/~tullio/IS-1/2020/Progetto/C4.pdf}}
	
	\item \textbf{Software Engineering - Ian Sommerville - 10th Edition}\\ Part 4: Software Management:
	\begin{itemize}
	\item Chapter 22 - Project management;
	\item Chapter 23 - Project planning.
	\end{itemize}		 
	
	\item \textbf{Slide T5 del corso Ingegneria del Software - Ciclo di vita del software}:\\
	\textcolor{blue}{\url{https://www.math.unipd.it/~tullio/IS-1/2020/Dispense/L05.pdf}}
	
	\item \textbf{Slide T6 del corso Ingegneria del Software - Gestione di Progetto}:\\
	\textcolor{blue}{\url{https://www.math.unipd.it/~tullio/IS-1/2020/Dispense/L06.pdf}}
\end{itemize}

\subsubsection{Scadenze}
Il gruppo \Gruppo{} si impegna a rispettare le seguenti scadenze per lo sviluppo del progetto:
\begin{itemize}
\item \textbf{Revisione dei Requisiti:} 11-01-2021;
\item \textbf{Revisione di Progettazione:} 01-03-2021;
\item \textbf{Revisione di Qualifica:} 02-04-2021;
\item \textbf{Revisione di Accettazione:} 03-05-2021.
\end{itemize}

