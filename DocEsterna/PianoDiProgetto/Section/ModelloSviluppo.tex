\section{Modello di sviluppo}
Il gruppo ha deciso di utilizzare il modello incrementale.
\subsection{Modello incrementale}
Il modello incrementale prevede rilasci multipli e successivi, ciascuno di questi realizza un incremento di funzionalità.
\'E richiesta dunque una classificazione preliminare dei requisiti atta ad individuare i più importanti, i quali devono essere sviluppati nei primi incrementi, così da avere fin da subito un prodotto funzionante, che verrà via via integrato.
L'adozione di questo modello comporta i seguenti vantaggi:
\begin{itemize}
\item Le funzionalità primarie hanno priorità nello sviluppo, così facendo queste vengono verificate più volte;
\item L'avere un prodotto funzionante già dai primi incrementi permette subito al proponente di valutarne le funzioni primarie;
\item Ogni incremento riduce il rischio di fallimento, con un approccio predisposto ai cambiamenti;
\item L'analisi dei requisiti può essere raffinata tramite la progettazione di dettaglio ad ogni incremento;
\item Le modifiche e la correzione degli errori sono più economiche;
\item Le fasi di verifica e test sono facilitate perché mirate a un singolo incremento.
\end{itemize}
\subsection{Incrementi individuati}
In seguito è riportata la tabella con indicati gli incrementi individuati durante la fase di analisi con il rispettivo obiettivo e i requisiti ad esso associati.
I requisiti riportati nella tabella includono tutti i requisiti figli. Tutti i requisiti non riportati sono da intendersi soddisfatti, in parte, da ogni incremento.
Ogni requisito è individuato dal suo codice identificativo, reperibile nel documento \AdRv{}.

\rowcolors{1}{coloreGrigietto}{colorePanna}
\begin{longtable}{C{3cm} C{7cm} C{2cm} C{2cm}}
\rowcolor{coloreRosso}
\textcolor{white}{\textbf{Incremento}} & 
\textcolor{white}{\textbf{Obiettivo dell'incremento}} & 
\textcolor{white}{\textbf{Requisiti}} &
\textcolor{white}{\textbf{Casi d'uso}}\\
\endfirsthead
\rowcolor{white}\multicolumn{4}{c}{\textit{Continua nella pagina successiva...}}\\
\endfoot
\rowcolor{white}\caption{Tabella degli incrementi}
\endlastfoot

%------------------------------------------
Incremento 0 & 
Caricamento dati tramite file e selezione delle dimensioni da utilizzare & 
R1F1 \newline R1F3 & 
UC1.1 \newline UC2\\
%-----------------------------------------
Incremento 1 &
Visualizzazione \glo{Scatter Plot Matrix} & 
R1F5.1 & 
UC3.1 \\
%------------------------------------------
Incremento 2 & 
Personalizzazione della visualizzazione Scatter Plot Matrix & 
R3F10 \newline R3F11 & 
UC4.1\\
%------------------------------------------
Incremento 3 & 
Visualizzazione \glo{Heat Map} e relativa personalizzazione & 
R1F5.2 \newline R1F5.2.1 \newline R3F10 \newline R3F11 & 
UC3.2 \newline UC4.2\\
%------------------------------------------
Incremento 4 & 
Visualizzazione \glo{Force Field} e relativa personalizzazione & 
R1F5.3 \newline R3F10 \newline R3F11 & 
UC3.3 \newline UC4.3\\
%------------------------------------------
Incremento 5 & 
Visualizzazione \glo{Proiezione Lineare Multi Asse} e relativa personalizzazione & 
R1F5.4 \newline R3F10 \newline R3F11& 
UC3.4 \newline UC4.4\\
%------------------------------------------
Incremento 6 & 
Gestione della sessione & 
R2F4 & 
UC5 \newline UC6\\
%------------------------------------------
Incremento 7 & 
Implementazione di un \glo{database} per il caricamento dati attraverso \glo{query} & 
R1F1.2 & 
UC1.2\\
%------------------------------------------
Incremento 8 & 
Perfezionamento del codice con correzioni ricevute durante la \textit{Technology Baseline} e dal proponente & 
Non saranno aggiunte nuove funzionalità software & 
Non saranno aggiunte nuove funzionalità software\\

\end{longtable}
