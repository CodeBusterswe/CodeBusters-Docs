\section{Introduzione}
\subsection{Scopo del documento}
Questo documento ha lo scopo di fornire tutte le informazioni relative al sistema di controllo di qualità per i processi ed i prodotti, basandosi su assunti misurabili ma adattati alle esigenze del proprio progetto.
Esso deve implementare degli standard che permettano il miglioramento continuo, tracciando periodicamente tramite misurazioni i risultati ottenuti sfruttandoli per definire azioni migliorative. All'interno del \textit{Piano di Qualifica} vengono anche raccolte le definizioni dei test, il loro stato e il loro tracciamento. 

\subsection{Scopo del capitolato}
Oggigiorno, anche i programmi più tradizionali gestiscono e memorizzano una grande mole di dati; di conseguenza servono software in grado di eseguire un'analisi e un'interpretazione delle informazioni.\\
Il \glo{capitolato} C4 ha come obiettivo quello di creare un'applicazione di visualizzazione di dati con numerose dimensioni modo da renderle comprensibili all'occhio umano.  Lo scopo del prodotto sarà quello di fornire all'utente diversi tipi di visualizzazioni e di algoritmi per la riduzione dimensionale in modo che, attraverso un processo esplorativo, l'utilizzatore del prodotto possa studiare tali dati ed evidenziarne degli eventuali \glo{cluster}. 

\subsection{Glossario}
Per evitare ambiguità relative alle terminologie utilizzate, è stato compilato il \Glossariov{1.0.0}. In questo documento sono riportati tutti i termini importanti e con un significato particolare. Questi termini sono evidenziati da una 'G' ad apice.

\subsection{Riferimenti}
\subsubsection{Riferimenti normativi}
\begin{itemize}	
\item \textbf{Norme di Progetto v1.0.0};
	
\end{itemize}

\subsubsection{Riferimenti informativi}
\begin{itemize}
\item \textbf{Capitolato d'appalto C4 - HD Viz: visualizzazione di dati multidimensionali}:\\
	\textcolor{blue}{\url{https://www.math.unipd.it/~tullio/IS-1/2020/Progetto/C4.pdf}};
	
	
	\item \textbf{Software Engineering - Ian Sommerville - 10 th Edition}: \\
	Part 4 - Software Management
	\begin{itemize}
	\item Chapter 24 - Quality Management;
	\end{itemize}
	
	\item \textbf{Slide T12 del corso Ingegneria del Software - Qualità di prodotto}:\\
	\textcolor{blue}{\url{https://www.math.unipd.it/~tullio/IS-1/2020/Dispense/L12.pdf}};
	\item \textbf{Slide T13 del corso Ingegneria del Software - Qualità di processo}:\\
	\textcolor{blue}{\url{https://www.math.unipd.it/~tullio/IS-1/2020/Dispense/L13.pdf}};
	\item \textbf{Slide T14 del corso Ingegneria del Software - Verifica e validazione}:\\
	\textcolor{blue}{\url{https://www.math.unipd.it/~tullio/IS-1/2020/Dispense/L14.pdf}};
	
	\item \textbf{Indice di Gulpease}:\\
	\textcolor{blue}{\url{https://it.wikipedia.org/wiki/Indice_Gulpease}}.
	
\end{itemize}