\section{Introduzione}
\subsection{Scopo del documento}
Questo documento ha lo scopo di fornire tutte le informazioni relative al sistema di controllo di qualità per i processi ed i prodotti, basandosi su assunti misurabili ma adattati alle esigenze del proprio progetto.
Esso deve implementare degli standard che permettano il miglioramento continuo, tracciando periodicamente tramite misurazioni i risultati ottenuti sfruttandoli per definire azioni migliorative. All'interno del \textit{Piano di Qualifica} vengono anche raccolte le definizioni dei test, il loro stato e il loro tracciamento. 

\subsection{Scopo del capitolato}
Oggigiorno, anche i programmi più tradizionali gestiscono e memorizzano una grande mole di dati; di conseguenza servono software in grado di eseguire un'analisi e un'interpretazione delle informazioni.\\
Il \glo{capitolato} C4 ha come obiettivo quello di creare un'applicazione di visualizzazione di dati con numerose dimensioni in modo da renderle comprensibili all'occhio umano.  Lo scopo del prodotto sarà quello di fornire all'utente diversi tipi di visualizzazioni e di algoritmi per la riduzione dimensionale in modo che, attraverso un processo esplorativo, l'utilizzatore del prodotto possa studiare tali dati ed evidenziarne degli eventuali \glo{cluster}. 

\subsection{Glossario}
Per evitare ambiguità relative alle terminologie utilizzate, è stato compilato il \Glossariov{2.0.0-0.2}. In questo documento sono riportati tutti i termini importanti e con un significato particolare. Questi termini sono evidenziati da una 'G' ad apice.

\subsection{Riferimenti}
\subsubsection{Riferimenti normativi}
\begin{itemize}	
\item \textbf{\NdPv{v 2.0.0-0.2}};
	
\item \textbf{Capitolato d'appalto C4 - HD Viz: visualizzazione di dati multidimensionali}:\\
	\textcolor{blue}{\url{https://www.math.unipd.it/~tullio/IS-1/2020/Progetto/C4.pdf}}

\end{itemize}

\subsubsection{Riferimenti informativi}
\begin{itemize}
	\item \textbf{Software Engineering - Ian Sommerville - 10 th Edition}: \\
	Parte 4 - Software Management
	\begin{itemize}
	\item Capitolo 24 - Quality Management:
		\begin{itemize}
			\item Paragrafo 24.1 - Software Quality (da pag. 703 a 705);
			\item Paragrafo 24.3 - Reviews and inspection (da pag. 710 a 714);
			\item Paragrafo 24.5 - Software measurement (da pag. 717 a 725).
		\end{itemize}
	\end{itemize}
	
	\item \textbf{Slide T12 del corso Ingegneria del Software - Qualità di prodotto}:\\
	\textcolor{blue}{\url{https://www.math.unipd.it/~tullio/IS-1/2020/Dispense/L12.pdf}}
	\begin{itemize}
		\item Slide 8 - I 7 principi del Sistema Qualità;
		\item Slide 12,13 - Cosa significa qualità SW;
		\item Slide 17 - Il processo di valutazione.
	\end{itemize}
	
	\item \textbf{Slide T13 del corso Ingegneria del Software - Qualità di processo}:\\
	\textcolor{blue}{\url{https://www.math.unipd.it/~tullio/IS-1/2020/Dispense/L13.pdf}}
		\begin{itemize}
		\item Slide 3 - Modello concettuale di processo;
		\item Slide 11 - I 5 livelli di maturità;
		\item Slide 23 - Riepilogo: la ricerca della qualità.
	\end{itemize}
	
	\item \textbf{Slide T14 del corso Ingegneria del Software - Verifica e validazione}:\\
	\textcolor{blue}{\url{https://www.math.unipd.it/~tullio/IS-1/2020/Dispense/L14.pdf}}
	\begin{itemize}
		\item Slide 6 - Verifica e validazione nello sviluppo; 
		\item Slide 15 - Analisi dinamica: tipi di test.
	\end{itemize}
	
	\item \textbf{Indice di Gulpease}:\\
	\textcolor{blue}{\url{https://it.wikipedia.org/wiki/Indice_Gulpease}}
	
	\item \textbf{Averege Cyclomatic complexity}:\\
	\textcolor{blue}{\url{https://eslint.org/docs/rules/complexity}}
	
	\item \textbf{ISO/IEC 9126}:\\
	\textcolor{blue}{\url{https://en.wikipedia.org/wiki/ISO/IEC_9126}}
	
	\item \textbf{ISO/IEC 12207}:\\
	\textcolor{blue}{\url{https://www.math.unipd.it/~tullio/IS-1/2009/Approfondimenti/ISO_12207-1995.pdf}}
	
\end{itemize}