\section{Introduzione}
\subsection{Scopo del documento}
Questo documento ha lo scopo di fornire tutte le informazioni relative al sistema di controllo di qualità per i processi ed i prodotti, basandosi su assunti misurabili ma adattati alle esigenze del proprio progetto.
Esso deve implementare degli standard che permettano il miglioramento continuo, tracciando periodicamente tramite misurazioni i risultati ottenuti sfruttandoli per definire azioni migliorative. All'interno del \textit{Piano di Qualifica} vengono anche raccolte le definizioni dei test, il loro stato e il loro tracciamento. 

\subsection{Scopo del capitolato}
Oggigiorno, anche i programmi più tradizionali gestiscono e memorizzano una grande mole di dati e di conseguenza serve un software in grado di eseguire un'analisi e un'interpretazione delle informazioni.\\
Il \glo{capitolato} ha come obiettivo quello di creare un'applicazione di visualizzazione di dati con numerose dimensioni in un formato comprensibile dall'occhio umano.  A questo scopo è necessario utilizzare algoritmi di intelligenza artificiale, o nel caso svilupparne di nuovi, che, agendo sulla distanza dei vari punti del grafico, riescano a sviluppare un modello semplificato che ne evidenzi i \glo{cluster}. 
L'applicazione dovrà inoltre agire su questi grafici creati evidenziando i dati ottenuti.

\subsection{Glossario}
Per evitare ambiguità relative alle terminologie utilizzate, è stato compilato il \Glossariov{1.0.0}. In questo documento sono riportati tutti i termini importanti e con un significato particolare. Questi termini sono evidenziati da una 'G' ad apice.

\subsection{Riferimenti}
\subsubsection{Riferimenti normativi}
\begin{itemize}	
	\item \textbf{Capitolato d'appalto C4 - HD Viz: visualizzazione di dati multidimensionali}:\\
	\textcolor{blue}{\url{https://www.math.unipd.it/~tullio/IS-1/2020/Progetto/C4.pdf}}
\end{itemize}

\subsubsection{Riferimenti informativi}
\begin{itemize}
	\item \textbf{Norme di Progetto v1.0.0};
	\item \textbf{Standard ISO/IEC 12207:1995}: \\
	\textcolor{blue}{\url{https://www.math.unipd.it/~tullio/IS-1/2009/Approfondimenti/ISO_12207-1995.pdf}}
	\item \textbf{Guide to the Software Engineering Body of Knowledge(SWEBOK), 2014} 
	
	\item \textbf{Software Engineering - Ian Sommerville - 10 th Edition (2010)}: \\
	(formato cartaceo);
	
	\item \textbf{Slide T3 del corso Ingegneria del Software - Ciclo di vita del software}:\\
	\textcolor{blue}{\url{https://www.math.unipd.it/~tullio/IS-1/2020/Dispense/L03.pdf}}
\end{itemize}