\subsection{Processi di supporto}



\subsubsection{Verifica}
Il processo di verifica si pone come obiettivo il controllo dello sviluppo software a livello di codifica. Successivamente sono riportate le metriche utilizzate, che possono essere consultate dettagliatamente nel documento \NdPv{v 1.0.0}.

\paragraph{Metriche}

\begin{itemize}
\item MPC8 - Code coverage (CC);
\item MPC9 - Passed test cases percentage (PTCP);
\item MPC10 - Failed test cases percentage (FTCP);
 
\end{itemize}

\paragraph{Valori ammissibili}
{
\rowcolors{2}{coloreGrigietto}{white}
\renewcommand{\arraystretch}{1.5}
\centering
\begin{longtable}{C{2.5cm} C{4cm} C{4cm}}
\rowcolor{coloreRosso}
\textcolor{white}{\textbf{Metrica}}&
\textcolor{white}{\textbf{Valori accettabile}}&
\textcolor{white}{\textbf{Valore ottimale}}\\	
\endhead

MPC8 & $ 75-85\% $  & $ 90-100 \% $ \\
MPC9 & $ \geq 90\% $  & $ 100 \% $ \\
MPC10 & $ \leq 10\% $  & $ 0 \% $ \\

\end{longtable}
}



\subsubsection{Gestione della qualità}
Il processo di gestione della qualità consiste nel garantire gli obiettivi di qualità del prodotto e dei servizi che offre. Successivamente sono riportate le metriche utilizzate, che possono essere consultate dettagliatamente nel documento \NdPv{v 1.0.0}.

\paragraph{Metriche}
\begin{itemize}
	\item MPC11 - Qaulity Metrics Satisfied (QMS)
\end{itemize}

\paragraph{Valori ammissibili}
{
\rowcolors{2}{coloreGrigietto}{white}
\renewcommand{\arraystretch}{1.5}
\centering
\begin{longtable}{C{2.5cm} C{4cm} C{4cm}}
\rowcolor{coloreRosso}
\textcolor{white}{\textbf{Metrica}}&
\textcolor{white}{\textbf{Valori accettabile}}&
\textcolor{white}{\textbf{Valore ottimale}}\\	
\endhead

MPC11 & $\geq 90\% $  & $ 100 \% $ \\

\end{longtable}
}

