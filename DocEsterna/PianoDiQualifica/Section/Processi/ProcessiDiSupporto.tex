\subsection{Processi di supporto}

\subsubsection{Verifica}
Il processo di verifica si pone come obiettivo il controllo dello sviluppo software a livello di codifica. Successivamente sono riportate le metriche utilizzate, che possono essere consultate dettagliatamente nel documento \NdPv{v 1.0.0}.

\subsubsection{Gestione della qualità}
Il processo di gestione della qualità consiste nel garantire gli obiettivi di qualità del prodotto e dei servizi che offre. Successivamente sono riportate le metriche utilizzate, che possono essere consultate dettagliatamente nel documento \NdPv{v 1.0.0}.

\subsubsection{Metriche di qualità dei processi di supporto}
{
\renewcommand{\arraystretch}{1.5}
\centering
\begin{longtable}{C{2.5cm} C{5.5cm} C{3.5cm} C{3.5cm}}
\rowcolor{coloreRosso}
\textcolor{white}{\textbf{Codice}}&
\textcolor{white}{\textbf{Nome metrica}}&
\textcolor{white}{\textbf{Valore accettabile}}&
\textcolor{white}{\textbf{Valore ottimale}}\\
\hline
\rowcolor{coloreRosso}
\multicolumn{4}{|c|}{\textcolor{white}{\textbf{Verifica}}} \\	
\endhead
\endfoot
\rowcolor{white}\caption{Metriche di qualità dei processi di supporto}
\endlastfoot

\textbf{MPC8} & Code coverage (CC) & $ 75-85\% $  & $ 90-100 \% $ \\
\textbf{MPC9} & Passed test cases percentage (PTCP) & $ \geq 90\% $  & $ 100 \% $ \\
\textbf{MPC10} & Failed test cases percentage (FTCP) & $ \leq 10\% $  & $ 0 \% $ \\
\rowcolor{coloreRosso}
\multicolumn{4}{|c|}{\textcolor{white}{\textbf{Gestione della qualità}}} \\	
\textbf{MPC11} & Quality Metrics Satisfied (QMS) & $\geq 90\% $  & $ 100 \% $ \\

\end{longtable}
}

