\subsection{Processi di supporto}

{
\renewcommand{\arraystretch}{1.5}
\centering
\begin{longtable}{C{2.5cm} | C{5.5cm} | C{3.5cm} | C{3.5cm}}
\rowcolor{coloreRosso}
\textcolor{white}{\textbf{Codice}}&
\textcolor{white}{\textbf{Nome metrica}}&
\textcolor{white}{\textbf{Valore accettabile}}&
\textcolor{white}{\textbf{Valore ottimale}}\\

\endhead
\endfoot
\rowcolor{white}\caption{Metriche di qualità dei processi di supporto}
\endlastfoot

\rowcolor{coloreRossoChiaro}
\textcolor{white}{\textbf{Verifica}} & \multicolumn{3}{C{13.34cm}}{\textcolor{white}{Processo che si pone come obiettivo il controllo dello sviluppo software a livello di codifica}} \\

\textbf{MPC8} & Code coverage (CC) & $ 75-85\% $  & $ 90-100 \% $ \\
\textbf{MPC9} & Passed test cases percentage (PTCP) & $ \geq 90\% $  & $ 100 \% $ \\
\textbf{MPC10} & Failed test cases percentage (FTCP) & $ \leq 10\% $  & $ 0 \% $ \\

\rowcolor{coloreRossoChiaro}
\textcolor{white}{\textbf{Gestione della qualità}} & \multicolumn{3}{C{13.34cm}}{\textcolor{white}{Processo che consiste nel garantire gli obiettivi di qualità del prodotto e dei servizi che offre}} \\

\textbf{MPC11} & Quality Metrics Satisfied (QMS) & $\geq 90\% $  & $ 100 \% $ \\

\end{longtable}
}

