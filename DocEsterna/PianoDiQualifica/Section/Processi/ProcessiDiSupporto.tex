\subsection{Processi di supporto}
\subsubsection{Documentazione}
Il processo di documentazione mira a registrare le informazioni prodotte da una attività del ciclo di vita sotto forma di documento. Successivamente sono riportate le metriche utilizzate, che possono essere consultate dettagliatamente nel documento \NdPv{v 1.0.0}.

\paragraph{Metriche}

\paragraph{Valori ammissibili}
{
\rowcolors{2}{coloreGrigietto}{white}
\renewcommand{\arraystretch}{1.5}
\centering
\begin{longtable}{c C{2.6cm} C{3cm} C{2.6cm} C{5cm}}
\rowcolor{coloreRosso}
\textcolor{white}{\textbf{Metrica}}&
\textcolor{white}{\textbf{Valori accettabile}}&
\textcolor{white}{\textbf{Valore ottimale}}\\	
\endhead

 
\end{longtable}
}

\subsubsection{Gestione della qualità}
Il processo di gestione della qualità consiste nel garantire gli obiettivi di qualità del prodotto e dei servizi che offre. Successivamente sono riportate le metriche utilizzate, che possono essere consultate dettagliatamente nel documento \NdPv{v 1.0.0}.

\paragraph{Metriche}
\begin{itemize}
	\item MPC5 - Qaulitt Metrics Satisfied (QMS)
\end{itemize}

\paragraph{Valori ammissibili}
{
\rowcolors{2}{coloreGrigietto}{white}
\renewcommand{\arraystretch}{1.5}
\centering
\begin{longtable}{c C{2.6cm} C{3cm} C{2.6cm} C{5cm}}
\rowcolor{coloreRosso}
\textcolor{white}{\textbf{Metrica}}&
\textcolor{white}{\textbf{Valori accettabile}}&
\textcolor{white}{\textbf{Valore ottimale}}\\	
\endhead

MPC5 & $\geq 90\% $  & $ 100 \% $ 
\end{longtable}
}