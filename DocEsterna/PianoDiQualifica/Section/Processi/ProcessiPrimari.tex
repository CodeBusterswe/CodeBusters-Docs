
Per garantire la qualità dei processi è stato utilizzato come riferimento lo standard ISO/IEC/IEEE 12207:1995. Tra i processi elencati dal modello, il gruppo ne ha scelti alcuni che sono stati semplificati e adattati alle necessità del progetto. Questa sezione espone i valori di qualità accettabili sulla base di metriche elencate nelle \NdPv{v 1.0.0}.\\ Successivamente sono riportati i processi selezionati con le relative metriche utilizzate, che possono essere consultate dettagliatamente nel documento \NdPv{v 1.0.0}.

\subsection{Processi Primari}

{
\renewcommand{\arraystretch}{1.5}
\centering
\begin{longtable}{C{2.5cm} | C{4cm} | C{5cm} | C{3.5cm}}
\rowcolor{coloreRosso}
\textcolor{white}{\textbf{Codice}}&
\textcolor{white}{\textbf{Nome metrica}}&
\textcolor{white}{\textbf{Valore accettabile}}&
\textcolor{white}{\textbf{Valore ottimale}}\\	

\endhead
\endfoot

\rowcolor{white}\caption{Metriche di qualità dei processi primari}
\endlastfoot

\rowcolor{coloreRossoChiaro}
\textcolor{white}{\textbf{Fornitura}} & \multicolumn{3}{C{13.34cm}}{\textcolor{white}{Processo che consiste nel scegliere le procedure e le risorse atte a perseguire lo sviluppo del progetto}} \\

\textbf{MPC1} & Schedule Variance (SV) & $\geq 5\%$ & $\geq 0\%$ \\
\textbf{MPC2} & Budget Variance (BV) & $\geq 5\%$ & $\geq 0\%$ \\
\textbf{MPC3} & Budget at Completion (BAC) & preventivo - 5\% $ \leq $ BAC \newline BAC $ \leq $ preventivo + 5\% & Pari al preventivo  \\
\textbf{MPC4} & Earned Value (EV) & $\geq 0$  & $\geq 0$ \\
\textbf{MPC5} & Planned Value (PV) & $\geq 0$  & $\geq 0$ \\

\rowcolor{coloreRossoChiaro}
\textcolor{white}{\textbf{Sviluppo}} & \multicolumn{3}{C{13.34cm}}{\textcolor{white}{Processo che contiene le attività e i compiti per realizzare il prodotto software richiesto}} \\

\textbf{MPC6} & Requirements stability index (RSI) & 70\% & 100\% \\
\textbf{MPC7} & Satisfied obligatory requirements (SOR) & 100\% & 100\%
\end{longtable}
}


