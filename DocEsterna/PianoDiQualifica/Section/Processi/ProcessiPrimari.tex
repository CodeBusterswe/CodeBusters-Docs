
Per garantire la qualità dei processi è stato utilizzato come riferimento lo standard ISO/IEC/IEEE 12207:1995. Tra i processi elencati dal modello, il gruppo ne ha scelti alcuni che sono stati semplificati e adattati alle necessità del progetto. Questa sezione espone i valori di qualità accettabili sulla base di metriche elencate nelle \NdPv{}.\\ Successivamente sono riportati i processi selezionati con le relative metriche utilizzate, che possono essere consultate dettagliatamente nel documento \NdPv{v 2.0.0-0.1}.

\subsection{Obbiettivi di qualità di processo}
{
\setlength\arrayrulewidth{0.95pt}
\renewcommand{\arraystretch}{1.5}
\begin{longtable}{C{3cm} | C{9.5cm} | C{2.5cm}}
\rowcolor{coloreRosso}

\textcolor{white}{\textbf{Processo}}&
\textcolor{white}{\textbf{Descrizione}}&
\textcolor{white}{\textbf{Metriche}} \\
\endfirsthead
\rowcolor{white}\multicolumn{3}{C{15cm}}{\textit{Continua nella pagina successiva...}}\\
\endfoot
\rowcolor{white}\caption{Obbiettivi di qualità di processo}
\endlastfoot
	
\rowcolor{coloreRossoChiaro}
 & \textcolor{white}{\textbf{Processi primari}} &  \\

	\textbf{Fornitura} &
	 Processo che consiste nel scegliere le procedure e le risorse atte a perseguire lo sviluppo del progetto. &
	MPC1, MPC2 \newline MPC3, MPC4 \newline MPC5, MPC6 \newline MPC7  \\
	
		\textbf{Sviluppo} &
	 Processo che contiene le attività e i compiti per realizzare il prodotto software richiesto. &
	MPC8, MPC9 \\
 
 \rowcolor{coloreRossoChiaro}
 & \textcolor{white}{\textbf{Processi di supporto}} &  \\
 
	\textbf{Verifica} & 
	Processo che si pone come obbiettivo il controllo dello sviluppo software a livello di codifica. &
	MPC10, MPC11 \newline MPC12 \\
	
	\textbf{Gestione qualità} &
	 Processo che consiste nel garantire gli obbiettivi di qualità del prodotto e dei servizi che offre. &
	MPC13 \\
	
	\rowcolor{coloreRossoChiaro}
 & \textcolor{white}{\textbf{Processi Organizzativi}} &  \\
 
	\textbf{Gestione organizzativa} & 
	Processo che si occupa di esporre le modalità di coordinamento del gruppo. &
	MPC14 \\
	
\end{longtable}	

}

\newpage

\subsection{Metriche utilizzate}


\subsubsection{Processi Primari}

{
\renewcommand{\arraystretch}{1.5}
\centering
\begin{longtable}{C{2.5cm} | C{4cm} | C{5cm} | C{3.5cm}}
\rowcolor{coloreRosso}
\textcolor{white}{\textbf{Codice}}&
\textcolor{white}{\textbf{Nome metrica}}&
\textcolor{white}{\textbf{Valore accettabile}}&
\textcolor{white}{\textbf{Valore ottimale}}\\	

\endhead
\endfoot

\rowcolor{white}\caption{Metriche di qualità dei processi primari}
\endlastfoot

\rowcolor{coloreRossoChiaro}
\multicolumn{4}{c}{\textcolor{white}{Fornitura}} \\

\textbf{MPC1} & Schedule Variance (SV) & $\geq -10\%$ & $\leq 0\%$ \\
\textbf{MPC2} & Budget Variance (BV) & $\geq -10\%$ & $\leq 0\%$ \\
\textbf{MPC3} & Estimated at Completion (EAC) & preventivo - 5\% $ \leq $ EAC \newline EAC $ \leq $ preventivo + 5\% & Pari al preventivo  \\
\textbf{MPC4} & Earned Value (EV) & $\geq 0$  & $\leq $ EAC \\
\textbf{MPC5} & Planned Value (PV) & $\geq 0$  & $\leq $ Budget at Completion\\
\textbf{MPC6} & Actual Cost (AC) & $\geq 0$  & $\leq $ EAC\\
\textbf{MPC7} & Estimate to Complete (ETC) & $\geq 0$  & $\leq $ EAC\\

\rowcolor{coloreRossoChiaro}
\multicolumn{4}{c}{\textcolor{white}{Sviluppo}} \\

\textbf{MPC8} & Requirements stability index (RSI) & 70\% & 100\% \\
\textbf{MPC9} & Satisfied obligatory requirements (SOR) & 100\% & 100\%
\end{longtable}
}


