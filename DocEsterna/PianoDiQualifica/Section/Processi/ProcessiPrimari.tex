
\subsection{Introduzione}
Per garantire la qualità dei processi è stato utilizzato come riferimento lo standard ISO/IEC/IEEE 12207:1995. Tra i processi elencati dal modello, il gruppo ne ha scelti alcuni che sono stati semplificati e adattati alle necessità del progetto. Questa sezione espone i valori di qualità accettabili sulla base di metriche elencate nelle \NdPv{v 1.0.0}. Di seguito sono esposti i processi selezionati.

\subsection{Processi Primari}
\subsubsection{Sviluppo}
Il processo di sviluppo contiene i compiti e le attività da svolgere col fine di presentare al proponente il prodotto richiesto
Successivamente sono riportate le metriche utilizzate, che possono essere consultate dettagliatamente nel documento \NdPv{v 1.0.0}.
\paragraph{Metriche}
\begin{itemize}
	\item MPC3 - Requirements stability index (RSI);
	\item MPC4 - Satisfied obligatory requirements (SOR);
\end{itemize}


\paragraph{Valori ammissibili}
{
\rowcolors{2}{coloreGrigietto}{white}
\renewcommand{\arraystretch}{1.5}
\centering
\begin{longtable}{c C{2.6cm} C{3cm} C{2.6cm} C{5cm}}
\rowcolor{coloreRosso}
\textcolor{white}{\textbf{Metrica}}&
\textcolor{white}{\textbf{Valori accettabile}}&
\textcolor{white}{\textbf{Valore ottimale}}\\	
\endhead

MPC3 &   &  \\
MPC4 & 100\% & 100\%
\end{longtable}
}



\subsubsection{Sviluppo}

\paragraph{Metriche}

\paragraph{Valori ammissibili}
{
\rowcolors{2}{coloreGrigietto}{white}
\renewcommand{\arraystretch}{1.5}
\centering
\begin{longtable}{c C{2.6cm} C{3cm} C{2.6cm} C{5cm}}
\rowcolor{coloreRosso}
\textcolor{white}{\textbf{Metrica}}&
\textcolor{white}{\textbf{Valori accettabile}}&
\textcolor{white}{\textbf{Valore ottimale}}\\	
\endhead

MPC3 &   &  \\
MPC4 & 100\% & 100\%
\end{longtable}
}