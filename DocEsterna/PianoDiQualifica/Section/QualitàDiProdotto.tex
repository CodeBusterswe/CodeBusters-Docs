\section{Qualità di prodotto}
Dallo standard ISO/IEC 9126, il gruppo \Gruppo ha identificato le qualità che ritiene necessarie nell'intero ciclo di vita del prodotto e ne ha tratto delle metriche e degli obiettivi da realizzare per perseguire la qualità del software. 
\subsection{Prodotti}
\subsubsection{Documenti}
I documenti dovranno essere comprensibili e corretti ortograficamente e sintatticamente.
\paragraph{Comprensione}
I documenti dovranno poter essere letti e compresi da tutti coloro che hanno una istruzione di base ovvero che abbiano completato la scuola secondaria di secondo grado.
\paragraph{Correttezza}
La correttezza dei documenti sarà compito del verificatore con l'ausilio di uno strumento di controllo ortografico.
\subsubsection{Software}
\paragraph{Funzionalità}
Capacità del prodotto di offrire tutte le funzioni individuate nell'\Adr. Gli obiettivi da perseguire sono:
\begin{itemize}
\item \textbf{Accuratezza}: il prodotto dovrà ottenere i risultati richiesti;
\item \textbf{Adeguatezza}: le funzionalità dovranno almeno equivalere le attese.
\end{itemize}
Per verificare che ciò avvenga dovranno essere coperti almeno tutti i requisiti obbligatori e desiderabili.
\paragraph{Affidabilità}
Con questo termine si intende la capacità del prodotto di riuscire a svolgere tutte le funzionalità presenti anche in caso di errori o problemi. L'esecuzione, per risultare affidabile, dovrà possedere queste caratteristiche:
\begin{itemize}
\item \textbf{Tolleranza agli errori}: la gestione degli errori dovrà essere tale da permettere di avere sempre un alto livello di prestazioni;
\item \textbf{Previdenza}: evitare che malfunzionamenti o operazioni illegali si manifestino.
\end{itemize}
\paragraph{Usabilità}
Il prodotto dovrà essere comprensibile e graficamente armonioso in modo da rendere piacevole l'esperienza dell'utente. Gli obiettivi di usabilità sono:
\
\subsection{Manutenibilità}