\section{Qualità di prodotto}
\subsection{Introduzione}
Dallo standard ISO/IEC 9126, il gruppo \Gruppo{} ha identificato le qualità che ritiene necessarie nell'intero ciclo di vita del prodotto e ne ha tratto delle metriche e degli obiettivi per perseguire la qualità del software. Ci sono due tipi di prodotto: i documenti e il software.

\subsection{Documenti}
I documenti devono essere comprensibili e corretti ortograficamente e sintatticamente. Infatti, devono poter essere letti da tutti coloro che hanno un'istruzione di base, ovvero che abbiano completato la scuola secondaria di secondo grado. La correttezza dei documenti redatti è compito dei \glo{verificatori}, che possono avvalersi dell'ausilio di strumenti di controllo ortografico. 

\subsubsection{Metriche di qualità dei prodotti documentali}
\renewcommand{\arraystretch}{1.5}
\begin{longtable}{C{2.5cm} C{5.5cm} C{3.5cm} C{3.5cm}}
\rowcolor{coloreRosso}
\textcolor{white}{\textbf{Codice}}&
\textcolor{white}{\textbf{Nome metrica}}&
\textcolor{white}{\textbf{Valore accettabile}}&
\textcolor{white}{\textbf{Valore ottimale}}\\	
\endhead
\endfoot
\rowcolor{white}\caption{Metriche di qualità dei prodotti documentali}
\endlastfoot
		\textbf{MPD1} & Indice di Gulpease & $\geq$ 60 & $\geq$ 80\\
		\textbf{MPD2} & Errori ortografici & 100\% corretto & 100\% corretto \\
\end{longtable}

\subsection{Software}
Il prodotto software deve perseguire alcune caratteristiche affinché possa essere considerato valido. È compito dei \glo{verificatori} accertare, attraverso le metriche adottate per il software, che tali caratteristiche siano raggiunte con valori accettabili.

\subsubsection{Caratteristiche del software}
\renewcommand{\arraystretch}{1.5}
\begin{longtable}{C{3cm} C{9.5cm} C{2.5cm}}
\rowcolor{coloreRosso}
\textcolor{white}{\textbf{Caratteristica}}&
\textcolor{white}{\textbf{Descrizione}}&
\textcolor{white}{\textbf{Metriche}} \\
\endfirsthead
\rowcolor{white}\multicolumn{3}{C{15cm}}{\textit{Continua nella pagina successiva...}}\\
\endfoot
\rowcolor{white}\caption{Caratteristiche da perseguire nei prodotti software}
\endlastfoot
	
	\textbf{Funzionalità} & 
	Capacità del prodotto di offrire tutte le funzioni individuate nell'\AdRv{}. Gli obiettivi da perseguire sono:
	\begin{itemize}
		\item \textbf{Accuratezza}: il prodotto dovrà ottenere i risultati richiesti;
		\item \textbf{Adeguatezza}: le funzionalità dovranno almeno equivalere alle attese.
	\end{itemize} &
	MPD3 \\
	\textbf{Affidabilità} &
	 Capacità del prodotto di riuscire a svolgere tutte le funzionalità anche in caso di errori o problemi. Gli obiettivi da perseguire sono:
	 \begin{itemize}
		\item \textbf{Tolleranza agli errori}: la gestione degli errori dovrà essere tale da permettere di avere sempre un alto livello di prestazioni;
		\item \textbf{Previdenza}: evitare che malfunzionamenti o operazioni illegali si manifestino.
	\end{itemize} &
	MPD4 \\
	\textbf{Usabilità} &
	Capacità di essere comprensibile e graficamente armonioso in modo da rendere piacevole l'esperienza dell'utente. Gli obiettivi da perseguire sono: 
		\begin{itemize}
		\item \textbf{Comprensibilità}: chi utilizza il prodotto deve comprendere facilmente quali sono le sue funzionalità in modo da ottenere i risultati voluti;
		\item \textbf{Facilità d'uso}: l'utente deve imparare senza troppe difficoltà come utilizzare l'applicazione;
		\item \textbf{Operabilità}: le funzioni devono essere compatibili con le aspettative dell'utente;
		\item \textbf{Attrattiva}: il software deve essere piacevole all'occhio.
	\end{itemize} &
    MPD5 \newline MPD6 \newline MPD7 \\
	\textbf{Efficienza} & 
	Capacità di raggiungere un fine con il minor utilizzo di tempo e risorse. Gli obiettivi da perseguire sono: 
		\begin{itemize}
		\item \textbf{Veloce}: le risposte all'input dell'utente devono essere quanto più possibili veloci e corrette;
		\item \textbf{Leggero}: il software dovrà utilizzare meno risorse possibili dell'utente.
	\end{itemize} &
	MPD8 \\
	\textbf{Manutenibilità} & 
	Capacità di permettere future correzioni e modifiche senza che ciò rischi di compromettere l'intero progetto. Gli obiettivi da perseguire sono: 
		\begin{itemize}
		\item \textbf{Analizzabilità}: l'individuazione degli errori deve essere facile;
		\item \textbf{Modificabilità}: la modifica o l'aggiunta di nuove parti deve essere permessa. Il codice deve essere leggibile così da poterla inserire facilmente;
		\item \textbf{Stabilità}: dopo la modifica non devono insorgere altri problemi relativi all'incompatibilità con altre parti di codice;
		\item \textbf{Testabilità}: i test sulle modifiche effettuate devono essere facilmente implementati.
	\end{itemize} &
 	MPD9 \\
	\textbf{Portabilità} & 
	Capacità di poter funzionare in diversi ambienti di esecuzione. Gli obiettivi da perseguire sono: 
		\begin{itemize}
		\item \textbf{Adattabilità}: il software dovrà poter essere eseguito in numerosi browser senza che debbano essere effettuate delle modifiche;
		\item \textbf{Sostituibilità}: il software deve poter sostituire un prodotto con lo stesso fine e che viene eseguito sullo stesso browser. 
	\end{itemize} &
	MPD10 \\
	
\end{longtable}	

\newpage
\subsubsection{Metriche di qualità dei prodotti software}
\renewcommand{\arraystretch}{1.5}
\begin{longtable}{C{2.5cm} C{5.5cm} C{3.5cm} C{3.5cm}}
\rowcolor{coloreRosso}
\textcolor{white}{\textbf{Codice}}&
\textcolor{white}{\textbf{Nome metrica}}&
\textcolor{white}{\textbf{Valore accettabile}}&
\textcolor{white}{\textbf{Valore ottimale}}\\
\hline
\rowcolor{coloreRosso}
\multicolumn{4}{|c|}{\textcolor{white}{\textbf{Funzionalità}}} \\	
\endhead
\endfoot
\rowcolor{white}\caption{Metriche di qualità dei prodotti software}
\endlastfoot
	\textbf{MPD3} & Copertura dei requisiti & 100\% dei requisiti obbligatori & 100\% di tutti i requisiti \\
\rowcolor{coloreRosso}
\multicolumn{4}{|c|}{\textcolor{white}{\textbf{Affidabilità}}} \\			
	\textbf{MPD4} & Densità di failure & 30\% & 20\% \\
\rowcolor{coloreRosso}
\multicolumn{4}{|c|}{\textcolor{white}{\textbf{Usabilità}}} \\
	\textbf{MPD5} & Average Cyclomatic Complexity & 15 & 10 \\		
	\textbf{MPD6} & Facilità di utilizzo & 6 & 5 \\
	\textbf{MPD7} & Facilità apprendimento funzionalità & 15 minuti & 10 minuti\\
\rowcolor{coloreRosso}
\multicolumn{4}{|c|}{\textcolor{white}{\textbf{Efficienza}}} \\
 	\textbf{MPD8} & Tempo medio di risposta & 5 secondi & 4 secondi \\
\rowcolor{coloreRosso}
\multicolumn{4}{|c|}{\textcolor{white}{\textbf{Manutenibilità}}} \\ 	
 	\textbf{MPD9} & Comprensione del codice & 60-100\% & 80-100\% \\
\rowcolor{coloreRosso}
\multicolumn{4}{|c|}{\textcolor{white}{\textbf{Portabilità}}} \\ 
	\textbf{MPD10} & Versioni del browser supportate & 80\% & 100\% \\
\end{longtable}