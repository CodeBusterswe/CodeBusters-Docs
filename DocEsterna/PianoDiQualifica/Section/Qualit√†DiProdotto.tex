\section{Qualità di prodotto}
Dallo standard ISO/IEC 9126, il gruppo \Gruppo ha identificato le qualità che ritiene necessarie nell'intero ciclo di vita del prodotto e ne ha tratto delle metriche e degli obiettivi da realizzare per perseguire la qualità del software. 
\subsection{Prodotti}
\subsubsection{Documenti}
I documenti dovranno essere comprensibili e corretti ortograficamente e sintatticamente.
\paragraph{Comprensione}
I documenti dovranno poter essere letti e compresi da tutti coloro che hanno una istruzione di base ovvero che abbiano completato la scuola secondaria di secondo grado.
\paragraph{Correttezza}
La correttezza dei documenti sarà compito del verificatore con l'ausilio di uno strumento di controllo ortografico.
\paragraph{Metrica utilizzata}

\textbf{MPD1 Indice Gulpease}: questo indice descrive la leggibilità del documento prodotto e affinché venga accettato il valore deve risiedere tra 50-100.
 
\subsubsection{Software}
\paragraph{Funzionalità}
Capacità del prodotto di offrire tutte le funzioni individuate nell'\Adr{}. Gli obiettivi da perseguire sono:
\begin{itemize}
\item \textbf{Accuratezza}: il prodotto dovrà ottenere i risultati richiesti;
\item \textbf{Adeguatezza}: le funzionalità dovranno almeno equivalere le attese.
\end{itemize}
\paragraph{Affidabilità}
Con questo termine si intende la capacità del prodotto di riuscire a svolgere tutte le funzionalità presenti anche in caso di errori o problemi. L'esecuzione, per risultare affidabile, dovrà possedere queste caratteristiche:
\begin{itemize}
\item \textbf{Tolleranza agli errori}: la gestione degli errori dovrà essere tale da permettere di avere sempre un alto livello di prestazioni;
\item \textbf{Previdenza}: evitare che malfunzionamenti o operazioni illegali si manifestino.
\end{itemize}
\paragraph{Usabilità}
Il prodotto dovrà essere comprensibile e graficamente armonioso in modo da rendere piacevole l'esperienza dell'utente. Gli obiettivi di usabilità sono:
\begin{itemize}
\item \textbf{Comprensibilità}: chi utilizza il prodotto deve comprendere facilmente quali sono le sue funzionalità in modo da ottenere i risultati voluti;
\item \textbf{Facilità d'uso}: l'utente deve imparare senza troppe difficoltà come utilizzare l'applicazione;
\item \textbf{Operabilità}: le funzioni devono essere compatibili con le aspettative dell'utente;
\item \textbf{Attrattiva}: il software deve essere piacevole all'occhio.
\end{itemize}
\paragraph{Efficienza}
L'efficienza è la capacità di raggiungere un fine con il minor utilizzo di tempo e risorse. Per quanto riguarda il prodotto questo dovrà essere:
\begin{itemize}
\item \textbf{Veloce}: le risposte all'input dell'utente devono essere quanto più possibili veloci e corrette;
\item \textbf{Leggero}: il software dovrà utilizzare meno risorse possibili dell'utente.
\end{itemize}
\paragraph{Manutenibilità}
Un prodotto, per avere tale capacità, dovrà permettere future correzioni e modifiche senza che ciò rischi di compromettere l'intero progetto. Le caratteristiche che il software deve avere sono:
\begin{itemize}
\item \textbf{Analizzabilità}: l'individuazione degli errori deve essere facile;
\item \textbf{Modificabilità}: la modifica o l'aggiunta di nuove parti deve essere permessa. Il codice deve essere leggibile così da poterla inserire facilmente;
\item \textbf{Stabilità}: dopo la modifica non devono insorgere altri problemi relativi alla incompatibilità con altre parti di codice;
\item \textbf{Testabilità}: i test sulle modifiche effettuate devono essere facilmente implementati.
\end{itemize}
\paragraph{Portabilità}
\'E la capacità di poter funzionare in diversi ambienti. Per avere tale capacità il prodotto dovrà avere le seguenti caratteristiche:
\begin{itemize}
\item \textbf{Adattabilità}: il software dovrà poter essere eseguito in numerosi browser senza che debbano essere effettuate delle modifiche;
\item \textbf{Sostituibilità}: il software deve poter sostituire un prodotto con lo stesso fine e che viene eseguito sullo stesso browser. 
\end{itemize}