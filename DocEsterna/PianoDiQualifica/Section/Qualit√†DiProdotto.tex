\section{Qualità di prodotto}

Dallo standard ISO/IEC 9126, il gruppo \Gruppo{} ha identificato le qualità che ritiene necessarie nell'intero ciclo di vita del prodotto e ne ha tratto delle metriche e degli obiettivi per perseguirne qualità.\\
Di seguito sono riportati gli obbiettivi di qualità con le relative metriche utilizzate, che possono essere consultate dettagliatamente nel documento \NdPv{v1.0.0}.


\subsection{Obbiettivi di qualità di prodotto}
\renewcommand{\arraystretch}{1.5}
\begin{longtable}{C{3cm} | C{9.5cm} | C{2.5cm}}
\rowcolor{coloreRosso}

\textcolor{white}{\textbf{Obbiettivo}}&
\textcolor{white}{\textbf{Descrizione}}&
\textcolor{white}{\textbf{Metriche}} \\
\endfirsthead
\rowcolor{white}\multicolumn{3}{C{15cm}}{\textit{Continua nella pagina successiva...}}\\
\endfoot
\rowcolor{white}\caption{Obbiettivi di qualità di prodotto}
\endlastfoot
	
\rowcolor{coloreRossoChiaro}
 & \textcolor{white}{\textbf{Monitoraggio documentazione}} &  \\

	\textbf{Comprensione} &
	 Tutti i documenti devono essere leggibili e comprensibili, qualità che derivano dalla correttezza lessicale e grammaticale. &
	MPD1 \newline MPD2 \\
 
 \rowcolor{coloreRossoChiaro}
 & \textcolor{white}{\textbf{Monitoraggio software}} &  \\
 
	\textbf{Funzionalità} & 
	Capacità del prodotto di offrire tutte le funzioni individuate nell'\AdRv{}, perseguendo accuratezza e adeguatezza. &
	MPD3 \\
	
	\textbf{Affidabilità} &
	 Capacità del prodotto di riuscire a svolgere tutte le funzionalità anche in caso di errori o problemi, cercando di evitare che si manifestino. &
	MPD4 \\
	
	\textbf{Usabilità} &
	Capacità di essere comprensibile e graficamente armonioso in modo da rendere piacevole l'esperienza dell'utente. Il prodotto deve essere comprensibile e le funzionalità devono essere compatibili con le aspettative. &
    MPD5 \newline MPD6 \newline MPD7 \\
    
	\textbf{Efficienza} & 
	Capacità di raggiungere un fine con il minor utilizzo di tempo e risorse, puntando alla velocità e alla leggerezza. &
	MPD8 \\
	
	\textbf{Manutenibilità} & 
	Capacità di permettere future correzioni e modifiche senza che ciò rischi di compromettere l'intero progetto; l'individuazione degli errori deve essere facile e deve essere possibile modificare o aggiungere nuove parti. &
 	MPD9 \\
 	
	\textbf{Portabilità} & 
	Capacità di poter funzionare in diversi ambienti di esecuzione. Gli obiettivi da perseguire sono adattabilità e sostituibilità. &
	MPD10 \\
	
\end{longtable}	

\newpage

\subsection{Metriche di qualità di prodotto}

\renewcommand{\arraystretch}{1.5}

\begin{longtable}{C{2.5cm} | C{5cm} | C{4cm} | C{4cm}}
\rowcolor{coloreRosso}
\textcolor{white}{\textbf{Codice}}&
\textcolor{white}{\textbf{Nome metrica}}&
\textcolor{white}{\textbf{Valore accettabile}}&
\textcolor{white}{\textbf{Valore ottimale}}\\	
\endhead
\endfoot
\rowcolor{white}\caption{Metriche di qualità di prodotto}
\endlastfoot

\rowcolor{coloreRossoChiaro}
\multicolumn{4}{c}{\textcolor{white}{\textbf{Documenti}}} \\

	\textbf{MPD1} & Indice di Gulpease & $\geq$ 60 & $\geq$ 80\\
	\textbf{MPD2} & Errori ortografici & 100\% corretto & 100\% corretto \\
	
\rowcolor{coloreRossoChiaro}
\multicolumn{4}{c}{\textcolor{white}{\textbf{Software}}} \\

	\textbf{MPD3} & Copertura dei requisiti & 100\% requisiti obbl. & 100\% di tutti i requisiti \\
		
	\textbf{MPD4} & Densità di failure & 30\% & 20\% \\

	\textbf{MPD5} & Average Cyclomatic Complexity & 15 & 10 \\		
	\textbf{MPD6} & Facilità di utilizzo & 6 & 5 \\
	\textbf{MPD7} & Facilità apprendimento funzionalità & 15 minuti & 10 minuti\\

 	\textbf{MPD8} & Tempo medio di risposta & 5 secondi & 4 secondi \\
	
 	\textbf{MPD9} & Comprensione del codice & 60-100\% & 80-100\% \\

	\textbf{MPD10} & Versioni del browser supportate & 80\% & 100\% \\
\end{longtable}


	

