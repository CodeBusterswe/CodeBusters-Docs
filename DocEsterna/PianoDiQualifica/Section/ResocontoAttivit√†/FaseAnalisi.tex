Durante questo periodo tutti i documenti prodotti dal gruppo \Gruppo{} sono stati sottoposti a verifica con l'obiettivo di mantenere un approccio costruttivo. Il rilevamento dei difetti è stato possibile grazie alla lettura critica ad ampio spettro del prodotto in esame e a controlli più mirati individuati dai \glo{verificatori}. Al termine dell'attività si sono applicate su ciascuno le metriche stabilite per il monitoraggio della qualità.

\subsubsection{Indice di Gulpease}
Per ogni documento prodotto, in base alla metrica \textbf{MPD1}, si è calcolato l'\glo{indice di Gulpease}. Per ottenere un valore idoneo non sono state prese in considerazioni parti dei documenti che potrebbero portare ad alterazioni del risultato.
\renewcommand{\arraystretch}{1.5}
\begin{longtable}{C{4cm} C{2cm} C{3cm}}
\rowcolor{coloreRosso}
\textcolor{white}{\textbf{Documento}}&
\textcolor{white}{\textbf{Valore}}&
\textcolor{white}{\textbf{Esito}}\\	
\endhead
\endfoot
\rowcolor{white}
\caption{Risultati indice di Gulpease periodo di Analisi}
\endlastfoot

Studio di Fattibilità & 73 & Superato \\

Analisi dei Requisiti & 70 & Superato \\

Piano di Qualifica & 67 & Superato \\

Piano di Progetto & 65 & Superato \\

Glossario & 71 & Superato \\

VI 27-10-2020 & 70 & Superato \\

VI 10-11-2020 & 66 & Superato \\

VI 26-11-2020 & 68 & Superato \\

VI 14-12-2020 & 70 & Superato \\

VI 20-12-2020 & 66 & Superato \\

VI 07-01-2021 & 68 & Superato \\

VE 17-12-2020 & 71 & Superato \\
 
\end{longtable}