\subsection{Test di integrazione}
%I test di integrazione verificano come le componenti software si integrino tra di loro.
%In questa prima versione del \PdQ{} il gruppo non è in grado di stabilire dei test di integrazione, non avendo individuato e testato le componenti del prodotto software.
Sigle utili per comprendere la tabella seguente:
\begin{itemize}
	\item \textbf{I}: test implementato;
	\item \textbf{NI}: test non implementato.
\end{itemize} 

\renewcommand{\arraystretch}{1.5}
\renewcommand\extrarowheight{1.5pt}
\begin{longtable}{ c|C{12cm}|c } 
		\rowcolor{coloreRosso}
		\textcolor{white}{\textbf{Codice}} & 
		\textcolor{white}{\textbf{Descrizione}} & 
		\textcolor{white}{\textbf{Stato}} \\
		\endfirsthead
		\rowcolor{white}\multicolumn{3}{c}{\textit{Continua nella pagina successiva...}}\\
	    \endfoot
	    \endlastfoot

\textbf{TI1} & 
Integrazione tra \NomeProgetto{} e database gestita correttamente & 
NI\\

\textbf{TI2} & 
\NomeProgetto{} accede ai dati del database & 
NI\\

\textbf{TI3} & 
\NomeProgetto{} accede ai dati del database utilizzando una query & 
NI\\

\textbf{TI4} & 
\NomeProgetto{} utilizza i dati caricati dal database & 
NI\\

\rowcolor{white}\caption{Test di integrazione}
\label{testIntegrazione}
\end{longtable}