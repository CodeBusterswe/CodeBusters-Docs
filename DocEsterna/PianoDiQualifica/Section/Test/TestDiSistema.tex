\subsection{Test di sistema}
%In base ai requisiti trovati e riportati nell'\AdRv{v 1.0.0}, il gruppo ha stilato una lista di test di sistema da effettuare per verificarne la correttezza.

{
\renewcommand{\arraystretch}{1.5}
\renewcommand\extrarowheight{1.5pt}
\setlength\arrayrulewidth{1pt}
\begin{longtable}{ C{2cm} | C{12cm}| C{1.5cm} } 
		\rowcolor{coloreRosso}
		\textcolor{white}{\textbf{Codice}} & 
		\textcolor{white}{\textbf{Descrizione}} & 
		\textcolor{white}{\textbf{Stato}} \\
		\endfirsthead
		\rowcolor{white}\multicolumn{3}{c}{\textit{Continua nella pagina successiva...}}\\
	    \endfoot
	    \endlastfoot

\textbf{TS1F1} & 
L'utente deve poter caricare dei dati nel sistema tramite file \glo{CSV}. & 
NI\\

\textbf{TS1F2} & 
L'utente deve poter caricare dei dati nel sistema tramite interrogazione al \glo{database}. & 
NI\\

\textbf{TS1F3} & 
L'utente deve visualizzare a schermo un messaggio d'esito dell'operazione. & 
NI\\
		
\textbf{TS2F4} & 
Verificare che gli aiuti all'utente siano utilizzabili e localizzabili. & 
NI\\		

\textbf{TS1F5} & 
L'utente deve poter selezionare le dimensioni che desidera utilizzare per l'analisi dal \glo{dataset} caricato. & 
NI\\	

\textbf{TS1F6} & 
L'utente deve poter calcolare la distanza tra le dimensioni caricate. & 
NI\\	

\textbf{} & 
L'utente deve poter calcolare la distanza tra dimensioni tramite distanza di \textit{\glo{Manhattan}}. & 
NI\\

\textbf{} & 
L'utente deve poter calcolare la distanza tra dimensioni tramite distanza di \textit{\glo{Canberra}}. & 
NI\\

\textbf{} & 
L'utente deve poter calcolare la distanza tra dimensioni tramite distanza di \textit{\glo{Chebyshev}}. & 
NI\\

\textbf{TS3F7} & 
L'utente deve poter utilizzare funzioni di calcolo della distanza diverse da quella \glo{Euclidea}. & 
NI\\	

\textbf{TS1F-} & 
L'utente deve poter ridurre le dimensioni caricate attraverso un algoritmo di riduzione dimensionale. & 
NI\\

\textbf{} & 
L'utente deve poter selezionare \textit{\glo{IsoMap}} come algoritmo di riduzione dimensionale. & 
NI\\

\textbf{} & 
L'utente deve poter selezionare \textit{\glo{LLE}} come algoritmo di riduzione dimensionale. & 
NI\\

\textbf{} & 
L'utente deve poter selezionare \textit{\glo{FastMap}} come algoritmo di riduzione dimensionale. & 
NI\\

\textbf{} & 
L'utente deve poter selezionare \textit{\glo{t-SNE}} come algoritmo di riduzione dimensionale. & 
NI\\

\textbf{} & 
L'utente deve poter modificare il tasso di apprendimento per l'algoritmo \textit{t-SNE}. & 
NI\\

\textbf{} & 
L'utente deve poter modificare la \textit{perplessità} per l'algoritmo \textit{t-SNE}. & 
NI\\

\textbf{} & 
L'utente deve poter modificare il numero di \textit{\glo{neighbors}} per gli algoritmi \textit{IsoMap} ed \textit{LLE}. & 
NI\\

\textbf{TS1F-} & 
L'utente deve poter scegliere la visualizzazione per i dati tra quelle disponibili. & 
NI\\

\textbf{TS1F-} & 
L'utente deve poter visualizzare le dimensioni originali, ridotte e di cui si è calcolata una distanza. & 
NI\\

\textbf{TS1F-} & 
L'utente deve poter scegliere la visualizzazione \glo{\textit{Scatter Plot Matrix}}. & 
NI\\

\textbf{} & 
L'utente deve poter personalizzare lo stile della visualizzazione scelta. & 
NI\\

\textbf{TS1F-.1} & 
In uno \glo{\textit{Scatter Plot Matrix}} l'utente deve poter associare le dimensioni agli assi tra quelle disponibili. & 
NI\\

\textbf{TS3F-.2} & 
In uno \textit{Scatter Plot Matrix} l'utente deve poter selezionare i punti per vedere la loro relazione in ciascun \glo{Scatter Plot} presente. & 
NI\\

\textbf{TS1F-} & 
L'utente deve poter scegliere la visualizzazione \glo{\textit{Heat Map}}. & 
NI\\

\textbf{TS1F-.1} & 
In un \textit{Heat Map} l'utente deve poter ordinare i punti in modo da evidenziare le strutture presenti. & 
NI\\

\textbf{TS1F-.2} & 
In un \textit{Heat Map} ordinato l'utente deve poter applicare un \glo{dendrogramma} ai bordi del grafico. & 
NI\\

\textbf{TS1F-} & 
L'utente deve poter scegliere la visualizzazione \glo{\textit{Force Field}}. & 
NI\\

\textbf{TS3F-.1} & 
In un \textit{Force Field} l'utente deve poter utilizzare funzioni di forza diverse da quelle previste in automatico dal grafico "forcebased" di \glo{D3.js}. & 
NI\\

\textbf{TS3F-.2} & 
In un \textit{Force Field} l'utente deve poter interagire con il grafico per rilevare punti d'interesse. & 
NI\\

\textbf{TS1F-} & 
L'utente deve poter scegliere la visualizzazione \glo{\textit{Proiezione Lineare Multi Asse}}. & 
NI\\

\textbf{TS1F-.1} & 
Nella visualizzazione \textit{Proiezione Lineare Multi Asse} l'utente deve poter interagire con gli assi per evidenziare i punti d'interesse. & 
NI\\

\textbf{TS3F-} & 
L'applicazione, tramite analisi automatiche, deve evidenziare situazioni di particolare interesse. & 
NI\\

\textbf{TS3F-} & 
L'applicazione, tramite specifici algoritmi, deve preparare i dati scartando quelli meno rilevanti per l'analisi. & 
NI\\

\textbf{TS3F-} & 
L'utente deve avere a disposizione una guida introduttiva. & 
NI\\

\textbf{TS2F-} & 
L'utente deve poter visualizzare contemporaneamente due grafici per un confronto. & 
NI\\

\textbf{TS2F-} & 
L'utente deve poter salvare la sessione di lavoro in corso. & 
NI\\

\textbf{TS2F-} & 
L'utente deve poter scaricare un file in formato \glo{JSON} contenente i dati della sessione di lavoro. & 
NI\\

\textbf{TS2F-} & 
L'utente deve poter ripristinare una sessione di lavoro tramite caricamento di un file in formato \glo{JSON} in suo possesso. & 
NI\\

\textbf{TS1F-} & 
Verificare che la web app sia compatibile con il browser \glo{Firefox} dalla versione 84. & 
NI\\

\textbf{TS1F-} & 
Si verifichi che la \textit{web app} sia compatibile con il browser \glo{Chrome} dalla versione 87. & 
NI\\

\textbf{TS1F-} & 
Verificare che la \textit{web app} sia compatibile con il browser \glo{Safari} dalla versione 13.1. & 
NI\\

\textbf{TS1F-} & 
Verificare che la \textit{web app} sia compatibile con il browser \glo{Edge} dalla versione 79. & 
NI\\
		   
\rowcolor{white}\caption{Test di sistema}
\label{testSistema}
\end{longtable}

\renewcommand{\arraystretch}{1.5}
\renewcommand\extrarowheight{1.5pt}
\begin{longtable}{C{2.5cm} | C{10cm} } 
		\rowcolor{coloreRosso}
		\textcolor{white}{\textbf{Codice Test}} & 
		\textcolor{white}{\textbf{Codici Requisiti}} \\
		\endfirsthead
		\rowcolor{white}\multicolumn{2}{C{12.5cm}}{\textit{Continua nella pagina successiva...}} \\
	    \endfoot
	    \rowcolor{white}\caption{Tracciamento test - requisiti funzionali}
	    \endlastfoot
		\hline
		\textbf{TS1F1} & R1F1, R1F1.1, R1F1.2 \\
		\textbf{TS2F2} & R2F4 \\
		\textbf{TS1F3} & R1F5 \\
		\textbf{TS2F4} & R2F2, R2F15 \\
		\textbf{TS1F5} &  R1F7 \\
		\textbf{TS1F5.1} & R1F7.1, R1F7.1.1 \\
		\textbf{TS1F5.2} & R1F5, R1F7.2, R1F7.2.1 \\
		\textbf{TS1F5.3} &  R1F5, R1F7.3\\
		\textbf{TS3F5.3.1} & R3F7.3.1, R3F7.6 \\
		\textbf{TS1F5.4} & R1F7.4 \\
		\textbf{TS3F5.6} & R3F7.5 \\
		\textbf{TS3F6} & R3F8 \\
		\textbf{TS3F7} & R3F9 \\
		\textbf{TS3F8} & R3F10 \\
		\textbf{TS2F9} & R2F11 \\
		\textbf{TS2F10} & R2F6 
\end{longtable}
}