\section{Informazioni generali}
\begin{itemize}
\item \textbf{Motivo della riunione}: esposizione del PoC al termine del primo incremento e chiarimento dubbi.
\item \textbf{Luogo riunione}: videoconferenza tramite \glo{Skype}.
\item \textbf{Data}: \Data{}
\item \textbf{Orario d'inizio}: 16:30 
\item \textbf{Orario di fine}: 17:20 
\item \textbf{Partecipanti}:
	\begin{itemize}
	\item Membri del gruppo \textit{CodeBusters};
	\item Dott. Piccoli Gregorio.
	\end{itemize}
\end{itemize}

\section{Resoconto}

\begin{itemize}

\item \textbf{Esposizione del PoC}: avendo ultimato l'implementazione delle prime funzionalità, il gruppo ha richiesto un meeting con il Dott. Piccoli Gregorio per verificare il soddisfacimento delle aspettative. La maggior parte delle domande si sono concentrate sulla visualizzazione del grafico \textit{\glo{Scatter Plot Matrix}}, obbiettivo del prossimo incremento. Il grafico in questione dovrà accettare qualsiasi tipologia di dimensione caricata o derivata dall'utente e la dimensione dedicata al colore potrà essere anche non categorica; in quest'ultimo caso il colore dovrà essere di un solo tipo, con graduazioni diverse a seconda dei valori (codice decisione \textbf{Esterno-3.1}). Il gruppo dovrà quindi gestire in modo dinamico l'assegnazione della corretta scala a seconda del dominio di ciascuna dimensione.

\item \textbf{Calcolo delle distanze tra dimensioni}: dopo aver fatto qualche prova, il gruppo ha riscontrato ulteriori dubbi nel processo di creazione di una nuova dimensione mediante calcolo delle distanze. Il Dott. Piccoli ha evidenziato come questo processo vada a creare una dimensione con più punti di quelli presenti nel dataset di partenza, producendo una \textit{matrice delle distanze}. Questa dimensione può essere eventualmente utilizzata anche per i grafici che non sono basati su tale concetto (codice decisione \textbf{Esterno-3.2}).

\item \textbf{Gestione delle dimensioni create}: la discussione si è poi soffermata sul modo più opportuno di gestire le dimensioni create dall'utente. Poiché si tratta di un processo esplorativo, l'utente avrà la necessità di mettere a confronto diverse combinazioni di dimensioni, in modo da trovare quella che risalta maggiormente dei possibili legami. Il consiglio fornito è quello di far scegliere all'utente come nominare le dimensioni create, in modo da evitare confusione (codice decisione \textbf{Esterno-3.3}).

\item \textbf{Richiesta di dataset specifici}: al termine dell'incontro il team ha richiesto qualche dataset per testare fin da subito il prodotto, soprattutto a livello prestazionale, in modo da garantire un buon livello qualitativo e per verificare la validità delle scelte implementative.  

\end{itemize}