\appendix
\section{Standard ISO/IEC 12207}
ISO/IEC 12207 è uno standard internazionale per i processi del ciclo di vita del software. Lo standard stabilisce i processi presenti nel ciclo di vita del software e, per ciascuno di essi, le attività da svolgere e i risultati da produrre.
I processi sono suddivisi dalla norma in tre categorie:

\begin{itemize}
	
	\item \textbf{Processi primari}: comprendono le attività direttamente legate allo sviluppo del software;
	
	\item \textbf{Processi di supporto}: includono la gestione dei documenti e dei processi di controllo della qualità;
	
	\item \textbf{Processi organizzativi}: coprono gli aspetti manageriali e di gestione delle risorse.
	
\end{itemize}

\subsection{Processi primari}
I processi primari definiti dallo standard ISO/IEC 12207 sono i seguenti.

\subsubsection{Acquisizione}
Definisce le attività dell'acquirente, l'organizzazione che acquisisce un sistema, un prodotto software o un servizio software.
Le attività sono le seguenti.

\paragraph{Iniziazione}
L'attività di iniziazione 

\paragraph{Preparazione della richiesta di proposta}

\paragraph{Preparazione e aggiornameto del contratto}

\paragraph{Monitoraggio dei fornitori}

\paragraph{Accettazione e completamento}




    

    


