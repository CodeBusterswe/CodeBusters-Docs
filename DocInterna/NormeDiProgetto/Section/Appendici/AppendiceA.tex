\appendix
\section{Standard ISO/IEC 12207}
ISO/IEC 12207 è uno standard internazionale per i processi del ciclo di vita del software. Lo standard stabilisce i processi presenti nel ciclo di vita del software e, per ciascuno di essi, le attività da svolgere e i risultati da produrre.
I processi sono suddivisi dalla norma in tre categorie:

\begin{itemize}
	
	\item Processi primari;
	
	\item Processi di supporto;
	
	\item Processi organizzativi. 
\end{itemize}

\subsection{Processi primari}
I processi primari definiti dallo standard ISO/IEC 12207 sono i seguenti.

\subsubsection{Acquisizione}
Definisce le attività dell'acquirente, l'organizzazione che acquisisce un sistema, un prodotto software o un servizio software.
Le attività sono le seguenti:

\begin{itemize}

\item Iniziazione;

\item Preparazione della richiesta di proposta;
\item Preparazione e aggiornamento del contratto;
\item Monitoraggio dei fornitori;

\item Accettazione e completamento.

\end{itemize}

\subsubsection{Fornitura}
Definisce le attività del fornitore, l'ente che fornisce il sistema, il prodotto software o il servizio software.
Le attività sono le seguenti:

\begin{itemize}

\item Iniziazione;

\item Preparazione della risposta;

\item Contratto;

\item Pianificazione;

\item Esecuzione e controllo;

\item Revisione e valutazione;

\item Rilascio e completamento.

\end{itemize}

\subsubsection{Sviluppo}
Il processo ha lo scopo di sviluppare un prodotto software, o un sistema basato sul software, che indirizzi le esigenze del cliente.
Le attività sono le seguenti:

\begin{itemize}

\item Iniziazione;

\item Analisi dei requisiti di sistema;

\item Progettazione architettura di sistema;

\item Analisi dei requisiti software;

\item Progettazione software;

\item Codifica;

\item Integrazione dei componenti;

\item Collaudo del software;

\item Integrazione del sistema;

\item Collaudo di sistema;

\item Installazione software.

\end{itemize}

\subsubsection{Esercizio}
Il processo di esercizio è svolto simultaneamente al processo di manutenzione. Il processo ha lo scopo di mantenere operativo il sistema e di fornire il supporto agli utenti. 
Le attività sono le seguenti:

\begin{itemize}

\item Implementazione dei processi;

\item Operational testing;

\item System operation;

\item Supporto utente.

\end{itemize}

\subsubsection{Manutenzione}
Le fase di manutenzione è svolta simultaneamente alla fase precedente di esercizio.\\
Il processo ha lo scopo di modificare il prodotto software dopo il suo rilascio per correggere i difetti, migliorare le sue prestazioni o altri attributi o adattarlo a cambiamenti nell'ambiente operativo.
Le attività sono le seguenti:

\begin{itemize}

\item Analisi del problema e delle modifiche;

\item Implementazione delle modifiche;

\item Revisione/accettazione della manutenzione;

\item Migrazione;

\item Ritiro del software.

\end{itemize}

\subsection{Processi di supporto}
I processi di supporto aiutano le attività di tutti gli altri processi dell'organizzazione a garantire il successo e la qualità del progetto.
Questi processi possono essere attivati da un processo primario o da un altro processo di supporto. \\
I processi definiti dallo standard sono i seguenti.

\subsubsection{Documentazione}
Il processo di documentazione garantisce lo sviluppo e la manutenzione delle informazioni prodotte e registrate relativamente al prodotto software. Il processo contiene una serie di attività per gestire i documenti:
\begin{itemize}

	\item Pianificazione;

	\item Progettazione e sviluppo;
	
	\item Produzione;
	
	\item Manutenzione.

\end{itemize}

\subsubsection{Gestione della configurazione}
Il processo di gestione della configurazione ha lo scopo di definire e mantenere l'integrità di tutti i componenti della configurazione e di renderli accessibili a chi ne ha diritto. Le sue attività sono:
\begin{itemize}

	\item Identificazione;
	
	\item Controllo della configurazione;
	
	\item Controllo; 	
	
	\item Valutazione della configurazione;
	
	\item Gestione del rilascio e distribuzione.

\end{itemize}

\subsubsection{Accertamento della qualità}
Il processo ha lo scopo di assicurare che tutti i prodotti di fase siano conformi con i piani e gli standard definiti. \\
Le sue attività sono:
\begin{itemize}

	\item Accertamento di prodotto;
	
	\item Accertamento di processo;
	
	\item Accertamento della qualità di sistema.

\end{itemize}

\subsubsection{Verifica}
Il processo di verifica è un processo per determinare se i prodotti software di un'attività soddisfano i requisiti o le condizioni loro imposti nelle attività precedenti. Per l'efficacia dei costi e delle prestazioni, la verifica deve essere integrata, il prima possibile, con il processo che la utilizza. Questo processo può includere analisi, revisione e test. Questo processo può essere eseguito con diversi gradi di indipendenza. Il grado di indipendenza può variare dalla stessa persona o persona diversa nella stessa organizzazione a una persona in un'altra organizzazione con diversi gradi di separazione. Nel caso in cui il processo venga eseguito da un'organizzazione indipendente dal fornitore, sviluppatore, operatore o manutentore, viene chiamato Processo di verifica indipendente.

\subsubsection{Validazione}
Il processo di validazione ha lo scopo di confermare che i requisiti siano rispettati quando uno specifico prodotto sia utilizzato nell'ambiente destinatario.

\subsubsection{Revisione congiunta}
Il processo di revisione congiunta ha lo scopo di rivedere con gli \glo{stakeholders} i processi eseguiti rispetto agli obiettivi definiti negli accordi e le cose da fare per assicurare lo sviluppo di un prodotto che soddisfi i requisiti concordati.
Le revisioni sono svolte durante l'intero ciclo di vita, sia a livello di progetto che a livello tecnico.
La revisione congiunta è svolta tra gli stessi componenti del team quando si revisiona un componente del prodotto oppure, tra fornitore e committente quando si revisiona l'intero prodotto. \\
Le attività che la compongono sono:
\begin{itemize}

	\item Revisioni di gestione del progetto;
	
	\item Revisioni tecniche.

\end{itemize}

\subsubsection{Audit}
Il processo di audit ha lo scopo di determinare in maniera indipendente la conformità di prodotti e processi selezionati ai requisiti, piani e accordi. L'attività di auditing è svolta da personale che non ha partecipato direttamente allo sviluppo dei prodotti, dei servizi o dei sistemi oggetto delle revisioni.

\subsubsection{Risoluzione dei problemi}
Il processo di risoluzione dei problemi è un processo per analizzare e risolvere i problemi, qualunque sia la loro natura o fonte, che vengono scoperti durante l'esecuzione di sviluppo, funzionamento, manutenzione o altri processi. L'obiettivo è fornire un mezzo tempestivo, responsabile e documentato per garantire che tutti i problemi rilevati siano analizzati e risolti.

\subsection{Processi organizzativi}
I processi organizzativi sono impiegati in  una organizzazione per stabilire e implementare una struttura per organizzare  e gestire i processi del ciclo di vita e  del personale e per il continuo miglioramento dei processi e della struttura stessa.

\subsubsection{Gestione organizzativa}
Il processo di gestione organizzativa ha lo scopo di organizzare, monitorare e controllare l'avvio e le prestazioni di un processo per il raggiungimento dei loro obiettivi in accordo con quelli di business dell'organizzazione. Il processo è stabilito da una organizzazione per assicurare la consistente applicazione di pratiche per l'uso dall'organizzazione e nei progetti.\\
Questo processo è composto da una serie di attività:
\begin{itemize}
	\item Inizializzazione e definizione dello scopo;
	
	\item Pianificazione;
	
	\item Esecuzione e controllo;
	
	\item Revisione e Valutazione;
	
	\item Chiusura.
\end{itemize}

\subsubsection{Gestione delle infrastrutture}
Il processo ha lo scopo di mantenere una infrastruttura stabile ed affidabile necessaria a supportare le prestazioni di qualsiasi processo. L'infrastruttura può includere hardware, software, metodi, tool, tecniche, standard ed utilità per lo sviluppo, operatività o manutenzione.\\
Il processo è composto dalle seguenti attività:
\begin{itemize}
	\item Implementazione del processo;
	
	\item Istituzione dell'infrastruttura;
	
	\item Mantenimento dell'infrastruttura.
\end{itemize}

\subsubsection{Miglioramento del processo}
Il processo ha lo scopo di stabilire, valutare, misurare, controllare e migliorare il ciclo di vita del software.\\
Le attività che lo compongono sono:
\begin{itemize}
	\item Istituzione del processo;
	
	\item Valutazione del processo;
	
	\item Miglioramento del processo.
\end{itemize}

\subsubsection{Formazione del personale}
Il processo di formazione ha lo scopo di fornire e mantenere personale qualificato. L'acquisizione, la fornitura, lo sviluppo e molti altri processi dipendono in gran parte da personale esperto e competente.\\
Il processo è composta dalle seguenti attività:
\begin{itemize}
	\item Implementazione del processo;
	
	\item Materiale per l'implementazione formazione;
	
	\item Pianificazione per l'implementazione della formazione.
\end{itemize}


