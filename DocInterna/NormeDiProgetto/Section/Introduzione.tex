\section{Introduzione}
\subsection{Scopo del documento}
Questo documento ha lo scopo di definire le linee guida di tutti i processi istanziati dal gruppo \Gruppo, inoltre contiene l'organizzazione e l'uso di tutte le risorse di sviluppo e le convenzioni che il gruppo decide di attuare sull'uso delle tecnologie, sullo stile di codifica e di scrittura. Ogni membro del gruppo è obbligato a tenere in considerazione questo documento per garantire maggiore uniformità e coerenza del materiale prodotto.

\subsection{Scopo del capitolato}
Oggigiorno, anche i programmi più tradizionali gestiscono e memorizzano una grande mole di dati e di conseguenza serve un software in grado di eseguire un'analisi e una interpretazione delle informazioni.\\
Il \glo{capitolato} C4 ha come obiettivo quello di creare un'applicazione di visualizzazione di dati con numerose dimensioni in un formato comprensibile dall'occhio umano.  A questo scopo è necessario utilizzare algoritmi di intelligenza artificiale, o nel caso svilupparne di nuovi, che, agendo sulla distanza dei vari punti del grafico, riescano a sviluppare un modello semplificato che ne evidenzi i \glo{cluster}. 
L'applicazione dovrà inoltre agire su questi grafici creati evidenziando i dati ottenuti.

\subsection{Glossario}
Per evitare ambiguità relative alle terminologie utilizzare, è stato compilato il \Glossariov{1.0.0}. In questo documento sono riportati tutti i termini di particolare importanza e con un significato particolare. Questi termini sono evidenziati da una 'G' ad apice.

\subsection{Riferimenti}
\subsubsection{Riferimenti normativi}
\begin{itemize}	
	\item \textbf{Capitolato d'appalto C4 - HD Viz: visualizzazione di dati multidimensionali}:\\
	\textcolor{blue}{\url{https://www.math.unipd.it/~tullio/IS-1/2020/Progetto/C4.pdf}}
\end{itemize}

\subsubsection{Riferimenti informativi}
\begin{itemize}
	\item \textbf{Standard ISO/IEC 12207:1995}: \\
	\textcolor{blue}{\url{https://www.math.unipd.it/~tullio/IS-1/2009/Approfondimenti/ISO_12207-1995.pdf}}
	\item \textbf{Guide to the Software Engineering Body of Knowledge(SWEBOK), 2014} \\
	
	\item \textbf{Software Engineering - Ian Sommerville - 10 th Edition (2010)}: \\
	(formato cartaceo);
	
	\item \textbf{Slide T3 del corso Ingegneria del Software - Ciclo di vita del software}:\\
	\textcolor{blue}{\url{https://www.math.unipd.it/~tullio/IS-1/2020/Dispense/L03.pdf}}
\end{itemize}

