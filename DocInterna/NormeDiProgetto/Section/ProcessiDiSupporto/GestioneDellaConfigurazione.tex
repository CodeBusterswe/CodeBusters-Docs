\subsection{Gestione della Configurazione}
\subsubsection{Scopo}
Lo scopo di questa sezione è definire come il gruppo ha deciso di attuare il processo di supporto di gestione della configurazione sui file prodotti.

\subsubsection{Aspettative}
Le aspettative del gruppo \Gruppo{} nell'utilizzo di questo processo sono:
\begin{itemize}
	\item Avere un vantaggio nell'individuazione e risoluzione di possibili conflitti o errori;
	\item Avere il tracciamento di ogni modifica;
	\item Avere il ripristino, se necessario, a una versione precedente;
	\item Avere la condivisione tra i membri del gruppo del materiale configurato.
\end{itemize}

\subsubsection{Descrizione}
Il processo di gestione della configurazione ha lo scopo di mantenere organizzata e tracciabile la documentazione redatta e il codice sviluppato, creando una storia per ogni file prodotto. In particolare si vuole gestire la struttura e la disposizione delle varie parti di ogni file all'interno di un \glo{repository} facilmente accessibile e navigabile.
Inoltre il processo si occupa anche di mantenere ordinati i \glo{repository} utilizzati, favorendo lo sviluppo di un senso dell'orientamento.

\subsubsection{Versionamento}
\paragraph{Codice di versione}
La storia di ogni documento avanza nel tempo attraverso una successione di modifiche, verifiche e approvazioni. Ciascuna di queste operazioni comporta il cambiamento della versione del documento stesso. Ogni versione è identificata tramite un codice numerico di tre cifre:
\begin{center}
\textbf{[X].[Y].[Z]} 
\end{center}
Singolarmente esse rappresentano:
\begin{itemize}
	\item \textbf{X}: versione stabile, ossia sottoposta ad approvazione del responsabile del documento;
  	\item \textbf{Y}: versione controllata, ossia sottoposta a verifica di un verificatore del documento;
  	\item \textbf{Z}: versione modificata, ossia sottoposta a modifica di un redattore del documento.
\end{itemize}
  
Tutte le cifre iniziano dal valore 0. \\ 
Ciascuna cifra aumenta di un'unità ogni volta che si compie un'operazione sul documento, inoltre:
\begin{description}
	\item[Se la cifra X è modificata] : le cifre Y e Z ritornano a 0 (per esempio da 0.2.6 a 1.0.0);
	\item[Se la cifra Y è modificata] : la cifra Z ritorna a 0 (per esempio da 0.2.6 a 0.3.0).
\end{description}

\paragraph{Sistemi software utilizzati}
Per gestire le versioni è stato deciso di utilizzare il \glo{version control system (\glo{VCS}) distribuito} \glo{Git}. 
Le motivazioni di questa scelta si racchiudono nei vantaggi di utilizzo rispetto a \glo{VCS} centralizzati:
\begin{itemize}
	\item Possibilità di lavorare in locale senza il supporto del nodo centrale remoto; 
	\item Possibilità di creare diversi flussi di lavoro (\glo{branch}) per lavorare a documenti differenti;
	\item Miglior gestione dei conflitti, a favore di una migliore collaborazione.
\end{itemize}
Per gestire i \glo{repository} \glo{Git} è stata scelto il servizio \glo{GitHub} per i seguenti motivi: 
\begin{itemize}
	\item Integrazione di un \glo{issue tracking system(ITS)};
	\item Possibilità di utilizzarlo tramite browser, applicazione desktop, applicazione mobile o linea di comando;
	\item Buona conoscenza di quest'ultimo da parte di tutti i membri del gruppo.
\end{itemize}

\subsubsection{Struttura dei repository}
\paragraph{Repository utilizzati}
Per favorire una migliore organizzazione e divisione del lavoro è stato deciso di creare due \glo{repository} pubblici distinti:
\begin{description}
	\item[CodeBusters-Docs] : per il versionamento dei documenti. \\
	Riferimento: \url{https://github.com/Panz99/CodeBusters-Docs}
	\item[CodeBusters-Codes] : per il versionamento del codice. Il suo utilizzo è successivo alla \textit{Revisione dei Requisiti}, perciò non è ancora stato creato. 
\end{description}
In questa prima versione delle \NdPv si tratta la struttura del solo \glo{repository} CodeBusters-Docs. 

\paragraph{CodeBuster-Docs}
L'organizzazione del lavoro collaborativo è così riassunta:
\begin{itemize}
	\item Ramo principale main in cui è presente la sola documentazione pronta alla revisione;	
	\item Ramo develop in cui effettuare periodicamente il \glo{merge} dai vari rami minori;
	\item Rami derivanti dal develop con nomi parlanti, ognuno dedicato alla stesura di uno specifico documento (l'idea fondante è quella del \glo{Git feature branch workflow});
	\item Ramo fixTemplate per evitare errori sul template al momento del merge dei rami minori nel develop.
\end{itemize}
Nel \textit{main} i file sono contenuti nella cartella RR (\textit{Revisione dei Requisiti}), con l'obiettivo di crearne una nuova per ogni revisione futura. \\
Il file \textit{.gitignore} è l'unico esterno a cartelle e dichiara esplicitamente l'estensione dei file automaticamente generati da \LaTeX{} da non tracciare, poiché poco utili allo scopo del \glo{repository}. \\
Gli altri file sono invece organizzati in ulteriori cartelle dentro la RR:
\begin{itemize}
	\item \textbf{DocEsterna}: contiene l'\AdRv, il \PdPv, il \PdQv, il \Glossariov, i \textit{verbali esterni} ;
	\item \textbf{DocInterna}: contiene lo \SdFv, le \NdPv, i \textit{verbali interni};
	\item \textbf{Utility}: contiene file di utilità generale come il template dei documenti, comandi \LaTeX{} per velocizzare la redazione e immagini utilizzate.
\end{itemize}
	
\paragraph{Gestione dei cambiamenti in CodeBuster-Docs}
La separazione del flusso di lavoro tra i vari documenti da redarre permette una notevole diminuzione dei conflitti e quindi delle modifiche da apportare per errore. Il punto focale è che il ramo main rimanga pulito da ogni tipo di errore, per cui non è utilizzabile da nessun membro del gruppo fino a che ciascun responsabile non abbia dato l'approvazione al corrispettivo documento. Solo in quel momento è permesso il \glo{merge} del ramo develop nel main. \\
I cambiamenti da gestire sui documenti possono essere:
\begin{itemize}
	\item \textbf{Modifiche minori}: riguardano errori grammaticali, lessicali o di sintassi, che possono essere corretti dai redattori senza l'approvazione del responsabile;
	\item \textbf{Modifiche generali}: riguardano cambiamenti più generali come la struttura del documento o convenzioni da utilizzare e richiedono il consulto con il responsabile che potrà accettare o declinare la proposta di modifica.
\end{itemize}