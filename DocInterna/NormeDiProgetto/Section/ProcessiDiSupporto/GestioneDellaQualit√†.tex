\subsection{Gestione della Qualità}
\subsubsection{Scopo}
Lo scopo di questa sezione è definire gli obiettivi che il gruppo si è posto nel processo di supporto della gestione della qualità. 

\subsubsection{Aspettative}
Le aspettative del gruppo \Gruppo{} nell'utilizzo di questo processo sono:
\begin{itemize}
	\item Avere un continuo accertamento sulla qualità del prodotto, in modo che sia conforme con quella richiesta dal \glo{proponente};
	\item Avere un'organizzazione dei documenti qualitativa, al fine di velocizzare possibili modifiche e manutenzioni future;
	\item Avere un livello quantificabile della qualità dei processi attuati.
\end{itemize}

\subsubsection{Descrizione}
Il documento \PdQv racchiude gli standard di qualità che il gruppo si è posto di mantenere, le misurazioni oggettive che descrivono gli stati di avanzamneto.
In generale per perseguire la qualità di un prodotto bisogna agire in modo \glo{sistematico}, fornendo per esempio un ruolo a ciascun componente del gruppo, gestendo risorse e procedure per ogni processo in atto, effettuando test statici e dinamici sul codice prodotto in modo non invasivo.

\subsubsection{Attività}
Le attività del processo sono principalmente tre:
\begin{description}
	\item[Quality Control (QC)] : rappresenta tutti i controlli di qualità \textit{product-oriented} che iniziano insieme all'inizio effettivo di produzione di documenti o codice;
	\item[Quality Assurance (QA)] : rappresenta tutti gli accertamenti preventivi \textit{process-oriented} che valutano il \glo{way of working} adottato, senza rallentare i processi di sviluppo;
	\item[Quality Planning (QP)] : rappresenta una visione generale sugli obiettivi di qulità e sulle risorse necessarie a conseguirli. Nel nostro caso questo corrisponde alla redazione del \PdQv.
\end{description}

Nello specifico ogni membro del gruppo deve:
\begin{itemize}
	\item Comprendere fin da subito l'obiettivo del file che sta scrivendo, che sia documentale o software;
	\item Porsi obiettivi incrementali, per ridurre la possibilità d'errore e perseguire correttezza;
	\item Utilizzare conoscenze personali o di altri componenti del gruppo, creando un avanzamento continuo della propria formazione;
	\item Rispettare gli standard di qualità del \PdQv{}\textit{v 1.0.0}.
\end{itemize}

\subsubsection{Strumenti utili}
Per le attività del processo di gestione della qualità gli strumenti sono stati scelti in base.... 

 

  
