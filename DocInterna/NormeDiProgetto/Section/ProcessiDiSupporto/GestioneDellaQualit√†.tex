\subsection{Gestione della Qualità}
\subsubsection{Scopo}
Lo scopo di questa sezione è definire gli obiettivi che il gruppo si è posto nel processo di supporto della gestione della qualità. 

\subsubsection{Aspettative}
Le aspettative del gruppo \Gruppo{} nell'utilizzo di questo processo sono:
\begin{itemize}
	\item Avere un continuo accertamento sulla qualità del prodotto, in modo che sia conforme con quella richiesta dal \glo{proponente};
	\item Avere un'organizzazione dei documenti qualitativa, al fine di velocizzare possibili modifiche e manutenzioni future;
	\item Avere un livello quantificabile della qualità dei processi attuati.
\end{itemize}

\subsubsection{Descrizione}
Il documento \PdQv{1.0.0} racchiude i livelli di qualità che il gruppo si è posto di mantenere e le misurazioni oggettive che descrivono gli stati di avanzamento.
In generale per perseguire la qualità di un prodotto bisogna agire in modo \glo{sistematico}, fornendo per esempio un ruolo a ciascun componente del gruppo, gestendo risorse e procedure per ogni processo in atto, effettuando test statici e dinamici sul codice prodotto in modo non invasivo.

\subsubsection{Attività}
Le attività del processo sono principalmente tre:
\begin{itemize}
	\item \textbf{Quality Control (QC)}: rappresenta tutti i controlli di qualità \textit{product-oriented} che incominciano insieme all'inizio effettivo di produzione di documenti o codice;
	\item \textbf{Quality Assurance (QA)}: rappresenta tutti gli accertamenti preventivi \textit{process-oriented} che valutano il \glo{way of working} adottato, senza rallentare i processi di sviluppo;
	\item \textbf{Quality Planning (QP)}: rappresenta una visione generale sugli obiettivi di qualità e sulle risorse necessarie a conseguirli. Nel nostro caso questo corrisponde alla redazione del \PdQ{}.
\end{itemize}
	
Nello specifico ogni membro del gruppo deve:
\begin{itemize}
	\item Comprendere fin da subito l'obiettivo del file che sta scrivendo, che sia documentale o software;
	\item Porsi obiettivi incrementali, per ridurre la possibilità d'errore e perseguire correttezza;
	\item Utilizzare conoscenze personali o di altri componenti del gruppo, creando un avanzamento continuo della propria formazione;
	\item Rispettare gli standard di qualità del \PdQ{}.
\end{itemize}

\subsubsection{Metriche}
Le metriche sono degli standard per la misurazione delle qualità che il prodotto (software o documentale che sia) deve avere. Lo scopo è permettere ai \glo{verificatori} di verificare e valutare in modo oggettivo non solo i file prodotti, ma anche il processo per ottenerli. 
Le metriche si dividono in due sottoinsiemi: 
\begin{itemize}
	\item \textbf{Metriche per i processi}: consentono di valutare i processi nel ciclo di vita di un prodotto, permettendo possibili modifiche sulla pianificazione di essi per migliorare alcuni aspetti (tempi, costi, risorse); 
	\item \textbf{Metriche per i prodotti}: permettono di valutare i prodotti finali, verificando se effettivamente rispettano gli obiettivi di qualità del gruppo e del \glo{proponente}.
\end{itemize}

\paragraph{Classificazione delle metriche}
Le metriche scelte ed utilizzate dal gruppo sono identificabili tramite un codice univoco così composto: 
\begin{center}
\textbf{M[Utilizzo][IdNumerico]}
\end{center}
Singolarmente ciascun campo rappresenta:
\begin{itemize}
	\item \textbf{Utilizzo}: se la metrica è per:
			\begin{itemize}
				\item \textbf{PC}: processo;
				\item \textbf{PD}: prodotto.
			\end{itemize}
	\item \textbf{IdNumerico}: codice numerico crescente che parte da 1 e distingue le metriche dello stesso sottoinsieme (PC o PD). 
\end{itemize}