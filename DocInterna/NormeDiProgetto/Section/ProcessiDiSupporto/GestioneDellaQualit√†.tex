\subsection{Gestione della Qualità}
\subsubsection{Scopo}
Lo scopo di questa sezione è definire gli obiettivi che il gruppo si è posto nel processo di supporto della gestione della qualità. 

\subsubsection{Aspettative}
Le aspettative del gruppo \Gruppo{} nell'utilizzo di questo processo sono:
\begin{itemize}
	\item Avere un continuo accertamento sulla qualità del prodotto, in modo che sia conforme con quella richiesta dal \glo{proponente};
	\item Avere un'organizzazione dei documenti qualitativa, al fine di velocizzare possibili modifiche e manutenzioni future;
	\item Avere un livello quantificabile della qualità dei processi attuati.
\end{itemize}

\subsubsection{Descrizione}
Il documento \PdQv{1.0.0} racchiude i livelli di qualità che il gruppo si è posto di mantenere e le misurazioni oggettive che descrivono gli stati di avanzamento.
In generale per perseguire la qualità di un prodotto bisogna agire in modo \glo{sistematico}, fornendo per esempio un ruolo a ciascun componente del gruppo, gestendo risorse e procedure per ogni processo in atto, effettuando test statici e dinamici sul codice prodotto in modo non invasivo.

\subsubsection{Attività}
Le attività del processo sono principalmente tre:
\begin{itemize}
	\item \textbf{Quality Control (QC)}: rappresenta tutti i controlli di qualità \textit{product-oriented} che iniziano insieme all'inizio effettivo di produzione di documenti o codice;
	\item \textbf{Quality Assurance (QA)}: rappresenta tutti gli accertamenti preventivi \textit{process-oriented} che valutano il \glo{way of working} adottato, senza rallentare i processi di sviluppo;
	\item \textbf{Quality Planning (QP)}: rappresenta una visione generale sugli obiettivi di qualità e sulle risorse necessarie a conseguirli. Nel nostro caso questo corrisponde alla redazione del \PdQv{}.
\end{itemize}
	
Nello specifico ogni membro del gruppo deve:
\begin{itemize}
	\item Comprendere fin da subito l'obiettivo del file che sta scrivendo, che sia documentale o software;
	\item Porsi obiettivi incrementali, per ridurre la possibilità d'errore e perseguire correttezza;
	\item Utilizzare conoscenze personali o di altri componenti del gruppo, creando un avanzamento continuo della propria formazione;
	\item Rispettare gli standard di qualità del \PdQv{}.
\end{itemize}

\subsubsection{Metriche}
Le metriche sono degli standard per la misurazione delle qualità che il prodotto (software o documentale che sia) deve avere. Lo scopo è permettere ai \glo{verificatori} di verificare e valutare in modo oggettivo non solo i file prodotti, ma anche il processo per ottenerli. 
Le metriche si dividono in due sottoinsiemi: 
\begin{itemize}
	\item \textbf{Metriche per i processi}: consentono di valutare i processi nel ciclo di vita di un prodotto, permettendo possibili modifiche sulla pianificazione di essi per migliorare alcuni aspetti (tempi, costi, risorse); 
	\item \textbf{Metriche per i prodotti}: permettono di valutare i prodotti finali, verificando se effettivamente rispettano gli obiettivi di qualità del gruppo e del \glo{proponente}.
\end{itemize}

\paragraph{Classificazione delle metriche}
Le metriche scelte ed utilizzate dal gruppo sono identificabili tramite un codice univoco così composto: 
\begin{center}
\textbf{M[Utilizzo][IdNumerico]}
\end{center}
Singolarmente ciascun campo rappresenta:
\begin{itemize}
	\item \textbf{Utilizzo}: se la metrica è per:
			\begin{itemize}
				\item \textbf{PC}: processo;
				\item \textbf{PD}: prodotto.
			\end{itemize}
	\item \textbf{IdNumerico}: codice numerico crescente che parte da 1 e distingue le metriche dello stesso sottoinsieme (PC o PD). 
\end{itemize}

\paragraph{Metriche utilizzate}
Alcuni parametri per comprendere la tabella seguente:
\begin{itemize}
	\item \textbf{Budget at Completion (BAC)}: numero intero, corrisponde al budget totale allocato per il progetto ;

	\item \textbf{Earned Value (EV) o Budget totale allocato per il progetto}: (\% del lavoro svolto fino al momento del calcolo) * (budjet a disposizione);
	
	\item \textbf{Planned Value (PV) o Budgeted Cost of Work Scheduled (BCWS)}: lavoro che si era pianificato di svolgere fino al momento del calcolo;
	
	\item \textbf{Actual Cost (AC)}: soldi spesi per il progetto fino al momento del calcolo;
	
	\item \textbf{Number of Changed (NoC)}: numero di requisiti cambiati;
	
	\item \textbf{Number of Deleted (NoD)}: numero di requisiti eliminati;
	
	\item \textbf{Number of Added (NoA)}: numero di requisiti aggiunti;
	
	\item \textbf{Total Number of Initial Requirement (TNIR)}: numero totale dei requisiti iniziali;
	
	\item \textbf{Number of Satisfied (NoS)}: numero totale di requisiti soddisfatti;
	
	\item \textbf{Total number of Requirement (TnR)}: numero totale di requisiti;
	
	\item \textbf{Number of Quality Metrics Satisfied (NQMS)}: numero di metriche di qualità soddisfatte;
	
	\item \textbf{Total number of Quality Metrics (TQM)}: numero totale di metriche di qualità;
	
	\item \textbf{Passed test cases percentage (PTCP)}: numero totale di metriche di qualità;
	
	\item \textbf{Failed test cases percentage (FTCP)}: numero totale di metriche di qualità.
	
\end{itemize}

\rowcolors{2}{coloreGrigietto}{white}
\renewcommand{\arraystretch}{1.5}
\renewcommand\extrarowheight{1.5pt}
\begin{longtable}{C{1.5cm} C{4.5cm} C{5cm} C{4.5cm}}
		\rowcolor{coloreRosso}
		\textcolor{white}{\textbf{Codice}} & 
		\textcolor{white}{\textbf{Nome}} & 
		\textcolor{white}{\textbf{Descrizione}} & 
		\textcolor{white}{\textbf{Formula}} \\
		\endfirsthead
	    \endfoot
	    \rowcolor{white}\caption{Metriche per i processi}
	    \endlastfoot
		\hline
		\textbf{MPC1} & 
		Schedule Variance (SV)  & 
		Descrive se il gruppo sta rispettando o meno i tempi prestabiliti per i processi. & 
		EV - PV \\
		
		\textbf{MPC2} & 
		Budget Variance (BV) & 
		Descrive se il gruppo sta rispettando o meno i costi prestabiliti per i processi. & 
		EV - AC \\
		
		\textbf{MPC3} &
		Budget at Completion (BAC) &
		Indica il budget totale allocato per il progetto. &
		Numero intero \\
		
		\textbf{MPC4} &
		Earned Value (EV) &
		Rappresenta il valore prodotto dal progetto fino al momento della misurazione in seguito alle attività svolte.  &
		\% completamento $ \cdot BAC $\\
				
		\textbf{MPC5} &
		Planned Value (PV) &
		Corrisponde al denaro che si dovrebbe guadagnare al momento della misurazione.  &
		\% lavoro pianificato $ \cdot BAC $ \\	
			
		
		\textbf{MPC6} &
		Requirements stability index (RSI) &
		Indica quanto i requisiti variano nel tempo. &
		$(1 - \frac{NoC + NoD + NoA}{TNIR}) \cdot 100$ \\
		
		\textbf{MPC7} &
		Satisfied obligatory requirements (SOR) &
		Descrive se il gruppo ha soddisfatto i requisiti obbligatori o meno. &
		$\frac{NoS}{TnR} \cdot 100$ \\ 
		
		\textbf{MPC8} &
		Code Coverage (CC) &
		Descrive quanto il codice prodotto è coperto dalla suite di test dinamici. & - \\ 
		
		\textbf{MPC9} &
		Passed test cases percentage (PTCP) &
		Corrisponde alla percentuale di test passati rispetto alla suite di test dinamici. &
		Test passati / Test totali \\ 
		
		\textbf{MPC10} &
		Failed test cases percentage (FTCP) &
		Corrisponde alla percentuale di test falliti rispetto alla suite di test dinamici. &
		Test falliti / Test Totali \\ 
		
		\textbf{MPC11} &
		Qaulity Metrics Satisfied (QMS) &
		Descrive la percentuale di metriche di qualità soddisfatte. &
		$\frac{NQMS}{TQM} \cdot 100$ \\
		
		\textbf{MPC12} &
		Non-calculated Risk (NR) &
		Indica il numero di rischi non preventivati. &
		- 
\end{longtable}

Alcuni parametri per comprendere la tabella seguente:
\begin{itemize}
	\item \textbf{Numero frasi (N\textsubscript{fr})}: numero di frasi nell'intero documento;
	\item \textbf{Numero lettere (N\textsubscript{lett})}: numero di lettere nell'intero documento;
	\item \textbf{Numero parole (N\textsubscript{par})}: numero di parole nell'intero documento;
	\item \textbf{Funzionalità mancanti (N\textsubscript{fm})}: numero di funzionalità non implementate;
	\item \textbf{Funzionalità individuate (N\textsubscript{fi})}: numero di funzionalità individuate;
\end{itemize}

\rowcolors{2}{coloreGrigietto}{white}
\renewcommand{\arraystretch}{1.5}
\renewcommand\extrarowheight{1.5pt}
\begin{longtable}{C{1.5cm} C{4.5cm} C{5.5cm} C{5cm}}
		\rowcolor{coloreRosso}
		\textcolor{white}{\textbf{Codice}} & 
		\textcolor{white}{\textbf{Nome}} & 
		\textcolor{white}{\textbf{Descrizione}} & 
		\textcolor{white}{\textbf{Formula}} \\
		\endfirsthead
	    \endfoot
	    \rowcolor{white}\caption{Metriche per i prodotti}
	    \endlastfoot
		\hline
		\textbf{MPD1} & 
		Indice Gulpease & 
		Descrive la leggibilità del documento. & 
		$ 89 + \frac{300 \cdot N_{fr} - 10 \cdot N_{lett}}{N_{par}}$ \\
		\textbf{MPD2} & 
		Errori ortografici (EO) & 
		Descrive la correttezza ortografica del documento. & 
		- \\
		\textbf{MPD3} & 
		Copertura dei requisiti (CDR) & 
		Descrive quanti requisiti sono stati implementati nel prodotto software. &
		$(1 - \frac{N_{fm}}{N_{fi}}) \cdot 100 $ \\
		\textbf{MPD4} & 
		Averege Cyclomatic complexity (ACC) & 
		Indica in numero di cammini indipendenti presenti nel programma (misurabile attraverso il grafo di controllo di flusso). &
		- \\
\end{longtable}
 

  
