\subsection{Validazione}
\subsubsection{Scopo}
Lo scopo di questa sezione è fissare come il gruppo ha deciso di attuare il processo di \glo{validazione}. Questo comprende le attività di controllo mirate a confrontare che il prodotto sia conforme ai requisiti accordati con il \glo{proponente}.

\subsubsection{Aspettative}
Le aspettative del gruppo \Gruppo{} nell'utilizzo di questo processo sono:
\begin{itemize}
	\item Dimostrare la correttezza delle attività svolte in fase di verifica; 
	\item Avere la certezza che il prodotto software rispetti i requisiti riportati nell'\AdRv.
\end{itemize}

\subsubsection{Descrizione}
Le attività di validazione seguono quelle di verifica e validano i risultati ottenuti dai test. Sono attività che possono essere svolte non solo a prodotto finito, ma anche integrate con il processo di sviluppo per validare i risultati ottenuti fino a quel momento. È compito del \textit{Responsabile di progetto} accettarli solo se rispettano le aspettative del gruppo e del \glo{proponente}. \\
\linebreak
In questa prima versione delle \NdPv non si ha ancora attivato questo processo non avendo ancora iniziato la fase di sviluppo del codice.

