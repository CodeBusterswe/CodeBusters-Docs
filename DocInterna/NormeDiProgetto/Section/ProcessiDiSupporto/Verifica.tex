\subsection{Verifica}
\subsubsection{Scopo}
Lo scopo di questa sezione è definire come il gruppo ha deciso di attuare il processo di \glo{verifica}. Questo accerta che non siano stati introdotti errori durante lo sviluppo. Essa è svolta ripetutamente su tutti i processi in esecuzione (a ogni incremento della \glo{baseline}).

\subsubsection{Aspettative}
Le aspettative del gruppo \Gruppo{} nell'utilizzo di questo processo sono:
\begin{itemize}
	\item Verificare ogni fase rispettando criteri precisi, consistenti e modificabili se necessario;
	\item Automatizzare il più possibile le attività del processo;
	\item Rispettare gli obiettivi di copertura indicati nel \PdQv;
	\item Verificare correttamente per ottenere successo in fase di validazione.
\end{itemize}

\subsubsection{Descrizione}
Il processo di verifica prevede due attività principali, svolte dai \glo{verificatori}:
\begin{itemize}
	\item \textbf{Analisi statica}: non richiede l'esecuzione dell'oggetto di verifica. Per questo motivo è applicabile ad ogni prodotto, accertando la conformità agli standard e convenzioni di stile;
	\item \textbf{Analisi dinamica}: richiede l'esecuzione dell'oggetto di verifica. Per questo motivo è applicabile al codice sviluppato ma non ai documenti. Accerta, tramite test, il funzionamento di ogni unità del codice presa singolarmente, ma anche dell'intero sistema nella sua complessità.
\end{itemize}

\subsubsection{Verifica della documentazione}
Il processo di verifica della documentazione consiste in un'analisi statica attraverso strumenti automatici o condotta a mano (\glo{desk check}) con, in questo caso, due possibili metodi di lettura:   

\rowcolors{2}{coloreGrigietto}{white}
\renewcommand{\arraystretch}{1.5}
\renewcommand\extrarowheight{1.5pt}
\begin{longtable}{C{3cm} | C{4cm} C{3cm} C{5cm}}
		\rowcolor{coloreRosso}
		\textcolor{white}{\textbf{Metodo}} & 
		\textcolor{white}{\textbf{Obiettivo}} & 
		\textcolor{white}{\textbf{Attori}} & 
		\textcolor{white}{\textbf{Caratteristica}} \\
		\endfirsthead
		\rowcolor{white}\multicolumn{3}{C{8cm}}{\textit{Continua nella pagina successiva...}}\\
	    \endfoot
	    \rowcolor{white}\caption{Metodi di lettura}
	    \endlastfoot
		\hline
		\textbf{Walkthrough} & 
		Rilevare errori attraverso letture ad ampio spettro. & 
		Verificatori \newline Redattori & 
		Un errore riscontrato dal verificatore comporta una discussione con il redattore "colpevole" riguardo a una possibile soluzione. \\
		\textbf{Inspection} & 
		Rilevare specifici errori attraverso letture mirate. & 
		Verificatori \newline Redattori & 
		Il verificatore utilizza una lista di controllo, ossia un elenco di cosa va verificato in modo selettivo. \\
\end{longtable}

\subsubsection{Verifica del codice}
Il processo di verifica del codice rappresenta l'unione delle attività di analisi statica e dinamica. 

\rowcolors{2}{coloreGrigietto}{white}
\renewcommand{\arraystretch}{1.5}
\renewcommand\extrarowheight{1.5pt}
\begin{longtable}{C{3cm} | C{4cm} C{3cm} C{5cm}}
		\rowcolor{coloreRosso}
		\textcolor{white}{\textbf{Analisi}} & 
		\textcolor{white}{\textbf{Obiettivo}} & 
		\textcolor{white}{\textbf{Attori}} & 
		\textcolor{white}{\textbf{Esempi}} \\
		\endfirsthead
		\rowcolor{white}\multicolumn{3}{C{8cm}}{\textit{Continua nella pagina successiva...}}\\
	    \endfoot
	    \rowcolor{white}\caption{Analisi del codice}
	    \endlastfoot
		\hline
		\textbf{Statica} & 
		Verificare che siano rispettate le regole di buona programmazione preimposte dal gruppo. & 
		Verificatori \newline Programmatori & 
		Analisi di flusso dei dati, verifica formale del codice ecc. \\
		\textbf{Dinamica} & 
		Trovare bug ed errori eseguendo il prodotto software.  & 
		Verificatori \newline Programmatori & 
		Test di unità, di integrazione, di sistema, di regressione, di accettazione. \\
\end{longtable}

\paragraph{Analisi di flusso dei dati} 
Tipo di analisi statica del codice che accerta che il programma in verifica non acceda in nessuna sua parte a variabili prive di valore, quindi non ancora scritte. Controlla l'assenza di dati globali.

\paragraph{Verifica formale} 
Tipo di analisi statica del codice che ne prova la sua correttezza rispetto ai requisiti imposti dalle specifiche. Lo scopo è quello di esplorare tutti i rami possibili di esecuzione, senza doverlo necessariamente eseguire. 

\paragraph{Test} 
La scrittura dei test è la parte essenziale dell'analisi dinamica del codice. Tale attività ha lo scopo di rivelare al programmatore errori o \glo{bug} riscontrabili a run-time. Di conseguenza i test sono utili solo se falliscono. \\
L'esecuzione dei test deve essere ripetibile. Ottima pratica è renderla il più possibile automatica. \\
L'obiettivo del gruppo è rispettare gli standard di qualità dettati nel \PdQv. \\
I test si distinguono in:
\begin{itemize}
	\item \textbf{Test di unità (TU)}: verificano l'unità, ossia la più piccola parte di codice verificabile indipendentemente dalle altre (singolo metodo o singola classe);
	\item \textbf{Test di integrazione (TI)}: verificano le componenti del sistema, rilevando errori in fase di progettazione;
	\item \textbf{Test di sistema (TS)}: verificano il comportamento dell'intero sistema, controllando se rispetta le specifiche tecniche;
	\item \textbf{Test di regressione (TR)}: verificano che nuove modifiche apportate non abbiano creato errori in altri parti del programma;
	\item \textbf{Test di accettazione (TA)}: verificano errori a livello di \glo{UI}, relativi a requisiti o casi d'uso concordati con il proponente.
\end{itemize}

Questi test sono identificati tramite un codice formato dai seguenti campi:
\begin{center}
\textbf{T[TipologiaTest][ImportanzaRequisito]*[TipologiaRequisito]*[IdNumerico]} 
\end{center}
Singolarmente essi rappresentano:
\begin{itemize}
	\item \textbf{TipologiaTest}: identifica la tipologia di test tra quelli prima citati:
		\begin{itemize}
			\item \textbf{U} : unità;
			\item \textbf{I} : integrazione;
			\item \textbf{S} : sistema;
			\item \textbf{R} : regressione;
			\item \textbf{A} : accettazione.			 
		\end{itemize}
	\item \textbf{ImportanzaRequisito \& TipologiaRequisito}: sono seguiti da '*' perché presenti solo nei codici per i test di sistema e accettazione. Questi campi identificano il requisito che si vuole testare (o più requisiti con la stessa importanza e stessa tipologia), seguendo ciò che è riportato al paragrafo \ref{para:requisiti};
	\item \textbf{IdNumerico}: codice numerico crescente che parte da 1 e distingue i test dello stesso tipo (U, I, S, R, A).
\end{itemize}
 
\subsubsection{Strumenti per la verifica}
Per il processo di verifica applicato ai documenti scritti in \LaTeX{} il gruppo ha deciso di utilizzare il correttore automatico dell'editor TeXmaker, che individua le parole grammaticalmente scorrette sottolineandole in rosso. I verificatori, attraverso i metodi di lettura sopra citati, devono occuparsi di rilevare possibili ripetizioni, errori di lessico o sintassi.  \\
\linebreak
Per il processo di verifica applicato al codice il gruppo non si è ancora riunito per decidere quali strumenti software utilizzare per effettuare analisi statica e dinamica. Ciò sarà sicuramente argomento di discussione nei prossimi incontri interni.
















