\subsubsection{Procedure}
Per il coordinamento e le comunicazioni durante la realizzazione del progetto, il gruppo adotterà le seguenti procedure: 
\begin{itemize}
\item\textbf{comunicazione interna}: coinvolgeranno tutti i membri del team;
\item\textbf{comunicazione esterna}: avverrà con il proponente e committente.
\end{itemize}

\paragraph{Gestione delle comunicazioni}

\subparagraph{Comunicazioni interne}
Le comunicazioni interne avvengono attraverso il canale chiamato \glo{Discord}. Questa applicazione consente la collaborazione a distanza e viene utilizzata anche in ambienti aziendali. Tale software permette al team di creare uno spazio di lavoro condiviso.
\subparagraph{Comunicazioni esterne}
Le comunicazioni con utenti esterni al gruppo sono gestite dal responsabile del progetto. Le modalità utilizzate sono le seguenti: 
\begin{itemize}
\item tramite la posta elettronica, dove viene utilizzato il seguente indirizzo \textbf{codebusterswe@gmail.com}; 
\item attraverso \glo{Skype} per colloqui con azienda Zucchetti.
\end{itemize} 
\paragraph{Gestione degli incontri}
\subparagraph{Incontri interni}
Il responsabile del progetto concorda con il team gli incontri interni. Egli ha il compito di specificare la data delle riunioni nel calendario e approvare i verbali redatti dal segretario. I membri del gruppo sono tenuti a partecipare alle riunioni, interagendo nel dibattito. Affinché una riunione sia ritenuta valida, devono essere presenti almeno cinque membri del gruppo.
\subparagraph{Verbali di riunioni interne}
In occasione di ogni incontro interno viene redatto un verbale dal segretario scelto dal responsabile. Il contenuto della riunione viene riportato nel verbale corrispondente e deve essere approvato dal responsabile.
\subparagraph{Incontri esterni}
Il responsabile del progetto organizza gli incontri esterni con il proponente. Il proponente o committente potrebbero richiedere incontri con il team, il responsabile è tenuto a proporre una data in accordo con le parti e la comunica attraverso i canali sopra citati.
Gli incontri esterni avvengono tra i membri del gruppo e il proponente e quanto discusso viene riportato nel verbale esterno corrispondente.
\subparagraph{Verbali di riunioni esterne}
In occasione di ogni incontro esterno viene redatto un verbale dal segretario scelto dal responsabile. Il contenuto della riunione verrà riportato nel verbale corrispondente e deve essere approvato dal responsabile.
\paragraph{Gestione degli strumenti e di coordinamento}
\subparagraph{Tickecting}
Il tickecting permette ai membri del gruppo di rimanere aggiornati sullo stato delle attività in corso. Tramite questo il responsabile assegna compiti ai membri del gruppo e controlla l'andamento dei \glo{task} assegnati. Lo strumento di ticketing scelto è \textbf{Planner}: si tratta di un'applicazione \glo{multi-piattaforma}, in cui è visibile a tutti i membri del team lo stato di avanzamento dei compiti assegnati, con la possibilità di aggiungerne nuovi. 

\paragraph{Gestione dei rischi}
Il responsabile è tenuto a individuare i rischi e renderli noti; tale attività verrà documentata nel \textit{Piano di Progetto}. La procedura per la gestione dei rischi è: 
\begin{itemize}
\item individuazione di nuovi problemi e monitoraggio di quelli già presenti;
\item menzione dei rischi individuati nel \textit{Piano di Progetto}; 
\item ridefinizione delle strategie per la gestione dei rischi in caso di necessità.
\end{itemize}
\subparagraph{Codifica dei rischi}
I rischi sono codificati nel modo seguente: 
\begin{itemize}
\item  \textbf{RT}: rischi tecnologici 
\item \textbf{RO}: rischi organizzativi
\item   \textbf{RI}: rischi interpersonali
\end{itemize}
\subsubsection{Strumenti}
Nel corso dello sviluppo del progetto, il gruppo utilizzerà i seguenti strumenti: 
\begin{itemize}
\item \textbf{Telegram\glo}: applicazione di messaggistica per la comunicazione rapida e la gestione del gruppo; 
\item \textbf{Github\glo}: piattaforma che permette la condivisione in remoto di tutti i file del progetto e del versionamento;
\item \textbf{Git\glo}: sistema di controllo delle versioni;
\item \textbf{Discord}: applicazione multipiattaforma utilizzata per le riunioni interne;
\item \textbf{Planner}: piattaforma per la gestione dei compiti assegnati;
\item \textbf{Skype}: applicazione che consente di effettuare delle videoconferenze, utilizzata per la comunicazione con il proponente;
\item \textbf{Google Drive}: server per la condivisione rapida delle documentazioni che riguardano l'attività del gruppo.
\end{itemize}