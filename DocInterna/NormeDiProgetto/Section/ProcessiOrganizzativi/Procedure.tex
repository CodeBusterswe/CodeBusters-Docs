\subsection{Procedure}
Per il coordinamento e le comunicazioni durante la realizzazione del progetto, il gruppo adotterà le seguenti procedure: 
\begin{itemize}
\item\textbf{comunicazione interna}: coinvolgeranno tutti i membri del team;
\item\textbf{comunicazione esterna}: avverrà con il proponente e committente.
\end{itemize}

\subsubsection{Gestione delle comunicazioni}
\paragraph{Comunicazioni interne}
Le comunicazioni interne avvengono attraverso il canale chiamato \glo{Discord}. Questa applicazione consente la collaborazione a distanza e viene utilizzata anche in ambienti aziendali. Tale software permette al team di creare uno spazio di lavoro condiviso. \newline \newline

\paragraph{Comunicazioni esterne}
Le comunicazioni con utenti esterni al gruppo sono gestite dal responsabile del progetto. Le modalità utilizzate sono le seguenti: 
\begin{itemize}
\item tramite la posta elettronica, dove viene utilizzato il seguente indirizzo \textbf{codebusterswe@gmail.com}; 
\item attraverso skype per colloqui con azienda Zucchetti.
\end{itemize}

\subsubsection{Gestione degli incontri}
\paragraph{Incontri interni} 
Il responsabile del progetto concorda con il team gli incontri interni. Egli ha il compito di specificare la data delle riunioni nel calendario e approvare i verbali redatto dal segretario. I membri del gruppo sono tenuti a partecipare alle riunioni, interagendo nel dibattito. Affinché una riunione sia ritenuta valida, devono essere presenti almeno cinque membri del gruppo. \newline \newline

\paragraph{Verbali di riunioni interni} 
In occasione di ogni incontro interno viene redatto un verbale dal segretario scelto dal responsabile. Il contenuto della riunione viene riportato nel verbale corrispondente e deve essere approvato dal responsabile.
\newline \newline

\paragraph{Incontri esterni} 
Il responsabile del progetto organizza gli incontri esterni con il proponente. Il proponente o committente potrebbero richiedere incontri con il team, il responsabile è tenuto a proporre una data in accordo con le parti e la comunica attraverso i canali sopra citati.
L'incontri esterno avvengo tra i membri del gruppo e il proponente e viene riportato nel verbale esterno corrispondente.
 \newline \newline

\paragraph{Verbali di riunioni esterni} 
In occasione di ogni incontro esterno viene redatto un verbale dal segretario scelto dal responsabile. Il contenuto della riunione verrà riportato nel verbale corrispondente e deve essere approvato dal responsabile.
\newline \newline
\subsubsection{Gestione degli strumenti e di coordinamento}
\paragraph{Tickecting}

Il tickecting permette ai  membri del gruppo di rimanere aggiornati sullo stato delle attività in corso. Attraverso il quale il responsabile assegna compiti ai membri del gruppo e controlla l'andamento dei \glo{task} assegnati. Lo strumento di ticketing scelto è \textbf{Planner}: si tratta di un'applicazione \glo{multipiattaforma}, in cui sono visibili a tutti i membri del team lo stato di avanzamento dei compiti assegnati e con la possibilità di aggiungere nuovi. 
\newline \newline
\subsubsection{Gestione dei rischi}
Il responsabile è tenuto a individuare i rischi e renderli noti, tale attività verrà documentata nel \textit{Piano di Progetto}. La procedura per la gestione dei rischi sono: 
\begin{itemize}
\item individuazione nuovi problemi e monitorare quelli già presenti;
\item menzionare i rischi individuati nel \textit{Piano di Progetto}; 
\item ridefinizione le strategie dei rischi in caso di necessità. \newline \newline

\paragraph{Codifica dei rischi} 
I rischi sono codificate nel modo seguente: 
\begin{itemize}
\item  \textbf{RT}: rischi tecnologici 
\item \textbf{RO}: rischi organizzativi
\item   \textbf{RI}: rischi interpersonali
\end{itemize}
\end{itemize}
\subsubsection{Strumenti}
Nel durante dello sviluppo del progetto, il gruppo utilizzerà i seguenti strumenti: 
\begin{itemize}
\item \textbf{Telegram\glo}: un'applicazione di messaggistica per la comunicazione rapida e gestione del gruppo; 
\item \textbf{Github\glo}: permette la condivisione in remoto di tutti i file del progetto e versionamento;
\item \textbf{Git\glo}: il sistema di controllo di versioni;
\item \textbf{Discord}: un'applicazione multipiattaforma utilizzata per le riunioni interni;
\item \textbf{Planner}: è una piattaforma per la gestione dei compiti assegnati;
\item \textbf{Skype}: un'applicazione consente di effettuare le videoconferenze, utilizzato per comunicazione con il proponente;
\item \textbf{Google Drive}: server per la condivisione rapida delle documentazioni che riguardano l'attività del gruppo.
\end{itemize}