\subsection{Gestione Organizzativa}
\subsubsection{Scopo}
In questa sezione vengono esposte le modalità di coordinamento adottate dal gruppo e lo scopo del processo, che sono le seguenti:
\begin{itemize}
\item Adottare un modello organizzativo per l'individuazione dei rischi che potrebbero verificarsi;
\item Definire un modello di sviluppo da adottare;
\item Pianificare il lavoro rispettando le scadenze;
\item Calcolo del prospetto economico in base ai ruoli;
\item Determinare un bilancio finale sulle spese.
\end{itemize}

\subsubsection{Aspettative}
Le attese riguardo al processo in questione sono le seguenti: 
\begin{itemize}
\item Ottenere una pianificazione ragionevole delle attività da seguire;
\item Coordinamento dell'attività del gruppo, assegnando ruoli, compiti e semplificando la comunicazione tra i membri;
\item Adoperare processi per regolare le attività e renderle economiche.
\end{itemize}

\subsubsection{Descrizione}
Le attività  di gestione sono: 
\begin{itemize}
\item Assegnazione dei ruoli e dei compiti;
\item Inizio e definizione dello scopo;
\item Istanziazione dei processi;
\item Pianificazione e stima di tempi, risorse e costi;
\item Esecuzione e controllo;
\item Revisione e valutazione periodica delle attività.
\end{itemize}

\subsubsection{Ruoli di Progetto}
Ogni membro del gruppo deve, a turno, ricoprire almeno una volta ciascun ruolo di progetto che corrisponde alle figure aziendali. I ruoli che ogni membro del gruppo è tenuto a rappresentare sono descritti di seguito.

\paragraph{Responsabile di progetto}
Il responsabile di progetto ricopre un ruolo fondamentale, in quanto si occupa delle comunicazioni con il proponente e committente. Inoltre, egli deve svolgere i seguenti compiti:
\begin{itemize}
\item pianificare;
\item gestire;
\item controllare;
\item coordinare.
\end{itemize}

\paragraph{Amministratore di progetto}
L'amministratore deve avere il controllo dell'ambiente di lavoro ed essere di supporto. Inoltre egli deve: 
\begin{itemize}
\item dirigere le infrastrutture di supporto;
\item controllare versioni e configurazioni;
\item risolvere i problemi che riguardano la gestione dei processi;
\item gestire la documentazione.
\end{itemize}

\paragraph{Analista}
L'analista si occupa dell'analisi dei problemi e del dominio applicativo. Questa figura ha le seguenti responsabilità:
\begin{itemize}
\item studio del dominio del problema; 
\item redazione della documentazione: Analisi dei Requisiti e Studio di Fattibilità;
\item definizione dei requisiti e della sua complessità.
\end{itemize}

\paragraph{Progettista}
Il progettista si occupa dell'aspetto tecnico e tecnologico del progetto, segue lo sviluppo e non la manutenzione del prodotto. Inoltre egli deve scegliere: 
\begin{itemize}
\item un'architettura adatta per il sistema del prodotto in base alle tecnologie scelte;
\item il modo più efficiente per ottimizzare l'aspetto tecnico del progetto.
\end{itemize}

\paragraph{Programmatore}
Il programmatore si occupa della parte di codifica in base alle specifiche fornite dal progettista, operando con ottica di manutenibilità del codice. Inoltre egli deve: 
\begin{itemize}
\item creare e gestire componenti di supporto per la verifica e la validazione del codice. 
\end{itemize}

\paragraph{Verificatore}
Il verificatore è presente durante tutta l'attività del progetto, deve controllare il lavoro svolto dai membro del gruppo. Il verificatore deve: 
\begin{itemize}
\item controllare i prodotti in fase di revisione, utilizzando le tecniche e gli strumenti definiti nelle Norme di Progetto; 
\item evidenziare gli errori e segnalarli all'autore del prodotto in questione.
\end{itemize}

\paragraph{Formazione}
La formazione di ogni membro del gruppo avviene attraverso studio autonomo delle tecnologie proposte dal proponente in occasione della presentazione del capitolato e incontri. 

\paragraph{Materiale utilizzato}
Si prende come riferimento, oltre al materiale indicato nella sotto sezione \textit{Riferimenti Informativi}, la seguente documentazione per l'utilizzo di: 
\begin{itemize}
\item\textbf{\LaTeX}: \textcolor{blue}{\url{https://www.latex-project.org/}};
\item\textbf{Texmaker}: \textcolor{blue}{\url{https://www.xm1math.net/texmaker/}};
\item\textbf{GitHub}: \textcolor{blue}{\url{https://github.com/}};
\item\textbf{React}: \textcolor{blue}{\url{https://it.reactjs.org/}};
\item\textbf{D3.js}: \textcolor{blue}{\url{https://d3js.org/}};
\item\textbf{Nodejs}: \textcolor{blue}{\url{https://nodejs.org/it/}};
\item\textbf{Git}: \textcolor{blue}{\url{https://git-scm.com/}}.
\end{itemize}