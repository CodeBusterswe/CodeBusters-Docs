\subsection{Gestione Organizzativa}
\subsubsection{Scopo}
In questa sezione vengono esposte le modalità  di coordinamento adottate dal gruppo che regolano gli incontri(interni o esterni) e le comunicazioni.
\begin{itemize}
\item \textbf{comunicazione}: interna con i membri del gruppo, quella esterna con l'azienda;
\item \textbf{incontri}: incontri interni con in membri del gruppo, esterno con l'azienda.

\end{itemize}
\subsubsection{Aspettative}
Le attese, riguardo al processo in questione sono i seguenti: 
\begin{itemize}
\item ottenere una pianificazione ragionevole delle attività da seguire;
\item coordinamento dell'attività  del gruppo, assegnando loro i ruoli, i compiti e semplificando la comunicazione tra loro;
\item adoperare processi per regolare le attività  e renderle economiche;

\end{itemize}

\subsubsection{Descrizione}
Le attività  di gestione sono: 
\begin{itemize}
\item assegnazione dei ruoli e dei compiti;
\item inizio e definizione dello scopo;
\item istanziazione dei processi;
\item pianificazione e stima di tempi, risorse e costi;
\item esecuzione e controllo;
\item revisione e valutazione periodica delle attività .
\end{itemize}

\subsection{Ruoli di Progetto}
Ogni membro del gruppo deve, a turno ricoprire almeno una volta ciascun ruolo di progetto che corrisponde alle figure aziendali. I ruoli che ogni membro del gruppo è tenuto a rappresentare sono descritti di seguito.



\subsubsection{Responsabile di progetto}
Il responsabile di progetto ricopre un ruolo fondamentale, in quanto si occupa delle comunicazioni con il proponente e committente. Inoltre, egli deve svolgere i seguenti compiti:
\begin{itemize}
\item pianificare;
\item gestire;
\item controllare;
\item coordinare,
\end{itemize}

\subsubsection{Amministratore di progetto}
L'amministratore deve avere il controllo dell'ambiente di lavoro ed essere di supporto,. Inoltre Egli deve: 
\begin{itemize}
\item dirigere le infrastrutture di supporto;
\item controllare versioni e configurazioni;
\item risolvere i problemi che riguardano la gestione dei processi;
\item gestire la documentazione;
\end{itemize}


\subsubsection{Analista}
L'analista si occupa dell'analisi dei problemi e del dominio applicativo. Questa figura ha le seguenti responsabilità:
\begin{itemize}
\item studio del dominio del problema; 
\item redazione della documentazione: Analisi dei Requisiti e Studio di Fattibilità ;
\item definizione dei requisiti e la sua complessità .
\end{itemize}
\subsubsection{Progettista}
Il progettista si occupa dell'aspetto tecnico e tecnologico del progetto, segue lo sviluppo e non la manutenzione del prodotto. Inoltre egli deve scegliere: 
\begin{itemize}
\item un'architettura adatta per il sistema del prodotto in base alle tecnologie scelte;
\item il modo più efficiente per ottimizzare l'aspetto tecnico del progetto.
\end{itemize}
\subsubsection{Programmatore}
Il programmatore si occupa della parte di codifica in base alle specifiche fornite dal progettista, operando con ottica di manutenibilità  del codice. Inoltre egli deve: 
\begin{itemize}
\item creare e gestire componenti di supporto per la verifica e la validazione del codice. 
\end{itemize}

\subsubsection{Verificatore}
Il verificatore è presente durante tutta l'attività   del progetto, deve controllare il lavoro svolto dai membro del gruppo. Il  verificatore deve: 
\begin{itemize}
\item controllare i prodotti in fase di revisione, utilizzando le tecniche e gli strumenti definiti  nelle Norme di Progetto; 
\item evidenziare gli errori e segnalarli all'autore del prodotto in questione.
\end{itemize}

\subsubsection{Formazione}
La formazione di ogni membro del gruppo avviene attraverso studio autonomo delle tecnologie proposte dal proponente in occasione della presentazione del capitolato e incontri.

