\subsection{Qualità}
\subsubsection{Descrizione}
In questa sezione si tratterà dei modi con cui si cercherà di mantenere la qualità nelle varie fasi del progetto. Queste sono delle dichiarazioni di intenti, dunque potrebbero subire aggiunte, in caso se ne ritenesse necessario, nelle prossime versioni del documento.
\subsubsection{Analisi dei requisiti}
Questa attività ha lo scopo di elencare e tracciare i requisiti e i casi d'uso richiesti dal proponente. Dopo averli formulati e approvati è necessario tracciare i loro cambiamenti. \\
Per verificare che il software sia di qualità è necessario che tutti i requisiti, almeno quelli obbligatori, siano rispettati.
\subsubsection{Progettazione}
Al termine della progettazione si dovrà avere un'architettura che ha tradotto i requisiti in unità di codice. \'E importante che ciascuna componente di codice si riferisca ad un requisito, in modo da verificare facilmente che venga soddisfatto. I compiti del programmatore dovranno essere organizzati in modo che ciascuno si occupi di un singolo modulo e non si creino interferenze.\\
\'E buona prassi scomporre le varie componenti in piccole parti per favorire la manutenibilità. 
\subsubsection{Codifica}
In questa fase di elaborazione vera e propria del prodotto dovranno essere seguite delle norme per rendere il codice più leggibile e favorire la manutenibilità:
\begin{itemize}
\item \textbf{Linee di commento}: un rapporto basso tra linee di codice e linee di commento può essere spia di carenza di informazioni;
\item \textbf{Profondità della gerarchia}: avere una gerarchia molto profonda aumenta il numero di dipendenze, meglio limitarla;
\item \textbf{Complessità dei moduli}: è preferibile avere moduli più semplici e scorporare moduli con più funzionalità per rendere il codice più leggibile e favorire le modifiche. 
\item \textbf{Numero dei parametri in un metodo}: un numero troppo elevato di parametri in un metodo potrebbe denotare un grado di complessità troppo elevato.
\end{itemize}