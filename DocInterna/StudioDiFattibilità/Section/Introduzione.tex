\section{Introduzione}
\subsection{Scopo del Documento}
Questo documento contiene la stesura dello studio di fattibilità riguardante i sette capitolati proposti. Per ciascuno di essi vengono evidenziati i seguenti aspetti:
\begin{itemize}
    \item \textbf{Descrizione generale};
    \item \textbf{Prerequisiti e tecnologie coinvolte};
    \item \textbf{Vincoli};
    \item \textbf{Aspetti positivi};
    \item \textbf{Aspetti critici}.
\end{itemize}
Per ogni capitolato vengono esposte le motivazioni e le ragioni per cui il gruppo ha deciso di non prenderlo in considerazione, scegliendo come progetto il capitolato \NomeProgetto{}.
\subsection{Glossario}
Il gruppo \Gruppo{} ha redatto un documento denominato \Glossariov{1.0.0} così da evitare ambiguità fra i termini presenti durante la lettura.
In tale documento, sono presenti tutti i termini tecnici, ambigui e specifici del progetto con le loro relative definizioni.
Un termine presente nel \Glossariov{1.0.0} e utilizzato in questo documento viene indicato con la lettera "G" ad apice alla fine della parola.

