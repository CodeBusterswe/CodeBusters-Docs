%DA FARE
\section{Capitolato C1}
\subsection{Titolo del capitolato}
Il capitolato in questione si chiama \textit{"BlockCOVID: supporto digitale al contrasto della pandemia"}, il proponente \`e l'azienda \textit{imola informatica} e i committenti sono \VT{} e \CR{}.

\subsection{Descrizione del capitolo}
Il capitolato uno propone di sviluppare un'applicazione per registrare le presenze in uffici in modo da permettere la pulizia delle postazioni ed evitare che la stessa sia utilizzata da più persone.
L'obiettivo finale è la creazione di un'applicazione mobile con due interfacce. La prima destinata agli amministratori per la gestione degli uffici e delle postazioni mentre la seconda per due categorie di utenti: il dipendente che registra la sua presenza e l'addetto alle pulizie, il quale riceve la lista delle postazioni da disinfettare.

\subsection{Tecnologie coinvolte}
((Sebbene l'azienda non impone tecnologie specifiche per lo sviluppo del server o della UI, vi sono comununque delle scelte preferenziali da considerare nello svolgimento del progetto))(?):
\begin{itemize}
\item	Java (versione 8 o superiori), Python o nodejs per lo sviluppo del server back-end;
\item	protocolli asincroni per le comunicazioni app mobile-server;
\item	un sistema blockchain per salvare con opponibilità a terzi i dati di sanificazione;
\item	IAAS Kubernetes o di un PAAS, Openshift o Rancher, per il rilascio delle componenti del server e la gestione della scalabilità orizzontale.
\item	Avere il server che esponga delle API Rest attraverso le quali sia possibile utilizzare l'applicativo. In alternativa è possibile utilizzare gRPC come soluzione alternativa al Rest.
\end{itemize}

\subsection{Vincoli}
DA FARE

\subsection{Aspetti positivi}
DA FARE

\subsection{Aspetti critici}
DA FARE

\subsection{Conclusioni}
Visto il contesto attuale di pandemia un'applicazione su questo campo è quantomai utile e dunque stimola interesse. La valutazione è positiva anche per quanto riguarda le tecnologie che si dovranno esplorare per portare a termine questo capitolato. Il lato negativo risiede nella complessità dello sviluppo dell'applicazione mobile nonché del lato server. 
