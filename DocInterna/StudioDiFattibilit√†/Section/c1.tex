\section{C1 - BlockCOVID: supporto digitale al contrasto della pandemia}
\subsection{Descrizione del capitolo}
Il capitolato propone di sviluppare un'applicazione per registrare le presenze in uffici in modo da permettere la pulizia delle postazioni ed evitare che la stessa sia utilizzata da più persone.
L'obiettivo finale è la creazione di un'applicazione mobile con due interfacce. La prima destinata agli amministratori per la gestione degli uffici e delle postazioni mentre la seconda per due categorie di utenti: il dipendente che registra la sua presenza e l'addetto alle pulizie, il quale riceve la lista delle postazioni da disinfettare.\\
L'azienda proponente è \textit{Imola Informatica}.

\subsection{Tecnologie coinvolte}
L' azienda non impone tecnologie specifiche per lo sviluppo, ma consiglia l'utilizzo di:
\begin{itemize}
\item \glo{Java} (versione 8 o superiori), \glo{Python} o \glo{Node.js} per lo sviluppo del server back-end;
\item	protocolli asincroni per le comunicazioni app mobile-server;
\item	un sistema \glo{blockchain} per salvare con opponibilità a terzi i dati di sanificazione;
\item	\glo{Kubernetes} o di un \glo{PAAS}, \glo{Openshift} o \glo{Rancher}, per il rilascio delle componenti del server e la gestione della scalabilità orizzontale.
\end{itemize}

\subsection{Vincoli}
Gli obiettivi sono:
\begin{itemize}
\item	Realizzare un server e un'applicazione mobile, dove è prevista la presenza di due macro-tipologie:
\begin{itemize}
\item	Amministratore di sistema, con i permessi di gestire e monitorare postazioni, stanze, utenti;
\item	Utente, che può essere il dipendente che scansiona il tag presente alla sua postazione e segnala la sua presenza (con la possibilità di prenotarla), o un addetto alle pulizie che può gestire e monitorare le stanze da igienizzare;
\end{itemize}
\item	E' richiesto il report dei test effettuati;
\item	E' richiesta una documentazione sulle scelte, le loro relative motivazioni ed eventuali problemi incontrati con relative soluzioni.
\end{itemize}

\subsection{Aspetti positivi}
\begin{itemize}
\item	Progetto con fine attuale ed utile alla società;
\item	Stimola molto interesse essendo che richiede l'uso di molti servizi;
\item	L'azienda risulta essere molto disponibile per seguire il gruppo nello sviluppo del progetto.
\end{itemize}
\subsection{Aspetti critici}
\begin{itemize}
\item	Per questo capitolato è presente molta concorrenza;
\item	Richiede l'apprendimento di molti servizi e linguaggi che i membri del gruppo non conoscono, andando ad incidere molto sul tempo che andrebbe dedicato per il loro apprendimento.
\end{itemize}

\subsection{Conclusioni}
Visto il contesto attuale di pandemia un'applicazione su questo campo risulta essere molto utile; anche le tecnologie coinvolte hanno suscitato un forte interesse. Però, dopo un'attenta valutazione, il capitolato in questione è stato escluso a causa della complessità, ritenuta più elevata rispetto alle altre proposte. 
