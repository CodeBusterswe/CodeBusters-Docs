\section{C2 - EmporioLambda: piattaforma di e-commerce in stile Serverless}

\subsection{Descrizione del capitolo}
Nel capitolato viene proposto di creare una piattaforma di \glo{e-commerce}.
Lo scopo finale è avere una piattaforma dove i clienti possano registrarsi, ricercare i prodotti, aggiungerli al "carrello" (ovvero una pagina che permette di rivedere e considerare i prodotti e l'ammontare complessivo dei prezzi) e acquistarli. La piattaforma deve offrire un'interfaccia per i venditori, dove essi possono aggiungere e modificare i prodotti destinati alla vendita.\\
L'azienda proponente è \textit{Red Babel}.

\subsection{Tecnologie coinvolte}
Il progetto prevede l'adozione di una bozza iniziale dell'architettura dettata dal proponente e di alcune scelte tecnologiche obbligatorie; il fornitore è libero e altamente incoraggiato ad esplorare diverse scelte.
L'applicazione deve essere costruita con tecnologia \glo{serverless} con una \glo{API} basata su \glo{HTTP}:
\begin{itemize}
\item	Si utilizzano \glo{AWS Lambda} e altre componenti di supporto (\glo{AWS API Gateway}, \glo{AWS DynamoDB}, \glo{AWS S3});
\item	\glo{CloudFormation} per la gestione delle risorse sopraelencate;
\item	Serverless \glo{Framework} per l'implementazione di applicazioni \glo{serverless}, facilitandone alcune difficili strutture (come permessi, sottoscrizioni o accesso);
\item	Il \glo{BFF}(Back end For Front end) sarà implementato in \glo{Next.js} e richiede l' uso di \glo{Typescript} come linguaggio principale;
\item	il codice sorgente dovrebbe essere pubblicato e aggiornato utilizzando \glo{GitHub} o \glo{GitLab}.
\end{itemize}

\subsection{Vincoli}
È richiesto l'uso di un numero minimo di ambienti di lavoro: uno locale di ogni sviluppatore, uno per i test (accessibile a tutti gli sviluppatori) e uno pubblico (accessibile agli utenti).
È richiesta l'implementazione di tutti i moduli di alto livello:
\begin{itemize}
\item	\textbf{EmporioLambda-frontend} (EML-FE): modulo che serve le pagine web richieste dal cliente;
\item	\textbf{EmporioLambda-backend} (EML-BE): modulo che espone i servizi dell'applicazione;
\item	\textbf{EmporioLambda-integration} (EML-I): rappresenta tutti i servizi di terze parti;
\item	\textbf{EmporioLambda-monitoring} (EML-MON): è il set di strumenti usati per monitorare lo stato dell'applicazione.
\end{itemize}
\'E richiesta la presenza delle seguenti pagine (con relative principali funzionalità):
\begin{itemize}
\item	Homepage;
\item	Lista dei prodotti;
\item	Dettaglio prodotto;
\item	Carrello;
\item	Account;
\item	Checkout;
\item	Controllo del venditore.
\end{itemize}
Il sito deve implementare i seguenti ruoli utilizzando \glo{AWS Cognito Identity}:
\begin{itemize}
\item	Amministratore;
\item	Venditore;
\item	Cliente.
\end{itemize}
L'unica integrazione obbligatoria richiesta è il provider di pagamento \glo{Stripe}.
\subsection{Aspetti positivi}
\begin{itemize}
\item	È richiesto l'uso di numerosi servizi forniti da \glo{AWS}, sia comuni che meno, i quali possono risultare utili al nostro futuro;
\item	Nel capitolato viene richiesto il pagamento elettronico, argomento attualmente di nicchia;
\item	Linguaggi come \glo{TypeScript} e \glo{JavaScript} sono molto utili anche in ambito lavorativo;
\item	Nel 2020 ormai gli e-commerce sono alla portata di tutti e fanno parte della vita quotidiana delle persone, perciò è un argomento familiare ai membri del gruppo.
\end{itemize}
\subsection{Aspetti critici}
Discutendo con il gruppo non abbiamo rilevato aspetti critici rilevanti.
\subsection{Conclusioni}
In questo caso si utilizzano tecnologie non troppo complesse e, prendendo in considerazione le altre piattaforme di e-commerce esistenti, al momento non risultano problemi evidenti che si potrebbero riscontrare nella fase di sviluppo. Tuttavia, il gruppo ha considerato questo capitolato poco appetibile.
