\section{C3 - GDP - Gathering Detection Platform}

\subsection{Descrizione del capitolo}
Il capitolato presentato si pone come obiettivo finale la realizzazione di una piattaforma in grado di dare all'utilizzatore una rappresentazione grafica delle informazioni sulla probabilità di assembramento nelle zone potenzialmente a rischio. Questo viene fatto sulla base di alcuni dati che vengono individuati, raccolti ed elaborati. \\
La piattaforma deve sfruttare tecniche di \glo{machine learning} e/o \glo{deep learning}.\\
L'azienda proponente è \textit{SyncLab}.

\subsection{Tecnologie coinvolte}
Sebbene l'azienda non impone tecnologie specifiche per lo sviluppo del server o dell'interfaccia, vi sono comunque delle scelte preferenziali da considerare nello svolgimento del progetto:
\begin{itemize}
\item Utilizzo di \glo{Java} e \glo{Angular} per lo sviluppo della parte di \glo{back-end} e di \glo{front-end} della componente web del sistema;
\item Il \glo{framework} \glo{Leaflet} per la gestione delle mappe (heatmap ecc.) (\url{https://leafletjs.com});
\item Utilizzo di \glo{protocolli asincroni} per le comunicazioni tra le diverse componenti;
\item Utilizzo del \glo{pattern} Publisher/Subscriber, e adozione del protocollo \glo{MQTT} ("MQ Telemetry Transport" o "Message Queue Telemetry Transport") di facile implementazione e ampia diffusione in applicazioni \glo{M2M} (MachineToMachine) e \glo{IoT} (InternetOfThings).
\end{itemize}
Per la parte di \glo{machine learning} l'azienda dà ampia libertà, sebbene ci siano delle tecnologie consigliate:
\begin{itemize}
\item \glo{Python} come linguaggio di programmazione;
\item \glo{TensorFlow}, \glo{Pytorch}, \glo{Keras} e \glo{Scikit-learn} come libreria per l'apprendimento automatico.
\end{itemize}
Per facilitare la comprensione delle librerie consigliate, l'azienda propone delle guide e delle risorse online.

\subsection{Vincoli}
L'azienda proponente suggerisce un'architettura in grado di implementare quello che si definisce essere un sistema reattivo, in grado cioè di soddisfare le seguenti caratteristiche: 
\begin{itemize}
\item \textbf{Responsive}: la richiesta di un servizio deve sempre avere una risposta, anche quando si verifica un guasto; 
\item \textbf{Resilient}: i servizi devono poter essere ripristinati a seguito di guasti; 
\item \textbf{Elastic}: i servizi devono poter essere scalati in base all'effettiva domanda; 
\item \textbf{Message-driven}: i servizi devono rispondere a quello che accade, non tentare di controllare ciò che fa.  
\end{itemize}
È richiesto che tutte le componenti applicative siano correlate da test unitari e d'integrazione. Inoltre il sistema deve essere testato nella sua interezza tramite test \textit{end-to-end}. 

\subsection{Aspetti positivi}
\begin{itemize}
\item Il capitolato ha come obiettivo la creazione di una piattaforma di grande aiuto a fronte della situazione che stiamo vivendo;
\item Nel capitolato si tratta l'argomento del \glo{machine learning}, molto interessante e che potrebbe risultare utile in futuro;
\item L'azienda sembra essere molto disponibile per seguire il gruppo nel percorso di sviluppo del progetto.
\end{itemize}

\subsection{Aspetti critici}
\begin{itemize}
\item Il capitolato richiede l'apprendimento di molti servizi, sconosciuti dai membri del gruppo, basati su argomenti non trattati nel percorso universitario. Gran parte del tempo andrebbe quindi dedicato allo studio di questi concetti e strumenti, per poi applicare le conoscenze acquisite;
\item Oltre al tempo impiegato per l'apprendimento di questi strumenti c'è da tenere in considerazione il tempo per addestrare la parte di \glo{machine learning}.
\end{itemize}

\subsection{Conclusioni}
Nonostante tale capitolato abbia destato particolare interesse all'interno del gruppo, specialmente per la possibilità di utilizzare tecnologie innovative, il team ha valutato la complessità di tale progetto come molto elevata e ha preferito orientarsi verso un'altra alternativa.
