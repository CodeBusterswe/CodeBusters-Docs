\section{C4 - HD Viz: visualizzazione di dati multidimensionali}

\subsection{Descrizione del capitolo}
Il capitolato C4 ha come obiettivo quello di creare un'applicazione di visualizzazione di grandi quantità di dati, con numerose dimensioni e in un formato comprensibile. A questo scopo è necessario utilizzare algoritmi di intelligenza artificiale che, agendo su concetti che variano dal grafico preso in considerazione, riescano a sviluppare un modello semplificato che ne evidenzi i \glo{cluster}. 
L'applicazione dovrà permettere l'interazione dell'utente con i grafici ottenuti.\\
L'azienda proponente è \textit{Zucchetti}.

\subsection{Tecnologie coinvolte}
L'applicazione “HD Viz” dovrà utilizzare le seguenti tecnologie:
\begin{itemize}
\item \glo{HTML}/\glo{CSS}/\glo{Javascript} con libreria \glo{D3.js} per la parte client;
\item Il \glo{database} (\glo{SQL} o \glo{NOSQL}) potrà essere sviluppato in \glo{Java} con server \glo{Tomcat} o in \glo{JavaScript} con server \glo{Node.js}.
\end{itemize}

\subsection{Vincoli}
I vincoli del capitolato C4 riguardano, in prima istanza, il numero di dimensioni dei dati da visualizzare. L'applicazione deve riuscire a visualizzare dati con almeno 15 dimensioni. 
Il formato dei dati da fornire a HD Viz deve essere una \glo{query} ad un \glo{database} o anche un file in formato \glo{CSV} preparato in precedenza.
HD Viz dovrà presentare le seguenti visualizzazioni: 
\begin{itemize}
\item \textbf{\glo{Scatter plot Matrix}} (fino ad un massimo di 5 dimensioni): è la presentazione a riquadri disposti a matrice di tutte le combinazioni di \glo{scatter plot}; 
\item \textbf{\glo{Force Field}}: converte dati in più dimensioni in un grafico a due/tre dimensioni rappresentandoli sulla base della loro distanza; 
\item \textbf{\glo{Heat Map}}: rappresenta la distanza tra i punti con colori più o meno intensi. In questa visualizzazione è necessario ordinare i dati per evidenziarli; 
\item \textbf{\glo{Proiezione Lineare Multi Asse}}: si rappresentano a due dimensioni dati multidimensionali, permettendo di spostare gli assi per rendere la visualizzazione più comprensibile.  
\end{itemize} 
È necessario che l'utente non percepisca le operazioni di riduzione dimensionale; inoltre può scegliere in quali dimensioni vuole visualizzare il grafico.
Il proponente lascia ampia libertà di esplorazione dell'argomento e accetta altri requisiti opzionali proposti dal fornitore come, ad esempio, la creazione di algoritmi per ridurre le dimensioni.

\subsection{Aspetti positivi}
\begin{itemize}
\item L'analisi dei \glo{Big Data} è una problematica più che mai attuale e che potrebbe risultare molto utile in ambito professionale;
\item Nel capitolato si tratta l'intelligenza artificiale, altro argomento molto interessante in un futuro ambito lavorativo;
\item Gli strumenti \glo{HTML}/\glo{CSS}/\glo{JavaScript} sono già stati visti nel corso di Tecnologie Web;
\item La libreria D3.js è ben documentata e largamente fornita di esempi sulle visualizzazioni richieste;
\item L'azienda ha dato disponibilità per ulteriori chiarimenti in corso d'opera;
\item L'azienda fornirà dei set di dati per testare il software.
\end{itemize}

\subsection{Aspetti critici}
\begin{itemize}
\item Il capitolato potrebbe risultare difficoltoso nel testare gli algoritmi;
\item Altri capitolati hanno per argomento situazioni più concrete in confronto all'analisi ed elaborazione dei dati;
\item La libreria \glo{D3.js} è molto potente e può richiedere diverso tempo per imparare a padroneggiarla. 
\end{itemize}

\subsection{Conclusioni}
La possibilità di applicare conoscenze del settore matematico a contesti reali ha riscosso fin da subito un notevole interesse da parte di tutti i membri.\\ In seguito ad alcuni chiarimenti forniti dall'azienda in questione, rivelatasi molto chiara e disponibile, il gruppo ha percepito l'importanza di approfondire il settore dei \glo{Big Data}. Esplorando questo campo, al giorno d'oggi molto florido nell'utilizzo e in continuo sviluppo, tutti i membri hanno l'opportunità di aggiungere al proprio bagaglio curricolare un'esperienza importante.\\
L'impiego della libreria \glo{D3.js}, una delle più usate in questo ambito, permette di conoscere strumenti moderni e potenti, nonostante richiedano un notevole impegno per padroneggiarli. Tuttavia l'utilizzare \glo{HTML}, \glo{Javascript} e \glo{CSS}, linguaggi già visti nel corso di Tecnologie Web, agevolerà e permetterà di dedicare molto più tempo per approfondirli.\\In seguito ad un'attenta valutazione, anche rispetto alle altre proposte, il gruppo ha scelto questo capitolato come prima preferenza, fiducioso nell'instaurare un buon rapporto con l'azienda.
