\section{C5 - PORTACS - Point Of Interest  Oriented Real-Time Anti Collision System}

\subsection{Descrizione del capitolo}
Il capitolato propone la realizzazione di un sistema per il monitoraggio e la gestione di unità presenti in una mappa.\\
Ogni unità (che può essere un robot, un muletto o un'auto a guida automatica) ha un punto di partenza nella mappa e una lista di \textit{Point Of Interest} (POI) che deve raggiungere.
Il sistema dovrà indicare ad ognuna di esse la prossima mossa che dovrà fare in base alla posizione, direzione e velocità  delle altre unità. Ognuna dovrà inviare costantemente questi parametri al sistema, in modo da rendere possibile pilotare e coordinare tutte le unità, evitando incidenti ed ingorghi.\\
L'azienda proponente è \textit{Sanmarco Informatica}

\subsection{Tecnologie coinvolte}
Il proponente non impone specifiche tecnologie per lo sviluppo della piattaforma ma consiglia l'utilizzo di :
\begin{itemize}
\item \glo{Node.js} per la parte di \glo{back-end} e \glo{React.js} o \glo{Next.js}  per lo sviluppo di front-end che permettono di creare una SPA (\textit{Single Page Application}) ;
\item \glo{Socket}, libreria per sviluppare applicazioni real-time facilmente integrabile con Node.js;
\end{itemize}

\subsection{Vincoli}
L'azienda proponente pone i seguenti obiettivi da raggiungere:

\begin{itemize}
\item Il sistema dovrà avere una visualizzazione in real-time della mappa e relativa posizione delle singole unità;
\item L'interfaccia utente che rappresenterà ogni singola unità  dovrà prevedere quattro frecce direzionali (si \textit{"accenderà"} quella suggerita dal sistema), il pulsante start/stop e l'indicatore della velocità attuale.
\end{itemize}

La versione finale del software dovrà essere in grado di accettare i seguenti input:
\begin{itemize}
\item Scacchiera oppure mappa con percorsi predefiniti e relativi vincoli (sensi unici, numero massimo di unità contemporanee);
\item Definizione dei POI (aree di carico/scarico e sosta).
\end{itemize}

\subsection{Aspetti positivi}
\begin{itemize}
\item Il proponente non obbliga l'implementazione di algoritmi di ricerca operativa per l'ottimizzazione dei percorsi;
\item Non si richiede la geo-localizzazione interna o esterna: questa verrà simulata, riducendo il tempo dedicato alla parte di codifica;
\item L'utilizzo di tecnologie per la gestione real-time è parso un argomento molto interessante e ha suscitato curiosità nel gruppo;
\item L'azienda proponente ha dimostrato disponibilità per ulteriori chiarimenti e/o informazioni.
\end{itemize}

\subsection{Aspetti critici}
L'azienda proponente non suggerisce tecnologie per lo sviluppo della piattaforma apprese in ambito universitario o conosciute dai membri del gruppo; inoltre l'ambito risulta avere una complessità elevata rispetto agli altri capitolati e ciò potrebbe comportare una dilatazione eccessiva del tempo dedicato allo sviluppo.

\subsection{Conclusioni}
Tale capitolato ha attirato l'attenzione e stimolato l'interesse di tutti i membri del gruppo, in particolar modo per il contesto applicativo che risulta essere un argomento attuale e molto utile per il futuro.
Per via del livello di difficoltà e dell'impegno stimato, il gruppo ha ritenuto troppo elevata la mole di lavoro necessaria e ha deciso di orientarsi verso altri capitolati. 

