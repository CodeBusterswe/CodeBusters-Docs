\section{C6 - RGP: Realtime Gaming Platform}

\subsection{Descrizione del capitolo}
L'obiettivo del capitolato è la realizzazione di un videogioco a scorrimento verticale, fruibile da cellulare, con due modalità di gioco:
\begin{itemize}
\item Single player;
\item Multiplayer.
\end{itemize}  
Il fulcro del progetto si basa sullo sviluppo della parte multiplayer, affinché si possa giocare in sessioni da 2 a 6 giocatori in modalità fantasma, ossia potendo vedere i movimenti degli avversari ma senza poter interagire con essi.
Riguardo il single player, è necessario creare una modalità "infinita", con livelli di difficoltà crescente, senza interruzioni tra un livello e il successivo.\\
L'azienda proponente è \textit{Zero12}. 

\subsection{Tecnologie coinvolte}
L'azienda suggerisce l'utilizzo di tecnologie differenti a seconda della componente da implementare:
\begin{itemize}
\item Tecnologie \glo{AWS} per la parte server-side, lasciando al fornitore l'analisi sul servizio più adatto per la realizzazione di giochi realtime (per esempio \glo{AWS GameLift}, \glo{AWS Appsync}). \\
Nel caso dell'utilizzo di servizi che richiedano lo sviluppo di codice è vincolato l'uso del linguaggio \glo{Node.js};
\item \glo{DynamoDB} come database \glo{NoSQL} per il salvataggio delle classifiche e informazioni a supporto dell'applicativo; 
\item Tecnologie native per la realizzazione della piattaforma mobile. \\ 
Il fornitore ha la possibilità di scegliere tra \glo{iOS} (preferenza del proponente) o \glo{Android} come sistema operativo dove sviluppare l'intefaccia mobile. \\
A seconda del sistema operativo i linguaggi da utilizzare sono:
\begin{itemize}
\item \glo{Swift} e framework \glo{SwiftUI} (engine \glo{SceneKit}/\glo{SpriteKit}) con target minimo iOS 13;
\item \glo{Kotlin} con target minimo Android 8.
\end{itemize}
\end{itemize}

\subsection{Vincoli}
I vincoli imposti dall'azienda proponente sono:
\begin{itemize}
\item Effettuare un'analisi preliminare circa la tecnologia da utilizzare per la componente server. Questa analisi deve essere documentata, giustificando la scelta finale. Il team di \textit{Zero12} si offre in aiuto al fornitore in caso di perplessità a riguardo;
\item \glo{API} sviluppate in \glo{Node.js};
\item Architettura di progetto basata a micro-servizi, ossia a tante funzioni base indipendenti, utilizzabili singolarmente senza compromettere la funzionalià delle altre;
\item Sviluppo della mobile app in ambiente \glo{iOS} o \glo{Android} per testare tutti i micro-servizi e le dinamiche gaming in realtime. 
\end{itemize}

\subsection{Aspetti positivi}
\begin{itemize}
\item Utilizzo di tecnologie innovative e molto usate nell'ambito lavorativo;
\item Possibilità di osservare i risultati di test dinamici direttamente nell'interfaccia di gioco.
\end{itemize}

\subsection{Aspetti critici}
\begin{itemize}
\item Il capitolato presenta un progetto molto specifico rispetto ad altri. L'ambito di sviluppo è quello gaming e perché si lavori al meglio deve essere d'interesse a tutti i membri del gruppo;
\item Le tecnologie consigliate sono nuove a tutti i membri del gruppo e richiedono quindi uno studio preliminare approfondito per essere utilizzate correttamente;
\item I vincoli imposti risultano essere per lo più generali e se da un lato potrebbe essere un bene per la creatività concessa al gruppo, dall'altro risulta essere anche un rischio durante lo sviluppo. 
\end{itemize}

\subsection{Conclusioni}
Per la principale causa del generale disinteresse riguardo il game development, è stato deciso di orientarsi verso un altro capitolato che potesse creare stimoli e curiosità a tutti i membri del gruppo.
