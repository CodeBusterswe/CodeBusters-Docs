\section{Informazioni generali}
\begin{itemize}
\item \textbf{Motivo della riunione}: il gruppo si è riunito per conoscenza reciproca e per creare dei primi canali di comunicazione, il nome del gruppo e dei possibili loghi come simbolo distintivo. Si sono esposti anche i pareri in merito ai capitolati presentati.
\item \textbf{Luogo riunione}: videoconferenza tramite server \glo{Discord}.
\item \textbf{Data}: 27/10/2020
\item \textbf{Durata}: 1 ora e 45 minuti.
\item \textbf{Partecipanti}:
	\begin{itemize}
	\item Baldisseri Michele
	\item Sassaro Giacomo
	\item Zenere Marco
	\item Scialpi Paolo
	\item Rago Alessandro
	\item Pirolo Alessandro
	\item Safdari Hossaine
	\end{itemize}
\end{itemize}
\newpage
\section{Resoconto}
\begin{itemize}
\item \textbf{Strumenti collaborativi}: il gruppo, in seguito ad una serie di proposte, ha deciso di utilizzare come canale di comunicazione interno \glo{Discord}, in modo da effettuare chiamate vocali o videochiamate collettive. Per comodità si è creato anche un gruppo \glo{Telegram}, poichè già utilizzato da tutti i membri in precedenza.\\ \'E stato in seguito creato l'indirizzo e-mail \href{mailto:codebusterswe@gmail.com}{\color{cyan}codebusterswe@gmail.com}, accessibile da tutti i membri, e di utilizzare \glo{GitHub} come strumento di versionamento.

\item \textbf{Nome e logo del gruppo}: ciascun membro ha esposto dei possibili nomi e loghi per il gruppo e la scelta è stata effettuata per votazione. Giacomo Sassaro si è proposto come volontario per la creazione del logo.

\item \textbf{Discussione dei capitolati}: in seguito ad un precedente studio individuale, tutti i membri hanno esposto le loro opinioni in merito ai capitolati presentando pregi e possibili problemi. Sempre per votazione, è stata creata una classifica in base al numero delle preferenze e sono stati selezionati i primi tre capitolati come punto di partenza. Il capitolato provvisorio che ha riscontrato maggiore interesse è il n°5 con il progetto \textit{PORTACS}, proposto dall'azienda \textit{San Marco Informatica}.
\end{itemize}