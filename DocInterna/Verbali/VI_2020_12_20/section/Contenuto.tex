\section{Informazioni generali}
\begin{itemize}
\item \textbf{Motivo della riunione}: alcuni componenti del gruppo si sono riuniti per organizzarsi e risolvere dei dubbi riguardo al documento \AdR.
\item \textbf{Luogo riunione}: videoconferenza tramite server \glo{Discord}.
\item \textbf{Data}: \Data{}
\item \textbf{Ore}: 13:30 - 16:00.
\item \textbf{Partecipanti}:
	\begin{itemize}
	\item \BM{}
	\item \SG{}
	\item \SP{}
	\item \SH{}
	\item \PA{}
	\end{itemize}
\end{itemize}
\newpage
\section{Resoconto}
\begin{itemize}
\item \textbf{Analisi dei Requisiti}: la maggior parte della riunione si è concentrata sulla discussione dei casi d'uso e dei requisiti per continuare la stesura dell'\AdRv{}.
Gli analisti hanno preso in considerazione il \glo{capitolato} cercando una visione collettiva di quel che sarà il prodotto finale, con l'obbiettivo di individuare le funzionalità che devono essere rese a disposizione dell'utente ed estrarre una lista di casi d'uso, da integrare con quelli individuati in precedenza. Di conseguenza, sono stati individuati diversi requisiti riportati nel registro dei tracciamenti.\\
Durante la discussione sono sorti alcuni dubbi riguardo a certi aspetti all'interno del capitolato dunque, si è deciso di contattare l'azienda in modo da chiarirli;

\item \textbf{Suddivisione dei lavori}: i membri presenti hanno suddiviso tra loro le attività individuate aggiungendo, al proprio calendario, una data limite per il completamento. In questo modo nell'incontro successivo i lavori saranno ultimati e sarà possibile nuovamente fare il punto della situazione.

\end{itemize}

 