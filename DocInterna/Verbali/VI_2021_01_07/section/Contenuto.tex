\section{Informazioni generali}
\begin{itemize}
\item \textbf{Motivo della riunione}: Punto della situazione sulla stesura, verifica e approvazione dei documenti. Gestione dei prossimi giorni e chiarimento dubbi riguardo il progetto.
\item \textbf{Luogo riunione}: videoconferenza tramite server \glo{Discord}.
\item \textbf{Data}: 07-01-2021
\item \textbf{Ore}: 14:00 - 15:30.
\item \textbf{Partecipanti}:
	\begin{itemize}
	\item \BM{}
	\item \SG{}
	\item \SP{}
	\item \SH{}
	\item \PA{}
	\item \ZM{}
	\item \RA{}
	\end{itemize}
\end{itemize}

\newpage
\section{Resoconto}
\begin{itemize}
\item \textbf{Situazione attuale della stesura dei documenti}: il gruppo si è riunito per confrontarsi sull'avanzamento del processo di stesura, verifica e approvazione dei documenti. La situazione attuale è nel complesso buona. Tutti i documenti sono, dal punto di vista del contenuto, completi. Ciò che si è notato è la presenza di alcune difformità rispetto alle norme imposte nelle \NdPv{}, riguardo:
	\begin{itemize}
		\item Formato della data;
		\item Posizionamento delle "caption" delle tabelle e colori utilizzati;
		\item Posizionamento della 'G' (glossario) sempre ad apice;
		\item Sezione riferimenti di alcuni documenti.
	\end{itemize} 
È stata quindi scritta una lista più dettagliata sulle cose da controllare, rivedere e in caso adattare per uniformare tutti i documenti. Sarà compito dei \glo{verificatori} di ciascun documento riscontrare che non siano presenti difformità. \\
Il gruppo s'impegnerà a consultare sempre quanto redatto nelle \NdPv{} per evitare nuovamente discrepanze future;

\item \textbf{Pianificazione fino alla consegna}: a questo punto il gruppo si è organizzato per i giorni seguenti, fino all'effettiva consegna alla \textit{Revisione dei Requisiti}:
	\begin{enumerate}
		\item Inserire il consuntivo finale nel \PdPv{};
		\item Controllare che tutti i termini segnati siano correttamente inseriti nel \textit{Glossario};
		\item Calcolo dell'\glo{indice di Gulpease} dei documenti prodotti e aggiornamento dei grafici nel \PdQv{};
		\item Per l' 08-01-2021 i verificatori devono aver eseguito tutti i controlli stilati;
		\item Gli approvatori entro il giorno 09-01-2021 devono aver approvato tutti i documenti;
		\item Il 10-01-2021, ad approvazioni effettuate, verrà eseguito il \glo{merge} di ogni \glo{branch} nel \glo{repository} di lavoro.
	\end{enumerate}
	
\item \textbf{E-mail al \glo{proponente}}: il gruppo ha poi deciso di scrivere una e-mail all'azienda proponente per aggiornarla sulla situazione attuale e per alcuni dubbi che riguardano la revisione successiva. L'e-mail risulta essere inoltre un modo per mostrare all'azienda il continuo interesse del gruppo al progetto e la volontà di mantenere un'interazione costante.
\end{itemize}
\newpage