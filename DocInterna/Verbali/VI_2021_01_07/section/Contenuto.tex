\section{Informazioni generali}
\begin{itemize}
\item \textbf{Motivo della riunione}: Punto della situazione sulla stesura, verifica e approvazione dei documenti. Gestione dei prossimi giorni e chiarimento dubbi riguardo il progetto.
\item \textbf{Luogo riunione}: videoconferenza tramite server \glo{Discord}.
\item \textbf{Data}: 07-01-2021
\item \textbf{Durata}: 1 ora e 30 minuti.
\item \textbf{Partecipanti}:
	\begin{itemize}
	\item \BM{}
	\item \SG{}
	\item \SP{}
	\item \SH{}
	\item \PA{}
	\item \ZM{}
	\item \RA{}
	\end{itemize}
\end{itemize}

\newpage
\section{Resoconto}
\begin{itemize}
\item \textbf{Situazione attuale della stesura dei documenti}: il gruppo si è riunito per confrontarsi sull'avanzamento del processo di stesura, verifica e approvazione dei documenti. La situazione attuale è nel complesso buona. Tutti i documenti sono, dal punto di vista del contenuto, completi. Ciò che si è notato è la presenza di alcune difformità rispetto alle norme imposte nelle \NdPv{}, riguardo:
	\begin{itemize}
		\item formato della data;
		\item posizionamento delle "caption" delle tabelle e colori utilizzati per le righe;
		\item posizionamento della 'G' (glossario) sempre ad apice (il comando utilizzato per inserirla, per motivo sconosciuto, a volte la mette come pedice);
		\item sezione riferimenti di alcuni documenti.
	\end{itemize} 
È stata quindi scritta una lista più dettagliata sulle cose da controllare, rivedere e in caso adattare. Sarà compito dei verificatori di ciascun documento riscontare che non siano presenti difformità. 

\item \textbf{Pianificazione fino alla consegna}: a questo punto il gruppo si è organizzato per i giorni seguenti, fino all'effettiva consegna alla \textit{Revisione dei Requisiti}. 
	\begin{enumerate}
		\item Per l' 8-01-2021 i verificatori devono aver eseguito tutti i controlli stilati;
		\item Gli approvatori hanno fino al 09-01-2021 per approvare tutti i documenti.
		\item Il 10-01-2021, ad approvazioni effettuate, verrà eseguito il \glo{merge} di ogni \glo{branch} nel ramo develop. Solo a questo punto si potrà "mergiare" il develop nel main.
	\end{enumerate}
	
\item \textbf{Email al proponente}: il gruppo si è poi soffermato su dubbi persistenti nella comprensione del progetto in tutte le sue parti. Si è quindi deciso di scrivere una email all'azienda proponente con delle domande specifiche riguardo:
	\begin{itemize}
		\item lingua dei manuali da redarre in futuro (solo italiana o anche inglese);
		\item come deve avvenire la gestione del \glo{database} da parte dell'amministratore (attraverso finestre specifiche nell'applicazione o direttamente da pannello di controllo del DB).
	\end{itemize}
L'email risulta essere inoltre un modo rapido di far capire il continuo interesse del gruppo al progetto e la volontà di mantenere un'interazione costante con l'azienda.
\end{itemize}
\newpage