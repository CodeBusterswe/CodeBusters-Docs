\section{Informazioni generali}
\begin{itemize}
\item \textbf{Motivo della riunione}: 
\begin{itemize}
\item Punto della situazione sui documenti e sul PoC;
\item Decisione di utilizzare la distanza euclidea;
\item Cambiamento della modalità di versionamento;
\item Requisiti prestazionali;
\item Programmazione delle attività da svolgere e divisione dei ruoli;
\item Valutazioni per il miglioramento.
\end{itemize}
\item \textbf{Luogo riunione}: videoconferenza tramite server \glo{Discord}.
\item \textbf{Data}: 23-02-2021
\item \textbf{Orario d'inizio}: 09:00;
\item \textbf{Orario d'inizio}: 10:20;
\item \textbf{Partecipanti}:
	\begin{itemize}
	\item \BM{}
	\item \SG{}
	\item \SP{}
	\item \SH{}
	\item \PA{}
	\item \ZM{}
	\item \RA{}
	\end{itemize}
\end{itemize}

\newpage
\section{Resoconto}
\begin{itemize}
\item \textbf{Proof of Concept}: raggiunto l'incremento I, il gruppo è a buon punto anche nella realizzazione dell'incremento II. Infatti è già possibile visualizzare i dati in uno \glo{Scatter Plot Matrix}. Non sono stati ancora aggiunti dei controlli per la configurazione dei parametri. Si dovrà inoltre effettuare qualche modifica del layout grafico per rendere il prodotto più godibile visivamente. 

\item\textbf{Documentazione}: l'\textit{Analisi dei Requisiti} è stata incrementata con nuovi grafici. Nel \textit{Piano di Qualifica} sono stati aggiunti dei grafici al cruscotto, dovranno essere aggiornati appena terminerà l'incremento II.

\item\textbf{Distanza euclidea}: è stato deciso di aggiungere la distanza euclidea alla lista degli algoritmi che possono essere applicati ai dataset per la creazione di nuove dimensioni.

\item\textbf{Versionamento}: il gruppo ha deciso di modificare il metodo di versionamento dei documenti, saranno dunque apportati dei cambiamenti ai registri delle modifiche di ciascun documento così da rendere più ordinato ed esplicativo il cambio di versione.

\item \textbf{Requisiti prestazionali}: a fronte della verifica di tempi di attesa lunghi per alcuni dataset, si è deciso di stabilire dei requisiti prestazionali riguardanti i tempi di risposta.

\item \textbf{Assegnazione dei ruoli}: il responsabile ha poi comunicato la nuova suddivisione dei ruoli. Si è deciso come suddividere i lavori da completare e sono state assegnate le \glo{tasks} per il prossimo incremento.

\item\textbf{Valutazioni per il miglioramento}: il gruppo ha svolto un'autovalutazione riguardo al lavoro svolto finora e le problematiche che si sono riscontrate. Per ogni problema si è dunque cercato un possibile modo per migliorare la situazione evitando che ciò possa riaccadere in futuro, come illustrato nelle seguenti tabelle.

\renewcommand{\arraystretch}{1.5}
\centering
\begin{longtable}{C{2.5cm} C{6cm} C{5cm}}
\rowcolor{coloreRosso}
\textcolor{white}{\textbf{Problema}} &
\textcolor{white}{\textbf{Descrizione}} &	
\textcolor{white}{\textbf{Soluzione}} \\	
\endhead

Covid-19 &
A causa della situazione di emergenza sanitaria dovuta al virus \textit{Covid-19}, fin dall'inizio gli incontri tra i membri si sono dovuti svolgere in modalità virtuale.&
Il gruppo continuerà ad usare \textit{Discord} e \textit{Telegram} per tenersi costantemente aggiornati sull'andamento dei lavori. \\

Esami universitari & 
Alcuni componenti del gruppo hanno dovuto svolgere esami universitari. &
Il gruppo si è organizzato in modo da redistribuire equamente il carico di lavoro. \\

Analisi dei Requisiti & 
Non era stata svolta un'\AdR{} dettagliata. &
Il gruppo ha studiato gli algoritmi di riduzione dimensionale da inserire. \\

\rowcolor{white}
 \caption{Valutazioni sull'organizzazione}
\end{longtable}

\renewcommand{\arraystretch}{1.5}
\centering
\begin{longtable}{C{2.5cm} C{6cm} C{5cm}}
\rowcolor{coloreRosso}
\textcolor{white}{\textbf{Problema}} &
\textcolor{white}{\textbf{Descrizione}} &	
\textcolor{white}{\textbf{Soluzione}} \\	
\endhead

\glo{React Hooks} e \glo{D3.js} & 
Riscontrati problemi nel passaggio a \glo{React Hooks} e nella nuova versione della libreria \glo{D3.js}. &
Tutto il gruppo si è adoperato nel trovare informazioni condividendo i risultati con gli altri membri.  \\

\glo{Poc} e \glo{GitHub Actions} & 
Sono state trovate difficoltà nell'ambiente di sviluppo del \glo{Poc} e per le \glo{GitHub Actions}. &
Tutto il gruppo si è adoperato nel trovare informazioni condividendo i risultati con gli altri membri.   \\

\rowcolor{white}
 \caption{Valutazioni sugli strumenti usati}
\end{longtable}

\end{itemize}

\newpage