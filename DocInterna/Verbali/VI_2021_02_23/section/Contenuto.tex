\section{Informazioni generali}
\begin{itemize}
\item \textbf{Motivo della riunione}: 
\begin{itemize}
\item Punto della situazione sui documenti e sul PoC;
\item Decisione di utilizzare la distanza euclidea;
\item Cambiamento della modalità di versionamento;
\item Requisiti prestazionali;
\item Programmazione delle attività da svolgere e divisione dei ruoli;
\item Valutazioni per il miglioramento.
\end{itemize}
\item \textbf{Luogo riunione}: videoconferenza tramite server \glo{Discord}.
\item \textbf{Data}: 23-02-2021
\item \textbf{Orario d'inizio}: 09:00;
\item \textbf{Orario d'inizio}: 10:20;
\item \textbf{Partecipanti}:
	\begin{itemize}
	\item \BM{}
	\item \SG{}
	\item \SP{}
	\item \SH{}
	\item \PA{}
	\item \ZM{}
	\item \RA{}
	\end{itemize}
\end{itemize}

\newpage
\section{Resoconto}
\begin{itemize}
\item \textbf{Proof of Concept}: il gruppo è a buon punto nella realizzazione dell'incremento II ed è già possibile visualizzare i dati in uno \glo{Scatter Plot Matrix}. Il prossimo obbiettivo è l'aggiunta dei controlli per la configurazione di alcuni parametri e della visualizzazione. Si dovrà inoltre effettuare qualche modifica del layout grafico per rendere il prodotto più ordinato e godibile visivamente. 

\item\textbf{Documentazione}: poiché l'attività di verifica dell'\textit{Analisi dei Requisiti} è quasi ultimata, verranno aggiunti i diagrammi dei nuovi casi d'uso. Anche coloro che stanno lavorando nel \textit{Piano di Qualifica} nei giorni a seguire dovranno iniziare ad aggiornare i grafici delle attività di verifica.

\item\textbf{Versionamento}: il gruppo ha deciso di modificare il metodo di versionamento dei documenti e del software, procedendo con la registrazione di sole versioni già verificate e garantendo così un prodotto sempre corretto e funzionante. Saranno dunque apportati dei cambiamenti ai registri delle modifiche di ciascun documento ed aggiornata l'apposita sezione delle \textit{Norme di Progetto}, così da rendere più ordinato ed esplicativo il cambio di versione.

\item \textbf{Requisiti prestazionali}: vista la notevole dimensione di alcuni dataset si è deciso di stabilire dei requisiti prestazionali riguardanti i tempi di risposta. Poiché l'azienda proponente ha messo a disposizione un dataset per fare dei test, il gruppo ha avuto modo di fare alcuni tentativi di possibile utilizzo con le funzionalità finora implementate. Grazie all'utilizzo di alcuni strumenti tutti i dati necessari sono stati registrati per integrare l'\textit{Analisi dei Requisiti}.

\item\textbf{Valutazioni per il miglioramento}: visto l'imminente termine della revisione di progettazione, il gruppo ha svolto un'autovalutazione riguardo al lavoro svolto finora e alle problematiche che si sono riscontrate. Per ogni problema si è dunque riflettuto su come è stato affrontato e su qualche altra possibile soluzione, in modo da evitare rallentamenti in futuro.

\renewcommand{\arraystretch}{1.5}
\centering
\begin{longtable}{C{3cm} C{6cm} C{6cm}}
\rowcolor{coloreRosso}
\textcolor{white}{\textbf{Problema}} &
\textcolor{white}{\textbf{Descrizione}} &	
\textcolor{white}{\textbf{Soluzione}} \\	
\endhead

Covid-19 &
Visto l'andamento della situazione di emergenza sanitaria gli incontri tra i membri si dovranno svolgere ancora in modalità virtuale.&
Il gruppo continuerà ad usare \textit{Discord} e \textit{Telegram} per tenersi costantemente aggiornato sull'andamento dei lavori. \\

Impegni universitari & 
Alcuni componenti del gruppo hanno dovuto svolgere esami universitari e dovranno seguire nuovi corsi. &
Il gruppo farà in modo di garantire una corretta distribuzione del carico di lavoro, evitando situazioni di stallo. \\

Documentazione & 
La stesura di alcuni documenti effettuata nel periodo precedente non era dettagliata. &
Presa visione dell'esito, il gruppo ha studiato maggiormente e ha capito il livello di approfondimento da raggiungere. \\

\rowcolor{white}
 \caption{Valutazioni sull'organizzazione}
\end{longtable}

\renewcommand{\arraystretch}{1.5}
\centering
\begin{longtable}{C{2.5cm} C{6cm} C{5cm}}
\rowcolor{coloreRosso}
\textcolor{white}{\textbf{Problema}} &
\textcolor{white}{\textbf{Descrizione}} &	
\textcolor{white}{\textbf{Soluzione}} \\	
\endhead

\glo{React} e \glo{D3.js} & 
Riscontrati problemi nell'apprendere le recenti funzionalità di React e quelle offerte dalla nuova versione della libreria D3.js. &
Tutto il gruppo si è adoperato nel trovare informazioni e a condividerle. Ciascun componente dovrà comunicare e rendere disponibile materiale utile individuato.  \\

\glo{Poc} e \glo{GitHub Actions} & 
Riscontrate difficoltà nel configurare l'ambiente di sviluppo del PoC e nell'uso delle GitHub Actions. &
I programmatori si sono dedicati alla risoluzione immediata del problema e s'impegnano nel perseguire anche in futuro la \glo{\textit{continuos integration}}.  \\

\rowcolor{white}
 \caption{Valutazioni sugli strumenti usati}
\end{longtable}

\end{itemize}

