\section*{C}
\markright{}
\addcontentsline{toc}{section}{C}
\subsection*{Cloud}
Vasta rete di server remoti ubicati in tutto il mondo, collegati tra loro e che operano come un unico ecosistema. Questi server possono archiviare e gestire dati, eseguire applicazioni o distribuire contenuti o servizi. L'accesso avviene online, da qualsiasi dispositivo con connessione Internet.
\subsection*{CloudFormation}
Strumento di Amazon Web Service che permette di modellare una raccolta di risorse AWS e gestirle nell'intero arco di ciclo di vita del prodotto.
\subsection*{Cluster}
Un cluster è un insieme di oggetti che presentano tra loro delle similarità, ma che, per contro, presentano delle dissimilarità con oggetti in altri cluster.
\subsection*{Code folding}
\'E una caratteristica di alcuni editor di testo e ambienti di sviluppo. Permette di nascondere delle porzioni di un file di codice mentre si lavora ad altre parti dello stesso file. Ciò permette agli sviluppatori di gestire più comodamente file molto lunghi all'interno di un'unica finestra.
\subsection*{CSS}
Acronimo di Cascading Style Sheets. Linguaggio usato per definire la formattazione di documenti HTML, XHTML e XML, ad esempio i siti web e relative pagine web.
\subsection*{CSV}
Acronimo di comma-separated values. Formato di file basato su file di testo utilizzato per l'importazione ed esportazione (ad esempio da fogli elettronici o database) di una tabella di dati. 
