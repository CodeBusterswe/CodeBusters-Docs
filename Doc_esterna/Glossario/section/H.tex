\section*{H}
\markright{}
\addcontentsline{toc}{section}{H}
\subsection*{Heat map}
Il grafico “Heat map” trasforma la distanza tra i punti i colori più o meno intensi, facendo così capire quali oggetti sono vicini tra loro e quali sono distanti. Per una buona visualizzazione è utile accompagnare la costruzione del grafico con l’ordinamento dei dati in modo che le strutture presenti siano evidenziate, inoltre, fatta questa operazione, è possibile associare un “dendrogramma” lungo i bordi della mappa. Questo grafico e la relativa operazione di ordinamento è facilmente reperibile tra gli esempi della libreria D3.
\subsection*{HTML}
Acronimo di HyperText Markup Language. Linguaggio di markup per la formattazione e impaginazione di documenti ipertestuali disponibili nel web.
\subsection*{HTTP}
Acronimo per HyperText Transfer Protocol. Protocollo a livello applicativo usato come principale sistema per la trasmissione d'informazioni sul web in una tipica architettura client-server.