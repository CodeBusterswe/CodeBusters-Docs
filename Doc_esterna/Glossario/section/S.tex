\section*{S}
\markright{}
\addcontentsline{toc}{section}{S}
\subsection*{SceneKit}
Framework a supporto del linguaggio di programmazione Swift che permette di costruire grafiche 3D per applicazioni \glo{iOS} e MacOS. Permette di inserire animazioni e comportamenti fisici realistici alle componenti dell'interfaccia dell'applicazione. 
\subsection*{Scikit-learn}
Libreria open source di apprendimento automatico per il linguaggio di programmazione Python. Contiene algoritmi di classificazione, regressione e clustering, macchine a vettori di supporto, regressione logistica, classificatore bayesiano, k-mean e DBSCAN. \'E progettato per operare con le librerie NumPy e SciPy. 
\subsection*{SpriteKit}
Framework a supporto del linguaggio di programmazione Swift che permette di costruire grafiche 2D per applicazioni \glo{iOS} e MacOS. Permette di animare figure, immagini, testi in ambienti a due dimensioni.
\subsection*{Serverless}
Si intende un network la cui gestione non viene incentrata su dei server, come spesso accade, ma viene dislocata fra i vari utenti che utilizzano il network stesso.
\subsection*{Swift}
Linguaggio di programmazione object-oriented (OOP) sviluppato da Apple che permette di creare nuove applicazioni \glo{iOS} e MacOS.  
\subsection*{SwiftUI}
Framework a supporto del linguaggio di programmazione Swift. Facilita lo sviluppo della User Interface (UI) per applicazioni \glo{iOS} e MacOS.