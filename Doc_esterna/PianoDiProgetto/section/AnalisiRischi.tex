\section{Analisi dei rischi}
Durante lo svolgimento di un progetto di una certa complessità bisogna fare molta attenzione ai possibili rischi a cui il gruppo può andare in contro. Questi possono avere conseguenze particolarmente negative sul lavoro e sul rispetto delle scadenze e risulta quindi necessaria un'attenta analisi, volta alla loro mitigazione.\\Questa attività richiede attenzione costante e ha l'obbiettivo di fare delle previsioni sui problemi che si potrebbero verificare durante tutto il percorso, classificandoli in base alla loro entità e apportando delle risoluzioni.\\
Di seguito è riportata una tabella che riassume tutte le informazioni, realizzata con le seguenti fasi:
\begin{itemize}
\item \textbf{Identificazione:} attività che permette d'individuare gli eventi che potrebbero causare problemi durante l'avanzamento del progetto;
\item \textbf{Analisi:} studio degli eventi rilevati ed assegnazione di un indice di gravità e di una probabilità di occorrenza, rilevando così l'impatto che avrebbero nel progetto;
\item \textbf{Controllo:} pianificazione di una metodologia per evitare che
si verifichino i rischi individuati, stabilendo come agire nel caso in cui questi si riscontrassero;
\item \textbf{Monitoraggio:} durante lo svolgimento del progetto viene costantemente eseguito un controllo per:
	\begin{itemize}
		\item rilevare eventuali nuovi indicatori di rischio;
		\item aggiornare le informazioni già presenti.
	\end{itemize}
Questo fase risulta essere fondamentale perché con il tempo gli effetti sui rischi possono variare ed è necessario riportarli periodicamente all'attenzione di tutto il gruppo.
\end{itemize}

\subsection{Categorie}
I rischi sono stati suddivisi nelle seguenti categorie:
\begin{itemize}
\item \textbf{Tecnologie scelte}
\item \textbf{Rapporti interpersonali}
\item \textbf{Organizzazione del lavoro}
\end{itemize}
I rischi sono identificati dal seguente codice:
\begin{center}
	\textbf{R\{Iniziale categoria\}\{Numero progressivo\}}
\end{center}

\subsubsection{Rischi tecnologici}
\rowcolors{2}{coloreGrigietto}{white}
\begin{longtable}{C{3cm} C{4cm} C{4cm} C{4cm}}
		\rowcolor{coloreRosso}
		\textcolor{white}{\textbf{Codice}} & 
		\textcolor{white}{\textbf{Descrizione}} & 
		\textcolor{white}{\textbf{Identificazione}} & 
		\textcolor{white}{\textbf{Grado}} \\
		\endfirsthead
	    \multicolumn{4}{|c|}{\textit{Continua nella pagina successiva...}}\\
	    \endfoot
	    \endlastfoot

RT1 \newline Scarsa esperienza &
Alcune delle tecnologie da adottare nello sviluppo del progetto sono nuove per tutti i membri del gruppo, di conseguenza potrebbero insorgere problemi operativi. & 
Il \textit{Responsabile} avrà il compito di rilevare conoscenze ed eventuali lacune dei vari componenti del team. Ciascun membro del gruppo inoltre provverrà a comunicare in assoluta trasparenza eventuali difficoltà. & 
Occorrenza \\

\multicolumn{4}{c}{\parbox{16cm}{\textbf{Piano di contingenza:} I compiti più onerosi, o che richiedono maggiori conoscenze tecnologiche, I compiti più onerosi, o che richiedono maggiori conoscenze tecnologiche} } \\

a & b & c & d \\

\multicolumn{4}{c}{Piano aaaaaaaaaaaaaaaa} \\

a & b & c & d \\

\multicolumn{4}{c}{Piano aaaaaaaaaaaaaaaa} \\


\end{longtable}