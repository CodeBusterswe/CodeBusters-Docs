\section{Analisi dei rischi}
Durante lo svolgimento di un progetto di una certa complessità bisogna fare molta attenzione ai possibili rischi a cui il gruppo può andare in contro. Questi possono avere conseguenze particolarmente negative sul lavoro e sul rispetto delle scadenze e risulta quindi necessaria un'attenta analisi, volta alla loro mitigazione.\\Questa attività richiede attenzione costante e ha l'obbiettivo di fare delle previsioni sui problemi che si potrebbero verificare durante tutto il percorso, classificandoli in base alla loro entità e apportando delle risoluzioni.\\
Di seguito è riportata una tabella che riassume tutte le informazioni, realizzata con le seguenti fasi:
\begin{itemize}
\item \textbf{Identificazione:} attività che permette d'individuare gli eventi che potrebbero causare problemi durante l'avanzamento del progetto;
\item \textbf{Analisi:} studio degli eventi rilevati ed assegnazione di un indice di gravità e di una probabilità di occorrenza, rilevando così l'impatto che avrebbero nel progetto;
\item \textbf{Controllo:} pianificazione di una metodologia per evitare che
si verifichino i rischi individuati, stabilendo come agire nel caso in cui questi si riscontrassero;
\item \textbf{Monitoraggio:} durante lo svolgimento del progetto viene costantemente eseguito un controllo per:
	\begin{itemize}
		\item rilevare eventuali nuovi indicatori di rischio;
		\item aggiornare le informazioni già presenti.
	\end{itemize}
Questo fase risulta essere fondamentale perché con il tempo gli effetti sui rischi possono variare ed è necessario riportarli periodicamente all'attenzione di tutto il gruppo.
\end{itemize}

\subsection{Categorie}
I rischi sono stati suddivisi nelle seguenti categorie:
\begin{itemize}
\item \textbf{Tecnologie scelte}
\item \textbf{Rapporti interpersonali}
\item \textbf{Organizzazione del lavoro}
\end{itemize}
I rischi sono identificati dal seguente codice:
\begin{center}
	\textbf{R\{Iniziale categoria\}\{Numero progressivo\}}
\end{center}

\subsubsection{Rischi relativi alle tecnologie}
\rowcolors{1}{coloreGrigietto}{colorePanna}
\begin{longtable}{C{2cm} C{5cm} C{5cm} C{3cm}}
		\rowcolor{coloreRosso}
		\textcolor{white}{\textbf{Codice}} & 
		\textcolor{white}{\textbf{Descrizione}} & 
		\textcolor{white}{\textbf{Identificazione}} & 
		\textcolor{white}{\textbf{Grado}} \\
		\endfirsthead
	    \multicolumn{4}{|c|}{\textit{Continua nella pagina successiva...}}\\
	    \endfoot
	    \endlastfoot

%--------------------------------------------
\textbf{RT1} \newline Scarsa esperienza &

Tutti i membri del gruppo non hanno ancora un'esperienza tale da affrontare un progetto di questa complessità senza l'insorgere di problemi operativi. & 

Ciascun membro del gruppo deve comunicare con assoluta trasparenza eventuali difficoltà incontrate. Il \textit{Responsabile} ha il compito di rilevare le varie lacune e favorire la condivisione delle conoscenze tra i componenti del team.  & 

\parbox{2.2cm}{
\begin{center}
Pericolosità: \textbf{Elevata} \newline Occorrenza: \textbf{Elevata} 
\end{center} } \\

\multicolumn{4}{c}{\parbox{16cm}{\textbf{Piano di contingenza:} I compiti con difficoltà maggiore verranno assegnati a più componenti, in modo da favorire l'assistenza reciproca.} } \\

%--------------------------------------------
\textbf{RT2} \newline Tecnologie da usare &

La documentazione disponibile per l'utilizzo delle tecnologie interessate è molto approfondita. Il tempo di apprendimento potrebbe causare dei ritardi nello svolgimento dei lavori. & 

Il \textit{Responsabile} ha il compito di monitorare la preparazione dei membri rispetto ai compiti assegnati.  & 

\parbox{2.2cm}{
\begin{center}
Pericolosità: \textbf{Elevata} \newline Occorrenza: \textbf{Media} 
\end{center} } \\

\multicolumn{4}{c}{\parbox{16cm}{\textbf{Piano di contingenza:} In casi di particolare difficoltà è prevista una ridistribuzione del carico di lavoro.} } \\

%--------------------------------------------
\textbf{RT3} \newline Strumenti software &

Il team si affida a strumenti software di terze parti e piattaforme online. Potrebbe esserci il rischio di perdita di dati o non operatività. & 

Qualsiasi membro ha il compito di avvisare il \textit{Responsabile} e gli altri componenti in caso di rilevamento di problemi.  & 

\parbox{2.2cm}{
\begin{center}
Pericolosità: \textbf{Media-Elevata} \newline Occorrenza: \textbf{Bassa} 
\end{center} } \\

\multicolumn{4}{c}{\parbox{16cm}{\textbf{Piano di contingenza:} effettuare un backup dei dati periodico su altre piattaforme.} } \\


%--------------------------------------------
\textbf{RT4} \newline Problemi hardware &

Tutti i componenti del gruppo utilizzano dispositivi personali per lavorare al progetto. Guasti hardware potrebbero causare notevoli disagi e perdite di tempo. & 

Ciascun membro dovrà, nei limiti del possibile, evitarli ed avvisare il \textbf{Responsabile} e gli altri componenti i problemi riscontrati.  & 

\parbox{2.2cm}{
\begin{center}
Pericolosità: \textbf{Media} \newline Occorrenza: \textbf{Bassa} 
\end{center} } \\

\multicolumn{4}{c}{\parbox{16cm}{\textbf{Piano di contingenza:} ogni componente deve rispettare l'utilizzo degli strumenti stabiliti nelle \textit{Norme di progetto} per evitare perdite di dati.} } \\
\end{longtable}