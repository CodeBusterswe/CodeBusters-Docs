\section{Modello di sviluppo}
Il gruppo ha deciso di utilizzare il modello incrementale.
\subsection{Modello incrementale}
Il modello incrementale prevede rilasci multipli e successivi, ciascuno di questi realizza un incremento di funzionalità.
É richiesta dunque una classificazione preliminare dei requisiti atta ad individuare i più importanti, i quali devono essere sviluppati nei primi incrementi, così da avere fin da subito un prodotto funzionante, che verrà via via integrato.
L'adozione di questo modello comporta i seguenti vantaggi:
\begin{itemize}
\item Le funzionalità primarie hanno priorità nello sviluppo, così facendo queste vengono verificate più volte;
\item L'avere un prodotto funzionante già dai primi incrementi permette subito al proponente di valutarne le funzioni primarie;
\item Ogni incremento riduce il rischio di fallimento, con un approccio predisposto ai cambiamenti;
\item L'analisi dei requisiti può essere raffinata tramite la progettazione di dettaglio ad ogni incremento;
\item Le modifiche e la correzione degli errori sono più economiche;
\item Le fasi di verifica e test sono facilitate perché mirate a un singolo incremento.
\end{itemize}
\subsection{Incrementi individuati}
In seguito è riportata la tabella con indicati gli incrementi individuati durante la fase di analisi con il rispettivo obiettivo e i requisiti ad esso associati.
I requisiti riportati nella tabella includono tutti i requisiti figli. Tutti i requisiti non riportati sono da intendersi soddisfatti, in parte, da ogni incremento.
Ogni requisito è individuato dal suo codice identificativo, reperibile nel documento \AdR
\begin{center}
Tabella 3.1: Tabella degli incrementi
\end{center}
\rowcolors{1}{coloreGrigietto}{colorePanna}
\begin{longtable}{C{3cm} C{5cm} C{3cm}}
\rowcolor{coloreRosso}
\textcolor{white}{\textbf{Incremento}} & 
\textcolor{white}{\textbf{Obiettivo dell'incremento}} & 
\textcolor{white}{\textbf{Requisiti}}\\
\endhead

%------------------------------------------
Incremento 0 & Caricamento dati & Requisito 1\\
Incremento 1 & Visualizzazione Scatter Plot Matrix & Requisito 2\\
Incremento 2 & Personalizzazione Scatter Plot Matrix & Requisito 3 \\
Incremento 3 & Visualizzazione Heat Map e Force Field & Requisito 4\\
Incremento 4 & Personalizzazione Heat Map e Force Field & Requisito 4a\\
Incremento 5 & Visualizzazione e Personalizzazione Proiezione Lineare Multi Asse & Requisito 5\\
Incremento 6 & Gestione della sessione & Requisito \\

\end{longtable}
