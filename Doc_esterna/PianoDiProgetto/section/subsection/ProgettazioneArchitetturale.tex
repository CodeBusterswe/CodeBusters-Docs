\subsection{Progettazione architetturale}
Questa fase comincia subito dopo la presentazione e finisce con la data di consegna per la Revisione di Progettazione, ovvero dal x/x/x all'x/x/x.\\
In questo periodo verrà individuata una soluzione architetturale che funga da sostegno per l'implementazione del prodotto. Deve soddisfare tutti i requisiti richiesti, oltre ad essere facilmente comprensibile ed attuabile. 
\begin{itemize}
\item \textbf{Incremento e verifica:} se fosse necessario, i documenti prodotti dal team verranno integrati.

 \item \glo{\textbf{Technology baseline}}: Viene fatta un'analisi ad alto livello per comprendere le tecnologie coinvolte, in due passi distinti:
\begin{itemize}
 \item \textbf{decomposizione del prodotto} in parti, in modo da essere realizzato con risorse sostenibili e costi compatibili;
 \item \textbf{analisi delle componenti individuate}, in modo da determinare come ciascuna interagisce con le altre.  
\end{itemize}
L'architettura verrà progettata sulla base di \glo{design patterns} esistenti e verrà realizzato un \glo{\textit{Proof of Concept}} da condividere con il proponente, in modo da verificare il corretto sviluppo del software. In particolare, quest'ultima attività riguarderà due incrementi:
\begin{itemize}
	\item \textbf{[codice incremento 1]}: questo periodo, che va dal x/x/x al xx(qualche giorno prima del secondo periodo in "periodi")/02/2020, ha l'obbiettivo d'implementare l'invio dei dati al \glo{back-end} con il formato richiesto. \textbf{[aggiungere riferimenti agli UC]}
	\item \textbf{[codice incremento 2]}: questo periodo, che va dal x/x/x all'xx(qualche giorno prima del secondo periodo in "periodi")/02/2020. si focalizza   sulla realizzazione dell'\glo{UI}, con la visualizzazione di un grafico di prova. \textbf{[aggiungere riferimenti agli UC]}
\end{itemize} 

\end{itemize}

\subsubsection{Periodi}
La pianificazione di questa fase è stata organizzata nel modo seguente:
\begin{itemize}
\item \textbf{18/01/2020 - xx/02/2020}: a causa della concomitanza con la sessione accademica, il team ha fissato la prima \glo{milestone} al termine di questo periodo. Alla scadenza, il gruppo dovrà aver iniziato lo studio delle tecnologie per la Technology baseline, oltre ad aver controllato buona parte della documentazione.

\item \textbf{xx/02/2020 - xx/02/2020}: per questa seconda milestone, il team s'impegnerà a terminare i lavori avviati nel periodo precedente. Inoltre l'obbiettivo è terminare il primo incremento previsto per il \textit{Proof of Concept}.

\item \textbf{xx/02/2020 - 08/03/2020}: per l'ultima milestone di questa fase, il gruppo prevede di terminare anche il secondo incremento relativo al \textit{Proof of Concept} e ultimato le verifiche.
\end{itemize}

GRAFICO