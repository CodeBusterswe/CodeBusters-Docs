\section{Introduzione}
\subsection{Scopo del Documento}
In questo documento verranno trattate le regole al fine di lavorare più efficientemente e in modo uniforme.
Di seguito verranno anche analizzati gli strumenti e come questi dovranno essere utilizzati in modo da rendere più armoniosa la collaborazione tra i membri.
\subsection{Scopo del Prodotto}
Il capitolato C4 ha come obiettivo quello di creare un'applicazione di visualizzazione di grandi quantità di dati con numerose dimensioni in un formato comprensibile dall'occhio umano. A questo scopo è necessario sviluppare algoritmi di intelligenza artificiale che, agendo sulla distanza dei vari punti del grafico, riescano a sviluppare un modello semplificato che ne evidenzi i \glo{cluster}. 
L'applicazione dovrà inoltre agire su questi grafici creati evidenziando i dati ottenuti.
\subsection{Glossario}
Al fine di evitare ambiguità fra i termini, e per avere chiare fra tutti gli stakeholder le terminologie utilizzate per la realizzazione del presente documento, il gruppo \Gruppo{} ha redatto un documento denominato \Glossariov{1.0.0}.
In tale documento, sono presenti tutti i termini tecnici, ambigui, specifici del progetto e scelti dai membri del gruppo con le loro relative definizioni.
Un termine presente nel \Glossariov{1.0.0} e utilizzato in questo documento viene indicato con un apice \ap{G} alla fine della parola.

\subsection{Riferimenti}

\subsubsection{Normativi}
\begin{itemize}
\item da completare.
\end{itemize}

\subsubsection{Informativi}

\begin{itemize}
\item \textbf {Piano di Progetto v 1.0.0}\\

\item \textbf {Piano di Qualifica v 1.0.0}\\

\item \textbf {Da completare}\\

\end{itemize}
