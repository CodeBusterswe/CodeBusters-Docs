\section{Processi Organizzativi}
\subsection{Gestione Organizzativa}
\subsubsection{Scopo}
In questa sezione vengono esposte le modalità di coordinamento adottate dal gruppo che regolano gli incontri(interni o esterni) e le comunicazioni.
\begin{itemize}
\item \textbf{comunicazione}: interna con i membri del gruppo, quella esterna con l'azienda;
\item \textbf{incontri}: incontri interni con in membri del gruppo, esterno con l'azienda.

\end{itemize}
\subsubsection{Aspettative}
Le attese, riguardo al processo in questione sono i seguenti: 
\begin{itemize}
\item ottenere una pianificazione ragionevole delle attività da seguire;
\item coordinamento dell'attività del gruppo, assegnando loro i ruoli, i compiti e semplificando la comunicazione tra loro;
\item adoperare processi per regolare le attività e renderle economiche;

\end{itemize}

\subsubsection{Descrizione}
Le attività di gestione sono: 
\begin{itemize}
\item assegnazione dei ruoli e dei compiti;
\item inizio e definizione dello scopo;
\item istanziazione dei processi;
\item pianificazione e stima di tempi, risorse e costi;
\item esecuzione e controllo;
\item revisione e valutazione periodica delle attività.
\end{itemize}

\subsection{Gestione organizzativa}
\subsubsection{Scopo}
In questa sezione vengono esposte le modalità di coordinamento adottate dal gruppo, che sono:
\begin{itemize}
\item Adottare un modello organizzativo per l'individuazione dei rischi che potrebbero verificarsi;
\item Definire un modello di sviluppo da adottare;
\item Pianificare il lavoro rispettando le scadenze;
\item Calcolo del prospetto economico in base ai ruoli;
\item Determinare un bilancio finale sulle spese.
\end{itemize}

\subsubsection{Aspettative}
\begin{itemize}
\item Ottenere una pianificazione ragionevole delle attività da seguire;
\item Avere un coordinamento delle attività, assegnando ruoli, compiti e semplificando la comunicazione tra i membri;
\item Riuscire a regolare le attività e renderle economiche.
\end{itemize}

\subsubsection{Descrizione}
Le attività  di gestione sono: 
\begin{itemize}
\item Assegnazione dei ruoli e dei compiti;
\item Inizio e definizione dello scopo;
\item Istanziazione dei processi;
\item Pianificazione e stima di tempi, risorse e costi;
\item Esecuzione e controllo;
\item Revisione e valutazione periodica delle attività.
\end{itemize}
\subsubsection{Istanziazione del processo}
\paragraph{Ruoli di progetto}
Ogni membro del gruppo deve, a rotazione, ricoprire almeno una volta ciascun ruolo di progetto per comprendere le differenze tra le diverse figure aziendali. Tali ruoli sono descritti di seguito.

\mbox{}

\textbf{Responsabile di progetto}\\
Il responsabile di progetto ricopre un ruolo fondamentale in quanto si occupa delle comunicazioni con il proponente e committente. Inoltre egli deve svolgere i seguenti compiti:
\begin{itemize}
\item Pianificare;
\item Gestire;
\item Controllare;
\item Coordinare.
\end{itemize}

\mbox{}

\textbf{Amministratore di progetto}\\
L'amministratore deve avere il controllo dell'ambiente di lavoro ed essere di supporto al team. Inoltre egli deve: 
\begin{itemize}
\item Dirigere le infrastrutture di supporto;
\item Controllare versioni e configurazioni;
\item Risolvere i problemi che riguardano la gestione dei processi;
\item Gestire la documentazione.
\end{itemize}

\mbox{}

\textbf{Analista}\\
L'analista si occupa dell'analisi dei problemi e del dominio applicativo. Questa figura ha le seguenti responsabilità:
\begin{itemize}
\item Studio del dominio del problema; 
\item Redazione della documentazione: \AdR{} e \SdF{};
\item Definizione dei requisiti e della sua complessità.
\end{itemize}

\mbox{}

\textbf{Progettista}\\
Il progettista si occupa dell'aspetto tecnico e tecnologico del progetto, segue lo sviluppo e non la manutenzione del prodotto. Inoltre egli deve scegliere: 
\begin{itemize}
\item Un'architettura adatta per il sistema del prodotto in base alle tecnologie scelte;
\item Il modo più efficiente per ottimizzare l'aspetto tecnico del progetto.
\end{itemize}

\mbox{}

\textbf{Programmatore}\\
Il programmatore si occupa della parte di codifica in base alle specifiche fornite dal progettista, operando con ottica di manutenibilità del codice. Inoltre egli deve creare e gestire componenti di supporto per la verifica e la validazione del codice. 

\mbox{}

\textbf{Verificatore}\\
Il \glo{verificatore} è presente durante tutta l'attività del progetto. Egli deve: 
\begin{itemize}
\item Controllare i prodotti in fase di revisione, utilizzando le tecniche e gli strumenti definiti nelle \NdP{}; 
\item Evidenziare gli errori e segnalarli all'autore del prodotto in questione.
\end{itemize}

Ogni membro del gruppo deve, a turno ricoprire almeno una volta ciascun ruolo di progetto che corrisponde alle figure aziendali. I ruoli che ogni membro del gruppo è tenuto a rappresentare sono descritti di seguito.



\subsubsection{Responsabile di progetto}
Il responsabile di progetto è un ruolo fondamentale, in quanto si occupa delle comunicazioni con il proponente e committente. Inoltre, egli deve svolgere i seguenti compiti:
\begin{itemize}
\item pianificare;
\item gestire;
\item controllare;
\item coordinare,
\end{itemize}

\subsubsection{Amministratore di progetto}
L'amministratore deve avere il controllo dell'ambiente di lavoro ed essere di supporto,. Inoltre Egli deve: 
\begin{itemize}
\item dirigere le infrastrutture di supporto;
\item controllare versioni e configurazioni;
\item risolvere i problemi che riguardano la gestione dei processi;
\item gestire la documentazione;
\end{itemize}


\subsubsection{Analista}
L'analista si occupa dell'analisi dei problemi e del dominio applicativo. Questa figura ha le seguenti responsabilità:
\begin{itemize}
\item studio del dominio del problema; 
\item redazione della documentazione: Analisi dei Requisiti e Studio di Fattibilità;
\item definizione dei requisiti e la sua complessità.
\end{itemize}
\subsubsection{Progettista}
Il progettista si occupa dell'aspetto tecnico e tecnologico del progetto, segue lo sviluppo e non la manutenzione del prodotto. Inoltre egli deve scegliere: 
\begin{itemize}
\item un'architettura adatta per il sistema del prodotto in base alle tecnologie scelte;
\item il modo più efficiente per ottimizzare l'aspetto tecnico del progetto.
\end{itemize}
\subsubsection{Programmatore}

\subsubsection{Verificatore}
\subsubsection{Procedure}
Per il coordinamento e le comunicazioni durante la realizzazione del progetto, il gruppo adotterà le seguenti procedure: 
\begin{itemize}
\item\textbf{comunicazione interna}: coinvolgeranno tutti i membri del team;
\item\textbf{comunicazione esterna}: avverrà con il proponente e committente.
\end{itemize}

\paragraph{Gestione delle comunicazioni}\mbox{}\\ \mbox{}\\
\textbf{Comunicazioni interne} \newline \newline
Le comunicazioni interne avvengono attraverso il canale chiamato \glo{Discord}. Questa applicazione consente la collaborazione a distanza e viene utilizzata anche in ambienti aziendali. Tale software permette al team di creare uno spazio di lavoro condiviso. \newline \newline

\textbf{Comunicazioni esterne} \newline \newline
Le comunicazioni con utenti esterni al gruppo sono gestite dal responsabile del progetto. Le modalità utilizzate sono le seguenti: 
\begin{itemize}
\item tramite la posta elettronica, dove viene utilizzato il seguente indirizzo \textbf{codebusterswe@gmail.com}; 
\item attraverso skype per colloqui con azienda Zucchetti.
\end{itemize} 

\paragraph{Gestione degli incontri}\mbox{}\\ \mbox{}\\
\textbf{Incontri interni} \newline \newline
Il responsabile del progetto concorda con il team gli incontri interni. Egli ha il compito di specificare la data delle riunioni nel calendario e approvare i verbali redatti dal segretario. I membri del gruppo sono tenuti a partecipare alle riunioni, interagendo nel dibattito. Affinché una riunione sia ritenuta valida, devono essere presenti almeno cinque membri del gruppo.\newline \newline

\textbf{Verbali di riunioni interne} \newline \newline
In occasione di ogni incontro interno viene redatto un verbale dal segretario scelto dal responsabile. Il contenuto della riunione viene riportato nel verbale corrispondente e deve essere approvato dal responsabile.

\textbf{Incontri esterni} \newline \newline
Il responsabile del progetto organizza gli incontri esterni con il proponente. Il proponente o committente potrebbero richiedere incontri con il team, il responsabile è tenuto a proporre una data in accordo con le parti e la comunica attraverso i canali sopra citati.
Gli incontri esterni avvengono tra i membri del gruppo e il proponente e quanto discusso viene riportato nel verbale esterno corrispondente.\newline \newline

\textbf{Verbali di riunioni esterne}  \newline \newline
In occasione di ogni incontro esterno viene redatto un verbale dal segretario scelto dal responsabile. Il contenuto della riunione verrà riportato nel verbale corrispondente e deve essere approvato dal responsabile.

\paragraph{Gestione degli strumenti e di coordinamento}\mbox{}\\ \mbox{}\\
\textbf{Tickecting}

Il tickecting permette ai membri del gruppo di rimanere aggiornati sullo stato delle attività in corso. Tramite questo il responsabile assegna compiti ai membri del gruppo e controlla l'andamento dei \glo{task} assegnati. Lo strumento di ticketing scelto è \textbf{Planner}: si tratta di un'applicazione \glo{multipiattaforma}, in cui è visibile a tutti i membri del team lo stato di avanzamento dei compiti assegnati, con la possibilità di aggiungerne nuovi. 

\paragraph{Gestione dei rischi}
Il responsabile è tenuto a individuare i rischi e renderli noti; tale attività verrà documentata nel \textit{Piano di Progetto}. La procedura per la gestione dei rischi è: 
\begin{itemize}
\item individuazione di nuovi problemi e monitoraggio di quelli già presenti;
\item menzione dei rischi individuati nel \textit{Piano di Progetto}; 
\item ridefinizione delle strategie per la gestione dei rischi in caso di necessità.\newline \newline

\textbf{Codifica dei rischi}\mbox{}\\ \mbox{}\\
I rischi sono codificati nel modo seguente: 
\begin{itemize}
\item  \textbf{RT}: rischi tecnologici 
\item \textbf{RO}: rischi organizzativi
\item   \textbf{RI}: rischi interpersonali
\end{itemize}
\end{itemize}
\subsubsection{Strumenti}
Nel corso dello sviluppo del progetto, il gruppo utilizzerà i seguenti strumenti: 
\begin{itemize}
\item \textbf{Telegram\glo}: applicazione di messaggistica per la comunicazione rapida e la gestione del gruppo; 
\item \textbf{Github\glo}: piattaforma che permette la condivisione in remoto di tutti i file del progetto e del versionamento;
\item \textbf{Git\glo}: sistema di controllo delle versioni;
\item \textbf{Discord}: applicazione multipiattaforma utilizzata per le riunioni interne;
\item \textbf{Planner}: piattaforma per la gestione dei compiti assegnati;
\item \textbf{Skype}: applicazione che consente di effettuare delle videoconferenze, utilizzata per la comunicazione con il proponente;
\item \textbf{Google Drive}: server per la condivisione rapida delle documentazioni che riguardano l'attività del gruppo.
\end{itemize}