\subsection{Fornitura}
\subsubsection{Descrizione}
In questa sezione si presenteranno le regole a cui si atterranno i membri del gruppo Codebusters nei processi che comprendono lo studio del capitolato fino alla candidatura a fornitore del progetto HD Viz del proponente Zucchetti e dei committenti \VT e \CR .  
\subsubsection{Scopo della fornitura}
Il fine del processo di fornitura è di scegliere le procedure e le risorse atte a perseguire lo sviluppo del progetto. Dopo aver ricevuto le richieste del proponente, il gruppo redige uno studio di fattibilità e la fornitura può essere avviata per completare tali richieste.\\
Il proponente e il fornitore stipuleranno un contratto per la consegna del prodotto.\\
Si dovrà poi sviluppare un piano di progetto partendo dalla determinazione delle procedure e delle risorse necessarie.
Da quel momento fino alla consegna del prodotto il piano di progetto scaglionerà le attività da svolgere. \\
 Il processo di fornitura è composto dalle seguenti fasi:
 \begin{enumerate}
 \item avvio; 
\item approntamento di risposte alle richieste;
\item contrattazione;
\item pianificazione;
\item esecuzione e controllo;
\item revisione e valutazione;
\item consegna e completamento.
 \end{enumerate}
\subsubsection{Proponente}
Codebusters vorrebbe avere un contatto costante con il proponente in modo da avere un riscontro:
\begin{itemize}
\item sulle soluzioni utilizzate;
\item sulle tempistiche di consegna del prodotto;
\item su eventuali dubbi;
\item stimare i costi;
\item su vincoli e requisiti.
\end{itemize}
\subsubsection{Documenti}
Di seguito sono descritti i documenti che saranno redatti durante questa fase.
\paragraph{Studio di fattibilità}
Documento che contiene la stesura dello studio di fattibilità riguardante i sette capitolati proposti, per ciascuno di essi vengono evidenziati i seguenti aspetti:
\begin{itemize}
    \item Descrizione generale;
    \item Prerequisiti e tecnologie coinvolte;
    \item Vincoli;
    \item Aspetti positivi;
    \item Aspetti critici.
\end{itemize}
Infine, per ogni capitolato vengono esposte le motivazioni e le ragioni per cui il gruppo ha scelto come progetto il capitolato C4 \NomeProgetto{} a discapito degli altri sei proposti.\\
\paragraph{Piano di qualifica}
Il piano di qualifica, redatto dal verificatore, contiene tutte le misure da adottare per garantire la qualità del prodotto. \'E suddiviso nelle seguenti parti:
\begin{itemize}
\item da inserire
\end{itemize} 
\paragraph{Piano di progetto}
Gli amministratori e il responsabile dovranno redigere questo documento che dovrà essere seguito durante tutto il corso del progetto. \'E suddiviso nelle seguenti sezioni:
\begin{itemize}
    \item Analisi dei rischi;
    \item Modello di sviluppo;
   \item Pianificazione;  
    \item Preventivo;
    \item Consuntivo;
    \item Organigramma.   
\end{itemize}
\subsubsection{Strumenti}
\paragraph{Excel}
Strumento utilizzato per creare grafici, fare calcoli e presentare tabelle organizzative.
\paragraph{Microsoft Planner}
Per gestire le risorse, le task che ciascun membro del gruppo deve completare e verificare se queste sono in svolgimento, i responsabili hanno utilizzato questo strumento. Permette di assegnare attività a risorse e di verificare che vengano completate in linea con i tempi previsti.