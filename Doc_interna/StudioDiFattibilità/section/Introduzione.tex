\section{Introduzione}
\subsection{Scopo del Documento}
Questo documento contiene la stesura dello studio di fattibilità riguardante i sette capitolati proposti, per ciascuno di essi vengono evidenziati i seguenti aspetti:
\begin{itemize}
    \item Descrizione generale;
    \item Prerequisiti e tecnologie coinvolte;
    \item Vincoli;
    \item Aspetti positivi;
    \item Aspetti critici.
\end{itemize}
Infine, per ogni capitolato vengono esposte le motivazioni e le ragioni per cui il gruppo ha scelto come progetto il capitolato C4 \NomeProgetto{} a discapito degli altri sei proposti.\\
I committenti di tutti i capitolati proposti sono \VT{} e \CR{}

\subsection{Glossario}
Il gruppo \Gruppo{} ha redatto un documento denominato \Glossariov{1.0.0} così da evitare ambiguità fra i termini, e per avere chiare fra tutti gli stakeholder le terminologie utilizzate per la realizzazione del presente documento.
In tale documento, sono presenti tutti i termini tecnici, ambigui, specifici del progetto e scelti dai membri del gruppo con le loro relative definizioni.
Un termine presente nel \Glossariov{1.0.0} e utilizzato in questo documento viene indicato con un apice \ap{G} alla fine della parola.

