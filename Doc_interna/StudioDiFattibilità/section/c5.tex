%DA FARE
\section{C5 - PORTACS - Point Of Interest  Oriented Real-Time Anti Collision System}

\subsection{Descrizione del capitolo}
Il capitolato propone la realizzazione di un sistema per il monitoraggio e la gestione di unità presenti in una mappa(griglia).\\
Ogni unità (può rappresentare un robot, un muletto, un'auto a guida automatica) ha punto di partenza nella griglia e una lista di POI \textit{"Point Of Interest"}  che deve raggiungere.
Il sistema dovrà indicare ad ognuna di esse la prossima mossa che dovrà fare, in base alla posizione, direzione e velocità  delle altre unità. Ogni unità dovrà inviare costantemente la propria posizione, direzione e velocità al sistema, così che sia possibile pilotare e coordinare tutte le unità evitando incidenti ed ingorghi.
L'azienda proponente è \textit{Sanmarco Informatica}

\subsection{Tecnologie coinvolte}
Il proponente non impone specifiche tecnologie per lo sviluppo della piattaforma ma invita all'utilizzo di :
\begin{itemize}
\item Nodejs per la parte di Back-end e Reactjs/next.js  per lo sviluppo di Front-end che permettono di creare un SPA \textit{"Single Page Application"} ;
\item Socket: è una libreria che consente di sviluppare applicazioni real-time e risulta facile da integrare con Nodejs;
\end{itemize}

\subsection{Vincoli}
L'azienda proponente pone i seguenti obiettivi da raggiungere:

\begin{itemize}
\item il sistema dovrà avere una visualizzazione in real-time della mappa e relativa posizione delle singole unità;
\item l'interfaccia utente che rappresenterà ogni singola unità  dovrà prevedere 4 frecce direzionali(si \textit{"accenderà"} quella suggerita dal sistema), il pulsante start/stop e l'indicatore della velocità attuale.
\end{itemize}

La versione finale del software dovrà essere in grado di accettare i seguenti input:
\begin{itemize}
\item Schacchiera oppure mappa con percorsi predefiniti e relativi vincoli (sensi unici, numero massimo di unità contemporanee);
\item definizione di POI (aree di carico/scarico e sosta).
\end{itemize}

\subsection{Aspetti positivi}
\begin{itemize}
\item Il proponente non obbliga l'implementazione di algoritmi di ricerca operativa per l'ottimizzazione dei percorsi;
\item Non si richiede la geolocalizzazione interna o esterna, verrà simulata dalla posizione, ciò riduce il tempo per la  codifica;
\item L'utilizzo di tecnologie per la gestione del real-time è parso un argomento molto interessante e ha suscitato curiosità nel gruppo;
\item L'azienda proponente ha dimostrato disponibilità per ulteriori chiarimenti e/o  informazioni.
\end{itemize}

\subsection{Aspetti critici}
L'azienda proponente non suggerisce tecnologie per lo sviluppo della piattaforma apprese in ambito universitario o conosciute dai membri del gruppo, ciò potrebbe comportare una dilatazione esagerata dei tempi per la scelta e lo studio delle tecnologie più adeguate.

\subsection{Conclusioni}
Tale capitolato ha attirato l'attenzione e stimolato l'interesse dei membri del gruppo in particolar modo per la possibilità di apprendere nuove tecnologie.

Il capitolato in questione presenta un livello di difficoltà elevato e richiede molto impegno, il gruppo ha ritenuto troppo elevata la mole di lavoro necessaria per svolgerlo e abbiamo deciso quindi di orientarci verso altri capitolati. 

