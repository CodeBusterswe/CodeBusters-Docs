%DA FARE
\section{Capitolato C2}
\subsection{Titolo del capitolato}
Il capitolato in questione si chiama \textit{" 	EmporioLambda: piattaforma di e-commerce in stile Serverless"}, il proponente \`e l'azienda \textit{Red Babel} e i committenti sono \VT{} e \CR{}.

\subsection{Descrizione del capitolo}
Nel capitolato due viene proposto di creare una piattaforma di E-Commerce, ovvero la compravendita on-line di prodotti.
Lo scopo finale del capitolato è avere una piattaforma dove i clienti possano registrarsi, ricercare i prodotti, aggiungerli al carrello (ovvero una pagina che permette di rivedere e considerare i prodotti e l'ammontare complessivo dei prezzi) e l'acquisto. La piattaforma dovrà offrire un'interfaccia per i venditori dove possono aggiungere e modificare i prodotti destinati alla vendita.

\subsection{Tecnologie coinvolte}
((Sebbene l'azienda non impone tecnologie specifiche per lo sviluppo del server o della UI, vi sono comunque delle scelte preferenziali da considerare nello svolgimento del progetto))(?):
\begin{itemize}
\item	Questa applicazione deve essere costruita con la tecnologia serverless con una API basata su HTTP. Per questo scopo si utilizza AWS Lambda e altre componenti di supporto (AWS API Gateway, AWS DynamoDB, AWS S3)
\item	CloudFormation per la gestione delle risorse sopraelencate
\item	Serverless Framework per l’implementazione di applicazioni serverless facilitandone alcune difficili strutture (come permessi, sottoscrizioni o accesso)
\item	Il BFF (Back end for Front end) richiede l’uso di Next.js
\end{itemize}

\subsection{Vincoli}
DA FARE
\subsection{Aspetti positivi}
DA FARE
\subsection{Aspetti critici}
DA FARE
\subsection{Conclusioni}
er il capitolato due si utilizzano tecnologie alla portata di tutti e, prendendo in considerazione le altre piattaforme di e-commerce esistenti, è relativamente facile progettarla. Proprio perché ci sono già moltissime piattaforme di e-commerce esistenti rende questo capitolato poco appetibile.
