%DA FARE
\section{Capitolato C2}
\subsection{Titolo del capitolato}
Il capitolato in questione si chiama \textit{" 	EmporioLambda: piattaforma di e-commerce in stile Serverless"}, il proponente \`e l'azienda \textit{Red Babel} e i committenti sono \VT{} e \CR{}.

\subsection{Descrizione del capitolo}
Nel capitolato due viene proposto di creare una piattaforma di E-Commerce, ovvero la compravendita on-line di prodotti.
Lo scopo finale del capitolato è avere una piattaforma dove i clienti possano registrarsi, ricercare i prodotti, aggiungerli al carrello (ovvero una pagina che permette di rivedere e considerare i prodotti e l'ammontare complessivo dei prezzi) e l'acquisto. La piattaforma dovrà offrire un'interfaccia per i venditori dove possono aggiungere e modificare i prodotti destinati alla vendita.

\subsection{Tecnologie coinvolte}
Il progetto prevede l' adozione di una bozza iniziale dell'architettura dettata dal proponente e di alcune scelte tecnologiche obbligatorie, il fornitore è libero e altamente incoraggiato ad esplorare diverse scelte.
Questa applicazione deve essere costruita con la tecnologia serverless con una API basata su HTTP:
\begin{itemize}
\item	Si utilizzano AWS Lambda e altre componenti di supporto (AWS API Gateway, AWS DynamoDB, AWS S3);
\item	CloudFormation per la gestione delle risorse sopraelencate;
\item	Serverless Framework per l’implementazione di applicazioni serverless facilitandone alcune difficili strutture (come permessi, sottoscrizioni o accesso);
\item	Il BFF (Back end for Front end) sarà implementato in Next.js e richiede l' uso di Typescript come linguaggio principale;
\item	Il codice sorgente dovrebbe essere pubblicato e aggiornato utilizzando GitHub o GitLab.
\end{itemize}

\subsection{Vincoli}
E' richiesto l'uso di un numero minimo di ambienti di lavoro: uno locale di ogni sviluppatore, uno per i test accessibile a tutti gli sviluppatori, e un ambiente pubblico accessibile dagli utenti.
E' richiesta l' implementazione di tutti i moduli di alto livello:
\begin{itemize}
\item	EmporioLambda-frontend (EML-FE): modulo che serve le pagine web richieste dal cliente;
\item	EmporioLambda-backend (EML-BE): modulo che espone i servizi dell' applicazione;
\item	EmporioLambda-integration (EML-I): rappresenta tutti i servizi di terze parti;
\item	 EmporioLambda-monitoring (EML-MON): è il set di strumenti usati per monitorare lo stato dell'applicazione.
\end{itemize}
E' richiesta la presenza di determinate minime, (ognuna con le proprie funzioni principali):
\begin{itemize}
\item	Homepage
\item	Product Listing Page
\item	Product Detail Pages
\item	Shopping Cart
\item	Account
\item	Checkout
\item	Merchant dashboard
\end{itemize}
Il sito deve implementare i seguenti ruoli utilizzando AWS Cognito Identity:
\begin{itemize}
\item	Amministratore
\item	Venditore
\item	Cliente
\end{itemize}
L'unica integrazione obbligatoria richiesta è il provider di pagamento, Stripe.
\subsection{Aspetti positivi}
\begin{itemize}
\item	E' richiesto l' uso di numerosi servizi forniti da AWS, sia comuni che meno; i quali possono risultare utili al nostro futuro;
\item	L' azienda sembra essere disponibile a seguire il gruppo visto che richiede delle scelte da fare insieme perciò viene tenuto un contatto con il proponente;
\item	Nel 2020 ormai gli e-commerce sono alla portata di tutti e fanno parte della vita quotidiana delle persone, perciò è un argomento familiare ai membri del gruppo.
\end{itemize}
\subsection{Aspetti critici}
\begin{itemize}
\item	La presentazione del capitolato è approssimativa visto che non è stato fatto alcun seminario di approfondimento.
\end{itemize}
\subsection{Conclusioni}
Per il capitolato due si utilizzano tecnologie non troppo complesse e, prendendo in considerazione le altre piattaforme di e-commerce esistenti (per esempio Amazon), è relativamente facile progettarla visto che l' e-commerce è ormai la quotidianità. Proprio perché ci sono già moltissime piattaforme di e-commerce esistenti rende questo capitolato poco appetibile.
