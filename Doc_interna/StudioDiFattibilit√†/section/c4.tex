\section{C4 - HD Viz: visualizzazione di dati multidimensionali}

\subsection{Descrizione del capitolo}
Il capitolato C4 ha come obiettivo quello di creare un'applicazione di visualizzazione di grandi quantità di dati con numerose dimensioni in un formato comprensibile dall'occhio umano. A questo scopo è necessario sviluppare algoritmi di intelligenza artificiale che, agendo sulla distanza dei vari punti del grafico, riescano a sviluppare un modello semplificato che ne evidenzi i \glo{cluster}. 
L'applicazione dovrà inoltre agire su questi grafici creati evidenziando i dati ottenuti.\\
L'azienda proponente è \textit{Zucchetti}

\subsection{Tecnologie coinvolte}
L'applicazione “HD Viz” dovrà utilizzare le seguenti tecnologie:
\begin{itemize}
\item \glo{HTML}/\glo{CSS}/\glo{Javascript} con libreria \glo{D3.js} per la parte client;
\item La parte server e un database (\glo{SQL} o \glo{NOSQL}) potranno essere sviluppati in \glo{Java} con server \glo{Tomcat} o in \glo{Javascript} con server \glo{Node.js}.
\end{itemize}

\subsection{Vincoli}
I vincoli del capitolato C4 riguardano, in prima istanza, il numero di dimensioni dei dati da visualizzare. L'applicazione deve riuscire a visualizzare dati con almeno 15 dimensioni. 
Il formato dei dati da fornire a HD Viz deve essere una query ad un database o anche un file in formato \glo{CSV} preparato in precedenza.
HD Viz dovrà presentare le seguenti visualizzazioni: 
\begin{itemize}
\item Scatter plot Matrix (fino ad un massimo di 5 dimensioni): è la presentazione a riquadri disposti a matrice di tutte le combinazioni di scatter plot; 
\item Force Field: converte dati in più dimensioni in un grafico a due/tre dimensioni rappresentandoli sulla base della loro distanza ; 
\item Heat Map: rappresenta la distanza tra i punti con colori più o meno intensi. In questa visualizzazione è necessario ordinare i dati per evidenziarli; 
\item Proiezione Lineare Multi Asse: si rappresentano a due dimensioni dati multidimensionali permettendo di spostare gli assi per rendere la visualizzazione più comprensibile.  
\end{itemize} 
\'E necessario che l'utente non veda le operazioni e gli algoritmi utilizzati per ridurre le dimensioni dei dati e che possa scegliere in quali dimensioni vuole visualizzare il grafico.
Il proponente lascia ampia libertà di esplorazione dell'argomento e accetta altri requisiti opzionali proposti dal fornitore come, ad esempio, l'inventare degli algoritmi per ridurre le dimensioni.

\subsection{Aspetti positivi}
\begin{itemize}
\item L'analisi dei Big Data è una problematica più che mai attuale e che potrebbe risultare molto utile in ambito professionale;
\item Nel capitolato si tratta l'intelligenza artificiale, altro argomento molto interessante in un futuro ambito lavorativo;
\item Gli strumenti \glo{HTML}/\glo{CSS}/\glo{Javascript} sono già stati visti nel corso di Tecnologie Web;
\item La libreria D3 è ben documentata e largamente fornita di esempi sulle visualizzazioni richieste;
\item L'azienda ha dato disponibilità per ulteriori chiarimenti in corso d'opera;
\item L'azienda fornirà dei set di dati per testare il software.
\end{itemize}

\subsection{Aspetti critici}
\begin{itemize}
\item Il capitolato potrebbe risultare difficoltoso nel testare gli algoritmi;
\item Altri capitolati hanno per argomento oggetti o situazioni più reali in confronto all'analisi ed elaborazione dei dati;
\item La libreria D3 è molto potente ed è necessario imparare ad usarla.
\end{itemize}

\subsection{Conclusioni}
Il capitolato C4 è tra i preferiti dal gruppo che infine ha deciso di sceglierlo. La squadra è stata convinta soprattutto dalla risposta dell'azienda ad alcuni dubbi che il gruppo si era posto e dalla chiarezza e disponibilità della stessa alla richiesta di informazioni. La possibilità di trattare un argomento come l'analisi dei Big data, ci permette di esplorare un campo molto florido per l'uso che se ne fa nei social network e nella ricerca di target per la pubblicità. L'utilizzo della libreria D3, una delle più usate in questo campo, ci permette di conoscere nuovi strumenti, nonostante questa libreria sia molto potente e sia necessario diverso tempo per padroneggiarla. L'utilizzare HTML, Javascript e CSS, linguaggi già visti nel corso di Tecnologie Web, ci agevolerà e ci permetterà di dedicare molto più tempo allo studio della libreria D3.
