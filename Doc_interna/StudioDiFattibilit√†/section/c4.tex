
\section{Capitolato C4}
\subsection{Titolo del capitolato}
Il capitolato in questione si chiama \textit{"HD Viz"}, il proponente \`e l'azienda \textit{"Zucchetti"} e i committenti sono \VT{} e \CR{}.

\subsection{Descrizione del capitolo}
Il capitolato C4 ha come obiettivo quello di creare un'applicazione di visualizzazione di grandi quantità di dati con numerose dimensioni in un formato comprensibile dall'occhio umano. A questo scopo è necessario sviluppare algoritmi di intelligenza artificiale che, agendo sulla distanza dei vari punti del grafico, riescano a sviluppare un modello semplificato che ne evidenzi i cluster. %cluster: gruppo di elementi omogenei in un insieme di dati 
L'applicazione dovrà inoltre agire su questi grafici creati evidenziando i dati ottenuti.

\subsection{Tecnologie coinvolte}
L'applicazione “HD Viz” dovrà utilizzare le seguenti tecnologie:
\begin{itemize}
\item HTML/CSS/Javascript con libreria D3.js per la parte client;
\item La parte server e i database (SQL o NOSQL) potranno essere sviluppati in Java con server Tomcat o in Javascript con server Node.js.
\end{itemize}

\subsection{Vincoli}
I vincoli del capitolato C4 riguardano, in prima istanza, il numero di dimensioni dei dati da visualizzare. L'applicazione deve riuscire a visualizzare dati con almeno 15 dimensioni ma se necessario anche dati a meno dimensioni. 
Il formato dei dati da fornire a HD Viz deve essere una query ad un database o anche un file in formato CSV preparato in precedenza.
HD Viz dovrà presentare le seguenti visualizzazioni: 
\begin{itemize}
\item Scatter plot Matrix (fino ad un massimo di 5 dimensioni): è la presentazione a riquadri disposti a matrice di tutte le combinazioni di scatter plot; 
\item Force Field: converte dati in più dimensioni in un grafico a due/tre dimensioni rappresentandoli sulla base della loro distanza ; 
\item Heat Map: rappresenta la distanza tra i punti con colori più o meno intensi. In questa visualizzazione è necessario ordinare i dati così da evidenziare i dati; 
\item Proiezione Lineare Multi Asse: si rappresentano a due dimensioni dati multidimensionali permettendo però di trascinare gli assi così da rendere più comprensibile la visualizzazione.  
\end{itemize} 
Il proponente lascia ampia libertà di esplorazione dell'argomento e accetta altri requisiti opzionali proposti dal fornitore.

\subsection{Aspetti positivi}
\begin{itemize}
\item L'analisi dei Big Data è una problematica più che mai attuale e che potrebbe risultare molto utile in ambito professionale;
\item Nel capitolato si tratta l'intelligenza artificiale, altro argomento molto interessante in un futuro ambito lavorativo;
\item Gli strumenti HTML/CSS e Javascript sono già stati visti nel corso di Tecnologie Web;
\item L'azienda ha dato disponibilità per ulteriori chiarimenti in corso d'opera.
\end{itemize}

\subsection{Aspetti critici}
\begin{itemize}
\item Il capitolato potrebbe risultare difficoltoso nella realizzazione degli algoritmi e nel testare gli stessi;
\item Alcuni altri capitolati hanno per argomento oggetti o situazioni più reali in confronto all'analisi statistica dei dati.
\end{itemize}

\subsection{Conclusioni}
Il capitolato C4 è tra i preferiti dal gruppo che infine ha deciso di (...)
