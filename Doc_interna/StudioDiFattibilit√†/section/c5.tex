\section{C5 - PORTACS - Point Of Interest  Oriented Real-Time Anti Collision System}

\subsection{Descrizione del capitolo}
Il capitolato si pone come obiettivo la realizzazione di una piattaforma in grado di fornire un'interfaccia grafica consenta all'utente utilizzatore il monitoraggio delle attività delle seguenti realtà come:\begin{itemize}
\item Il Robot cameriere: percorre percorsi predefiniti, ritiro e consegna dell'ordine;
\item Geolocalizzazione del magazzino: ogni mezzo e persona si spostano all'interno del magazzino ed è georeferenziato;  i mezzi devono rispettare il senso di percorrenza della corsia per evitare possibili collisioni,; ogni persona a piedi o su mezzo deve raggiunger varie ubicazioni per poter preparare ordini ;
\item Geolocalizzazione nei trasporti: ogni mezzo ha un punto di partenza con una lista di consegne deve effettuare; velocità media e deve evitare collisioni; il responsabile è in grado di monitorare lo stato di ogni mezzo in tempo reale;
\item Auto a guida automatica: bisogna definire la destinazione dell'auto; ogni auto può controllare la propria velocità così da evitare possibili collisioni.
\end{itemize}
Queste realtà hanno dei punti in comune: mappatura della posizione in tempo reale e tenere conto dei punti di partenza e destinazione, possibilità di monitorare la velocità ottimale e velocità massima, evitare gli ostacoli/collisioni. 


Ogni unità (può rappresentare un robot, un muletto, un'auto) ha punti di partenza nella griglia e una lista di POI \textit{"Point Of Interest"} da raggiungere, con una velocità massimae deve inviare al sistema: 
\begin{itemize}
\item la propria posizione costantemente;
\item la direzione e velocità, in modo tale che il sistema centrale possa coordinare e pilotare tutte le unità  per evitare incidenti e ingorghi.

\end{itemize}

La User Interface conterrà ogni singola unità  che dovrà provvedere delle 4 frecce direzionali (si \textit{"accenderà"} quella suggerita dal sistema), il pulsante start/stop e l'indicatore della velocità attuale (limite massimo è  la velocità anagrafica).

Il sistema è corredato da un display in real-time della mappa e relativa posizione di ogni singola unità e inoltre il sistema indicherà ad ogni unità la prossima mossa da compiere, in funzione di: 
\begin{itemize}
\item prossimo POI da raggiungere ;
\item posizione delle altre unità evitando collisioni e vanno rispettati vincoli dimensionali (limite delle corsie);
\item Plus: posizioni dei pedoni per evitare collisioni.
\end{itemize}
Il software dovrà essere in grado di accettare i seguenti input:\begin{itemize}
\item Schacchiera oppure mappa con percorsi predefiniti e relativi vincoli (sensi unici, numero massimo di unità contemporaneamente);
\item definizio di POI (aree di carico e scarico, sosta).
\end{itemize}
L'azienda proponente è \textit{Sanmarco Informatica}

\subsection{Tecnologie coinvolte}
Il proponente non impone specifiche tecnologie per lo sviluppo della piattaforma ma invita all'utilizzo di quelle più funzionali e moderne.
Tra queste si segnala: 
\begin{itemize}
\item utilizzo di Nodejs per la parte di Back-end e Reactjs/next.js  per lo sviluppo di Front-end che permettono di creare un SPA \textit{"Single Page Application"} ;
\item Socket: è una libreria che consente di sviluppare applicazioni real-time e risulta facile da integrare con il Nodejs;
\end{itemize}

\subsection{Vincoli}
L'azienda proponente non pone vincoli per la scelta degli strumenti e tecnologie di sviluppo ma vi sono alcuni vincoli legati al funzionamento della piattaforma: 
\begin{itemize}
\item il sistema dovrà avere una visualizzazione in real-time la mappa e la posizione delle singole unità;
\item ogni unità dovrà inviare al sistema centrale costantemente la propria posizione, direzione e velocità;
\item il sistema deve accettare come input: una Scacchiera o mappa con percorsi.
\end{itemize}


\subsection{Aspetti positivi}
L'obbiettivo del capitolato è quello di creare una piattaforma real-time che facilita il monitoraggio dello stato delle attività di ogni singola unità (robot, muletto, automobile); inoltre il progetto proposto consente all'utente sviluppatore di apprendere le nuove tecnologie che verranno utilizzate per implementare le varie funzionalità, in particolare l'implementazione del meccanismo real-time.

Un altro aspetto positivo è dato dal fatto che l'azienda proponente ha dimostrato disponibilità per ulteriori chiarimenti e/o  informazioni.


\subsection{Aspetti critici}
L'azienda proponente non segnala tecnologie per lo sviluppo della piattaforma conosciute e apprese in ambito universitario; ciò potrebbe comportare una dilatazione dei tempi per la scelta delle tecnologie più adeguate e per la formazione e l'autoapprendimento da parte dei membri del gruppo.


\subsection{Conclusioni}
Ciò nonostante, tale capitolato ha attirato l'attenzione e stimolato l'interesse dei membri del gruppo in particolar modo per la possibilità di apprendere nuove tecnologie moderne.

Il capitolato in questione presenta un livello di difficoltà elevato e richiede impegno; alla luce di questo il gruppo si rende disponibile a sviluppare quanto richiesto dal proponente. 

