%DA FARE
\section{Capitolato C5}
\subsection{Titolo del capitolato}
Il capitolato in questione si chiama \textit{"PORTACS - Point Of Interest  Oriented Real-Time Anti Collision System"}, il proponente \`e l'azienda \textit{Sanmarco Informatica} e i committenti sono \VT{} e \CR{}.

\subsection{Descrizione del capitolo}
Il capitolato si pone come obiettivo finale la realizzazione di una piattaforma real time in grado di dare all'utente utilizzatore una rappresentazione grafica sullo stato delle attività di diverse unità. Ogni unità (può rappresentare un robot, un muletto, un'auto a guida automatica) ha punto di partenza nella griglia e una lista di POI \textit{"Point Of Interest"}  che deve raggiungere. Il sistema centrale dovrà indicare ad ogni unità la prossima mossa da fare in base alla posizione, direzione e velocità che riceve dalle singole unità, in modo tale che il sistema possa pilotare e coordinare tutte le unità per evitare incidenti e ingorghi.



\subsection{Tecnologie coinvolte}
Il proponente non impone specifiche tecnologie per lo sviluppo della piattaforma ma invita all'utilizzo di quelle più funzionali e moderne.
Tra queste si segnala: 
\begin{itemize}
\item utilizzo di Nodejs per la parte di Back-end e Reactjs/next.js  per lo sviluppo di Front-end che permettono di creare un SPA \textit{"Single Page Application"} ;
\item Socket: è una libreria che consente di sviluppare applicazioni real-time e risulta facile da integrare con il Nodejs;
\end{itemize}

\subsection{Vincoli}
L'azienda proponente pone i seguenti obiettivi da raggiungere:

\begin{itemize}
\item il sistema dovrà avere una visualizzazione in real-time della mappa e relativa posizione delle singole unità;
\item ogni unità dovrà inviare al sistema centrale costantemente la propria posizione, direzione e velocità.
\item l'interfaccia utente rappresenterà ogni singola unità che dovrà prevedere delle 4 frecce direzionali(si \textit{"accenderà"} quella suggerita dal sistema), il pulsante start/stop e l'indicatore della velocità attuale (limite massimo è  la velocità anagrafica).
\end{itemize}

La versione finale del software dovrà essere in grado di accettare i seguenti input:
\begin{itemize}
\item Schacchiera oppure mappa con percorsi predefiniti e relativi vincoli (sensi unici, numero massimo di unità contemporaneamente);
\item definizione di POI (aree di carico/scarico e sosta).
\end{itemize}


\subsection{Aspetti positivi}
\begin{itemize}
\item Il proponente non richiede l'implementazione di algoritmi di ricerca per l'ottimizzazione dei percorsi;
\item non richiede la geolocalizzazione interna o esterna e verrà simulata dalla posizione, ciò riduce il tempo per la  codifica;
\item l progetto proposto consente all'utente sviluppatore di apprendere le nuove tecnologie che verranno utilizzate per implementare le varie funzionalità, in particolare l'implementazione del meccanismo real-time;
\item l'azienda proponente ha dimostrato disponibilità per ulteriori chiarimenti e/o  informazioni.
\end{itemize}



\subsection{Aspetti critici}
L'azienda proponente non segnala tecnologie per lo sviluppo della piattaforma conosciute e apprese in ambito universitario; ciò potrebbe comportare una dilatazione dei tempi per la scelta delle tecnologie più adeguate e per la formazione e l'autoapprendimento da parte dei membri del gruppo.


\subsection{Conclusioni}
Ciò nonostante, tale capitolato ha attirato l'attenzione e stimolato l'interesse dei membri del gruppo in particolar modo per la possibilità di apprendere nuove tecnologie moderne.

Il capitolato in questione presenta un livello di difficoltà elevato e richiede impegno; alla luce di questo il gruppo si rende disponibile a sviluppare quanto richiesto dal proponente. 

