%DA FARE
\section{C7 - SSD: soluzioni di sincronizzazione desktop}

\subsection{Descrizione del capitolo}
Il capitolato presentato si pone come obiettivo finale la realizzazione di un algoritmo di sincronizzazione Desktop in grado di garantire il salvataggio in \glo{cloud} di qualsiasi lavoro e contemporaneamente la sincronizzazione dei cambiamenti presenti in cloud. In aggiunta a ciò, lo sviluppo di un'interfaccia multipiattaforma per l'uso dell'algoritmo e l'utilizzo di quest'ultimo con il prodotto dell'azienda \textit{Zextras Drive}.\\
L'azienda proponente è \textit{Zextras}.

\subsection{Tecnologie coinvolte}
L'azienda non impone tecnologie specifiche per lo sviluppo della parte \glo{back-end} o della \glo{UI}, ma vi sono comunque delle scelte preferenziali da considerare nello svolgimento del progetto:
\begin{itemize}
\item utilizzo di \glo{Python} per lo sviluppo dell'algoritmo di sincronizzazione;
\item utilizzo del framework \glo{Qt} per lo sviluppo dell'interfaccia e del controller d'architettura;
\end{itemize}

\subsection{Vincoli}
La proponente richiede di rispettare alcuni vincoli sull'architettura del progetto: 
\begin{itemize}
\item l’algoritmo di sincronizzazione e l’interfaccia utente devono essere utilizzabili nei tre pricipali sistemi operativi desktop (Mac, Windows, Linux) senza richiedere all’utente l’installazione manuale di ulteriori prodotti per il funzionamento; 
\item forte distinzione tra interfaccia e algoritmo di sincronizzazione; 
\item utilizzare il pattern architetturale \glo{MVC}.
\end{itemize}

\subsection{Aspetti positivi}
\begin{itemize}
\item Il linguaggio proposto, Python, è in grande ascesa negli ultimi anni e sono molte le tecnologie che lo stanno implementando o che lo hanno già fatto. Riteniamo, quindi, che sia di grande aiuto apprenderlo;
\item Il capitolato propone l'utilizzo del framework Qt, che è già familiare a tutto il gruppo di progetto.
\end{itemize}

\subsection{Aspetti critici}
\begin{itemize}
\item Lo sviluppo di un algoritmo di sincronizzazione a partire da zero;
\item Il framework Qt è già conosciuto, questo rende il capitolato meno interessante, in quanto questo progetto non viene visto solo come un'occasione per fare esperienza nel campo del Software Engineering, ma anche come un'opportunità per apprendere tecnologie che non si studiano durante il percorso di studi.
\end{itemize}

\subsection{Conclusioni}
Il capitolato non ha destato molto la curiosità del gruppo; sebbene l'utilizzo del linguaggio Python sia l'unica nota di interesse, questo è proposto anche in altri capitolati che hanno catturato maggiormente la nostra attenzione.
