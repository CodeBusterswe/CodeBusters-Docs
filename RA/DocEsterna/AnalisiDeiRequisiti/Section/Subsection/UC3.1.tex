\subsubsection{UC3.1 - Selezione dell'algoritmo di riduzione dimensionale}
\begin{itemize}
	\item \textbf{Attore primario}: Utente;
	\item \textbf{Precondizioni}: L'utente ha caricato i dati e le dimensioni nel sistema [UC1];
	\item \textbf{Postcondizioni}: I nuovi dati vengono inseriti nel sistema e sono disponibili all'utente per la visualizzazione [UC6];
	\item \textbf{Scenario principale}: L'utente seleziona un algoritmo di riduzione dimensionale tra quelli resi disponibili dal sistema. I parametri di personalizzazione dell'algoritmo saranno impostati automaticamente a dei valori di default.
	\item \textbf{Generalizzazioni}: L'utente seleziona una delle seguenti opzioni:
	\begin{enumerate}[(a)]
		\item \glo{\textit{Isometric Mapping (IsoMap)}} [UC3.1.1];
		\item \glo{\textit{Locally Linear Embedding (LLE)}} [UC3.1.2];
		\item \glo{\textit{Fast Mapping (FastMap)}} [UC3.1.3];
		\item \glo{\textit{T-distributed Stochastic Neighbor Embedding (t-SNE)}} [UC3.1.4];
		\item \glo{\textit{Uniform Manifold Approximation and Projection (UMAP)}} [UC3.1.5];
		\item \glo{\textit{Principal Component Analysis (PCA)}} [UC3.1.6].
		
	\end{enumerate}
\end{itemize}