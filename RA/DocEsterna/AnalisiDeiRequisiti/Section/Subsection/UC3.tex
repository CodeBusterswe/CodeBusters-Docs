\newpage
\subsection{UC3 - Riduzione dimensionale}
\begin{figure}[h]
\includegraphics[width=16cm]{section/Images/UC3.png}
\centering
\caption{UC3 - Riduzione dimensionale}
\end{figure}
\begin{itemize}
	\item \textbf{Attore primario}: Utente;
	\item \textbf{Precondizioni}: L'utente ha caricato i dati e le dimensioni nel sistema [UC1];
	\item \textbf{Postcondizioni}: I nuovi dati vengono inseriti nel sistema e sono disponibili all'utente per la visualizzazione;
	\item \textbf{Scenario principale}: L'utente può creare nuovi dati, a partire dalle dimensioni caricate [UC1] ed eventualmente scremate [UC2] tramite:
	\begin{enumerate}[1.]
		\item Algoritmo di riduzione dimensionale [UC3.1];
		\item Calcolo delle distanze tra i valori delle dimensioni [UC3.2].
	\end{enumerate}
	L'utente potrà selezionare le dimensioni interessate dalla riduzione per poi premere il tasto di conferma.
\end{itemize}