\section*{P}
\markright{}
\addcontentsline{toc}{section}{P}
\subsection*{PaaS}
Acronimo di Platform as a Service. Piattaforme di elaborazione che permettono di sviluppare, sottoporre a test, implementare e gestire le applicazioni aziendali senza i costi e la complessità associati all'acquisto, alla configurazione, all'ottimizzazione e alla gestione dell'hardware e del software di base. 

\subsection*{Pattern}
Rappresenta uno schema ricorrente, una struttura ripetitiva in uno specifico contesto. 

\subsection*{PCA}
Acronimo di Principal Component Analysis. \'E una tecnica per la semplificazione dei dati utilizzata
nell'ambito della statistica multivariata. Lo scopo della tecnica è quello di ridurre il numero più o meno elevato di variabili che descrivono un insieme di dati a un numero minore di variabili latenti, limitando il più possibile la perdita di informazioni.

\subsection*{Perplessità}
Parametro dell'algoritmo \glo{t-SNE} che indica la capacità di bilanciare gli aspetti locali del dato con quelli globali. Esso può essere considerato una supposizione sul numero di \glo{neighbors} che ha ciascun punto e dipende dal \glo{manifold}.  

\subsection*{PostgreSQL}
È un completo DBMS ad oggetti rilasciato con licenza libera, è una alternativa sia rispetto ad altri prodotti liberi come \glo{MySQL}, Firebird SQL e MaxDB che a quelli a codice chiuso come Oracle, IBM Informix o DB2 ed offre caratteristiche uniche nel suo genere che lo pongono per alcuni aspetti all'avanguardia nel settore delle basi di dati.

\subsection*{Proiezione lineare multi asse}
Grafico che posiziona i punti dello spazio multidimensionale in un piano cartesiano (con assi "draggabili"), effettuando quindi sui dati una riduzione a due dimensioni. Questo grafico non è tra quelli presenti negli esempi di \glo{D3.js}, ma è visibile nel programma di data mining \glo{Orange Canvas} o nello strumento di visualizzazione \glo{ggobi}.

\subsection*{Product baseline}
Presenta la \glo{baseline} architetturale del prodotto (\glo{design patterns} adottati) e va mandata tramite "allegato tecnico", ovvero una cartella compressa inviata tramite e-mail, con diagrammi delle classi e di sequenza. In questo momento deve esistere un prodotto, idealmente "finito", e bisogna essere pronti alla \glo{validazione}.

\subsection*{Proof of Concept}
Una realizzazione incompleta o abbozzata di un determinato progetto o metodo, allo scopo di provarne la fattibilità o dimostrare la fondatezza di alcuni principi o concetti costituenti. Un esempio tipico è quello di un prototipo. 

\subsection*{Proponente}
Ente o azienda che compie l'atto di proporre il \glo{capitolato} d'appalto per un progetto.

\subsection*{Protocollo asincrono}
Protocollo per la trasmissione di dati, un byte alla volta (un carattere). Si definiscono protocolli start-stop perché l'informazione da inviare deve essere accompagnata da un bit all'inizio (start della trasmissione) e un bit alla fine (stop della trasmissione).

\subsection*{Python}
Linguaggio di programmazione orientato a oggetti, adatto, tra gli altri usi, a sviluppare applicazioni distribuite, scripting, computazione numerica e system testing.

\subsection*{Pytorch}
Libreria \glo{open source} di machine-learning sviluppata da Facebook e basata sulla libreria Torch. È utilizzata in campi come la computer vision e per l'elaborazione del linguaggio naturale.

