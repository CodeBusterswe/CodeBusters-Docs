\section{Tecnologie coinvolte}

{
\setlength\arrayrulewidth{0.95pt}
\renewcommand{\arraystretch}{1.5}
\begin{longtable}{C{3cm} | C{2cm} | C{10cm}}
\rowcolor{coloreRosso}

\textcolor{white}{\textbf{Tecnologia}}&
\textcolor{white}{\textbf{Versione}}&
\textcolor{white}{\textbf{Descrizione}} \\
\endfirsthead
\rowcolor{white}\multicolumn{3}{C{15cm}}{\textit{Continua nella pagina successiva...}}\\
\endfoot
\rowcolor{white}\caption{Tecnologie coinvolte}
\endlastfoot
	
\rowcolor{coloreRossoChiaro}
\multicolumn{3}{|c|}{\textcolor{white}{\textbf{Linguaggi}}} \\

	\textbf{JavaScript} & 
	\glo{ES6} &
	Utilizzato per la creazione di effetti dinamici e interattivi tramite eventi invocati dall'utente. È il linguaggio della libreria \glo{React}.\\

	\textbf{HTML} &
	5 &
	Utilizzato nel progetto assieme a React per definire l'interfaccia grafica. \\
	
	\textbf{CSS} &
	3 & 
	Utilizzato per definire la formattazione dei documenti HTML5 e lo stile. \\
	
 \rowcolor{coloreRossoChiaro}
\multicolumn{3}{|c|}{\textcolor{white}{\textbf{Strumenti}}} \\

	\textbf{PostgreSQL} & 
	13.x &
	\glo{DBMS} ad oggetti utilizzato nel progetto per il caricamento nell'applicazione di basi di dati preesistenti. \\
 
 	\textbf{Npm} & 
	7.x &
	Gestore di pacchetti utilizzato per effettuare le operazioni di build del codice. \\
 
% \textbf{ESLint} & 
%	7.15.0  &
%	ESLint è uno strumento di analisi statica per identificare i modelli problematici trovati nel codice. Le regole in ESLint sono configurabili e personalizzabili a seconda delle esigenze del progetto.
%È stato utilizzato questo strumento principalmente per la segnalazione degli errori di sintassi, per avere regole  d'indentazione uguali in tutti i file e per notifiche su variabili o funzioni non utilizzati. \\
 
	\textbf{Babel} & 
	7.11.0 &
	Transcompiler JavaScript utilizzato per convertire il codice ECMAScript 2015+ in una versione retro compatibile per browser non aggiornati. \\
  
 \rowcolor{coloreRossoChiaro}
\multicolumn{3}{|c|}{\textcolor{white}{\textbf{Librerie e framework}}}\\
 
	\textbf{React} & 
	17.0.1 &
	Libreria per la creazione di UI scelta per facilitare lo sviluppo del front-end e avere performance migliori grazie al suo metodo di renderizzazione dei componenti grafici.\\ 
 
 	\textbf{D3.js} & 
	6.x &
	Libreria per creare visualizzazioni dinamiche ed interattive partendo da dati organizzati.
\\

	\textbf{Node.js} & 
	14.16.0 &
	Runtime system orientato agli eventi scelto come strumento per l'utilizzo di Javascript lato server.\\
 
	\textbf{React Bootstrap} & 
	1.5.2 &
	\glo{Framework} che fornisce componenti React con uno stile già integrato e che permette la creazione di applicazioni web responsive.\\
		
	\textbf{MobX} & 
	6.1.x &
	Libreria per React che permette la gestione dello \glo{state} dei componenti  e l'implementazione del design pattern Observer.\\
	
	\textbf{Druid.js} & 
	0.3.5 &
	Libreria usata per implementare il processo di riduzione dimensionale attraverso algoritmi lineari e non lineari.\\
	
	\textbf{ml-distance.js} & 
	3.0.0 &
	Libreria usata per implementare il processo di calcolo delle distanze.\\

	\textbf{Express} & 
	4.17.1 &
	\glo{Framework} per Node.js che fa da intermediario tra web app e database, agevolandone il collegamento.\\

%	\textbf{Jest} & 
%	26.x &
%	Jest è un \glo{framework} open-source di test JavaScript gestito da Facebook. Funziona con progetti che utilizzano Babel, TypeScript, Node.js, React, Angular, Vue.js e Svelte. In \NomeProgetto{} è utilizzato per l'analisi dinamica del codice.\\
%
%	\textbf{React Testing Library} & 
%	11.2.x &
%	La libreria React Testing Library è una libreria apposita per testare i componenti React. Anch'essa viene utilizzata in \NomeProgetto{} per la stesura dei test.\\
\end{longtable}	
}



