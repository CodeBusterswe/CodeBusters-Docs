\appendix
\section{Glossario}
\section*{A}

\subsection*{Adjacency matrix}
La matrice delle adiacenze o matrice di connessione costituisce una particolare struttura dati comunemente utilizzata nella rappresentazione dei grafi finiti.

\section*{B}

\subsection*{Browser}
Applicazione per l'acquisizione, la presentazione e la navigazione di risorse sul web.
Il programma implementa le funzionalità di client per il protocollo HTTP, che regola il download delle risorse dai server web, e quelle di visualizzazione dei contenuti ipertestuali e di riproduzione di contenuti multimediali.

\section*{C}

\subsection*{Canberra}
La distanza di Canberra è una misura numerica della distanza tra coppie di punti in uno spazio vettoriale.

\subsection*{Chebyshev}
In matematica, la distanza di Čebyšëv, conosciuta anche come distanza della scacchiera o distanza di Lagrange, è una distanza su spazi vettoriali tale per cui la distanza tra due vettori è il valore massimo della loro differenza lungo gli assi.

\subsection*{Cluster}
Un cluster è un insieme di oggetti che presentano tra loro delle similarità e, allo stesso modo, delle dissimilarità con oggetti in altri cluster.
	
\subsection*{CSV}
Acronimo di Comma-Separated Values. Formato di file basato su file di testo utilizzato per l'importazione ed esportazione (ad esempio da fogli elettronici o \glo{database}) di una tabella di dati.

\section*{D}

\subsection*{D3.js}
È una libreria JavaScript per creare visualizzazioni dinamiche ed interattive partendo da dati organizzati, visibili attraverso un comune \glo{browser}.


\subsection*{Database}
Letteralmente "base di dati". Rappresenta la versione digitale di un archivio di informazioni, ossia memorizza e organizza grandi moli di dati all'interno di dischi rigidi.

\subsection*{Dataset}
Collezione di dati ordinati in tabella in cui ogni colonna rappresenta una dimensione e ogni riga un membro del dataset.

\section*{E}

\subsection*{Euclidea}
In matematica, la distanza euclidea è una distanza tra due punti, in particolare è una misura della lunghezza del segmento avente per estremi i due punti.

\section*{F}

\subsection*{Fastmap}
È un algoritmo di riduzione dimensionale per la visualizzazione di \glo{dataset} tradizionali e multimediali. Si basa sulla mappatura di oggetti in punti in uno spazio k-dimensionale.

\subsection*{Force field}
Grafico che traduce le distanze nello spazio a molte dimensioni in forze di attrazione e repulsione tra i punti proiettati nello spazio bidimensionale (o anche tridimensionale). Questo grafico esegue una riduzione dimensionale preservando, o addirittura evidenziando, le strutture presenti nei dati.

\subsection*{Form}
Il termine form indica la parte di interfaccia utente di un'applicazione che consente all'utente di inviare uno o più dati liberamente inseriti dallo stesso. Per descriverlo può essere utile la metafora della "scheda da compilare" per l'inserimento di dati. 

\section*{H}

\subsection*{Heatmap}
Grafico che trasforma la distanza tra i punti in colori più o meno intensi, permettendo così di comprendere quali oggetti sono più vicini tra loro. Per una buona visualizzazione è utile accompagnarlo con l'ordinamento dei dati in modo che le strutture presenti siano evidenziate. Questo grafico e la relativa operazione di ordinamento sono reperibili nella libreria \glo{D3.js}.

\section*{I}

\subsection*{Isomap}
È un efficiente algoritmo non lineare per la riduzione dimensionale. L'algoritmo offre un semplice metodo per stimare un \glo{manifold} basandosi su una stima dei \glo{neighbors} di ciascun punto.

\section*{L}

\subsection*{LLE}
Acronimo di Locally-Linear Embedding. È un algoritmo di riduzione dimensionale non lineare. Si basa sui più vicini \glo{neighbors} di ciascun punto e da un'ottimizzazione ad autovettori. 

\section*{M}

\subsection*{Manhattan}
In matematica, distanza di Manhattan (chiamata anche geometria del taxi) è un concetto geometrico secondo il quale la distanza tra due punti è la somma del valore assoluto delle differenze delle loro coordinate. 

\subsection*{Manifold}
Detta anche varietà differenziabile, è una struttura matematica su cui si basano alcuni metodi di riduzione della dimensionalità. Si tratta di uno spazio matematico dove localmente viene a ricrearsi uno spazio euclideo (di una specifica dimensione).

\section*{N}

\subsection*{Neighbors}
Sono il numero di vicini che vengono determinati sul \glo{manifold} in base alla distanza Euclidea. Possono essere scelti per gli algoritmi \glo{IsoMap} ed \glo{LLE}.

\section*{P}
\subsection*{PCA}
Acronimo di Principal Component Analysis. \'E una tecnica per la semplificazione dei dati utilizzata nell'ambito della statistica multivariata. Lo scopo della tecnica è quello di ridurre il numero più o meno elevato di variabili che descrivono un insieme di dati a un numero minore di variabili latenti, limitando il più possibile la perdita di informazioni.

\subsection*{PLMA}
Acronimo di Proiezione Lineare Multi Asse. Grafico che posiziona i punti dello spazio multidimensionale in un piano cartesiano (con assi "draggabili"), effettuando quindi sui dati una riduzione a due dimensioni.

\section*{R}

\subsection*{Repository}
Ambiente di un sistema informativo, in cui vengono gestiti i metadati, attraverso tabelle relazionali; l'insieme di tabelle, regole e motori di calcolo tramite cui si gestiscono i metadati prende il nome di metabase.

\section*{S}

\subsection*{Scatter plot}
Grafico di dispersione a due dimensioni. I dati sono riportati su uno spazio cartesiano: una dimensione sull'asse delle ascisse e una su quello delle ordinate.

\subsection*{Scatter plot matrix}
Grafico formato da più \glo{scatter plot} disposti a matrice, dove in ognuno vengono messe in relazione due dimensioni diverse. Esso aiuta l'analista a trovare dimensioni con forti correlazioni e più  dimensioni che danno la stessa informazione. Il grafico è reperibile nella libreria \glo{D3.js}.

\section*{T}

\subsection*{TSNE}
Acronimo di T-Distributed Stochastic Neighbor Embedding. È un algoritmo di riduzione dimensionale non lineare che si presta particolarmente all'embedding di \glo{dataset} ad alta dimensionalità in uno spazio a due o tre dimensioni, nel quale possono essere visualizzati tramite un grafico di dispersione. L'algoritmo modella i punti in modo che oggetti vicini nello spazio originale risultino vicini nello spazio a dimensionalità ridotta e oggetti lontani risultino lontani, cercando di preservare la struttura locale.

\section*{U}

\subsection*{UMAP}
Algoritmo di riduzione dimensionale per aumentare le prestazioni del clustering basato sulla densità.
Come \glo{t-SNE}, non preserva completamente la densità.
