\subsection{Test d'integrazione}
%I test di integrazione verificano come le componenti software si integrino tra di loro.
%In questa prima versione del \PdQ{} il gruppo non è in grado di stabilire dei test di integrazione, non avendo individuato e testato le componenti del prodotto software.
{

\renewcommand{\arraystretch}{1.5}
\renewcommand\extrarowheight{1.5pt}
\setlength\arrayrulewidth{1pt}
\begin{longtable}{ C{1.5cm} | C{12cm}| C{1.5cm} } 
		\rowcolor{coloreRosso}
		\textcolor{white}{\textbf{Codice}} & 
		\textcolor{white}{\textbf{Descrizione}} & 
		\textcolor{white}{\textbf{Stato}} \\
		\endfirsthead
		\rowcolor{white}\multicolumn{3}{c}{\textit{Continua nella pagina successiva...}}\\
	    \endfoot
	    \endlastfoot

\textbf{TI1} & 
Si verifica che il collegamento con il \glo{database} avvenga correttamente. & 
S\\

\textbf{TI2} & 
Si verifica che tutti i \glo{dataset} presenti nel \glo{database} siano raggiungibili. & 
S\\

\textbf{TI3} & 
Si verifica che le \glo{query} per il recupero dei dati avvengano con successo. & 
S\\

\textbf{TI4} & 
Si verifica che la chiusura del collegamento con il \glo{database} avvenga correttamente. & 
S\\

\textbf{TI5} & 
Si verifica che l'integrazione con la libreria di visualizzazione dei grafici sia gestita correttamente. & 
S\\

\textbf{TI6} & 
Si verifica che l'integrazione con la libreria di riduzione dimensionale sia gestita correttamente. & 
S\\

\rowcolor{white}\caption{Test di integrazione}
\label{testIntegrazione}
\end{longtable}
}
