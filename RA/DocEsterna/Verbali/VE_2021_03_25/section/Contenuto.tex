\section{Informazioni generali}
\begin{itemize}
\item \textbf{Motivo della riunione}: esposizione dell'applicazione allo stato attuale e chiarimento di alcuni dubbi.
\item \textbf{Luogo riunione}: videoconferenza tramite \glo{Skype}.
\item \textbf{Data}: \Data{}
\item \textbf{Orario d'inizio}: 11:00 
\item \textbf{Orario di fine}: 11:45 
\item \textbf{Partecipanti}:
	\begin{itemize}
	\item Membri del gruppo \textit{CodeBusters};
	\item Dott. Piccoli Gregorio.
	\end{itemize}
\end{itemize}

\section{Resoconto}
\begin{itemize}
 	\item \textbf{Esposizione dell'applicazione}: il gruppo inizialmente ha mostrato al Dott. Piccoli l'applicazione in ogni sua parte, alla ricerca di consigli su possibili migliorie da applicare nelle prossime settimane. Il proponente è risultato molto soddisfatto del lavoro svolto fino ad ora, soprattutto per quanto riguarda l'impianto grafico della web app e il fatto che tutti i grafici obbligatori siano stati implementati correttamente.

 	\item  \textbf{Proposta di sostituzione e aggiunta grafico}: il gruppo ha esposto al proponente l'idea di sostituire il grafico Heat Map con Adjacency Matrix, rendendoli indipendenti e leggermente diversi, motivando la scelta. La proposta è stata accolta in modo positivo. Ci si è poi soffermati sul grafico appena introdotto per risolvere i dubbi riguardanti all'applicare l'ordinamento per cluster dei dati. Il proponente ha lasciato piena libertà di scelta su questo campo.
 	
   \item \textbf{Ulteriori indicazioni}: un consiglio che il proponente ha fornito è stato quello di implementare più widget in aiuto all'utente, affinché possa costantemente sapere quale dataset ha caricato, quali riduzioni dimensionali ha applicato e quali matrici delle distanze ha già calcolato, senza dover ogni volta ricercarle nel menu. Questi saranno implementati sicuramente in seguito all'ingresso in RQ, come da \PdP{}.

\end{itemize}

\newpage