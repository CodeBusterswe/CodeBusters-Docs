\section{Informazioni generali}
\begin{itemize}
\item \textbf{Motivo della riunione}: Visualizzazione dell'applicazione.
\item \textbf{Luogo riunione}: videoconferenza tramite \glo{Skype}.
\item \textbf{Data}: \Data{}
\item \textbf{Orario d'inizio}: 9:30 
\item \textbf{Orario di fine}: 10:30 
\item \textbf{Partecipanti}:
	\begin{itemize}
	\item Membri del gruppo \textit{CodeBusters};
	\item Dott. Piccoli Gregorio.
	\end{itemize}
\end{itemize}

\section{Resoconto}
\begin{itemize}
		\item \textbf{Visualizzazione dell'applicazione}: il gruppo ha mostrato al dott. Piccoli l'applicazione allo stato corrente per avere un riscontro su possibili funzionalità mancanti. Il proponente ne è risultato soddisfatto e non ha identificato aggiunte necessarie o requisiti mancanti. Per questo motivo il gruppo si ritiene a sua volta soddisfatto del lavoro svolto e, da questo punto di vista, ritiene concluso il prodotto. 
		\item \textbf{Correzione all'impianto grafico}: il proponente ha comunque suggerito alcune modifiche stilistiche da apportare al prodotto prima della consegna in RA:
		\begin{itemize}
			\item Mostrando il grafico Scatterplot Matrix il proponente ci ha fatto notare come le coordinate dei punti, mostrate a video una volta portato il cursore sopra il grafico, siano di un colore poco coerente con il resto dell'impianto grafico dell'applicazione, oltre ad essere disallineate rispetto la legenda dei colori sopra di esse.
			\item Mostrando la funzionalità di salvataggio e caricamento della sessione il proponente ha ritenuto necessario soffermarsi sulla corrispettiva icona nel menu. Questa infatti rappresenta il logo del formato JSON, appunto per indicare l'utilizzo di tale formato per salvare e caricare la sessione. Questo però è effettivamente poco user friendly e suggerisce quindi di utilizzare l'icona di un disco rigido o floppy disk, già utilizzate da altri programmi e ormai conosciute da tutti gli utenti.  
		\end{itemize}
		La modifica di questi aspetti, seppur secondari alle reali funzionalità del prodotto, migliorano sicuramente la user experience. Il gruppo ha quindi compreso le correzioni suggerite e ne riconosce l'importanza, per questo nei prossimi giorni apporterà tali cambiamenti.
\end{itemize}
\newpage