\section{Introduzione}
\subsection{Scopo del documento}
Questo documento vuole definire le linee guida di tutti i processi istanziati dal gruppo \Gruppo{}. Sono presenti l'organizzazione e l'uso di tutte le risorse di sviluppo, oltre alle convenzioni che il gruppo decide di attuare sull'uso delle tecnologie, sullo stile di codifica e di scrittura. Ogni membro è obbligato a tenere in considerazione questo documento per garantire maggiore uniformità e coerenza del materiale prodotto.

\subsection{Scopo del capitolato}
Oggigiorno, anche i programmi più tradizionali gestiscono e memorizzano una grande mole di dati; di conseguenza servono software in grado di eseguire un'analisi e un'interpretazione delle informazioni.\\
Il \glo{capitolato} C4 ha come obiettivo quello di creare un'applicazione di visualizzazione di dati con numerose dimensioni in modo da renderle comprensibili all'occhio umano.  Lo scopo del prodotto sarà quello di fornire all'utente diversi tipi di visualizzazioni e di algoritmi per la riduzione dimensionale in modo che, attraverso un processo esplorativo, l'utilizzatore del prodotto possa studiare tali dati ed evidenziarne degli eventuali \glo{cluster}. 

\subsection{Glossario}
Per evitare ambiguità relative alle terminologie utilizzate, è stato compilato il \Glossariov{v4.0.0-1.8}. In questo documento sono riportati tutti i termini di particolare importanza e con un significato particolare. Questi termini sono evidenziati da una 'G' ad apice.

\subsection{Riferimenti}
\subsubsection{Riferimenti normativi}
\begin{itemize}	
	\item \textbf{Capitolato d'appalto C4 - HD Viz: visualizzazione di dati multidimensionali}:\\
	\textcolor{blue}{\url{https://www.math.unipd.it/~tullio/IS-1/2020/Progetto/C4.pdf}}.
\end{itemize}

\subsubsection{Riferimenti informativi}
\label{riferimenti}
\begin{itemize}
	\item \textbf{Standard ISO/IEC 12207:1995}: \\
	\textcolor{blue}{\url{https://www.math.unipd.it/~tullio/IS-1/2009/Approfondimenti/ISO_12207-1995.pdf}}
	\begin{itemize}
		\item Capitolo 5 - Primary life cycle processes (da pag. 10 a 24);
		\item Capitolo 6 - Supporting life cycle processes (da pag. 28 a 41);
		\item Capitolo 7 - Organizational life cycle processes (da pag. 42 a 47).
	\end{itemize}
	\item \textbf{ISO/IEC 9126}: \\
	\textcolor{blue}{\url{http://www.colonese.it/00-Manuali_Pubblicatii/07-ISO-IEC9126_v2.pdf}}
	\begin{itemize}
		\item Capitolo 2 - Il modello ISO/IEC 9126 (da pag. 12 a 24);
		\item Capitolo 4 - Utilizzo del modello:
		\begin{itemize}
			\item Paragrafo 4.1 - Processo di sviluppo e qualità (da pag. 31 a 34);
			\item Paragrafo 4.2 - Implementazione del modello in progetti reali (da pag. 34 a 37);
			\item Paragrafo 4.3 - Esempi di metriche (pag. 37).
		\end{itemize}
	\end{itemize}
	\item \textbf{Metriche}:\\ \textcolor{blue}{\url{https://it.wikipedia.org/wiki/Metriche_di_progetto}}
		\item \textbf{Slide T3 del corso Ingegneria del Software - Processi di ciclo di vita}:\\
	\textcolor{blue}{\url{https://www.math.unipd.it/~tullio/IS-1/2020/Dispense/L03.pdf}}
	\item \textbf{Slide FC1 del corso Ingegneria del Software - Amministrazione di progetto}:\\
	\textcolor{blue}{\url{https://www.math.unipd.it/~tullio/IS-1/2020/Dispense/FC1.pdf}}
	
	\item \textbf{Software Engineering - Ian Sommerville - 10 th Edition}: \\
	Parte 4 - Software management:
	\begin{itemize}
		\item Capitolo 25 - Configuration management:
		\begin{itemize}
			\item Paragrafo 25.1 - Version management (da pag. 735 a 740);
			\item Paragrafo 25.2 - System building (da pag. 740 a 745).
		\end{itemize}
	\end{itemize}
		
	\item \textbf{Guida Latex}:\\ \textcolor{blue}{\url{http://www.lorenzopantieri.net/LaTeX_files/LaTeXimpaziente.pdf}}.
	
	
\end{itemize}

	

