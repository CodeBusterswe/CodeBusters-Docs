\subsection{Gestione della qualità}
\subsubsection{Scopo}
Si occupa di stabilire una metrica precisa per tutti i servizi nell’ambito della verifica e della validazione, mantenendo un dato livello di qualità che rimanga uniforme e misurabile durante tutto il ciclo di vita del software.

\subsubsection{Aspettative}
\begin{itemize}
	\item Avere un continuo accertamento sulla qualità del prodotto, in modo che sia conforme con quella richiesta dal \glo{proponente};
	\item Avere un'organizzazione dei documenti qualitativa, al fine di velocizzare possibili modifiche e manutenzioni future;
	\item Avere un livello quantificabile della qualità dei processi attuati.
\end{itemize}

\subsubsection{Descrizione}
Il documento \PdQv{v4.0.0-1.8} racchiude i livelli di qualità che il gruppo si è posto di mantenere e le misurazioni oggettive che descrivono gli stati di avanzamento.
Il gruppo vuole perseguire la qualità del prodotto agendo in modo \glo{sistematico}, fornendo quindi un ruolo a ciascun componente del gruppo, gestendo risorse e procedure per ogni processo in atto, effettuando analisi statica e dinamica sul codice prodotto in modo non invasivo.
\subsubsection{Istanziazione del processo}
\paragraph{Classificazione delle metriche}
Le metriche scelte ed utilizzate dal gruppo sono identificabili tramite un codice univoco così composto: 
\begin{center}
\textbf{M[Utilizzo][IdNumerico]}
\end{center}
Singolarmente ciascun campo rappresenta:
\begin{itemize}
	\item \textbf{Utilizzo}: se la metrica è per:
			\begin{itemize}
				\item \textbf{PC}: processo;
				\item \textbf{PD}: prodotto.
			\end{itemize}
	\item \textbf{IdNumerico}: codice numerico crescente che parte da 1 e distingue le metriche dello stesso sottoinsieme (PC o PD). 
\end{itemize}

\subsubsection{Metriche}
Il processo di gestione della qualità non fa uso di metriche qualitative particolari.

\subsubsection{Strumenti}
Non sono stati identificati degli strumenti particolari per la gestione di qualità.