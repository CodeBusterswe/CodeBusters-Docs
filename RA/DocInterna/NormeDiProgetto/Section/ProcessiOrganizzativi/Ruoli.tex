\subsection{Gestione organizzativa}
\subsubsection{Scopo}
In questa sezione vengono esposte le modalità di coordinamento adottate dal gruppo, che sono:
\begin{itemize}
\item Adottare un modello organizzativo per l'individuazione dei rischi che potrebbero verificarsi;
\item Definire un modello di sviluppo da adottare;
\item Pianificare il lavoro rispettando le scadenze;
\item Calcolo del prospetto economico in base ai ruoli;
\item Determinare un bilancio finale sulle spese.
\end{itemize}

\subsubsection{Aspettative}
\begin{itemize}
\item Ottenere una pianificazione ragionevole delle attività da seguire;
\item Avere un coordinamento delle attività, assegnando ruoli, compiti e semplificando la comunicazione tra i membri;
\item Riuscire a regolare le attività e renderle economiche.
\end{itemize}

\subsubsection{Descrizione}
Le attività  di gestione sono: 
\begin{itemize}
\item Assegnazione dei ruoli e dei compiti;
\item Inizio e definizione dello scopo;
\item Istanziazione dei processi;
\item Pianificazione e stima di tempi, risorse e costi;
\item Esecuzione e controllo;
\item Revisione e valutazione periodica delle attività.
\end{itemize}
\subsubsection{Istanziazione del processo}
\paragraph{Ruoli di progetto}
Ogni membro del gruppo deve, a rotazione, ricoprire almeno una volta ciascun ruolo di progetto per comprendere le differenze tra le diverse figure aziendali. Tali ruoli sono descritti di seguito.

\mbox{}

\textbf{Responsabile di progetto}\\
Il responsabile di progetto ricopre un ruolo fondamentale in quanto si occupa delle comunicazioni con il proponente e committente. Inoltre egli deve svolgere i seguenti compiti:
\begin{itemize}
\item Pianificare;
\item Gestire;
\item Controllare;
\item Coordinare.
\end{itemize}

\mbox{}

\textbf{Amministratore di progetto}\\
L'amministratore deve avere il controllo dell'ambiente di lavoro ed essere di supporto al team. Inoltre egli deve: 
\begin{itemize}
\item Dirigere le infrastrutture di supporto;
\item Controllare versioni e configurazioni;
\item Risolvere i problemi che riguardano la gestione dei processi;
\item Gestire la documentazione.
\end{itemize}

\mbox{}

\textbf{Analista}\\
L'analista si occupa dell'analisi dei problemi e del dominio applicativo. Questa figura ha le seguenti responsabilità:
\begin{itemize}
\item Studio del dominio del problema; 
\item Redazione della documentazione: \AdR{} e \SdF{};
\item Definizione dei requisiti e della sua complessità.
\end{itemize}

\mbox{}

\textbf{Progettista}\\
Il progettista si occupa dell'aspetto tecnico e tecnologico del progetto, segue lo sviluppo e non la manutenzione del prodotto. Inoltre egli deve scegliere: 
\begin{itemize}
\item Un'architettura adatta per il sistema del prodotto in base alle tecnologie scelte;
\item Il modo più efficiente per ottimizzare l'aspetto tecnico del progetto.
\end{itemize}

\mbox{}

\textbf{Programmatore}\\
Il programmatore si occupa della parte di codifica in base alle specifiche fornite dal progettista, operando con ottica di manutenibilità del codice. Inoltre egli deve creare e gestire componenti di supporto per la verifica e la validazione del codice. 

\mbox{}

\textbf{Verificatore}\\
Il \glo{verificatore} è presente durante tutta l'attività del progetto. Egli deve: 
\begin{itemize}
\item Controllare i prodotti in fase di revisione, utilizzando le tecniche e gli strumenti definiti nelle \NdP{}; 
\item Evidenziare gli errori e segnalarli all'autore del prodotto in questione.
\end{itemize}
