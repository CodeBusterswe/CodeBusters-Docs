\subsection{Fornitura}
\subsubsection{Scopo}
\begin{itemize}
\item Capire quali strumenti e competenze sono necessarie per la realizzazione dei prodotti;

\item Redigere un documento che descriva nel dettaglio come si intenda organizzare il lavoro che porterà alla realizzazione del prodotto;

\item Stabilire se il materiale prodotto rispetti determinati vincoli e sia di qualità.

\end{itemize}

\subsubsection{Aspettative}
\begin{itemize}
	\item Avere una chiara struttura dei documenti;
	\item Definire i tempi di lavoro;
	\item Chiarire dubbi e stabilire vincoli con il proponente.
\end{itemize}

\subsubsection{Descrizione}
Nel processo di fornitura si scelgono le procedure e le risorse atte a perseguire lo sviluppo del progetto. Dopo aver ricevuto le richieste del proponente, il gruppo redige uno studio di fattibilità e la fornitura può essere avviata per soddisfare tali richieste.\\
Il proponente e il fornitore stipuleranno un contratto per la consegna del prodotto.\\
Si dovrà poi sviluppare un piano di progetto partendo dalla determinazione delle procedure e delle risorse necessarie.
Da quel momento, fino alla consegna del prodotto, il \PdP{} scaglionerà le attività da svolgere.
\subsubsection{Istanziazione del processo}
\paragraph{Inizializzazione} 
Gli analisti devono effettuare una valutazione dei capitolati proposti e redigere un documento per ciascuno di essi. Il documento deve evidenziare gli aspetti negativi e positivi del capitolato, argomentando le motivazioni che portano il gruppo a scartarlo o a sceglierlo. Infine, deve essere indicato il capitolato scelto dal gruppo.

\paragraph{Preparazione alla risposta}
Il gruppo deve redigere una lettera di presentazione con cui si dichiara l'impegno nella realizzazione del prodotto proposto dal capitolato scelto.


\paragraph{Pianificazione}
Il gruppo deve redigere un piano per la gestione del progetto e della qualità. A \ref{pdp}  vengono descritte le sezioni che compongono i due documenti e quindi i loro contenuti.

\paragraph{Esecuzione e controllo}
Il gruppo deve seguire ed implementare il piano di progetto e controllare l'avanzamento, i costi e la qualità dei prodotti in tutto il loro ciclo di vita.

\paragraph{Revisione e valutazione}
Il gruppo deve coordinare le revisioni delle attività svolte, deve eseguire la verifica e la validazione del prodotto garantendo che questo rispetti le aspettative del proponente.

\paragraph{Collaudo e rilascio del prodotto}
Il gruppo deve rilasciare un prodotto conforme alle aspettative, presentando quindi al proponente e al committente:
\begin{itemize}
	\item Codice sorgente del prodotto;
	\item Documentazione di prodotto, alla quale si aggiungono: 
	\begin{itemize}
		\item Manuale utente;
		\item Manuale sviluppatore.
	\end{itemize}
\end{itemize}

Durante il collaudo il gruppo dimostrerà al committente che:
\begin{itemize}
	\item I test presenti nel \textit{Piano di Qualifica v4.0.0-1.8} sono stati eseguiti e hanno ottenuto risultati conformi alle metriche;
	\item I requisiti obbligatori sono stati rispettati e le aspettative minime del proponente sono state soddisfatte;
	\item Altri requisiti non obbligatori sono stati implementati.
\end{itemize}

\paragraph{Rapporti con l’azienda proponente Zucchetti S.p.A.}
Tale rapporto permette di:
\begin{itemize}
	\item Comprendere meglio il dominio e chiarire dubbi su di esso;
	\item Individuare strategie di attuazione per soddisfare le aspettative del proponente;
	\item Determinare i requisiti;
	\item Confrontarsi sulla \glo{product baseline};
	\item Accordarsi sulla qualifica del prodotto;
	\item Confrontarsi continuamente sugli incrementi del prodotto;
	\item Discutere e rivedere, se necessario, i test di accettazione.
\end{itemize}
Avvenuta la consegna del prodotto, il gruppo \textit{\textbf{non seguirà}} l'attività di manutenzione dello stesso.

\subsubsection{Metriche}
Il processo di fornitura non fa uso di metriche qualitative particolari.

\subsubsection{Strumenti}
Gli strumenti utilizzati in questo processo comprendono:
\begin{itemize}
	\item \textbf{Excel}: utilizzato per creare grafici, eseguire calcoli e presentare tabelle organizzative; 
	\begin{center}
		\textcolor{blue}{\url{https://www.microsoft.com/it-it/microsoft-365/excel}}
	\end{center}
	\item \textbf{Microsoft Planner}: utilizzato per gestire le \glo{task} che ciascun membro del gruppo deve svolgere; 
	\begin{center}
		\textcolor{blue}{\url{https://www.microsoft.com/it-it/microsoft-365/business/task-management-software}}
	\end{center}
	\item \textbf{GanttProject}: utilizzato per creare i grafici di \glo{Gantt} relativi alla pianificazione.
	\begin{center}
		\textcolor{blue}{\url{https://www.ganttproject.biz/}}
	\end{center}
\end{itemize}
