\subsection{Sviluppo}
\subsubsection{Scopo}
Il processo di sviluppo definisce i compiti e le attività da intraprendere per ottenere il prodotto finale richiesto dal proponente.
\subsubsection{Aspettative}
\begin{itemize}
\item Determinare vincoli tecnologici;
\item Determinare gli obiettivi di sviluppo;
\item Determinare vincoli di design;
\item Realizzare un prodotto finale che superi i test e soddisfi requisiti e richieste del proponente.
\end{itemize}
\subsubsection{Descrizione}
Il processo di sviluppo comprende le attività e i compiti dello sviluppatore, tra cui quelle per l'analisi dei requisiti, la progettazione, la codifica, l'integrazione, l'installazione e l'accettazione del prodotto software.

\subsubsection{Istanziazione del processo}

\paragraph{Analisi dei requisiti}
\textbf{Scopo} \mbox{} \\
Lo scopo dell'analisi dei requisiti è individuare e poi riportare in un documento tutti i requisiti individuati e i casi d'uso del sistema.\\ \mbox{} \\
\textbf{Aspettative} \mbox{} \\
L'obiettivo dell'attività di analisi dei requisiti consiste nella creazione della documentazione formale contenente tutti i requisiti richiesti dal proponente.\\ \mbox{} \\
\textbf{Requisiti} \mbox{} \\
I requisiti si raccolgono tramite:
\begin{itemize}
\item Lettura del capitolato;
\item Visione della presentazione del capitolato;
\item Confronto interno tra i membri del gruppo;
\item Confronto esterno con il proponente.
\end{itemize}
\textbf{Casi d'uso} \mbox{} \\
È un diagramma che descrive uno scenario di utilizzo del prodotto. È costituito da:
\begin{itemize}
\item Codice identificativo e titolo;
\item Attore primario;
\item Precondizioni;
\item Postcondizioni;
\item Scenario principale;
\item Generalizzazioni;
\item Estensioni.
\end{itemize}
\textbf{Codice identificativo casi d'uso} \mbox{} \\
Un caso d'uso è così identificato:

\begin{center}
\textbf{UC[Numero caso d'uso].[Caso d'uso figlio] - [Titolo caso d'uso}]
\end{center}
		
dove "caso d'uso figlio" è il sottocaso del caso d'uso principale. \\ \mbox{} \\
\textbf{Struttura dei requisiti} \label{para:requisiti}
\begin{itemize}
\item \textbf{Codice identificativo:} 
\begin{center}
\textbf{R[Importanza][Tipologia][Codice]}
\end{center}
 		
Per \textbf{Importanza} s'intende un numero da 1 a 3 che rappresenta:
\begin{enumerate}
\item Requisito obbligatorio;
\item Requisito desiderabile ma non essenziale per il funzionamento;
\item Requisito opzionale.
\end{enumerate}
Per \textbf{Tipologia} s'intende una lettera che rappresenta la natura del requisito:
\begin{itemize}
\item \textbf{V}: Vincolo;
\item \textbf{P}: Prestazionale;
\item \textbf{Q}: Qualitativo;
\item \textbf{F}: Funzionale.
\end{itemize}

Per \textbf{Codice} s'intende un identificativo univoco del requisito espresso in forma gerarchica padre/figlio.
\item \textbf{Classe:} per rendere la tabella più esplicativa viene riportata nuovamente l'importanza del requisito, nonostante sia già scritta nel codice identificativo;
\item \textbf{Descrizione:} una sintetica descrizione del requisito;
\item \textbf{Fonti:} vi sono diverse fonti da cui possono derivare i requisiti; l'origine dei requisiti viene quindi qui specificata. 
\end{itemize}

\paragraph{Progettazione}
\textbf{Scopo} \mbox{} \\
Questa attività ha la funzione di definire una soluzione al problema proposto dal capitolato, basandosi sull'analisi dei requisiti.
Mentre quest'ultima divide il problema nei requisiti da soddisfare, la progettazione incorpora le parti specificando le funzionalità dei sottosistemi e riconducendo ad un'unica soluzione. \\ 


\textbf{Aspettative} \mbox{} \\
Riuscire ad arrivare, al termine di questa attività, ad una architettura di sistema.\\ \mbox{}

\textbf{Tecnology baseline} \mbox{} \\
Motiva le tecnologie, i \glo{framework} e le librerie selezionate per la realizzazione del prodotto. Sarà il progettista ad occuparsene e dovrà contenere:
\begin{itemize}
\item Tecnologie adottate, motivando le scelte;
\item La relazione tra ciascuna componente e il requisito che soddisfa, per avere un tracciamento;  
\item \glo{\textit{Proof of Concept}}, ovvero un prototipo per dimostrare le funzionalità del prodotto.
\end{itemize}

\mbox{}

\textbf{Product baseline} \mbox{} \\
La product baseline deve includere:
\begin{itemize}
	\item \glo{Design pattern} utilizzati, accompagnati da una descrizione;
	\item Diagrammi UML delle classi, di attività, di sequenza e dei package;
	\item Una definizione delle classi, evitando nomi e funzionalità ridondanti;
	\item Tracciamento delle classi, in modo che ciascun requisito sia soddisfatto da una classe;
	\item Test di unità su ogni componente, in modo da verificare il corretto funzionamento.
\end{itemize}
\paragraph{Codifica}

\textbf{Scopo}  \mbox{} \\
L'attività di codifica ha il fine di concretizzare la progettazione con la programmazione del software vero e proprio.\\

\textbf{Aspettative} \mbox{} \\
Questa attività dovrà avere come risultato un prodotto software avente le caratteristiche e i requisiti concordati con il proponente. Il codice generato dovrà rispettare alcune norme per poter essere leggibile e per facilitarne la  manutenzione, modifica, verifica e validazione.\\

\textbf{Stile della codifica}
\begin{itemize}
\item \textbf{Indentazione}: i blocchi di codice innestati dovranno avere un'indentazione di quattro spazi;
\item \textbf{Parentesi}: le parentesi devono essere sempre usate, anche se il corpo del costrutto è vuoto o ha una sola istruzione. La parentesi aperta dovrà essere inserita nella stessa riga di dichiarazione del costrutto, separate da uno spazio; 
\item \textbf{Metodi}: il nome dei metodi dovrà iniziare con lettera minuscola e, se composto da più parole, le successive dovranno iniziare con lettera maiuscola. È preferibile mantenere metodi brevi, con poche righe di codice;
\item \textbf{Classi}: il nome delle classi dovrà sempre iniziare con la lettera maiuscola e, come per i metodi, se composto da più parole, le successive dovranno iniziare con la lettere maiuscola;
\item \textbf{Variabili}: il nome delle variabili deve sempre essere scritto in minuscolo e in inglese. Se il nome è composta da più parole, la seconda dovrà iniziare con la lettera maiuscola;
\item \textbf{Costanti}: il nome deve essere sempre scritto in maiuscolo e in inglese. Se il nome è composto da più parole, queste dovranno essere separate dal carattere "\_";
\item \textbf{Univocità dei nomi}: tutti i costrutti dovranno avere nomi univoci e significativi;
\item \textbf{Commenti}: i commenti dovranno essere inseriti prima dell'inizio del costrutto e presentati in lingua italiana;
\item \textbf{File}: dovranno avere un nome che inizia per lettera maiuscola che ne specifichi il contenuto;
\item \textbf{Altro}:
	\begin{itemize}
		\item Utilizzare la \glo{method chaining} per l'invocazione dei metodi. Se la catena di metodi è formata da più di due metodi, scrivere un metodo per riga.
	\end{itemize}
\end{itemize}
\textbf{Ricorsione} \mbox{} \\
Onde evitare un'eccessiva allocazione di memoria è preferibile, quando possibile, evitare la ricorsione.

\newpage

\subsubsection{Metriche}
\paragraph{Metriche per la portabilità}
Alcuni parametri per capire meglio la tabella seguente:
\begin{itemize}
\item \textbf{\glo{Browser} supportati (B\textsubscript{sup})}: browser dove il software può essere eseguito;
	\item \textbf{\glo{Browser} richiesti (B\textsubscript{ric})}: numero di browser compatibili con il programma richiesti dal proponente.
\end{itemize}

\renewcommand{\arraystretch}{1.5}
\renewcommand\extrarowheight{1.5pt}
\begin{longtable}{C{1.5cm} C{4.5cm} C{5.5cm} C{5cm}}
		\rowcolor{coloreRosso}
		\textcolor{white}{\textbf{Codice}} & 
		\textcolor{white}{\textbf{Nome}} & 
		\textcolor{white}{\textbf{Descrizione}} & 
		\textcolor{white}{\textbf{Formula}} \\
		\endfirsthead
	    \endfoot
	    \rowcolor{white}\caption{Metriche per garantire la portabilità del prodotto}
	    \endlastfoot
		\textbf{MPD10} & 
		Versioni del browser supportate & 
		La percentuale di versioni di browser supportate dal prodotto. &
		$(\frac{B_{sup}}{B_{ric}}) \cdot 100 $  \\
\end{longtable} 

\paragraph{Metriche per la funzionalità}

Alcuni parametri per capire meglio la tabella seguente:
\begin{itemize}
\item \textbf{Funzionalità mancanti (N\textsubscript{fm})}: numero di funzionalità non implementate;
\item \textbf{Funzionalità individuate (N\textsubscript{fi})}: numero di funzionalità individuate.

\end{itemize}
\renewcommand{\arraystretch}{1.5}
\renewcommand\extrarowheight{1.5pt}
\begin{longtable}{C{1.5cm} C{4.5cm} C{5.5cm} C{5cm}}
		\rowcolor{coloreRosso}
		\textcolor{white}{\textbf{Codice}} & 
		\textcolor{white}{\textbf{Nome}} & 
		\textcolor{white}{\textbf{Descrizione}} & 
		\textcolor{white}{\textbf{Formula}} \\
		\endfirsthead
	    \endfoot
	    \rowcolor{white}\caption{Metriche per garantire che i requisiti siano rispettati}
	    \endlastfoot
		\textbf{MPD3} & 
		Copertura dei requisiti (CDR) & 
		Descrive quanti requisiti sono stati implementati nel prodotto software. &
		$(1 - \frac{N_{fm}}{N_{fi}}) \cdot 100 $ \\
\end{longtable} 

\paragraph{Metriche per la manutenibilità}
Alcuni parametri per comprendere la tabella seguente:
\begin{itemize}
	\item \textbf{Linee di commento (N\textsubscript{com})}: numero di righe di commento;
	\item \textbf{Linee di commento (N\textsubscript{cod})}: numero di righe di codice.
\end{itemize}
\renewcommand{\arraystretch}{1.5}
\renewcommand\extrarowheight{1.5pt}
\begin{longtable}{C{1.5cm} C{4.5cm} C{5.5cm} C{5cm}}
		\rowcolor{coloreRosso}
		\textcolor{white}{\textbf{Codice}} & 
		\textcolor{white}{\textbf{Nome}} & 
		\textcolor{white}{\textbf{Descrizione}} & 
		\textcolor{white}{\textbf{Formula}} \\
		\endfirsthead
	    \endfoot
	    \rowcolor{white}\caption{Metriche per garantire manutenibilità del prodotto}
	    \endlastfoot
		\textbf{MPD5} & 
		Average Cyclomatic complexity (ACC) & 
		Indica in numero di cammini indipendenti presenti nel programma. & Misurabile attraverso il grafo di controllo di flusso.\\
		\textbf{MPD9} & 
		Comprensione del codice & 
		Può essere dedotta dal rapporto tra linee di codice e linee di commenti. &
		$(\frac{N_{com}}{N_{cod}}) \cdot 100 $ \\
		
\end{longtable}
\paragraph{Metrica per l'efficienza}
\renewcommand{\arraystretch}{1.5}
\renewcommand\extrarowheight{1.5pt}
\begin{longtable}{C{1.5cm} C{4.5cm} C{5.5cm} C{5cm}}
		\rowcolor{coloreRosso}
		\textcolor{white}{\textbf{Codice}} & 
		\textcolor{white}{\textbf{Nome}} & 
		\textcolor{white}{\textbf{Descrizione}} & 
		\textcolor{white}{\textbf{Formula}} \\
		\endfirsthead
	    \endfoot
	    \rowcolor{white}\caption{Metrica per garantire efficienza del prodotto}
	    \endlastfoot
		\textbf{MPD8} & 
		Tempo medio di risposta & 
		Tempo medio impiegato dal software per rispondere a una richiesta utente o svolgere un'attività di sistema. &
		- \\
\end{longtable}		
\paragraph{Metriche per l'usabilità}
\renewcommand{\arraystretch}{1.5}
\renewcommand\extrarowheight{1.5pt}
\begin{longtable}{C{1.5cm} C{4.5cm} C{5.5cm} C{5cm}}
		\rowcolor{coloreRosso}
		\textcolor{white}{\textbf{Codice}} & 
		\textcolor{white}{\textbf{Nome}} & 
		\textcolor{white}{\textbf{Descrizione}} & 
		\textcolor{white}{\textbf{Formula}} \\
		\endfirsthead
	    \endfoot
	    \rowcolor{white}\caption{Metriche per garantire usabilità del prodotto}
	    \endlastfoot
		\textbf{MPD6} & 
		Facilità di utilizzo & 
		Numero di click necessari con cui l'utente raggiunge la funzionalità cercata. &
		- \\
		\textbf{MPD7} & 
		Facilità apprendimento funzionalità & 
		Numero di minuti necessari all'utente per apprendere le funzionalità del prodotto software. & - \\
\end{longtable}

\newpage

\paragraph{Metrica per l'affidabilità}
Alcuni parametri per comprendere la tabella seguente:
\begin{itemize}
\item \textbf{Test falliti (T\textsubscript{fal})}: test eseguiti sul programma ma falliti;
	\item \textbf{Test eseguiti (T\textsubscript{ese})}: test totali eseguiti sul programma.
\end{itemize}
\renewcommand{\arraystretch}{1.5}
\renewcommand\extrarowheight{1.5pt}
\begin{longtable}{C{1.5cm} C{4.5cm} C{5.5cm} C{5cm}}
		\rowcolor{coloreRosso}
		\textcolor{white}{\textbf{Codice}} & 
		\textcolor{white}{\textbf{Nome}} & 
		\textcolor{white}{\textbf{Descrizione}} & 
		\textcolor{white}{\textbf{Formula}} \\
		\endfirsthead
	    \endfoot
	    \rowcolor{white}\caption{Metrica per garantire affidabilità del prodotto}
	    \endlastfoot
		\textbf{MPD4} & 
		Densità di failure & 
		Indica l'affidabilità del prodotto e si può ricavare dalla percentuale di test falliti sui test eseguiti. &
		$(\frac{T_{fal}}{T_{ese}}) \cdot 100 $ \\
\end{longtable}		

\subsubsection{Strumenti}
Gli strumenti utilizzati in questo processo comprendono:
\begin{itemize}
	\item \textbf{DruidJS}: libreria per la riduzione dimensionale sui dati caricati nella \glo{web app}. Essa fornisce diversi algoritmi a questo scopo, tra cui i più interessanti e utilizzati in ambienti reali di analisi dei dati;
	\begin{center}
		\textcolor{blue}{\url{https://github.com/saehm/DruidJS}}
	\end{center}	  
	\item \textbf{D3.js}: libreria per la manipolazione del DOM (Document Object Model). Permette la realizzazione di molti tipi diversi di grafici da dati reali;
	\begin{center}
		\textcolor{blue}{\url{https://d3js.org/}}
	\end{center}
	\item \textbf{Node.js}: runtime \glo{JavaScript} utilizzato per creare applicazioni di rete scalabili;
	\begin{center}
		\textcolor{blue}{\url{https://nodejs.org/it/}}
	\end{center}
	\item \textbf{Express.js}: \glo{framework} standard per le applicazioni che utilizzano \glo{Node.js} per la comunicazione tra \glo{front-end} e \glo{back-end};
	\begin{center}
		\textcolor{blue}{\url{https://expressjs.com/it/}}
	\end{center}
	\item \textbf{PostgreSQL}: \glo{database} scelto per la sua efficienza e affidabilità nel lungo periodo;
		\begin{center}
		\textcolor{blue}{\url{https://www.postgresql.org/}}
	\end{center}
	\item \textbf{Babel}: transcompiler {JavaScript} per compilare tutte le nuove istruzioni introdotte con ES6 in istruzioni retrocompatibili;
	\begin{center}
		\textcolor{blue}{\url{https://babeljs.io/}}
	\end{center}
	\item \textbf{Npm}: packet manager per il linguaggio \glo{JavaScript};
	\begin{center}
		\textcolor{blue}{\url{https://www.npmjs.com/}}
	\end{center}
	\item \textbf{Bootstrap}: \glo{framework} \glo{CSS} che velocizza la scrittura della parte di presentazione della \glo{web app} con classi predefinite per elementi \glo{HTML} specifici;
	\begin{center}
		\textcolor{blue}{\url{https://getbootstrap.com/}}
	\end{center}
	\item \textbf{React}: libreria \glo{JavaScript} per facilitare la creazione di interfacce utente, migliorare i tempi di renderizzazione della pagina, aumentare la manutenibilità e la facilità di testing;
	\begin{center}
		\textcolor{blue}{\url{https://it.reactjs.org/}}
	\end{center}
	\item \textbf{Visual Studio Code}: IDE versatile ed estendibile scelto per lo sviluppo del codice.
	\begin{center}
		\textcolor{blue}{\url{https://code.visualstudio.com/}}
	\end{center}
\end{itemize}

