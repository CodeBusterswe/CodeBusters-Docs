\section{C5 - PORTACS - Point Of Interest Oriented Real-Time Anti Collision System}

\subsection{Descrizione del capitolo}
Il capitolato propone la realizzazione di un sistema per il monitoraggio e la gestione di unità presenti in una mappa.\\
Ogni unità (che può essere un robot, un muletto o un'auto a guida automatica) ha un punto di partenza nella mappa e una lista di POI (Point Of Interest) che deve raggiungere.
Il sistema dovrà indicare ad ogni unità la prossima mossa che dovrà fare in base alla posizione, direzione e velocità di tutte le altre. Ognuna dovrà inviare costantemente questi parametri al sistema, in modo da evitare incidenti e ingorghi.\\
L'azienda proponente è \textit{Sanmarco Informatica}

\subsection{Tecnologie coinvolte}
Il proponente non impone specifiche tecnologie per lo sviluppo della piattaforma ma consiglia l'utilizzo di :
\begin{itemize}
\item \glo{Node.js} per la parte di \glo{back-end};
\item \glo{React.js} o \glo{Next.js} per lo sviluppo di \glo{front-end};
\item \glo{Socket}, ossia una libreria per sviluppare applicazioni \glo{real-time}, facilmente integrabile con \glo{Node.js}.
\end{itemize}

\subsection{Vincoli}
L'azienda proponente pone i seguenti obiettivi da raggiungere:

\begin{itemize}
\item Il sistema deve avere una visualizzazione in \glo{real-time} della mappa e della posizione delle singole unità;
\item L'interfaccia utente (presente in ogni unità) deve prevedere quattro frecce direzionali, un pulsante start/stop e un indicatore della velocità attuale.
\end{itemize}

La versione finale del software deve essere in grado di accettare i seguenti input:
\begin{itemize}
\item Scacchiera oppure mappa con percorsi predefiniti e relativi vincoli (sensi unici, numero massimo di unità contemporanee);
\item Definizione dei POI (aree di carico/scarico e sosta).
\end{itemize}

\subsection{Aspetti positivi}
\begin{itemize}
\item Il proponente non obbliga l'implementazione di algoritmi di ricerca operativa per l'ottimizzazione dei percorsi;
\item Non si richiede la geo-localizzazione interna o esterna. Questa verrà simulata, riducendo il tempo dedicato alla parte di codifica;
\item L'utilizzo di tecnologie per la gestione \glo{real-time} è parso un argomento molto interessante e ha suscitato curiosità nel gruppo;
\item L'azienda proponente ha dimostrato disponibilità per ulteriori chiarimenti.
\end{itemize}

\subsection{Aspetti critici}
\begin{itemize}
	\item  L'azienda proponente non suggerisce tecnologie apprese in ambito universitario o già conosciute dai membri del gruppo;
	\item  Complessità elevata rispetto ad altri capitolati.
\end{itemize}

\subsection{Conclusioni}
Il capitolato ha attirato l'attenzione e stimolato l'interesse di tutti i membri del gruppo, in particolar modo per il contesto applicativo che risulta essere un argomento attuale, seppur molto complesso.
Per via del livello di difficoltà e dell'impegno stimato, il gruppo ha ritenuto troppo elevata la mole di lavoro necessaria e ha deciso di orientarsi verso altri capitolati. 

