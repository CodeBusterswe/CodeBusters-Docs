\section{Informazioni generali}
\begin{itemize}
\item \textbf{Motivo della riunione}: 
\begin{itemize}
	\item Individuazione nuovi requisiti e suddivisione del lavoro;
	\item Organizzazione e suddivisione dei lavori.
\end{itemize}
\item \textbf{Luogo riunione}: videoconferenza tramite server \glo{Discord}.
\item \textbf{Data}: \Data{};
\item \textbf{Orario d'inizio}: 9:30;
\item \textbf{Orario di fine}: 11:00;
\item \textbf{Partecipanti}:
	\begin{itemize}
	\item \BM{}
	\item \SG{}
	\item \SP{}
	\item \ZM{}
	\item \PA{}
	\item \RA{}
	\item \SH{}
	\end{itemize}
\end{itemize}

\newpage
\section{Resoconto}
\begin{itemize}
	\item \textbf{Punto della situazione}: in seguito all'esito della \textit{Revisione di Qualifica} il gruppo si è riunito per fare il punto della situazione sullo stato dell'applicazione e per discutere su come proseguire nelle prossime settimane, fino alla consegna in RA.
	L'app è praticamente completa, soddisfa infatti tutti i requisti obbligatori e la maggior parte di quelli desiderabili e opzionali individuati dal proponente e dal gruppo stesso. Nell'ultima settimana si è inoltre raggiunto il 90\% di code coverage, sanando i problemi nella fase di testing riscontrati in RQ. 
Avendo ancora tempo a disposizione il gruppo ha quindi deciso di introdurre degli ulteriori requisiti, per aggiungere valore al prodotto. 
	
	\item \textbf{Discussione sui nuovi requisiti da aggiungere}: il gruppo, sfruttando la facile estensibilità della famiglia di algoritmi di riduzione dimensionale utilizzati, ha deciso di aggiungere l'algoritmo \glo{PCA (principal component analysis)}. Inoltre si è deciso di introdurre il processo di normalizzazione, procedimento che permette di mantenere la proporzione dei dati e allo stesso tempo li limita in uno spazio più contenuto, in modo da non perdere informazione ma aumentarne la leggibilità nel grafico.
	Entrambi i requisiti dovrebbero essere abbastanza semplici da soddisfare in poco tempo, così da poterci dedicare anche alla correzione e alla redazione dei documenti.
	
	\item \textbf{Suddivisione del lavoro}: il gruppo si è quindi organizzato riguardo a come procedere con il lavoro nelle prossime settimane. Visto il buono stato dell'applicazione ci si vuole concentrare in particolare sulla correzione dei documenti per sistemare tutti i problemi identificabili.
Nella prossima settimana si vogliono inoltre soddisfare i nuovi requisiti e, tempo permettendo, introdurre nuovi test per preparare al meglio il prodotto al collaudo. 
A parere comune sarà necessario anche dedicarsi a fare debugging per riconoscere possibili errori non ancora individuati.
	
\end{itemize}

\newpage