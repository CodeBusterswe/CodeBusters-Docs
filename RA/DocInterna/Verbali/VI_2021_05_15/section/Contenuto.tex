\section{Informazioni generali}
\begin{itemize}
\item \textbf{Motivo della riunione}: 
\begin{itemize}
	\item Discussione sull'incontro con il proponente;
	\item Situazione dei documenti;
	\item Preparazione presentazione per la RA.
\end{itemize}
\item \textbf{Luogo riunione}: videoconferenza tramite server \glo{Discord}.
\item \textbf{Data}: \Data{};
\item \textbf{Orario d'inizio}: 10:00;
\item \textbf{Orario di fine}: 11:30;
\item \textbf{Partecipanti}:
	\begin{itemize}
	\item \BM{}
	\item \SG{}
	\item \SP{}
	\item \ZM{}
	\item \PA{}
	\item \RA{}
	\item \SH{}
	\end{itemize}
\end{itemize}

\newpage
\section{Resoconto}
\begin{itemize}
	\item  \textbf{Discussione sull'incontro con il proponente}: il gruppo si è soffermato inizialmente sull'incontro avuto con il proponente il giorno prima (14-05-2021). In generale il gruppo si ritiene molto soddisfatto del lavoro svolto e, durante la chiamata stessa, sono state apportate le modifiche stilistiche sul prodotto suggerite dal dott. Piccoli. 
	A questo punto si ritiene che il prodotto sia completo, sia per quanto riguarda le funzionalità fornite, che per l'impianto grafico, e quindi pronto per la \textit{Revisione di Accettazione}.
	\item \textbf{Situazione dei documenti}: il gruppo ha poi fatto il punto della situazione sui documenti. Gli errori individuati dal professore nella valutazione della RQ sono stati corretti. Ciò che ha richiesto più tempo è stata la modifica delle \NdP{}, le quali hanno subito un importante cambiamento strutturale, alla ricerca di una maggiore uniformità. Sono poi state sostituite parti testuali con diagrammi più esplicativi, per facilitare la lettura del documento.  
	Infine sono stati aggiornati i manuali con le funzionalità introdotte nell'ultimo periodo e il \PdQ{} con il cruscotto di valutazione aggiornato con i nuovi valori delle metriche.
	Anche dal punto di vista della redazione dei documenti il gruppo si ritiene soddisfatto del lavoro svolto. Restano da compiere alcune verifiche complessive e le relative approvazioni, le quali verranno svolte tra oggi e domani (15-05-2021 e 16-05-2021).
	\item \textbf{Preparazione presentazione per la RA}: il gruppo ha infine discusso come dividersi il lavoro per la presentazione della RA. 
	Parte fondamentale sarà l'elevator pitch, utile ad introdurre il prodotto al committente e al proponente, riassumendone le funzionalità e i suoi reali scopi. Di questo si occuperanno \ZM{} e \SP{}.
	Per la demo del prodotto, parte centrale della presentazione, saranno \SG{}, \BM{}, \RA{} e \SH{} ad occuparsene.
	Infine la parte di consuntivo finale spetterà a \PA{}.
	Il gruppo rimarrà comunque in contatto per confronti, richieste d'aiuto e per provare la presentazione stessa. 
\end{itemize}

\newpage