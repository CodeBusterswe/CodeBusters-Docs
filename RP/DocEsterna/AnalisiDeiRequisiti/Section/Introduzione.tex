\section{Introduzione}
\subsection{Scopo del Documento}
Questo documento contiene la descrizione degli attori del sistema, definendo poi tutti i casi d'uso individuati a partire dai requisiti, fornendo una visione chiara ai progettisti sul problema da trattare. Tutti le informazioni facenti parte il seguente documento derivano dallo studio e dalla conseguente comprensione del \glo{capitolato}, oltre che dagli incontri avvenuti con l'azienda proponente \Proponente{}.
\subsection{Scopo del Prodotto}
Oggigiorno, anche i programmi più tradizionali gestiscono e memorizzano una grande mole di dati; di conseguenza servono software in grado di eseguire un'analisi e un'interpretazione delle informazioni.\\
Il \glo{capitolato} C4 ha come obiettivo quello di creare un'applicazione di visualizzazione di dati con numerose dimensioni in modo da renderle comprensibili all'occhio umano.  Lo scopo del prodotto sarà quello di fornire all'utente diversi tipi di visualizzazioni e di algoritmi per la riduzione dimensionale in modo che, attraverso un processo esplorativo, l'utilizzatore del prodotto possa studiare tali dati ed evidenziarne degli eventuali \glo{cluster}. 
\subsection{Glossario}
Per evitare ambiguità relative alle terminologie utilizzate, è stato compilato il \Glossariov{v2.0.0-0.2}. In questo documento sono riportati tutti i termini importanti e con un significato particolare. Questi termini sono evidenziati da una 'G' ad apice.
\subsection{Riferimenti}
\subsubsection{Riferimenti normativi}
\begin{itemize}
	\item \textbf{\NdPv{v2.0.0-0.2}};
	\item \textbf{Verbale interno 14-12-2020};
	\item \textbf{Verbale esterno 17-12-2020};
	\item \textbf{Verbale interno 20-12-2020};
	\item \textbf{Verbale interno 07-01-2021};
	\item \textbf{Verbale esterno 05-02-2021};
	\item \textbf{Verbale interno 08-02-2021};
	\item \textbf{Verbale interno 17-02-2021};
	\item \textbf{Verbale esterno 18-02-2021};
	\item \textbf{Verbale interno 23-02-2021};
	\item \textbf{Capitolato d'appalto C4 - HD Viz: visualizzazione di dati multidimensionali}:\\
	\textcolor{blue}{\url{https://www.math.unipd.it/~tullio/IS-1/2020/Progetto/C4.pdf}}
\end{itemize}

\subsubsection{Riferimenti informativi}
\begin{itemize}
	\item \textbf{Slide T7 del corso Ingegneria del Software - Analisi dei requisiti}:\\
	\textcolor{blue}{\url{https://www.math.unipd.it/~tullio/IS-1/2020/Dispense/L07.pdf}}
	\item \textbf{Slide E3 del corso Ingegneria del Software - Diagrammi dei casi d'uso}:\\
	\textcolor{blue}{\url{https://www.math.unipd.it/~rcardin/swea/2021/Diagrammi\%20Use\%20Case_4x4.pdf}}

\item \textbf{Software Engineering - Ian Sommerville - 10th Edition}\\ Parte 1: Introduction to Software Engineering:
	\begin{itemize}
	\item Capitolo 4 - Requirements engineering:
		\begin{itemize}
			\item Paragrafo 4.1 - Functional and non-functional requirements (da pag. 105 a 111);
			\item Paragrafo 4.4 - Requirements specification (da pag. 120 a 128);
			\item Paragrafo 4.5 - Requiriments validation (da pag. 129 a 130);
			\item Paragrafo 4.6 - Requirements change (da pag. 130 a 134).
		\end{itemize}
	\end{itemize}
	\item \textbf{Documentazione libreria D3.js}: \\
	\textcolor{blue}{\url{https://github.com/d3/d3/wiki}}
	
	\item \textbf{Documentazione libreria React}: \\
	\textcolor{blue}{\url{https://it.reactjs.org/docs/getting-started.html}}
	

\end{itemize}
	
	

