\section*{L}
\markright{}
\addcontentsline{toc}{section}{L}
\subsection*{\LaTeX{}}
\LaTeX{} è un \glo{linguaggio di markup} per la preparazione di testi, basato sul programma di composizione tipografica \TeX{}.
\subsection*{Linguaggio di markup}
È un insieme di regole che descrivono i meccanismi di rappresentazione (strutturali, semantici, presentazionali) o layout di un testo; facendo uso di convenzioni rese standard, tali regole sono utilizzabili su più supporti. Esempi di linguaggio di markup sono: \LaTeX{}, \glo{JSON}, \glo{CSV}. 
\subsection*{Leaflet}
Libreria \glo{JavaScript} per sviluppare mappe geografiche interattive.
\subsection*{LLE}
Acronimo di Locally-Linear Embedding. È un algoritmo di riduzione dimensionale non lineare. Si basa sui più vicini \glo{neighbors} di ciascun punto e da un'ottimizzazione ad autovettori. 