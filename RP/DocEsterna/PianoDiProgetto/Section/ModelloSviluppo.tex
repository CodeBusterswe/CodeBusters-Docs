\section{Modello di sviluppo}

\subsection{Modello incrementale}
Il modello incrementale prevede rilasci multipli e successivi, ciascuno dei quali realizza un incremento di funzionalità.
È richiesta dunque una classificazione preliminare dei requisiti atta ad individuare i più importanti, i quali devono essere sviluppati nei primi incrementi, così da avere fin da subito un prodotto funzionante, che verrà via via integrato.
L'adozione di questo modello comporta i seguenti vantaggi:
\begin{itemize}
\item Le funzionalità primarie hanno priorità nello sviluppo, così facendo queste vengono verificate più volte;
\item L'avere un prodotto funzionante già dai primi incrementi permette subito al \glo{proponente} di valutarne le funzioni primarie;
\item Ogni incremento riduce il rischio di fallimento, con un approccio predisposto ai cambiamenti;
\item L'analisi dei requisiti può essere raffinata tramite la progettazione di dettaglio ad ogni incremento;
\item Le modifiche e la correzione degli errori sono più economiche;
\item Le fasi di verifica e test sono facilitate perché mirate a un singolo incremento.
\end{itemize}

\subsection{Incrementi individuati}
In seguito è riportata la tabella con indicati gli incrementi individuati, con il rispettivo obiettivo e i requisiti ad esso associati. 
I requisiti riportati includono tutti i requisiti figli. Tutti i requisiti non riportati sono da intendersi soddisfatti, in parte, da ogni incremento.
Ogni requisito è individuato dal suo codice identificativo; per maggiori informazioni fare riferimento all'\AdR{}.

\newpage
{
\setlength\arrayrulewidth{1pt}
\setlength\colA{1.5cm}
\setlength\colB{8cm}
\setlength\colC{4cm}
\setlength\colD{2cm}
\setlength\total{\dimexpr\colA+\colB+\colC+\colD+6\tabcolsep\relax}
\begin{longtable}{C{\colA} | C{\colB} | C{\colC} | C{\colD}}
		\rowcolor{coloreRosso}
		\textcolor{white}{\textbf{Incr.}} & 
		\textcolor{white}{\textbf{Obiettivo}} & 
		\textcolor{white}{\textbf{Requisiti}} & 
		\textcolor{white}{\textbf{Casi d'uso}} \\
		\endfirsthead
	    \rowcolor{white}\multicolumn{4}{C{\total}}{\textit{Continua nella pagina successiva...}}\\
	    \endfoot
	    \rowcolor{white}\caption{Tabella degli incrementi}
	    \endlastfoot


\rowcolor{coloreRossoChiaro}\multicolumn{4}{c}{\textcolor{white}{\textbf{Fase di progettazione architetturale}}}\\
%------------------------------------------
I &
Caricamento dati tramite file e selezione delle dimensioni da utilizzare.& 
R1F1.1, R1F5 & 
UC1.1.1 \newline UC2\\
%------------------------------------------
II &
Visualizzazione \glo{Scatter Plot Matrix} e relativa personalizzazione.& 
R1F7.1, R1F7.1.1, R3F12, R3F13 & 
UC5.1, UC6.1\\
%-----------------------------------------
\rowcolor{coloreRossoChiaro}\multicolumn{4}{c}{\textcolor{white}{\textbf{Fase di progettazione di dettaglio e codifica}}}\\
%------------------------------------------
III & 
Implementazione di una sezione per l'applicazione di una riduzione dimensionale dei dati. & 
R2F15, R2F2, R1F15, R3F15.1, R3F15.2, R3F15.3, R3F15.4, R3F15.4.1, R3F15.4.2, R3F15.5, R3F15.6, R3F16& 
UC3, UC4\\
%------------------------------------------
IV & 
Visualizzazione \glo{Heat Map} e relativa personalizzazione. & 
R1F7.2, R1F7.2.1, R3F7.6, R3F12, R3F13 & 
UC5.2, UC6.2\\
%------------------------------------------
V & 
Visualizzazione \glo{Force Field} e relativa personalizzazione. & 
R1F7.3, R1F7.3.1, R3F7.6, R3F12, R3F13& 
UC5.3, UC6.3\\
%------------------------------------------
VI & 
Visualizzazione \glo{Proiezione Lineare Multi Asse} e relativa personalizzazione. & 
R1F7.4, R1F7.4.1, R3F12, R3F13& 
UC5.4, UC6.4\\
%------------------------------------------
VII & 
Implementazione di un \glo{database} per il caricamento dati attraverso \glo{query}. & 
R1F1.2 & 
UC1.1.2\\

\rowcolor{coloreRossoChiaro}\multicolumn{4}{c}{\textcolor{white}{\textbf{Fase di validazione e collaudo}}}\\
%------------------------------------------
VIII & 
Gestione della sessione. & 
R2F6 & 
UC1.2, UC7\\
%------------------------------------------
IX & 
Implementazione della guida introduttiva e di \glo{widget} di aiuto all'utente. & 
R2F4, R3F10& 
\\


\end{longtable}
}