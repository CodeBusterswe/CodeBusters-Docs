\section{Specifica dei test}
I test costituiscono il cuore delle attività di analisi dinamica sul codice. Come già riportato nelle \NdPv{v2.0.0-0.2} questi possono essere di vari tipi, ognuno con lo scopo di individuare diversi difetti software. \\
Il gruppo \Gruppo{} ha deciso che, per perseguire correttezza continua del prodotto, il processo di verifica si svolgerà in parallelo con quello di sviluppo (\glo{Modello a V}), facendo sì che questo non rallenti in alcun modo la produzione. L'obiettivo è quindi rendere i test il più possibile automatici con l'utilizzo di software appositi. \\
Il codice utilizzato per l'identificazione dei test è specificato dettagliatamente nelle \NdPv{v2.0.0-0.2}, mentre delle sigle utili per comprendere le tabelle seguenti sono:
\begin{itemize}
	\item \textbf{I}: test implementato;
	\item \textbf{NI}: test non implementato.
\end{itemize} 

\subsection{Test di unità}

{

\renewcommand{\arraystretch}{1.5}
\renewcommand\extrarowheight{1.5pt}
\setlength\arrayrulewidth{1pt}
\begin{longtable}{ C{1.5cm} | C{12cm}| C{1.5cm} } 
		\rowcolor{coloreRosso}
		\textcolor{white}{\textbf{Codice}} & 
		\textcolor{white}{\textbf{Descrizione}} & 
		\textcolor{white}{\textbf{Stato}} \\
		\endfirsthead
		\rowcolor{white}\multicolumn{3}{c}{\textit{Continua nella pagina successiva...}}\\
	    \endfoot
	    \endlastfoot

\textbf{TU1} & 
Si verifica che venga visualizzato un messaggio d'errore se
    i dati non sono stati inseriti nel sistema. & 
S\\

\textbf{TU2} & 
Si verifica che i dati inseriti siano caricati nel sistema. & 
S\\

\textbf{TU3} & 
Si verifica che la lettura del file CSV avvenga correttamente. & 
S\\

\textbf{TU4} & 
Si verifica che il parsing del file CSV pulisca i dati non idonei alla visualizzazione. & 
S\\

\textbf{TU5} & 
Si verifica che le matrici delle distanze vengano calcolate correttamente. & 
S\\

\textbf{TU6} & 
Si verifica che il grafico Adjacency Matrix venga renderizzato correttamente. & 
S\\

\textbf{TU7} & 
Si verifica che l'ordinamento dei dati per clustering avvenga correttamente. & 
S\\

\textbf{TU8} & 
Si verifica che il grafico Force Field venga renderizzato correttamente. & 
S\\

\textbf{TU9} & 
Si verifica che il grafico Heat Map venga renderizzato correttamente. & 
S\\

\textbf{TU10} & 
Si verifica che il grafico Scatterplot Matrix venga renderizzato correttamente. & 
S\\

\textbf{TU11} & 
Si verifica che il grafico Proiezione Lineare Multi Asse venga renderizzato correttamente. & 
S\\

\textbf{TU12} & 
Si verifica che le preferenze dell'utente nel grafico Scatter Plot Matrix siano salvate correttamente nel sistema. & 
S\\

\textbf{TU13} & 
Si verifica che le preferenze dell'utente nel grafico Heat Map siano salvate correttamente nel sistema. & 
S\\

\textbf{TU14} & 
Si verifica che le preferenze dell'utente nel grafico Adjacency Matrix siano salvate correttamente nel sistema. & 
S\\

\textbf{TU15} & 
Si verifica che le preferenze dell'utente nel grafico Force Field siano salvate correttamente nel sistema. & 
S\\

\textbf{TU16} & 
Si verifica che le preferenze dell'utente nel grafico Proiezione Lineare Multi Asse siano salvate correttamente nel sistema. & 
S\\

\textbf{TU17} & 
Si verifica che le dimensioni da visualizzare siano settate correttamente nel grafico \glo{PLMA}. & 
S\\

\textbf{TU18} & 
Si verifica che le dimensioni del dataset siano caricate correttamente nel sistema. & 
S\\

\textbf{TU19} & 
Si verifica che le dimensioni ottenute dalla riduzione dimensionale siano caricate correttamente nel sistema. & 
S\\

\textbf{TU20} & 
Si verifica che le matrici delle distanze calcolate dall'utente siano caricate correttamente nel sistema. & 
S\\

\textbf{TU21} & 
Si verifica che la riduzione dimensionale tramite algoritmo LLE avvenga correttamente. & 
S\\

\textbf{TU22} & 
Si verifica che la riduzione dimensionale tramite algoritmo IsoMap avvenga correttamente. & 
S\\

\textbf{TU23} & 
Si verifica che la riduzione dimensionale tramite algoritmo t-SNE avvenga correttamente. & 
S\\

\textbf{TU24} & 
Si verifica che la riduzione dimensionale tramite algoritmo Fastmap avvenga correttamente. & 
S\\

\textbf{TU25} & 
Si verifica che la riduzione dimensionale tramite distanza Euclidea avvenga correttamente. & 
S\\

\textbf{TU26} & 
Si verifica che la riduzione dimensionale tramite distanza di Chebychev avvenga correttamente. & 
S\\

\textbf{TU27} & 
Si verifica che la riduzione dimensionale tramite distanza di Canberra avvenga correttamente. & 
S\\

\textbf{TU28} & 
Si verifica che la riduzione dimensionale tramite distanza di Manhattan avvenga correttamente. & 
S\\

\textbf{TU29} & 
Si verifica che l'algoritmo sia settato correttamente nella Strategy. & 
S\\

\textbf{TU30} & 
Si verifica il render della vista. & 
S\\

\textbf{TU31} & 
Si verifica che i parametri degli algoritmi inseriti siano settati correttamente. & 
S\\

\textbf{TU32} & 
Si verifica che il nome scelto per le dimensioni ridotte sia settato correttamente. & 
S\\

\textbf{TU33} & 
Si verifica che il nome scelto per una nuova matrice delle distanze sia settato correttamente. & 
S\\

\textbf{TU34} & 
Si verifica che le dimensioni categoriche vengano selezionate correttamente. & 
S\\

\textbf{TU35} & 
Si verifica che le dimensioni numeriche vengano selezionate correttamente. & 
S\\

\rowcolor{white}\caption{Test d'unità}
\label{testIntegrazione}
\end{longtable}
}

\newpage

\subsubsection{Tracciamento test d'unità}

{

\renewcommand{\arraystretch}{1.5}
\renewcommand\extrarowheight{1.5pt}
\setlength\arrayrulewidth{1pt}
\begin{longtable}{C{1.1cm} | L{15cm}} 
		\rowcolor{coloreRosso}
		\textcolor{white}{\textbf{ID}} & 
		\textcolor{white}{\textbf{Metodo}} \\
		\endfirsthead
		\rowcolor{white}\multicolumn{2}{c}{\textit{Continua nella pagina successiva...}}\\
	    \endfoot
	    \endlastfoot
	  		
		\textbf{TU1} & src/components/UI/burgerMenuUI/ModalContent/LoadCsvVM.js : handleDismiss()\\
		\textbf{TU2} & src/components/UI/burgerMenuUI/ModalContent/LoadCsvVM.js : loadDataAndDims()\\
		\textbf{TU3} & src/components/UI/burgerMenuUI/ModalContent/CsvReaderVM.js : handleOnDrop()\\
		\textbf{TU4} & src/components/UI/burgerMenuUI/ModalContent/CsvReaderVM.js : handleOnDrop()\\
		\textbf{TU5} & src/components/UI/burgerMenuUI/ModalContent/DistanceCalcVM.js : handleSubmit()\\
		\textbf{TU6} & src/components/UI/graphUI/charts/AmChartVM.js : renderChart()\\
		\textbf{TU7} & src/components/UI/graphUI/charts/AmChartVM.js : sort()\\
		\textbf{TU8} & src/components/UI/graphUI/charts/FfChartVM.js : renderChart()\\
		\textbf{TU9} & src/components/UI/graphUI/charts/HmChartVM.js : renderChart()\\
		\textbf{TU10} & src/components/UI/graphUI/charts/SpmChartVM.js : renderChart()\\
		\textbf{TU11} & src/components/UI/graphUI/charts/PlmaChartVM.js : renderChart()\\
		\textbf{TU12} & src/components/UI/graphUI/preferences/SpmPrefVM.js : handleSelectChange()\\
		\textbf{TU13} & src/components/UI/graphUI/preferences/HmPrefVM.js : setHmPreferences(id,value)\\
	\textbf{TU14} & src/components/UI/graphUI/preferences/SpmPrefVM.js : setAmPreferences(id,value)\\	
	\textbf{TU15} & src/components/UI/graphUI/preferences/FfPrefVM.js : setFfPreferences(id,value)\\
	\textbf{TU16} & src/components/UI/graphUI/preferences/PlmaPrefVM.js : setPlmaPreferences(id,value)\\
\textbf{TU17} & src/components/UI/graphUI/preferences/PlmaPrefVM.js : handleMultiSelChange(id,val)\\
\textbf{TU18} & src/stores/datasetStore.js : loadDimensions(dimensions)\\
\textbf{TU19} & src/stores/datasetStore.js : addDimensionsToDataset(dimensions)\\
\textbf{TU20} & src/stores/distancesMatrixStore.js : addDistanceMatrix(matrix)\\
\textbf{TU21} & src/components/UI/burgerMenuUI/ModalContent/DimsRedVM.js : handleSubmit()\\
\textbf{TU22} & src/components/UI/burgerMenuUI/ModalContent/DimsRedVM.js : handleSubmit()\\
\textbf{TU23} & src/components/UI/burgerMenuUI/ModalContent/DimsRedVM.js : handleSubmit()\\
\textbf{TU24} & src/components/UI/burgerMenuUI/ModalContent/DimsRedVM.js : handleSubmit()\\

\textbf{TU25} & src/components/UI/burgerMenuUI/ModalContent/DistCalcVM.js : handleSubmit()\\
\textbf{TU26} & src/components/UI/burgerMenuUI/ModalContent/DistCalcVM.js : handleSubmit()\\
\textbf{TU27} & src/components/UI/burgerMenuUI/ModalContent/DistCalcVM.js : handleSubmit()\\
\textbf{TU28} & src/components/UI/burgerMenuUI/ModalContent/DistCalcVM.js : handleSubmit()\\

\textbf{TU29} & src/components/UI/burgerMenuUI/ModalContent/StrategyDimRed/DimRed.js : setStrategy(alg)\\

\textbf{TU30} & src/components/View.jsx : render()\\

\textbf{TU31} & src/components/UI/burgerMenuUI/ModalContent/StrategyDimRed/DimRed.js : executeStrategy(params, data)\\ 

\textbf{TU32} & src/stores/datasetStore.js : getselectedDimensions()\\

\textbf{TU33} & src/stores/distancesMatrixStore.js : getDistanceMatrixByName(name\\

\textbf{TU34} & src/stores/datasetStore.js : getCategoricCheckedDimensions()\\

\textbf{TU35} & src/stores/datasetStore.js : getNumericDimensions()\\


		
		
		\rowcolor{white}
		\caption{Tracciamento test d'unità - metodi}
\end{longtable}
}
\subsection{Test d'integrazione}
%I test di integrazione verificano come le componenti software si integrino tra di loro.
%In questa prima versione del \PdQ{} il gruppo non è in grado di stabilire dei test di integrazione, non avendo individuato e testato le componenti del prodotto software.
{

\renewcommand{\arraystretch}{1.5}
\renewcommand\extrarowheight{1.5pt}
\setlength\arrayrulewidth{1pt}
\begin{longtable}{ C{1.5cm} | C{12cm}| C{1.5cm} } 
		\rowcolor{coloreRosso}
		\textcolor{white}{\textbf{Codice}} & 
		\textcolor{white}{\textbf{Descrizione}} & 
		\textcolor{white}{\textbf{Stato}} \\
		\endfirsthead
		\rowcolor{white}\multicolumn{3}{c}{\textit{Continua nella pagina successiva...}}\\
	    \endfoot
	    \endlastfoot

\textbf{TI1} & 
Verificare che il collegamento con il database avvenga correttamente & 
NI\\

\textbf{TI2} & 
Verificare che l'elenco dei dataset presenti nel database sia raggiungibile & 
NI\\

\textbf{TI3} & 
Verificare che le \glo{query} per il recupero dei dati avvengano con successo & 
NI\\

\textbf{TI4} & 
Verificare che la chiusura del collegamento con il database avvenga correttamente & 
NI\\

\rowcolor{white}\caption{Test di integrazione}
\label{testIntegrazione}
\end{longtable}
}

\subsection{Test di sistema}
In base ai requisiti trovati e riportati nell'\AdRv{}, il gruppo ha stilato un lista di test di sistema da effettuare per verificarne la correttezza.
Il codice utilizzato per l'identificazione dei test è specificato nelle \NdP, mentre delle sigle utili per comprendere la tabella seguente sono:
\begin{itemize}
	\item \textbf{I}: test implementato;
	\item \textbf{NI}: test non implementato.
\end{itemize} 

\newpage
\renewcommand{\arraystretch}{1.5}
\rowcolors{2}{coloreGrigietto}{white}
\renewcommand{\arraystretch}{1.5}
\renewcommand\extrarowheight{1.5pt}
\begin{longtable}{C{2.5cm} | C{12cm} C{2cm}} 
		\rowcolor{coloreRosso}
		\textcolor{white}{\textbf{Codice}} & 
		\textcolor{white}{\textbf{Descrizione}} & 
		\textcolor{white}{\textbf{Stato}} \\
		\endfirsthead
		\rowcolor{white}\multicolumn{3}{C{8cm}}{\textit{Continua nella pagina successiva...}}\\
	    \endfoot
	    \rowcolor{white}\caption{Test di sistema}
	    \endlastfoot
		\hline
\textbf{TS1F1} & L'utente deve poter caricare dei dati nel sistema. Verificare che l'utente possa: 
					\begin{itemize}
						\item Visusalizzare correttamente la schermata di inserimento dei dati;
						\item Inserire i dati attravero file CSV tramite apposito bottone;
						\item Inserire i dati attraverso una interrogazione al DB tramite query.	
					\end{itemize}					 			    
			  & NI\\
\textbf{TS1F1.1} & Caricamento dati attraverso l'invio di un file CSV. Verificare che:
					\begin{itemize}
						\item Il file CSV inserito sia sintatticamente corretto; 
						\item I dati siano correttamente prelevati dal file e pronti per la visualizzazione;
						\item Sia visualizzato a schermo un messaggio di corretto inserimento dei dati.
					\end{itemize}													
		         & NI \\
\textbf{TS1F1.2} & Caricamento dati attraverso l'interrogazione a un DB. Verificare che:
					\begin{itemize}
						\item L'interrogazione sia stata effettuata correttamente; 
						\item I dati siano stati individuati e ritornati;
						\item Sia visualizzato a schermo un messaggio che notifica che sono stati trovati i dati.
					\end{itemize}
 				 & NI \\ 
\textbf{TS1F1.3} &  Il caricamento fallisce se il file CSV non è corretto o l'interrogazione al DB fallisce. Verificare che:
					\begin{itemize}
						\item Sia visualizzato a schermo un messaggio d'errore parlante; 
						\item Sia possibile reinserire i dati tramite file CSV o interrogazione al DB dalla stessa schermata.
					\end{itemize}	
 				 & NI \\ 
\textbf{TS2F2} &  Aiuti all'utente attraverso widget. Verificare che:
					\begin{itemize}
						\item Siano tutti facilmente localizzabili dall'utente;
						\item Siano tutti facilmente utilizzabili dall'utente;
						\item Siano effettivamente utili per l'utente.
					\end{itemize}	
 			   & NI \\ 
\textbf{TS1F3} &  L'applicazione deve fornire diverse visualizzazioni per i dati. Verificare che: 
					\begin{itemize}
						\item Siano mostrati a video dei radio button per la selezione del grafico da utilizzare;
						\item I radio button abbiano nomi parlanti;
						\item Ogni radio button sia selezionabile e permetta la visualizzazione del grafico corretto;
						\item Sia possibile selezionare due grafici diversi nel caso in cui si voglia compararli.
					\end{itemize}
			   &  NI \\
\textbf{TS1F3.1} & L'applicazione deve permettere la scelta delle dimensioni da visualizzare. Verificare che:
 					\begin{itemize}
 						\item Siano mostrati a video dei bottoni per la selezione delle dimensioni da visualizzare;
 						\item I bottoni devono avere nomi parlanti;
 						\item Coerenza tra dimensioni visualizzabili e grafico scelto;
 						\item Ogni bottone deve ridurre/aumentare le dimensioni correttamente rispetto al grafico visualizzato.
 					\end{itemize}
           	     & NI \\
\textbf{TS1F3.2} & L'applicazione deve fornire la visualizzazione \glo{Scatter plot Matrix}. Verificare che:
					\begin{itemize}
						\item Gli scatter plot creati siano disposti a matrice;
						\item Ogni scatter plot sia corretto rispetto ai dati da rappresentare;
						\item Ogni scatter plot metta in relazione variabile diverse;
						\item Ogni variabile sia in relazione con tutte le altre nei deversi scatter plot;
						\item L'utente possa selezionare quali scatter plot evidenziare perché più rilevanti.
					\end{itemize}	
				  & NI \\
\textbf{TS1F3.3} & L'applicazione deve fornire la visualizzazione \glo{Heat Map}. Verificare che:
					\begin{itemize}
						\item Ogni variabile sia in relazione con tutte le altre; 
						\item I colori utilizzati rappresentano effettivamente la distanza tra i punti da misurare;
						\item Sia presente un \glo{dendrogramma} ai bordi della mappa.
					\end{itemize}	
				 & NI \\
\textbf{TS1F3.3.1} & L'applicazione deve ordinare i punti nella visualizzazione \glo{Heat Map}. Verificare che:
					\begin{itemize}
						\item L'utente possa ordinare i punti attraverso parametri specifici;
						\item Il parametro d'odrinamento sia selezionabile da un rudio button;
						\item Il grafico effettivamente cambi aspetto in real time;
						\item Il valore dei dati non cambi, ma solo la loro visualizzazione;
					\end{itemize}	
				   & NI \\
\textbf{TS1F3.4} & L'applicazione deve fornire la visualizzazione \glo{Force Field}. Verificare che:
					\begin{itemize}
						\item Ogni variabile abbia il suo corrispettivo nodo nel grafico;
						\item Ogni relazione sia identificata da una linea di collegamento tra i nodi;
						\item Le distanze nello spazio a molte dimensioni siano correttamente tradotte in forze di attrazione e repulsione tra i nodi;
						\item Il grafico sia bidimensionale (o tridimensionale).
					\end{itemize}	
				 & NI \\
\textbf{TS3F3.4.1} & Utilizzo di funzioni di "forza" diverse da quelle previste in autmatico dal grafico "forcebased" di \glo{D3.js}.
Verificare che:
					\begin{itemize}
						\item 
						\item 
						\item 
					\end{itemize}	
				   & NI \\
\textbf{TS1F3.5} & L'applicazione deve fornire la visualizzazione \glo{Proiezione Lineare Multi Asse}. Verificare che:
					\begin{itemize}
						\item I punti dello spazio multidimensionale siano correttamente posizionati nel piano cartesiano;
						\item La riduzione a due dimensioni sia corretta;
						\item L'utente possa spostare gli assi per individuare le strutture di dati di suo interesse;
						\item Lo spostamento degli assi avvenga spostando manualmente le frecce degli assi.
					\end{itemize}	
				   & NI \\
\textbf{TS3F3.6} & L'applicazione deve fornire altre visualizzazioni con più di tre dimensioni. Verificare che:
					\begin{itemize}
						\item 
						\item 
						\item 
					\end{itemize}	
				   & NI \\
\textbf{TS3F3.7} & Utilizzo di funzioni di calcolo della distanza diverse dalla distanza “Euclidea” in tutte le visualizzazioni che dipendono da tale concetto. Verificare che:
					\begin{itemize}
						\item 
						\item 
						\item 
					\end{itemize}	
				   & NI \\
\textbf{TS3F4} & Implementare analisi automatiche per evidenziare situazioni di particolare interesse. Verificare che: 
					\begin{itemize}
						\item In ogni tipo di grafico sia possibile la visualizzazione dei dati rilevanti;
						\item Vengano effettivamente esclusi tutti i dati outlier;
						\item Il variare delle dimensioni non modifichi i dati rilevanti.
					\end{itemize}		
			   & NI \\
\textbf{TS3F5} & Utilizzo di algoritmi di preparazione del dato per la visualizzazione. Verificare che:
					\begin{itemize}
						\item 
						\item 
						\item 
					\end{itemize}	
			   & NI \\
\textbf{TS3F6} & Presenza di una guida introduttiva per l'utente. Verificare che:
					\begin{itemize}
						\item L'utente possa facilmente trovare e consultare la guida;
						\item La guida sia scritta un italiano corretto;
						\item La guida sia fruibile da ogni tipo di utente, anche quello meno esperto; 
						\item La guida spieghi tutti gli utilizzi dell'applicazione.
					\end{itemize}
			   & NI \\
\textbf{TS2F7} & Possibilità di visualizzare contemporaneamente due grafici per confronti. Verificare che: 
					\begin{itemize}
						\item Sia presente un bottone che permetta la visualizzazione di un secondo grafico;
						\item L'utente possa effettivamente selezionare due grafici diversi nel radio button;
						\item Il secondo grafico corrisponda a quello selezionato;
						\item I dati rilevanti siano evidenziati correttamente anche nel secondo grafico.
					\end{itemize}	
			   & NI \\
\textbf{TS2F8} & L'utente può salvare la sessione in corso per ripristinarla in un secondo momento. Verificare che: 
					\begin{itemize}
						\item Sia disponibile un bottone per il salvataggio della sessione; 
						\item Sia disponibile un bottone per il ripristino della sessione;
						\item La sessione sia salvata correttamente;
						\item La sessione sia effettivamente ripristinabile;
					\end{itemize}	
			   & NI \\
\label{testSistema}
\end{longtable}
Come si può notare lo stato di tutti i test è NI ("non implementato") visto che il gruppo non ha ancora iniziato la scrittura del codice. L'implementazione dei test e i loro risultati saranno oggetto della prossima verisone del \PdQv.
\subsection{Test di accettazione}
I test di accettazione rappresentano il collaudo del prodotto software, verificando che corrisponda con quello atteso dal proponente. Sono l'unione dei test di sistema già svolti dal gruppo durante lo sviluppo ed ulteriori test finali.
In questa prima versione del \PdQ{} il gruppo non è in grado di stilare dei test aggiuntivi oltre quelli di sistema già riportati nella tabella \ref{testSistema}.
\subsection{Test di regressione}
%I test di regressione permettono di individuare errori causati da modifiche apportate a nuove versioni del prodotto. In questa prima versione del \PdQv{} il gruppo non è in grado di stabilire dei test di regressione utili.

