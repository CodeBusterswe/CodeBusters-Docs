\subsection{Gestione della Configurazione}
\subsubsection{Scopo}
Lo scopo di questa sezione è definire come il gruppo ha deciso di attuare il processo di supporto di gestione della configurazione sui file prodotti.

\subsubsection{Aspettative}
Le aspettative del gruppo \Gruppo{} nell'utilizzo di questo processo sono:
\begin{itemize}
	\item Avere un vantaggio nell'individuazione e risoluzione di possibili conflitti o errori;
	\item Avere il tracciamento di ogni modifica;
	\item Avere il ripristino, se necessario, a una versione precedente;
	\item Avere la condivisione tra i membri del gruppo del materiale configurato.
\end{itemize}

\subsubsection{Descrizione}
Il processo di gestione della configurazione ha lo scopo di mantenere organizzata e tracciabile la documentazione redatta e il codice sviluppato, creando una storia per ogni file prodotto. In particolare si vuole gestire la struttura e la disposizione delle varie parti di ogni file all'interno di \glo{repository} facilmente accessibili e navigabili.
Inoltre il processo si occupa anche di mantenere ordinati tali \glo{repository}, favorendo lo sviluppo di un senso dell'orientamento.

\subsubsection{Versionamento}
\paragraph{Codice di versione}
Ogni versione è identificata tramite un codice numerico di cinque cifre:
\begin{center}
\textbf{[X].[Y].[Z]-[A].[B]} 
\end{center}
Le prime tre rappresentano:
\begin{itemize}
	\item \textbf{X}: versione stabile, ossia sottoposta ad approvazione del responsabile del documento;
  	\item \textbf{Y}: versione controllata, ossia sottoposta a revisione complessiva da parte del verificatore;
  	\item \textbf{Z}: versione modificata dal redattore con relativa verifica.
\end{itemize}
  
Dall'inizio della progettazione e codifica del prodotto software il gruppo ha deciso d'integrare ulteriori due cifre, in questo  modo:
\begin{center}
\textbf{\textit{Codice}-[A].[B]} 
\end{center}
dove: 
\begin{itemize}
	\item \textbf{A}: indica una versione completa e funzionante del prodotto che supera tutti i test, soddisfa le metriche e implementa i requisiti obbligatori;
	\item \textbf{B}: cresce al raggiungimento degli obiettivi degli incrementi pianificati nel \PdPv{v2.0.0-0.2}. 
\end{itemize}

Tutte le cifre iniziano dal valore 0. \\ 
Ciascuna cifra aumenta di un'unità ogni volta che si compie un'operazione sul documento; inoltre:
\begin{description}
	\item[Se la cifra X è modificata] : le cifre Y e Z ritornano a 0 (per esempio da 1.2.6-0.0 a 2.0.0-0.0);
	\item[Se la cifra Y è modificata] : la cifra Z ritorna a 0 (per esempio da 1.2.6-0.0 a 1.3.0-0.0).
\end{description}
Le cifre A e B non vengono mai annullate.

\paragraph{Sistemi software utilizzati}
Per gestire le versioni è stato deciso di utilizzare il \glo{version control system (\glo{VCS}) distribuito} \glo{Git}. 
Le motivazioni di questa scelta si racchiudono nei vantaggi di utilizzo rispetto a \glo{VCS} centralizzati:
\begin{itemize}
	\item Possibilità di lavorare in locale senza il supporto del nodo centrale remoto; 
	\item Possibilità di creare diversi flussi di lavoro (\glo{branch}) per lavorare a documenti differenti;
	\item Miglior gestione dei conflitti, a favore di una migliore collaborazione.
\end{itemize}
Per gestire i \glo{repository} \glo{Git} è stata scelto il servizio \glo{GitHub} per i seguenti motivi: 
\begin{itemize}
	\item Integrazione di un \glo{issue tracking system(ITS)};
	\item Possibilità di utilizzarlo tramite browser, applicazione desktop, applicazione mobile o linea di comando;
	\item Buona conoscenza di quest'ultimo da parte di tutti i membri del gruppo.
\end{itemize}

\subsubsection{Struttura dei repository}
\paragraph{Repository utilizzati}
Per favorire una migliore organizzazione e divisione del lavoro è stato deciso di creare due \glo{repository} pubblici distinti:
\begin{description}
	\item[CodeBusters-Docs] : per il versionamento dei documenti. \\
	Riferimento: \textcolor{blue}{\url{https://github.com/CodeBusterswe/CodeBusters-Docs.git}}
	\item[CodeBusters-HDviz] : per il versionamento del codice. \\
	Riferimento: \textcolor{blue}{\url{https://github.com/CodeBusterswe/CodeBusters-HDviz.git}}
\end{description}


\paragraph{CodeBuster-Docs}
L'organizzazione del lavoro collaborativo è così riassunta:
\begin{itemize}
	\item Ramo principale main in cui è presente la sola documentazione pronta alla revisione;	
	\item Ramo develop in cui effettuare periodicamente il \glo{merge} dai vari rami minori;
	\item Rami derivanti dal develop con nomi parlanti, ognuno dedicato alla stesura di uno specifico documento (l'idea fondante è quella del \glo{Git feature branch workflow});
	\item Ramo fixTemplate per evitare errori sul template al momento del merge dei rami minori nel develop.
\end{itemize}
Nel \textit{main} i file sono contenuti in diverse cartelle, a seconda della revisione per cui sono stati prodotti. Attualmente sono presenti le cartelle RR (\textit{Revisione dei Requisiti}) e RP(\textit{Revisione di Progettazione}). \\
Il file \textit{.gitignore} è l'unico esterno a cartelle e dichiara esplicitamente l'estensione dei file automaticamente generati da \LaTeX{} da non tracciare, poiché poco utili allo scopo del \glo{repository}. \\
I file nelle cartelle principali del main sono organizzati in sottocartelle:
\begin{itemize}
	\item \textbf{DocEsterna}: contiene l'\AdR{}, il \PdP{}, il \PdQ{}, il \textit{Glossario}, i \textit{verbali esterni};
	\item \textbf{DocInterna}: contiene lo \SdF{}, le \NdP{}, i \textit{verbali interni};
	\item \textbf{Utility}: contiene file di utilità generale come il template dei documenti, comandi \LaTeX{} per velocizzare la redazione e immagini utilizzate.
\end{itemize}
	
\paragraph{Gestione dei cambiamenti in CodeBuster-Docs}
La separazione del flusso di lavoro tra i vari documenti da redarre permette una notevole diminuzione dei conflitti. Il punto focale è che il ramo main rimanga pulito da ogni tipo di errore, per cui non è utilizzabile da nessun membro del gruppo fino a che ciascun responsabile non abbia dato l'approvazione al corrispettivo documento. Solo in quel momento è permesso il \glo{merge} del ramo develop nel main. \\
I cambiamenti da gestire sui documenti possono essere:
\begin{itemize}
	\item \textbf{Modifiche minori}: riguardano errori grammaticali, lessicali o di sintassi, che possono essere corretti dai redattori senza l'approvazione del responsabile;
	\item \textbf{Modifiche generali}: riguardano cambiamenti più generali come la struttura del documento o convenzioni da utilizzare e richiedono il consulto con il responsabile, il quale potrà accettare o declinare la proposta di modifica.
\end{itemize}

\paragraph{CodeBusters-HDviz}
Il \glo{repository} è composto da un solo \glo{branch} main che contiene tutti i file della piattaforma che \Gruppo{} sta progettando. I file sono suddivisi nel seguente modo:
\begin{itemize}
	\item \textbf{src}: la cartella che contiene i file sorgente del codice. È composta da:
	\begin{itemize}
		\item \textbf{components}: la cartella contiene al suo interno una cartella per ogni componente del programma, dove al suo interno ci sarà il file sorgente e il file del test di unità;
		\item \texttt{\_\_}\textbf{snapshots}\texttt{\_\_}: la cartella viene creata dall'utilizzo della React Testing Library e contiene gli \glo{snapshot} test dei singoli componenti. Ognuno di questi è nominato in questo modo:
				\begin{center}
					\textbf{App.\textit{test.js.snap}}
				\end{center}
				dove \textbf{App} è il nome del componente.

	\end{itemize}
	\item \textbf{test}: la cartella contiene i file dei test di unità. È composta da:
	\begin{itemize}
		\item \textbf{integrazione}: contiene i file dei test di integrazione;
		\item \textbf{sistema}: contiene i file dei test di sistema.
	\end{itemize}
\end{itemize}

I singoli file di sorgente sono denominati nel seguente modo:
\begin{itemize}
	\item \textbf{Componenti}: il nome del file sorgente deve indicare la sua funzione o quello che rappresenta;
	\item \textbf{Test}: il nome del file sorgente del test segue la seguente nomenclatura:
	\begin{center}
		\textbf{[Componente].[Categoria].\textit{test.js}}
	\end{center}
	
	dove:
	\begin{itemize}
		\item \textbf{Componente}: indica il nome del componente a cui il test si riferisce;
		
		\item \textbf{Categoria}: la categoria del test che può essere:
		\begin{itemize}
			\item \textbf{unit}: unità;
			
			\item \textbf{int}: integrazione;
			
			\item \textbf{syst}: sistema.
		\end{itemize}
	\end{itemize}
\end{itemize}

\paragraph{Gestione dei cambiamenti in CodeBuster-HDviz}
I cambiamenti apportati alla \glo{repository} non devono includere codice "sporco", di debug o qualsiasi forma di codice provvisorio.

\subsubsection{Metriche}
Il processo di gestione della configurazione non fa uso di metriche qualitative particolari.

\subsubsection{Strumenti}
Non sono stati identificati degli strumenti particolari per la gestione della configurazione.