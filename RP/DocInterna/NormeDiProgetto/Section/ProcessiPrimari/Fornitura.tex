\subsection{Fornitura}
\subsubsection{Scopo}
In questa sezione vengono descritti i documenti che compongono il processo di fornitura, la loro struttura e gli strumenti utilizzati.

\subsubsection{Aspettative}
Le aspettative nell'applicazione del processo di fornitura sono:
\begin{itemize}
	\item Avere una chiara struttura dei documenti;
	\item Definire i tempi di lavoro;
	\item Chiarire dubbi e stabilire vincoli con il proponente.
\end{itemize}

\subsubsection{Descrizione}
Nel processo di fornitura si scelgono le procedure e le risorse atte a perseguire lo sviluppo del progetto. Dopo aver ricevuto le richieste del proponente, il gruppo redige uno studio di fattibilità e la fornitura può essere avviata per completare tali richieste.\\
Il proponente e il fornitore stipuleranno un contratto per la consegna del prodotto.\\
Si dovrà poi sviluppare un piano di progetto partendo dalla determinazione delle procedure e delle risorse necessarie.
Da quel momento fino alla consegna del prodotto il \PdP{} scaglionerà le attività da svolgere. \\
 Il processo di fornitura è composto dalle seguenti fasi:
 \begin{enumerate}
 \item Avvio; 
\item Approntamento di risposte alle richieste;
\item Contrattazione;
\item Pianificazione;
\item Esecuzione e controllo;
\item Revisione e valutazione;
\item Consegna e completamento.
 \end{enumerate}
\subsubsection{Proponente}
\Gruppo{} vorrebbe avere un contatto costante con il proponente in modo da avere un riscontro:
\begin{itemize}
\item Sulle soluzioni utilizzate;
\item Sulle tempistiche di consegna del prodotto;
\item Su eventuali dubbi;
\item Sulla stima dei costi;
\item Su vincoli e requisiti.
\end{itemize}
\subsubsection{Documenti}
Di seguito sono descritti i documenti che sono redatti durante questa fase.
\paragraph{Studio di fattibilità}
Documento che contiene la stesura dello studio di fattibilità riguardante i sette capitolati proposti; per ciascuno di essi vengono evidenziati i seguenti aspetti:
\begin{itemize}
    \item Descrizione generale;
    \item Prerequisiti e tecnologie coinvolte;
    \item Vincoli;
    \item Aspetti positivi;
    \item Aspetti critici.
\end{itemize}
Infine, per ogni capitolato vengono esposte le motivazioni e le ragioni per cui il gruppo ha scelto come progetto il \glo{capitolato} \NomeProgetto{} a discapito degli altri sei proposti.\\
\paragraph{Piano di qualifica}
Il \PdQ{}, redatto dal \glo{verificatore}, contiene tutte le misure da adottare per garantire la qualità del prodotto e dei processi. È suddiviso nelle seguenti parti:
\begin{itemize}
\item Qualità di processo;
\item Qualità di prodotto;
\item Specifica dei test;
\item Resoconto attività di verifica.
\end{itemize} 
\paragraph{Piano di progetto}
Gli amministratori e il responsabile dovranno redigere questo documento che dovrà essere seguito durante tutto il corso del progetto. È suddiviso nelle seguenti sezioni:
\begin{itemize}
    \item Analisi dei rischi;
    \item Modello di sviluppo;
    \item Pianificazione;  
    \item Preventivo;
    \item Consuntivo;
    \item Organigramma;
    \item Attualizzazione dei rischi. 
\end{itemize}

\subsubsection{Metriche}
Il processo di fornitura non fa uso di metriche qualitative particolari.

\subsubsection{Strumenti}
Gli strumenti utilizzati in questo processo comprendono:
\begin{itemize}
	\item \textbf{Excel}: utilizzato per creare grafici, eseguire calcoli e presentare tabelle organizzative; 
	\begin{center}
		\textcolor{blue}{\url{https://www.microsoft.com/it-it/microsoft-365/excel}}
	\end{center}
	\item \textbf{Microsoft Planner}: utilizzato per gestire le \glo{task} che ciascun membro del gruppo deve svolgere. Permette di assegnare attività a specifici membri, suddividerle per categorie ed applicare molte altre personalizzazioni. Grazie a questa applicazione è possibile verificare che tutte le attività siano completate in linea con i tempi previsti; 
	\begin{center}
		\textcolor{blue}{\url{https://www.microsoft.com/it-it/microsoft-365/business/task-management-software}}
	\end{center}
	\item \textbf{GanttProject}: utilizzato per creare i grafici di Gantt relativi alla pianificazione.
	\begin{center}
		\textcolor{blue}{\url{https://www.ganttproject.biz/}}
	\end{center}
\end{itemize}
