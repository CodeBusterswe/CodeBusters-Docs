\section*{B}
\markright{}
\addcontentsline{toc}{section}{B}

\subsection*{Back-end}
Parte del software che permette il funzionamento delle interazioni dell'utente e che elabora i dati.

\subsection*{Baseline}
Letteralmente "linea guida". Permette al team di lavoro di avere un riferimento concreto su obiettivi da raggiungere, tempi da utilizzare per ogni fase, risorse disponibili. 

\subsection*{BFF}
Acronimo di Backend For Frontend. Parte del software che si occupa della gestione dell'interfaccia utente.

\subsection*{Big Data}
Concetto che identifica una raccolta di dati molto estesa e la capacità di analizzarli, ovvero estrapolarli e metterli in relazione. Lo scopo di tale studio è scoprire i legami tra fenomeni diversi e prevedere gli andamenti.

\subsection*{Blockchain}
Struttura dati definita come un registro digitale le cui voci sono raggruppate in "blocchi", concatenati in ordine cronologico e la cui integrità è garantita dall'uso della crittografia.

\subsection*{Bootstrap}
Bootstrap è una raccolta di strumenti liberi per la creazione di siti e applicazioni per il Web. Essa contiene modelli di progettazione basati su \glo{HTML} e \glo{CSS}, sia per la tipografia, che per le varie componenti dell'interfaccia, come moduli, pulsanti e navigazione, così come alcune estensioni opzionali di \glo{JavaScript}.

\subsection*{Branch}
Rappresenta una linea di sviluppo indipendente e serve come astrazione per il processo di modifica, stage o \glo{commit}.

\subsection*{Browser}
Applicazione per l'acquisizione, la presentazione e la navigazione di risorse sul web.
Il programma implementa le funzionalità di client per il protocollo HTTP, che regola il download delle risorse dai server web, e quelle di visualizzazione dei contenuti ipertestuali e di riproduzione di contenuti multimediali.

\subsection*{Bug}
Problema che porta al malfunzionamento del software, per esempio producendo un risultato inatteso o errato, tipicamente dovuto a un errore nella scrittura del codice sorgente di un programma. 

\subsection*{Build}
Processo di trasformazione del codice sorgente in un artefatto eseguibile.




