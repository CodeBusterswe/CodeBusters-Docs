\section*{M}
\markright{}
\addcontentsline{toc}{section}{M}

\subsection*{M2M}
Acronimo di Machine-to-machine. Tecnologie e applicazioni di telemetria e telematica che utilizzano le reti wireless.

\subsection*{Machine learning}
Branca dell'intelligenza artificiale che utilizza metodi statistici per migliorare la performance di un algoritmo nell'identificare \glo{pattern} nei dati.

\subsection*{Manifold}
Detta anche varietà differenziabile, è una struttura matematica su cui si basano alcuni metodi di riduzione della dimensionalità. Si tratta di uno spazio matematico dove localmente viene a ricrearsi uno spazio euclideo (di una specifica dimensione).

\subsection*{Merge}
Operazione per la quale un \glo{branch} viene unificato con un altro all'interno di uno stesso \glo{repository}. Il fine è trasportare modifiche eseguite in un ramo d'esecuzione in un altro, mettendoli quindi in pari.

\subsection*{Method chaining}
Il concatenamento dei metodi è una sintassi comune in linguaggi di programmazione orientati agli oggetti. È utilizzata per invocare più metodi su un unico oggetto.

\subsection*{Metrica}
Una metrica software è uno standard per la misura di alcune proprietà del software o delle sue specifiche.

\subsection*{Microsoft Edge}
È un \glo{browser} web rilasciato nel 2015 e sviluppato da Microsoft, disponibile su Windows, macOS, Linux, Android e iOS e incluso come browser predefinito in Windows 10, Windows 10 Mobile e Xbox One. 

\subsection*{Microsoft Planner}
Strumento di Microsoft Office con cui è possibile creare nuovi piani, organizzare e assegnare attività, condividere file, discutere di un progetto in chat e ricevere aggiornamenti sullo stato di avanzamento.
A ogni piano è associata una specifica bacheca in cui è possibile organizzare le attività dividendole in contenitori.

\subsection*{Milestone}
Una milestone indica importanti traguardi intermedi nello svolgimento del progetto. Spesso hanno una durata prefissata in giorni.

\subsection*{MobX}
Libreria JavaScript per l'implementazione dell'observer pattern in architetture basate su React.

\subsection*{Modello a V}
Rappresenta il processo di sviluppo software che può essere considerata una estensione del modello a cascata. Invece di scendere verso il basso in modo lineare, le fasi del processo, dopo la fase di codifica, sono piegate verso l'alto per formare la tipica forma a V. Il modello illustra la relazione tra ogni fase del ciclo di vita di sviluppo e la fase di collaudo ad essa associata.

\subsection*{Mozilla Firefox}
Browser web libero \glo{multipiattaforma}, mantenuto dalla Mozilla Foundation. Creato nel 2002 con il nome di "Phoenix", ad oggi è uno dei browser web più utilizzati in tutto il mondo.

\subsection*{MQTT}
Acronimo di Message Queue Telemetry Transport. Protocollo ISO standard (ISO/IEC PRF 20922) di messaggistica leggero di tipo publish-subscribe posizionato in cima a TCP/IP. È stato progettato per le situazioni in cui è richiesto un basso impatto e dove la banda è limitata. 

\subsection*{Multipiattaforma}
Riferito ad un linguaggio di programmazione, ad un'applicazione software o ad un dispositivo hardware che funziona su più di un sistema o appunto, piattaforma.

\subsection*{MVC}
Acronimo di Model-view-controller. Pattern architetturale molto diffuso, in particolare nell'ambito della programmazione orientata agli oggetti e in applicazioni web, che prevede la separazione della logica di presentazione dei dati dalla logica.

\subsection*{MySQL}
MySQL è un sistema open source di gestione di database relazionali SQL.