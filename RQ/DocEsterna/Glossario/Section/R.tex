\section*{R}
\markright{}
\addcontentsline{toc}{section}{R}

\subsection*{Rancher}
Piattaforma che si occupa di gestire multipli sistemi \glo{Kubernetes}.

\subsection*{React.js}
Libreria \glo{JavaScript} utilizzata per facilitare al programmatore la creazione di \glo{UI}.

\subsection*{Real-time}
In informatica, un sistema real-time è un calcolatore in cui la correttezza del risultato delle sue computazioni dipende non solo dalla correttezza logica ma anche dalla correttezza temporale espressa in tempo massimo di risposta.

\subsection*{Repository}
Ambiente di un sistema informativo, in cui vengono gestiti i metadati, attraverso tabelle relazionali; l'insieme di tabelle, regole e motori di calcolo tramite cui si gestiscono i metadati prende il nome di metabase.  

\subsection*{Redattore}
Ha in compito la redazione di un documento, deve inoltre essere a disposizione per eventuali modifiche richieste dall'\glo{approvatore} e si occupa di correggere errori segnalati dal \glo{verificatore}.

\subsection*{Renderizzare}
Conversione di un componente React in HTML e inserimento nel DOM per la sua visualizzazione.