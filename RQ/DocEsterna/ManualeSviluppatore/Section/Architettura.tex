\section{Architettura}
L'architettura di \textit{HDViz} è basata sul \glo{design pattern architetturale} \glo{\textit{Model-View-ViewModel (MVVM)}}, derivato dal più comune \glo{Model-View-Controller (MVC)}. 
Abbiamo quindi sviluppato un \textit{ViewModel} per ogni componente React della \textit{View} che necessita di interagire con il \textit{Model}, e delegato a tali \textit{ViewModel} la modifica del \textit{Model} in base agli input dell'utente. 
Il modello corrisponde al RootStore, che istanzia i tre diversi store utilizzati in \NomeProgetto{}. \\ \mbox{}\\
È stato scelto il design MVVM per i seguenti motivi: 
\begin{itemize}
	\item Favorisce la separazione tra \glo{\textit{business logic}} e \glo{\textit{presentation logic}}, facendo comunicare \textit{Model} e \textit{View} solo attraverso un \textit{ViewModel};  
	\item Permette di non avere un unico controller con cui dover gestire tutta l'\glo{\textit{application logic}}. Essa è infatti contenuta nei vari \textit{ViewModel} dei componenti della vista, fornendo diversi vantaggi: 
	\begin{itemize}
		\item Minor numero di conflitti in fase di codifica (non si deve accedere ad uno stesso file dove è contenuta tutta la logica);
		\item Performance migliori (viene renderizzato solo il componente che effettivamente subisce modifiche del proprio \glo{stato interno}, gestito anch'esso nel rispettivo \textit{ViewModel}).
	\end{itemize}
	\item Adatto per le web application la cui interfaccia utente è sviluppata con React.
\end{itemize}

\begin{figure}[hb]
\includegraphics[width=11.8cm,height=8cm]{Extra/MVVMPattern}
\centering
\caption{Model-View-ViewModel di \textit{HDViz}}
\end{figure}

\newpage
\subsection{Diagrammi dei package}
\begin{figure}[hb]
\includegraphics[width=15.8cm]{Images/Allegato Tecnico-Package}
\centering
\caption{Diagramma dei package client side}
\end{figure}

Il diagramma sopra riportato descrive a livello di package le parti che compongono il lato client dell'applicazione. In particolare si può notare che:
\begin{itemize}
	\item \textit{View} e \textit{Model} non siano direttamente collegati, ma il passaggio delle informazioni avviene attraverso il \textit{ViewModel} e il componente \glo{\textit{Context Provider}} fornito da React;
	\item MobX ha il ruolo fondamentale di applicare \glo{l'Oberserver design pattern}, ossia rendere i dati contenuti nel \textit{Model} \glo{observable} e i componenti della vista che li utilizzano (grafici e form per la loro personalizzazione) \glo{observer};
	\item Druid.js è importata nelle classi per eseguire la riduzione dimensionale e D3.js solo nei componenti della vista dedicati all'implementazione delle varie visualizzazioni.
\end{itemize}  %componente fornito da React per consumare il contesto creato



%\begin{figure}[hb]
%\includegraphics[width=15.8cm]{Images/Allegato Tecnico-Package 2}
%\centering
%\caption{Connessione tra client side e server side}
%\end{figure}

\newpage
\subsection{Diagrammi delle classi}
\subsubsection{View}
\begin{figure}[hb]
\includegraphics[width=\linewidth]{Images/Allegato Tecnico-Class View}
\centering
\caption{Diagramma delle classi della vista}
\end{figure}
Il diagramma delle classi della vista è costituito da tutti i componenti React che compongono l'interfaccia utente di \NomeProgetto{}.
Visitando dall'alto la gerarchia di componenti, il padre è \textbf{View}, il quale crea l'\textbf{Header} e i più importanti \textbf{Menu} e \textbf{Chart}. Riguardo a quest'ultimi due componenti:
\begin{itemize}
	\item Menu rappresenta per l'utente il punto d'accesso a tutte le funzionalità fornite dall'applicazione. Da qui sono infatti creati i vari \glo{modal} per il caricamento dei dati da file CSV (\textbf{LoadCSV}) e dal database (\textbf{LoadDataFromDB}), per il calcolo della distanza (\textbf{DistanceCalculation}) e per la riduzione dimensionale (\textbf{DimensionalReduction});
	\item Chart è il componente che crea i grafici e le relative form per settarne le preferenze. La scelta del grafico avviene nel Menu e comprende Scatterplot Matrix, Adjacency Matrix, Force Field, Heatmap e PLMA. 
\end{itemize}
La maggior parte dei componenti hanno un attributo ViewModel: questo indica che hanno una dipendenza verso di esso, usata per ottenere le informazioni di cui hanno bisogno contenute negli store (per esempio i dati o le dimensioni), oppure semplicemente come contenitore della propria \textit{presentation logic}. \\
In questo modo la vista risulta essere completamente separata dalla \textit{business logic} e si occupa solo di ritornare i vari elementi HTML che la costituiscono. 

\newpage
\subsubsection{Model}
\begin{figure}[hb]
\includegraphics[width=\linewidth]{Images/Allegato Tecnico-Store}
\centering
\caption{Diagramma delle classi del modello}
\end{figure}
Il diagramma delle classi del modello è costituito da un \textbf{RootStore}, il quale istanzia i tre store utilizzati in \NomeProgetto{}:
\begin{itemize}
	\item \textbf{DatasetStore} che contiene i dati caricati e le dimensioni da visualizzare, fornendo i metodi per recuperarli e aggiornarli;
	\item \textbf{PreferencesStore} che conserva le preferenze di visualizzazione di tutti i grafici utilizzati dall'utente; 
	\item \textbf{DistanceMatricesStore} che contiene le matrici delle distanze.
\end{itemize}
\textbf{Dimension} e \textbf{DistanceMatrix} sono i tipi che definiscono rispettivamente le dimensioni e le matrici delle distanze create dall'utente attraverso le varie funzioni di calcolo della distanza fornite (Euclidea, Chebyshev, Manhattan e Canberra). \\ \mbox{}\\
\textbf{PreferencesSPM}, \textbf{PreferencesFF}, \textbf{PreferencesAM}, \textbf{PreferencesPLMA}, \textbf{PreferencesHM} sono i tipi che definiscono le preferenze dei diversi grafici.
Gli store contengono gli attributi \glo{observable} che, nel momento in cui vengono modificati, grazie all'utilizzo di MobX, causano la rirenderizzazione dei componenti \glo{observer} della vista. 


\newpage
\subsection{Diagrammi di sequenza}
\begin{figure}[hb]
\includegraphics[width=12cm]{Images/Allegato Tecnico-Sequenza-DR}
\centering
\caption{Diagramma di sequenza che modella il processo di riduzione dimensionale}
\end{figure}
\newpage
\begin{figure}[hb]
\includegraphics[width=16cm]{Images/Allegato Tecnico-Sequenza-PLMApref}
\centering
\caption{Diagramma di sequenza che modella .............}
\end{figure}
\newpage
\begin{figure}[hb]
\includegraphics[width=15.5cm]{Images/Allegato Tecnico-Sequenza-PLMA}
\centering
\caption{Diagramma di sequenza che modella .............}
\end{figure}
\newpage
\newpage
\subsection{Architettura di dettaglio: Strategy pattern} 
\begin{figure}[hb]
	\includegraphics[width=17cm]{Images/StrategyPattern}
	\centering
	\caption{Strategy pattern per la scelta dell'algoritmo di riduzione dimensionale}
\end{figure}
Nella figura sopra è mostrata la funzionalità di riduzione dimensionale sui dati caricati dall'utente.
Per l'implementazione dei vari algoritmi viene utilizzato lo strategy pattern. In particolare:
\begin{itemize}
	\item \textbf{DimensionalReduction} è il componente React che si occupa solo di ritornare gli elementi HTML che compongono questa parte di vista;
	\item \textbf{DimensionalReductionVM} è il rispettivo view-model che ne contiene tutta la logica. Questo prende i dati dal RootStore (nello specifico dal \textit{DatasetStore}) e crea il contesto DimReduction;
	\item \textbf{DimReduction} si occupa di creare i parametri con i valori impostati dall'utente dalla vista. A seconda del valore del suo attributo \texttt{strategy} crea la corretta classe di parametri e chiama attraverso \texttt{executeStrategy()} il metodo \texttt{startDR()} sull'algoritmo scelto;
	\item \textbf{AlgorithmStrategy} è l'interfaccia implementata dalle famiglia di classi concrete degli algoritmi, facilitandone l'estensibilità.
\end{itemize} 

Infine \texttt{startDR()} esegue la riduzione dimensionale sui dati utilizzando i metodi forniti dalla libreria Druid.js e ritorna le nuove dimensioni.

