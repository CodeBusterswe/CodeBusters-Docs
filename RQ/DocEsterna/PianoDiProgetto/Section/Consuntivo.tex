\section{Consuntivo di periodo}

%Di seguito vengono indicate le spese effettivamente sostenute per ciascun periodo.

\subsection{Analisi dei requisiti}

Le ore di lavoro accumulate durante questa fase non sono rendicontate e vengono considerate come investimento. Pertanto la spesa non sarà a carico del proponente.

{
\setlength\arrayrulewidth{1pt}
\begin{longtable}{ C{4cm} C{3cm} C{3.5cm}} 
 	\rowcolor{coloreRosso}
 	\color{white}{\textbf{Ruolo}} &
 	\color{white}{\textbf{Ore}} &
 	\color{white}{\textbf{Costo €}} \\
 	
 	Responsabile & 25 \color{coloreRosso}{\textbf{(+5)}} & 900 \color{coloreRosso}{\textbf{(+150 €)}}\\
 	Amministratore & 45 & 900\\
 	Analista & 78 \color{coloreRosso}{\textbf{(+9)}} & 2175 \color{coloreRosso}{\textbf{(+225 €)}}\\
 	Progettista & - & -\\
 	Programmatore & - & -\\
 	Verificatore & 62 & 930\\
 	
	\hline 	
 	
 	\textbf{Totale preventivo} &
	210 &
 	4530 € \\		
 	
 	\textbf{Totale consuntivo} &
	224 &
 	4905 € \\	
 	
 	\textbf{Differenza} &
	+14 &
 	+375 € \\	
 	
 	\rowcolor{white}
 	\caption{Consuntivo del periodo di analisi dei requisiti}
\end{longtable}
}

\subsubsection{Ragione degli scostamenti}
Come si può notare dalla tabella, durante questa fase sono state richieste più ore rispetto a quelle preventivate.
\begin{itemize}
\item \textbf{Responsabile}: come riportato nella \S B, a causa dell'inesperienza, la seguente figura ha riscontrato problemi nella suddivisione del carico del lavoro, richiedendo un'analisi più minuziosa e diversi cambiamenti durante questo periodo;
\item \textbf{Analista}: come riportato nella \S B, durante la fase di analisi sono sorti diversi dubbi che hanno richiesto sia uno studio più approfondito, sia una comunicazione diretta con il proponente.
\end{itemize}

\subsubsection{Considerazioni rispetto al preventivo}
Nonostante la variazione del bilancio, non è necessario adottare alcuna contromisura poiché le ore di lavoro non sono rendicontate. Tuttavia il gruppo s'impegna ad evitare in futuro situazioni che potrebbero causare nuove variazioni al monte ore preventivato.

\subsection{Progettazione della technology baseline}
Il team si è impegnato ad apportare le modifiche segnalate alla precedente revisione sulla maggior parte dei documenti, oltre ad integrarli e ad individuare le tecnologie necessarie allo sviluppo del prodotto.

{
\setlength\arrayrulewidth{1pt}
\begin{longtable}{ C{4cm} C{3cm} C{3.5cm}} 
 	\rowcolor{coloreRosso}
 	\color{white}{\textbf{Ruolo}} &
 	\color{white}{\textbf{Ore}} &
 	\color{white}{\textbf{Costo €}} \\
 	
 	Responsabile & 4 & 120 €\\
 	Amministratore & 8 & 160 €\\
 	Analista & 19 \color{coloreRosso}{\textbf{(-2)}} & 475 € \color{coloreRosso}{\textbf{(-50 €)}}\\
 	Progettista & 24 & 528 €\\
 	Programmatore & 4 \color{coloreRosso}{\textbf{(+4)}} & 60 € \color{coloreRosso}{\textbf{(+60 €)}}\\
 	Verificatore & 27 \color{coloreRosso}{\textbf{(-5)}} & 405 € \color{coloreRosso}{\textbf{(-75 €)}}\\
 	
	\hline 	
 	
 	\textbf{Totale preventivo} &
	86 &
 	1748 € \\		
 	
 	\textbf{Totale consuntivo} &
	83 &
 	1683 € \\	
 	
 	\textbf{Differenza} &
	-3 &
 	-65 € \\	
 	
 	\rowcolor{white}
 	\caption{Consuntivo del periodo di progettazione della TB}
\end{longtable}
}

\subsubsection{Ragione degli scostamenti}
 
\begin{itemize}
\item \textbf{Analista (-2 ore)}: grazie alla buona base prodotta nella fase precedente, sono servite meno ore di quelle preventivate in quanto, l'\textit{Analisi dei Requisiti}, è stata per lo più integrata;

\item \textbf{Programmatore (+4 ore)}: la seguente figura ha occupato più tempo di quello che si era previsto per testare le tecnologie individuate;

\item \textbf{Verificatore (-5 ore)}: poiché il lavoro degli analisti e degli amministratori è stato svolto con precisione e in breve tempo, anche il lavoro dei verificatori è stato più veloce del previsto.
\end{itemize}

\subsubsection{Considerazioni rispetto al preventivo}

Il bilancio è positivo rispetto al preventivo per questo periodo. Non si ritiene comunque necessaria alcuna ripianificazione del prossimo incremento in quanto la somma risparmiata non è significativa. Inoltre, avendo raggiunto tutti gli obiettivi, l'avanzamento delle attività non ha subito rallentamenti.

\newpage

\subsection{Incremento I}
Durante questo incremento il gruppo si è occupato di terminare i lavori in fase di conclusione e ad iniziare lo sviluppo del \glo{PoC}. Tutti gli obiettivi sono stati raggiunti, rispettando le milestone interne.
{
\setlength\arrayrulewidth{1pt}
\begin{longtable}{ C{4cm} C{3cm} C{3.5cm}} 
 	\rowcolor{coloreRosso}
 	\color{white}{\textbf{Ruolo}} &
 	\color{white}{\textbf{Ore}} &
 	\color{white}{\textbf{Costo €}} \\
 	
 	Responsabile & 2 \color{coloreRosso}{\textbf{(+1)}} & 60 € \color{coloreRosso}{\textbf{(+30 €)}}\\
 	Amministratore & 12 & 240 €\\
 	Analista & 20 & 500 € \\
 	Progettista & 24 \color{coloreRosso}{\textbf{(-2)}} & 528 € \color{coloreRosso}{\textbf{(-44 €)}}\\
 	Programmatore & 14 & 210 € \\
 	Verificatore & 9  & 135 €\\
 	
	\hline 	
 	
 	\textbf{Totale preventivo} &
	81 &
 	1673 € \\		
 	
 	\textbf{Totale consuntivo} &
	80 &
 	1659 € \\	
 	
 	\textbf{Differenza} &
	-1 &
 	-14 € \\	
 	
 	\rowcolor{white}
 	\caption{Consuntivo dell'incremento I}
\end{longtable}
}

\subsubsection{Ragione degli scostamenti}

\begin{itemize}
\item \textbf{Responsabile (+1 ora)}: il tempo richiesto per ultimare il \textit{Piano di Progetto} è stato leggermente più del dovuto. Parte del rallentamento è stato causato anche dal complicato coordinamento delle attività da svolgere in questa fase;

\item \textbf{Progettista (-2 ore)}: la progettazione necessaria all'incremento, grazie all'esperienza maturata in precedenza, ha richiesto meno ore del previsto.

\end{itemize}

\subsubsection{Considerazioni rispetto al preventivo}

In base al risultato ottenuto, si è potuto constatare che le ore pianificate per questo incremento sono risultate nel complesso corrette. Nonostante le piccole variazioni, il bilancio risulta essere positivo rispetto al preventivo per questo periodo. Non si ritiene comunque necessaria alcuna ripianificazione in quanto la somma risparmiata non è significativa. Inoltre, visto il raggiungimento di tutti gli obiettivi, le attività future, per il momento, non subiranno modifiche.

\newpage

\subsection{Incremento II}
Durante questo incremento il team si è concentrato ad ultimare il \glo{PoC}, ovvero nell'implementazione del primo grafico che il prodotto dovrà offrire. Tutti gli obiettivi sono stati raggiunti.%, rispettando le milestone interne.

{
\setlength\arrayrulewidth{1pt}
\begin{longtable}{ C{4cm} C{3cm} C{3.5cm}} 
 	\rowcolor{coloreRosso}
 	\color{white}{\textbf{Ruolo}} &
 	\color{white}{\textbf{Ore}} &
 	\color{white}{\textbf{Costo €}} \\
 	
 	Responsabile & 6 & 180 € \\
 	Amministratore & 10 & 200 €\\
 	Analista & 4 \color{coloreRosso}{\textbf{(-1)}} & 100 € \color{coloreRosso}{\textbf{(-25 €)}}\\
 	Progettista & 12 & 264 € \\
 	Programmatore & 16 \color{coloreRosso}{\textbf{(+2)}} & 240 € \color{coloreRosso}{\textbf{(+30 €)}}\\
 	Verificatore & 9 \color{coloreRosso}{\textbf{(+4)}} & 135 \color{coloreRosso}{\textbf{(+60)}}€\\
 	
	\hline 	
 	
 	\textbf{Totale preventivo} &
	57 &
 	1119 € \\		
 	
 	\textbf{Totale consuntivo} &
	62 &
 	1194 € \\	
 	
 	\textbf{Differenza} &
	+5 &
 	+75 € \\	
 	
 	\rowcolor{white}
 	\caption{Consuntivo dell'incremento II}
\end{longtable}
}

\subsubsection{Ragione degli scostamenti}

\begin{itemize}
\item \textbf{Analista (-1 ore)}: grazie ad una comunicazione costante con il proponente, gli analisti hanno individuato rapidamente degli algoritmi di calcolo della distanza tra i dati;

\item \textbf{Programmatore (+2 ore)}: sono state riscontrate alcune difficoltà nell'implementare la visualizzazione \glo{\textit{Scatter Plot Matrix}} in modo dinamico, ovvero nella creazione automatica dei singoli \glo{\textit{Scatter Plot}} in base al set di dati caricato;

\item \textbf{Verificatore (+4 ore)}: quasi tutte le ore che non erano state utilizzate nel primo incremento sono state impiegate per garantire una maggiore qualità dei documenti.
\end{itemize}
\subsubsection{Considerazioni rispetto al preventivo}

Sulla base degli obiettivi prefissati, il consumo di risorse durante questo incremento è stato notevole e il bilancio risulta negativo rispetto al preventivo; tuttavia non si ritiene necessaria alcuna ripianificazione in quanto sono state utilizzate le ore risparmiate negli incrementi precedenti, mantenendo un bilancio in positivo rispetto al totale. Visto il raggiungimento di tutti gli obiettivi prefissati non sono previsti interventi sulla pianificazione del prossimo incremento.

\newpage

\subsection{Incremento III}
Durante questo incremento il team si è concentrato nel consolidare il design pattern architetturale scelto e ad implementare una componente per la riduzione dimensionale. Tutti gli obiettivi pianificati dal team sono stati raggiunti.
{
\setlength\arrayrulewidth{1pt}
\begin{longtable}{ C{4cm} C{3cm} C{3.5cm}} 
 	\rowcolor{coloreRosso}
 	\color{white}{\textbf{Ruolo}} &
 	\color{white}{\textbf{Ore}} &
 	\color{white}{\textbf{Costo €}} \\
 	
 	Responsabile & 4 & 120 € \\
 	Amministratore & 5 & 100 €\\
 	Analista & -& - \\
 	Progettista & 13 \color{coloreRosso}{\textbf{(+2)}} & 286 € \color{coloreRosso}{\textbf{(+44 €)}}\\
 	Programmatore & 23 & 345 € \\
 	Verificatore & 20 \color{coloreRosso}{\textbf{(-5)}} & 300 € \color{coloreRosso}{\textbf{(-75 €)}}\\
 	
	\hline 	
 	
 	\textbf{Totale preventivo} &
	65 &
 	1151 € \\		
 	
 	\textbf{Totale consuntivo} &
	62 &
 	1120 € \\	
 	
 	\textbf{Differenza} &
	-3 &
 	-31 € \\	
 	
 	\rowcolor{white}
 	\caption{Consuntivo dell'incremento III}
\end{longtable}
}

\subsubsection{Ragione degli scostamenti}


\begin{itemize}
\item \textbf{Progettista (+2 ore)}: sono state riscontrate varie difficoltà nel capire come suddividere la logica delle componenti grafiche e nell'integrare \glo{Mobx} nell'architettura;
\item \textbf{Verificatore (-5 ore)}: la quantità di codice prodotto non è stata significativa, di conseguenza non sono stati necessari molti interventi da parte della seguente figura.
\end{itemize}

\subsubsection{Considerazioni rispetto al preventivo}

Nonostante il lieve aumento nel consumo delle risorse da parte dei progettisti, il non utilizzo di tutte le ore da verificatore ha portato ad un esito positivo del bilancio. Tuttavia non si ritiene necessaria alcuna ripianificazione poiché la somma non è significativa e sarà certamente recuperata negli incrementi successivi.\\ 
Avendo raggiunto tutti gli obiettivi l'avanzamento delle attività non ha subito rallentamenti.

\newpage

\subsection{Incremento IV}
Durante questo incremento il team si è concentrato nell'ultimare la riduzione dimensionale tramite calcolo della distanza e nella realizzazione del grafico \glo{Adjacency Matrix}. Tutti gli obiettivi prefissati dal gruppo sono stati raggiunti.
{
\setlength\arrayrulewidth{1pt}
\begin{longtable}{ C{4cm} C{3cm} C{3.5cm}} 
 	\rowcolor{coloreRosso}
 	\color{white}{\textbf{Ruolo}} &
 	\color{white}{\textbf{Ore}} &
 	\color{white}{\textbf{Costo €}} \\
 	
 	Responsabile & 4 & 120 € \\
 	Amministratore & 5 & 100 €\\
 	Analista & - & - \\
 	Progettista & 23 & 506 € \\
 	Programmatore & 28 \color{coloreRosso}{\textbf{(+3)}} & 420 € \color{coloreRosso}{\textbf{(+45 €)}}\\
 	Verificatore & 14 & 210 €\\
 	
	\hline 	
 	
 	\textbf{Totale preventivo} &
	74 &
 	1356 € \\		
 	
 	\textbf{Totale consuntivo} &
	77 &
 	1401 € \\	
 	
 	\textbf{Differenza} &
	+3 &
 	+45 € \\	
 	
 	\rowcolor{white}
 	\caption{Consuntivo dell'incremento IV}
\end{longtable}
}

\subsubsection{Ragione degli scostamenti}

\begin{itemize}
\item \textbf{Programmatore (+3 ore)}: il calcolo delle distanze e l'implementazione dell'ordinamento della matrice nella nuova versione della libreria \glo{D3.js} hanno causato alcuni rallentamenti. 
\end{itemize}

\subsubsection{Considerazioni rispetto al preventivo}

In base al risultato ottenuto, si è potuto constatare che le ore pianificate per questo incremento sono risultate nel complesso corrette. Tuttavia il lieve aumento nel consumo delle risorse ha causato un esito negativo sul bilancio rispetto al preventivo.\\
Non si ritiene necessaria alcuna ripianificazione poiché la somma non è significativa ed è in parte bilanciata da quanto risparmiato nell’incremento precedente. Inoltre, avendo raggiunto tutti gli obiettivi l'avanzamento delle attività non ha subito rallentamenti. Vista la portata limitata del prossimo incremento, il team conta di bilanciare nelle prossime settimane l'eccesso di risorse consumate.

\newpage

\subsection{Incremento V}
Durante questo incremento il team si è concentrato principalmente nell'implementazione del grafico \glo{Force Field}.
{
\setlength\arrayrulewidth{1pt}
\begin{longtable}{ C{4cm} C{3cm} C{3.5cm}} 
 	\rowcolor{coloreRosso}
 	\color{white}{\textbf{Ruolo}} &
 	\color{white}{\textbf{Ore}} &
 	\color{white}{\textbf{Costo €}} \\
 	
 	Responsabile & 3 & 90 € \\
 	Amministratore & 6 & 120 €\\
 	Analista & - & - \\
 	Progettista & 20 \color{coloreRosso}{\textbf{(-3)}} & 440 € \color{coloreRosso}{\textbf{(-66 €)}}\\
 	Programmatore & 27 \color{coloreRosso}{\textbf{(-3)}} & 405 € \color{coloreRosso}{\textbf{(-45 €)}}\\
 	Verificatore & 12 & 180 €\\
 	
	\hline 	
 	
 	\textbf{Totale preventivo} &
	68 &
 	1235 € \\		
 	
 	\textbf{Totale consuntivo} &
	62 &
 	1124 € \\	
 	
 	\textbf{Differenza} &
	-6 &
 	-111 € \\	
 	
 	\rowcolor{white}
 	\caption{Consuntivo dell'incremento V}
\end{longtable}
}

\subsubsection{Ragione degli scostamenti}

\begin{itemize}
\item \textbf{Progettista (-3 ore)}: vista la similitudine a livello logico ed implementativo del grafico rispetto al precedente introdotto, non sono state utilizzate tutte le ore pianificate;
\item \textbf{Programmatore (-3 ore)}: grazie all'esperienza maturata nell'utilizzo della libreria D3.js, anche per la seguente figura non sono state consumate tutte le ore pianificate.
\end{itemize}

\subsubsection{Considerazioni rispetto al preventivo}

Il non utilizzo di tutte le ore pianificate ha portato ad un esito positivo del bilancio rispetto al preventivo di questo incremento. Tuttavia non si ritiene necessaria alcuna ripianificazione poiché la somma va a bilanciare in parte le spese eccessive sostenute negli incrementi precedenti. Inoltre parte della differenza sarà sicuramente recuperata nel prossimo incremento, vista la notevole portata. 

Degli obiettivi pianificati non è stato portato al grado di avanzamento desiderato la stesura dei manuali. Questo compito avrà massima priorità nel prossimo incremento, eventualmente assegnando più ore di lavoro. Ciò nonostante la codifica del prodotto software è a un buon punto e, grazie anche alla padronanza acquisita nell'utilizzo della libreria D3.js, il team prevede di riuscire a contenere i costi e a rispettare il preventivo nel prossimo incremento.

\newpage

\subsection{Incremento VI}
Durante questo incremento il team si è concentrato ad implementare i grafici \glo{Proiezione Lineare Multi Asse} e \glo{Heat Map}.

{
\setlength\arrayrulewidth{1pt}
\begin{longtable}{ C{4cm} C{3cm} C{3.5cm}} 
 	\rowcolor{coloreRosso}
 	\color{white}{\textbf{Ruolo}} &
 	\color{white}{\textbf{Ore}} &
 	\color{white}{\textbf{Costo €}} \\
 	
 	Responsabile & 3 \color{coloreRosso}{\textbf{(+1)}} & 60 € \color{coloreRosso}{\textbf{(+30 €)}}\\
 	Amministratore & 5 & 100 €\\
 	Analista & - & - \\
 	Progettista & 11 & 242 € \\
 	Programmatore & 33 \color{coloreRosso}{\textbf{(-2)}} & 495 € \color{coloreRosso}{\textbf{(-30 €)}}\\
 	Verificatore & 16 & 240 €\\
 	
	\hline 	
 	
 	\textbf{Totale preventivo} &
	68 &
 	1167 € \\		
 	
 	\textbf{Totale consuntivo} &
	67 &
 	1167 € \\	
 	
 	\textbf{Differenza} &
	-1 &
 	- € \\	
 	
 	\rowcolor{white}
 	\caption{Consuntivo dell'incremento VI}
\end{longtable}
}

\subsubsection{Ragione degli scostamenti}

\begin{itemize}
\item \textbf{Responsabile (+1 ora)}: vista la decisione, in seguito al colloquio con il proponente, nell'aggiunta di un nuovo grafico la seguente figura ha avuto bisogno di più tempo del previsto per assegnare i compiti da svolgere in modo ottimale;
\item \textbf{Programmatore (-2 ore)}: l'implementazione della visualizzazione \glo{PLMA}, che il gruppo riteneva complessa, è stata svolta utilizzando meno ore del previsto. 
\end{itemize}

\subsubsection{Considerazioni rispetto al preventivo}
Tutti gli obiettivi sono stati raggiunti con successo. Il bilancio dell'incremento risulta comunque essere in pari, nonostante il consumo delle risorse non sia stato pienamente rispettato.

L'andamento delle attività di progetto non ha subito rallentamenti e il gruppo è riuscito ad implementare quanto era stato previsto per l'incremento IX. Per questo motivo si è deciso di attuare una ripianificazione per i prossimi incrementi. Le ore recuperate verranno utilizzate in parte durante la fase di verifica e collaudo, mentre le rimanenti saranno usate nel prossimo incremento. Al termine dell'incremento VII infatti, il gruppo dovrà sostenere la verifica sulla \textit{product baseline} e, in caso di raffinamenti, potrebbero essere necessarie delle ore aggiuntive rispetto a quelle pianificate in precedenza. 

\subsection{Incremento VII}
Durante questo incremento il gruppo si è concentrato principalmente ad implementare la connessione con il database per il recupero dei dati.
{
\setlength\arrayrulewidth{1pt}
\begin{longtable}{ C{4cm} C{3cm} C{3.5cm}} 
 	\rowcolor{coloreRosso}
 	\color{white}{\textbf{Ruolo}} &
 	\color{white}{\textbf{Ore}} &
 	\color{white}{\textbf{Costo €}} \\
 	
 	Responsabile & 8 & 240 € \\
 	Amministratore & 10 & 200 €\\
 	Analista & - & - \\
 	Progettista & 20 & 440 € \\
 	Programmatore & 40 \color{coloreRosso}{\textbf{(-6)}} & 600 € \color{coloreRosso}{\textbf{(-90 €)}}\\
 	Verificatore & 19 \color{coloreRosso}{\textbf{(+6)}} & 285 € \color{coloreRosso}{\textbf{(+90 €)}}\\
 	
	\hline 	
 	
 	\textbf{Totale preventivo} &
	97 &
 	1765 € \\		
 	
 	\textbf{Totale consuntivo} &
	97 &
 	1765 € \\	
 	
 	\textbf{Differenza} &
	- &
 	- € \\	
 	
 	\rowcolor{white}
 	\caption{Consuntivo dell'incremento VII}
\end{longtable}
}

\subsubsection{Ragione degli scostamenti}

\begin{itemize}
\item \textbf{Programmatore (-6 ore)}: poiché durante il periodo di progettazione di \textit{technology baseline} le tecnologie usate erano già state testate, la seguente figura ha occupato meno tempo del previsto; 

\item \textbf{Verificatore (+6 ore)}: i raffinamenti attuati a livello architetturale hanno richiesto un intervento maggiore da parte del verificatore.
\end{itemize}

\subsubsection{Considerazioni rispetto al preventivo}

In base al risultato ottenuto, nonostante il consumo delle risorse previsto non sia stato rispettato, il bilancio rispetto al preventivo di questo incremento risulta essere in pari. Rispetto al bilancio complessivo, questa fase termina in positivo, seppur di una quantità poco significativa. \\
Il gruppo non è completamente soddisfatto della parte di test implementata fino a questo momento. Questa attività avrà una priorità maggiore nelle prossime settimane, eventualmente assegnando più ore di lavoro. Ciò nonostante il team ha raggiunto dei risultati appaganti e prevede di riuscire a contenere i costi e a rispettare i preventivi dei prossimi incrementi, oltre alle scadenze dettate dalla nuova pianificazione.

\newpage

\subsection{Riepilogo}

Di seguito viene riportata la tabella con il riepilogo dei preventivi e consuntivi riferiti agli incrementi sostenuti fino a questo momento, con relativo impatto sull'importo complessivo.
{
\setlength\arrayrulewidth{1pt}
\begin{longtable}{ C{1.5cm} | C{3.4cm} | C{3.4cm} | C{3cm} | C{3.4cm}} 
   \rowcolor{coloreRosso}
   \color{white}{\textbf{Incr.}} &
   \color{white}{\textbf{Preventivo}} &
   \color{white}{\textbf{Consuntivo}} & 
   \color{white}{\textbf{Sostenuto}} &
   \color{white}{\textbf{Scostamento}} \\
   
   TB & 1.748,00€ , 86 ore & 1.683,00 € , 83 ore & \begin{LARGE}\redcheck \end{LARGE} & -65,00€ , -3 ore\\
   
   I & 1.673,00€ , 81 ore & 1.659,00€ , 80 ore & \begin{LARGE}\redcheck \end{LARGE} & -14,00€ , -1 ora\\
   
   II & 1.119,00€ , 57 ore & 1.194,00€ , 62 ore & \begin{LARGE}\redcheck \end{LARGE} & +75,00€ , +5 ore\\
   
   III & 1.151,00€ , 65 ore & 1.120,00€ , 62 ore & \begin{LARGE}\redcheck \end{LARGE} & -31,00€ , -3 ore\\
   
   IV & 1.356,00€ , 74 ore & 1.401,00€ , 77 ore & \begin{LARGE}\redcheck \end{LARGE} & +45,00€ , +3 ore\\
   
   V & 1.235,00€ , 68 ore & 1.124,00€ , 62 ore & \begin{LARGE}\redcheck \end{LARGE} & -111,00€ , -6 ore\\
   
   VI & 1.167,00€ , 68 ore & 1.167,00€ , 67 ore & \begin{LARGE}\redcheck \end{LARGE} & 0,00€ , -1 ora\\
   
   VII & 1.765,00€ , 97 ore & 1.765,00€ , 97 ore & \begin{LARGE}\redcheck \end{LARGE} & 0,00 € , 0 ore\\
   
   \rowcolor{coloreRossoChiaro}
   \color{white}\textbf{Totale} & \color{white}11.214,00€ , 596 ore & \color{white}11.113,00€ , 590 ore & & \color{white}-101,00€ , -6 ore\\
   
   VIII & 1.331,00€ , 73 ore &  &  & \\
   
   V/C & 810,00€ , 45 ore & &  & \\
   
   \textbf{Totale} & \textbf{13.355,00€} \newline \textbf{714 ore} & \textbf{13.254,00€} \newline \textbf{708 ore} &  &\\
   
   \rowcolor{white}
   \caption{Riepilogo preventivo e consuntivo}
\end{longtable}
}

\newpage