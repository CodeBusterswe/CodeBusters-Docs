\section{Informazioni generali}
\begin{itemize}
\item \textbf{Motivo della riunione}: Discussione in merito ai test di accettazione.
\item \textbf{Luogo riunione}: videoconferenza tramite \glo{Skype}.
\item \textbf{Data}: \Data{}
\item \textbf{Orario d'inizio}: 14:30 
\item \textbf{Orario di fine}: 15:00 
\item \textbf{Partecipanti}:
	\begin{itemize}
	\item Membri del gruppo \textit{CodeBusters};
	\item Dott. Piccoli Gregorio.
	\end{itemize}
\end{itemize}

\section{Resoconto}
\begin{itemize}
 	\item \textbf{Discussione in merito ai test di accettazione}:
 	 il team ha fissato l'incontro per visionare con il proponente la lista di test di accettazione preparata, necessari nel momento del collaudo. Dopo aver discusso ciascun test, per alcuni di essi si è deciso di modificare la descrizione per renderla più precisa. Inoltre, su suggerimento del proponente, il gruppo \textit{CodeBusters} ha deciso di aggiungere qualche test per migliorare ulteriormente la qualità del prodotto e i requisiti da soddisfare.
 	  
	\item \textbf{Discussione sugli algoritmi implementati}:
	successivamente il gruppo ha presentato al proponente tutti gli algoritmi che sono stati implementati nel prodotto, sia quelli per la riduzione dimensionale che quelli per il calcolo della distanza. Per ognuno sono state discusse le caratteristiche ai fini delle visualizzazione, in modo da comprendere la bontà della scelta effettuata.  

 	\item \textbf{Aggiunta nuovo algoritmo}:
il proponente ha suggerito di aggiungere un nuovo algoritmo di riduzione dimensionale non lineare, \glo{UMAP}. Questo algoritmo permetterà di andare a coprire alcuni difetti di cui l'algoritmo di riduzione dimensionale T-SNE (correntemente applicato nel nostro prodotto) soffre con particolari tipologie di dataset. Il gruppo ha dunque deciso che la nostra applicazione può beneficiare dell'aggiunta di questa nuova funzionalità e l'algoritmo UMAP sarà implementato al prossimo incremento del prodotto.

\end{itemize}

\newpage