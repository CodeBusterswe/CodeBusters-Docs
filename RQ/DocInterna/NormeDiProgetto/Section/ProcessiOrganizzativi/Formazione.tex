\subsection{Formazione}

\subsubsection{Scopo}
Lo scopo di questa sezione è definire le norme che riguardano la formazione dei membri del gruppo e quindi lo studio delle tecnologie utilizzate per produrre i documenti e costruire il prodotto richiesto. 

\subsubsection{Aspettative}
Le aspettative sono:
\begin{itemize}
	\item Ottenere una buona conoscenza di \LaTeX{};
	\item Ottenere una buona conoscenza delle librerie, degli strumenti e del linguaggio utilizzati per la codifica.
\end{itemize}

\subsubsection{Descrizione}
Il processo di formazione è un processo per fornire e mantenere i componenti del gruppo qualificati.

\subsubsection{Modalità di formazione}
La formazione di ogni membro del gruppo avviene attraverso studio autonomo delle varie tecnologie, sia quelle proposte dal proponente, che quelle prese in considerazione dal gruppo. 

\subsubsection{Metriche}
Il processo di formazione non fa uso di metriche qualitative particolari. 

\subsubsection{Strumenti}
Alcuni strumenti, librerie, \glo{framework} necessitano di una ampia formazione. In particolare:
\begin{itemize}
	\item \textbf{React}: libreria \glo{JavaScript} per la creazione di interfacce utente;
	\item \textbf{D3.js}: libreria per la manipolazione del DOM (Document Object Model);
	\item \textbf{DruidJS}: libreria per la riduzione dimensionale sui dati caricati nella \glo{web app};
	\item \textbf{Node.js}: runtime \glo{JavaScript} utilizzato per creare applicazioni di rete scalabili.
\end{itemize}