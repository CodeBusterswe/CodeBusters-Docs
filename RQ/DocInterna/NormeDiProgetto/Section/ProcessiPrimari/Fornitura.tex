\subsection{Fornitura}
\subsubsection{Scopo}
In questa sezione vengono descritti i documenti che compongono il processo di fornitura, la loro struttura e gli strumenti utilizzati.

\subsubsection{Aspettative}
Le aspettative nell'applicazione del processo di fornitura sono:
\begin{itemize}
	\item Avere una chiara struttura dei documenti;
	\item Definire i tempi di lavoro;
	\item Chiarire dubbi e stabilire vincoli con il proponente.
\end{itemize}

\subsubsection{Descrizione}
Nel processo di fornitura si scelgono le procedure e le risorse atte a perseguire lo sviluppo del progetto. Dopo aver ricevuto le richieste del proponente, il gruppo redige uno studio di fattibilità e la fornitura può essere avviata per completare tali richieste.\\
Il proponente e il fornitore stipuleranno un contratto per la consegna del prodotto.\\
Si dovrà poi sviluppare un piano di progetto partendo dalla determinazione delle procedure e delle risorse necessarie.
Da quel momento fino alla consegna del prodotto il \PdP{} scaglionerà le attività da svolgere.

\subsubsection{Attività}
 Il processo di fornitura è composto dalle seguenti attività:
 \begin{itemize}
 \item Inizializzazione; 
\item Preparazione alla risposta;
\item Pianificazione;
\item Esecuzione e controllo;
\item Revisione e valutazione;
\item Rilascio.
 \end{itemize}

\subsubsection{Inizializzazione} 
Gli analisti devono effettuare una valutazione dei capitolati proposti e redigere un documento per ciascuno di essi. Il documento deve evidenziare gli aspetti negativi e positivi del capitolato, argomentando le motivazioni che portano il gruppo a scartarlo o a sceglierlo. Infine, deve essere indicato il capitolato scelto dal gruppo.

\subsubsection{Preparazione alla risposta}
Il gruppo deve redigere una lettera di presentazione con cui si dichiara l'impegno nella realizzazione del prodotto proposto dal capitolato scelto.

\subsubsection{Pianificazione}
Il gruppo deve redigere un piano per la gestione del progetto e della qualità. Alle sezioni \ref{pdq} e \ref{pdp} vengono descritte le sezioni che compongono i due documenti e quindi i loro contenuti.

\subsubsection{Esecuzione e controllo}
Il gruppo deve seguire ed implementare il piano di progetto e controllare l'avanzamento, i costi e la qualità dei prodotti attraverso tutto il loro intero ciclo di vita.

\subsubsection{Revisione e valutazione}
Il gruppo deve coordinare le revisioni delle attività svolte, deve eseguire la verifica e la validazione del prodotto garantendo che questo rispetti le aspettative del proponente.

\subsubsection{Rilascio}
Il gruppo deve rilasciare un prodotto conforme alle aspettative, presentando quindi al proponente ed al committente:
\begin{itemize}
	\item Codice sorgente del prodotto;
	\item Documentazione di prodotto.
\end{itemize}

\subsubsection{Metriche}
Il processo di fornitura non fa uso di metriche qualitative particolari.

\subsubsection{Strumenti}
Gli strumenti utilizzati in questo processo comprendono:
\begin{itemize}
	\item \textbf{Excel}: utilizzato per creare grafici, eseguire calcoli e presentare tabelle organizzative; 
	\begin{center}
		\textcolor{blue}{\url{https://www.microsoft.com/it-it/microsoft-365/excel}}
	\end{center}
	\item \textbf{Microsoft Planner}: utilizzato per gestire le \glo{task} che ciascun membro del gruppo deve svolgere. Permette di assegnare attività a specifici membri, suddividerle per categorie ed applicare molte altre personalizzazioni. Grazie a questa applicazione è possibile verificare che tutte le attività siano completate in linea con i tempi previsti; 
	\begin{center}
		\textcolor{blue}{\url{https://www.microsoft.com/it-it/microsoft-365/business/task-management-software}}
	\end{center}
	\item \textbf{GanttProject}: utilizzato per creare i grafici di \glo{Gantt} relativi alla pianificazione.
	\begin{center}
		\textcolor{blue}{\url{https://www.ganttproject.biz/}}
	\end{center}
\end{itemize}
