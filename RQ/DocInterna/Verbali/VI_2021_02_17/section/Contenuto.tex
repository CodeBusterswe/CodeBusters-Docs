\section{Informazioni generali}
\begin{itemize}
\item \textbf{Motivo della riunione}: 
\begin{itemize}
\item Punto della situazione sui documenti e sul PoC;
\item Decisione degli algoritmi di riduzione dimensionale da utilizzare;
\item Programmazione delle attività da svolgere e divisione dei ruoli.
\end{itemize}
\item \textbf{Luogo riunione}: videoconferenza tramite server \glo{Discord}.
\item \textbf{Data}: 17-02-2021
\item \textbf{Orario d'inizio}: 10:15;
\item \textbf{Orario di fine}: 12:00;
\item \textbf{Partecipanti}:
	\begin{itemize}
	\item \BM{}
	\item \SG{}
	\item \SP{}
	\item \SH{}
	\item \PA{}
	\item \ZM{}
	\item \RA{}
	\end{itemize}
\end{itemize}

\newpage
\section{Resoconto}
\begin{itemize}
\item \textbf{Proof of Concept}: visto l'imminente termine del primo incremento, il gruppo si è riunito per fare il punto della situazione in merito al PoC. Tutti gli obbiettivi fissati sono stati raggiunti: la lettura dei \glo{dataset} avviene correttamente e l'utente ha modo di selezionare le dimensioni interessate dall'analisi. Il team ha poi discusso su quali fossero le strutture dati più adatte per la gestione dei dati e come migliorare l'interfaccia grafica. 

\item\textbf{Documentazione}: successivamente si è fatto il punto della situazione in merito alla documentazione. L'\textit{Analisi dei Requisiti} ha subito ulteriori integrazioni sia in termini di requisiti che di casi d'uso, aumentando il livello di approfondimento. Alle \textit{Norme di Progetto} sono state aggiunte tutte le tecnologie scelte durante il periodo di progettazione della \glo{technology baseline}. I documenti rimanenti sono quasi ultimati, ad eccezione di qualche grafico di resoconto del \textit{Piano di Qualifica} da ultimare.

\item\textbf{Discussione sugli algoritmi di riduzione dimensionale}: durante questo periodo alcuni componenti si sono dedicati allo studio degli algoritmi presenti nella libreria \glo{Druid.js}, scelta per la riduzione dimensionale. La decisione è stata per l'utilizzo di algoritmi di riduzione non lineari, perché in grado di mantenere le strutture geometriche intrinseche dei dati e quindi analizzare \glo{dataset} più complessi. Una volta scelti gli algoritmi, sono stati individuati anche i parametri di personalizzazione specifici d'ognuno, in modo da rendere disponibile all'utente la loro impostazione.\\ Per quanto riguarda il calcolo delle distanze, il gruppo ha deciso di contattare il \glo{proponente} per risolvere alcuni dubbi a livello implementativo. Dopo aver inviato l' e-mail citata, sono stati scelti alcuni algoritmi per il calcolo delle distanze da rendere disponibili in futuro per la creazione di nuove dimensioni.

\item \textbf{Assegnazione dei ruoli}: il responsabile ha poi comunicato la nuova suddivisione dei ruoli. Si è deciso come suddividere i lavori da completare e sono state assegnate le \glo{tasks} per il prossimo incremento.
\end{itemize}

\newpage