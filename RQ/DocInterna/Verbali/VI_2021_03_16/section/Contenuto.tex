\section{Informazioni generali}
\begin{itemize}
\item \textbf{Motivo della riunione}: 
\begin{itemize}
\item Punto della situazione dopo la prima settimana;
\item Recupero incrementi del \PdP{};
\item Divisione del lavoro.
\end{itemize}
\item \textbf{Luogo riunione}: videoconferenza tramite server \glo{Discord}.
\item \textbf{Data}: \Data{};
\item \textbf{Orario d'inizio}: 09:00;
\item \textbf{Orario di fine}: 10:30;
\item \textbf{Partecipanti}:
	\begin{itemize}
	\item \SG{}
	\item \BM{}
	\item \SP{}
	\item \RA{}
	\item \SH{}
	\item \PA{}
	\item \ZM{}
	\end{itemize}
\end{itemize}

\newpage
\section{Resoconto}
\begin{itemize}
	\item \textbf{Punto della situazione dopo la prima settimana}: seppur con qualche difficoltà il gruppo è riuscito ad ottenere  l'architettura MVVM che ricercava e un'idea precisa della struttura della vista. Inoltre sono state individuate e codificate alcune delle classi che compongono il modello e il view-moodel, con i rispettivi metodi;
	\item \textbf{Scostamento dall'incremento in corso}: sfortunatamente il gruppo ora come ora si trova indietro rispetto all'incremento in corso. Questo è dovuto principalmente all'errata stima sul tempo necessario per la scelta e implementazione del design architetturale. Nonostante ciò le ore a disposizione sono tali da permettere al gruppo di riallinearsi nel breve termine.
	\item  \textbf{Divisione del lavoro}: a questo punto il gruppo è completamente operativo per lo sviluppo di tutti i componenti dell'applicazione, dal modello alla vista. È stato ritenuto opportuno mettere da parte momentaneamente la redazione dei documenti, in attesa delle valutazioni della RP e quindi delle correzioni da apportare. \\
	Il lavoro è stato suddiviso in questo modo:
	\begin{itemize}
		\item \BM{} e \SG{} si occuperanno del proseguimento della codifica del modello e view-model;
		\item \SH{} si occuperà del collegamento al database;
		\item \PA{} si occuperà della realizzazione dei primi grafici in D3.js;
		\item \ZM{} inizierà fin da subito con la fase di test, per rispettare il parallelismo tra sviluppo e analisi dinamica del prodotto;
		\item \SP{} e \RA{} si occuperanno della realizzazione dei componenti della vista.
	\end{itemize}
	La divisione dei compiti rimane comunque molto lasca. L'obiettivo è infatti quello che tutti possano sapere tutto del prodotto in via di sviluppo. L'utilizzo di Telegram e Discord è quindi fondamentale per continui confronti, veloci riunioni e possibili richieste d'aiuto in caso di difficoltà. \\
	L'idea è quella di ritrovarsi fra tre giorni con vista, model e view-model pronti (o comunque in uno stato funzionante e accettabile), per potersi quindi mettere in più persone nella realizzazione dei grafici obbligatori.      
\end{itemize}


