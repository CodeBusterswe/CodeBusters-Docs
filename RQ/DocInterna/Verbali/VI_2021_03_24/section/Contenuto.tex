\section{Informazioni generali}
\begin{itemize}
\item \textbf{Motivo della riunione}: 
\begin{itemize}
	\item Requisiti soddisfatti fino ad ora;
	\item Individuazione nuovi requisiti;
	\item Problemi riscontrati e attuali;
	\item Avanzamento nella scrittura dei documenti;
	\item Richiesta video chiamata con il proponente.
\end{itemize}
\item \textbf{Luogo riunione}: videoconferenza tramite server \glo{Discord}.
\item \textbf{Data}: \Data{};
\item \textbf{Orario d'inizio}: 09:00;
\item \textbf{Orario di fine}: 11:00;
\item \textbf{Partecipanti}:
	\begin{itemize}
	\item \BM{}
	\item \SG{}
	\item \SP{}
	\item \SH{}
	\item \PA{}
	\item \ZM{}
	\item \RA{}
	\end{itemize}
\end{itemize}

\newpage
\section{Resoconto}
\begin{itemize}
	\item \textbf{Requisiti soddisfatti fino ad ora}: il gruppo nell'ultima settimana si è mosso in modo molto dinamico ed efficace nella codifica delle principali funzionalità del prodotto. Sono stati infatti soddisfatti diversi requisiti obbligatori, in particolare quelli riguardanti la visualizzazione dei grafici. Sono stati codificati i componenti principali che costituiscono la vista, la classe del ViewModel e del Model. Il gruppo è riuscito a lavorare in modo coordinato grazie soprattutto ad una comunicazione giornaliera per evitare che nessuno si trovasse con niente da fare oppure bloccato nel suo compito.
	\item \textbf{Individuazione nuovi requisiti}: l'inizio della codifica effettiva del prodotto ha portato all'introduzione di nuovi requisiti. In particolare il gruppo ha deciso di rinominare il grafico fino ad ora conosciuto come "Heat Map" in "Adcency Matrix". \\Nonostante il grafico Heat Map possa svolgere la funzione del grafico Adjacency Matrix, il gruppo ha ritenuto opportuno implementarli separatamente, in modo da non limitare l'utilizzo di particolari dataset dell'utente. L'Heat Map risulta adesso un nuovo grafico (già implementato), leggermente diverso, che comunque aumenta il valore del prodotto sviluppato. Adjacency Matrix verrà implementato nell'immediato, in modo da esporre la proposta al proponete ed avere un giudizio;
	\item \textbf{Problemi riscontrati e attuali}: i problemi principali che il gruppo ha riscontrato durante l'ultimo periodo sono stati:
	\begin{itemize}
		\item Difficoltà nella realizzazione dei grafici richiesti, in particolare la gestione di grandi moli di dati che appesantiscono notevolmente la visualizzazione grafica e la gestione della matrice delle distanze per i grafici che dipendono da tale concetto;
		\item Implementazione del grafico PLMA. È stata infatti richiesta una lunga fase di ricerca e sperimentazione per arrivare ad un risultato soddisfacente. Questo ha costretto ad un rallentamento generale che però attualmente è stato risanato;
		\item Difficoltà nell'utilizzo di Jest e React-testing-library per lo svolgimento dei test. L'analisi dinamica è infatti tutt'ora leggermente indietro rispetto allo sviluppo.
	\end{itemize}
	Il problema dei test rimane attuale e verrà risolto sicuramente prima dell'ingresso in RQ, con l'obiettivo di arrivare almeno alla soglia minima di \textit{code coverage} settata al 60\%. \\
	Per il resto rimangono da sistemare alcuni bug, mentre tutti i requisiti non soddisfatti fino ad ora sono appositamente mantenuti in stallo, in attesa del resoconto del prof. Cardin in \textit{Product Baseline} sull'architettura da noi implementata;
	\item \textbf{Avanzamento nella scrittura dei documenti}: arrivato ad un buon punto nella fase di codifica, il gruppo ora è pronto a rimettere mano ai documenti, occupandosi di apportare:
	\begin{itemize}
		\item Le modifiche richieste in seguito alle valutazioni dell'RP;
		\item L'integrazione dei nuovi requisti e la modifica di alcuni casi d'uso nell'\AdRv{};
		\item L'aggiunta di alcune norme riguardo la scrittura di codice nelle \NdP{};
		\item L'aggiornamento del cruscotto di valutazione e dei test nel \PdQv{}.
	\end{itemize}	  
	È inoltre già operativa la scrittura dei manuali e dell'allegato tecnico. Quest'ultimo in particolare ha prevalenza essendo la sua consegna più vicina;
	\item \textbf{Richiesta meeting con il proponente}: il gruppo ha ritenuto opportuno richiedere un incontro con il proponente per mostrare il prodotto nello stato attuale, con l'idea di risolvere gli ultimi dubbi. 
\end{itemize}
