\section*{D}
\markright{}
\addcontentsline{toc}{section}{D}

\subsection*{D3.js}
È una libreria \glo{JavaScript} per creare visualizzazioni dinamiche ed interattive partendo da dati organizzati, visibili attraverso un comune browser.

\subsection*{Database}
Letteralmente "base di dati". Rappresenta la versione digitale di un archivio di informazioni, ossia memorizza e organizza grandi moli di dati all'interno di dischi rigidi.

\subsection*{Dataset}
Collezione di dati ordinati in tabella in cui ogni colonna rappresenta una dimensione e ogni riga un membro del dataset.

\subsection*{Deep learning}
Insieme di tecniche basate su reti neurali artificiali organizzate in diversi strati, dove ogni strato calcola i valori per quello successivo affinché l'informazione venga elaborata in maniera sempre più completa.

\subsection*{Dendrogramma}
Nelle tecniche di clustering viene utilizzato per fornire una rappresentazione grafica del processo di raggruppamento delle istanze.

\subsection*{Design pattern}
Un design pattern descrive una soluzione generale a un problema di progettazione ricorrente, gli attribuisce un nome, astrae e identifica gli aspetti principali della struttura utilizzata per la soluzione del problema, identifica le classi e le istanze partecipanti e la distribuzione delle responsabilità, descrive quando e come può essere applicato. 

\subsection*{Desk check}
Un desk check è un processo informale non computerizzato per la verifica della programmazione e della logica di un algoritmo, prima che il programma venga avviato, per l' individuazione di errori e bug che ne impedirebbero il corretto funzionamento.

\subsection*{Discord}
Applicazione \glo{multipiattaforma} basta sulla tecnologia VoIP (Voice over Internet Protocol) che consente di creare gruppi di messaggistica e videoconferenze. Questa applicazione funziona sia sui dispositivi mobili sia su browser.

\subsection*{Distanza Euclidea}
Distanza tra due punti corrispondente alla lunghezza del segmento avente per estremi gli stessi punti. La particolarità è la possibilità di calcolarla tra più punti di 'n' dimensioni differenti.

\subsection*{Docker}
Progetto \glo{open source} che automatizza il rilascio di applicazioni all'utente all'interno di contenitori software.


