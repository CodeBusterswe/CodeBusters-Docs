\section*{G}
\markright{}
\addcontentsline{toc}{section}{G}

\subsection*{Gantt}
Un diagramma di Gantt è uno strumento utile per la pianificazione dei progetti. Attraverso una panoramica dei compiti programmati, tutte le parti interessate sono a conoscenza dei compiti e delle rispettive scadenze. Un diagramma di Gantt mostra le date di inizio e fine di un progetto, da quali attività è composto il progetto, le attività assegnate a ciascuna persona, le date previste per l'inizio e la fine delle attività, una stima di quanto tempo durerà ogni attività, come le attività si sovrappongono e/o sono collegate tra loro.

\subsection*{Ggobi}
Programma \glo{open source} per la visualizzazione di dati con tante dimensioni, fornendo grafici dinamici e interattivi. Può essere utilizzato anche come libreria in diversi linguaggi di scripting.

\subsection*{Git}
Software di controllo di versione distribuito utilizzabile da interfaccia a riga di comando, creato da Linus Torvalds nel 2005. 

\subsection*{GitHub}
Servizio di hosting per sviluppatori. Fornisce uno strumento di controllo versione e permette lo sviluppo distribuito del software.

\subsection*{GitLab}
Piattaforma web \glo{open source}che permette la gestione di \glo{repository} \glo{Git} e di funzioni trouble ticket. Fornisce uno strumento di controllo versione e permette lo sviluppo distribuito del software.

\subsection*{Google Chrome}
Browser web \glo{multipiattaforma}, basato su browser Chronium, creato e rilasciato da Google nel 2008. Dal 2016 è il browser web più utilizzato (poco meno del 70\% dell'utenza).

\subsection*{Google Drive}
Servizio web, in ambiente cloud computing, di memorizzazione e sincronizzazione online introdotto da Google. I suoi servizi comprendono il file hosting, il file sharing e la modifica collaborativa di documenti. Può essere usato via Web, caricando e visualizzando i file tramite il web browser, oppure tramite l'applicazione installata su computer, che sincronizza automaticamente una cartella locale del file system con quella condivisa.