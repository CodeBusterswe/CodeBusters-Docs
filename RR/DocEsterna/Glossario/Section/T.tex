\section*{T}
\markright{}
\addcontentsline{toc}{section}{T}

\subsection*{Task}
Complesso di azioni dirette alla realizzazione di un obiettivo.

\subsection*{Technology baseline}
Dimostrare, al committente e a noi stessi, di disporre delle tecnologie, librerie e \glo{framework} necessari per lo sviluppo del prodotto, portando una \glo{baseline}. Ne dimostriamo l'adeguatezza tramite \glo{Proof of Concept} ed è soggetta a verifica Agile.

\subsection*{TensorFlow}
Libreria software \glo{open source} per l'apprendimento automatico, che fornisce moduli sperimentati e ottimizzati, utili nella realizzazione di algoritmi per diversi tipi di compiti percettivi e di comprensione del linguaggio.

\subsection*{Telegram}
Applicazione \glo{multipiattaforma} che permette di effettuare chiamate ed inviare messaggi a più utenti in tempo reale.

\subsection*{Texmaker}
È un editor \LaTeX gratuito, moderno e \glo{multipiattaforma} per sistemi Linux, macOS e Windows, che integra molti strumenti necessari per sviluppare documenti con \LaTeX, in una sola applicazione.

\subsection*{Tomcat}
Server web che fornisce una piattaforma software per l'esecuzione di applicazioni web sviluppate in linguaggio \glo{Java}. La sua distribuzione standard include anche le funzionalità di web server tradizionale, che corrispondono al prodotto Apache. 

\subsection*{Typescript}
Linguaggio di programmazione \glo{open source} sviluppato da Microsoft. Si tratta di un Super-set di \glo{JavaScript} che basa le sue caratteristiche su ECMAScript 6.

