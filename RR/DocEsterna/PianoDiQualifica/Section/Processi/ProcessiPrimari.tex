
\subsection{Introduzione}
Per garantire la qualità dei processi è stato utilizzato come riferimento lo standard ISO/IEC/IEEE 12207:1995. Tra i processi elencati dal modello, il gruppo ne ha scelti alcuni che sono stati semplificati e adattati alle necessità del progetto. Questa sezione espone i valori di qualità accettabili sulla base di metriche elencate nelle \NdPv{v 1.0.0}. Di seguito sono esposti i processi selezionati.

\subsection{Processi Primari}

\subsubsection{Fornitura}
Il processo di fornitura consiste nel scegliere le procedure e le risorse atte a perseguire lo sviluppo del progetto. Successivamente sono riportate le metriche utilizzate, che possono essere consultate dettagliatamente nel documento \NdPv{v 1.0.0}.

\paragraph{Metriche}
\begin{itemize}
	\item MPC1 - Schedule Variance (SV);
	\item MPC2 - Budget Variance (BV);
	\item MPC3 - Budget at Completion (BAC);
	\item MPC4 - Earned Value (EV) o Budgeted Cost of Work Performed (BCWP);
	\item MCP5 - Planned Value (PV) o BCWS (Budgeted Cost of Work Scheduled).
\end{itemize}

\paragraph{Valori ammissibili}
{
\renewcommand{\arraystretch}{1.5}
\centering
\begin{longtable}{C{2.5cm} C{7cm} C{4cm}}
\rowcolor{coloreRosso}
\textcolor{white}{\textbf{Metrica}}&
\textcolor{white}{\textbf{Valore accettabile}}&
\textcolor{white}{\textbf{Valore ottimale}}\\	
\endhead
\endfoot
\rowcolor{white}\caption{Metriche di qualità del processo di fornitura}
\endlastfoot

MPC1 & $\geq 5\%$ & $\geq 0\%$ \\
MPC2 & $\geq 5\%$ & $\geq 0\%$ \\
MPC3 & prev. - 5 \% $ \leq $ BAC $ \leq $ prev. + 5\% & Pari al preventivo  \\
MPC4 & $\geq 0$  & $\geq 0$ \\
MPC5 & $\geq 0$  & $\geq 0$ \\
\end{longtable}
}



\subsubsection{Sviluppo}
Il processo di sviluppo contiene le attività e i compiti per realizzare il prodotto software richiesto. Successivamente sono riportate le metriche utilizzate, che possono essere consultate dettagliatamente nel documento \NdPv{v 1.0.0}.

\paragraph{Metriche}
\begin{itemize}
	\item MPC6 - Requirements stability index (RSI);
	\item MPC7 - Satisfied obligatory requirements (SOR).
\end{itemize}


\paragraph{Valori ammissibili}
{
\renewcommand{\arraystretch}{1.5}
\centering
\begin{longtable}{C{2.5cm} C{4cm} C{4cm}}
\rowcolor{coloreRosso}
\textcolor{white}{\textbf{Metrica}}&
\textcolor{white}{\textbf{Valore accettabile}}&
\textcolor{white}{\textbf{Valore ottimale}}\\	
\endhead
\endfoot
\rowcolor{white}\caption{Metriche di qualità del processo di sviluppo}
\endlastfoot

MPC6 & 70\% & 100\% \\
MPC7 & 100\% & 100\%
\end{longtable}
}

