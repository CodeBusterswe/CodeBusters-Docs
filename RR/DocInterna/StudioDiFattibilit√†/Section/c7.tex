\section{C7 - SSD: soluzioni di sincronizzazione desktop}

\subsection{Descrizione del capitolo}
Il capitolato presentato pone come obiettivo finale la realizzazione di un algoritmo di sincronizzazione desktop in grado di garantire il salvataggio in \glo{cloud} di qualsiasi lavoro, sincronizzando contemporaneamente i cambiamenti. In aggiunta a ciò, lo sviluppo di un'interfaccia \glo{multipiattaforma} per l'uso dell'algoritmo e l'utilizzo di quest'ultimo con il prodotto dell'azienda \textit{Zextras Drive}.\\
L'azienda proponente è \textit{Zextras}.

\subsection{Tecnologie coinvolte}
L'azienda non impone tecnologie specifiche per lo sviluppo della parte \glo{back-end} o della \glo{UI}, ma vi sono comunque delle scelte preferenziali da considerare nello svolgimento del progetto:
\begin{itemize}
\item Linguaggio \glo{Python} per lo sviluppo dell'algoritmo di sincronizzazione;
\item \glo{Framework} \glo{Qt} per lo sviluppo dell'interfaccia e del controller d'architettura.
\end{itemize}

\subsection{Vincoli}
Il proponente richiede di rispettare alcuni vincoli sull'architettura del progetto: 
\begin{itemize}
\item L'algoritmo di sincronizzazione e l'interfaccia utente devono essere utilizzabili nei tre principali sistemi operativi desktop (Mac, Windows, Linux) senza richiedere all'utente l'installazione manuale di ulteriori prodotti per il funzionamento; 
\item Forte distinzione tra interfaccia e algoritmo di sincronizzazione; 
\item Utilizzare il pattern architetturale \glo{MVC}.
\end{itemize}

\subsection{Aspetti positivi}
\begin{itemize}
\item Il linguaggio proposto, \glo{Python}, è in grande ascesa negli ultimi anni e sono molte le tecnologie che lo stanno implementando o che lo hanno già fatto. Il gruppo ritiene che sia di grande aiuto apprenderlo;
\item Il capitolato propone l'utilizzo del \glo{framework} \glo{Qt}, che è già familiare a tutto il gruppo di progetto.
\end{itemize}

\subsection{Aspetti critici}
\begin{itemize}
\item Lo sviluppo di un algoritmo di sincronizzazione a partire da zero;
\item Il \glo{framework} \glo{Qt} è già conosciuto. Questo rende il capitolato meno interessante per quanto riguarda l'opportunità di apprendere nuove tecnologie.
\end{itemize}

\subsection{Conclusioni}
Il capitolato non ha destato molto la curiosità del gruppo; l'utilizzo del linguaggio \glo{Python} è l'unica nota d'interesse sollevata dal gruppo, ma questo è proposto anche in altri capitolati che hanno catturato maggiormente l'attenzione.